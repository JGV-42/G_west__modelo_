\begin{abstract}
    Este artículo presenta un modelo de inteligencia artificial diseñado para predecir la cantidad de carbono capturado en proyectos de forestación y reforestación, a partir de variables estructurales, climáticas y espectrales. La motivación principal parte del hecho de que, durante su desarrollo, los árboles capturan dióxido de carbono ($CO_2$) en su biomasa; por tanto, modelar el crecimiento forestal permite inferir la cantidad de carbono absorbido a lo largo del tiempo.

    \medskip

    El modelo desarrollado responde a la siguiente pregunta: \textit{Dado un cultivo forestal con características concretas de vegetación, clima y terreno, ¿cuánto $CO_2$ contendrá pasados unos años?} Esta capacidad de predicción permite, además, integrarlo en marcos de optimización forestal, abordando cuestiones clave como:
    
    \begin{itemize}
        \item ¿Qué especie plantar en un terreno específico para maximizar la captura de carbono?
        \item ¿En qué terreno plantar una especie determinada para alcanzar mayor rendimiento en fijación de carbono?
    \end{itemize}

    \medskip

    El modelo se entrena sobre datos reales multifuente: registros históricos del Inventario Forestal Nacional (IFN2, IFN3 e IFN4), variables climáticas derivadas de Copernicus y métricas espectrales procedentes de imágenes Landsat. Los datos estan limitados al territorio español, dando lugar a un modelo adaptado a las características ambientales y  orográficas del país. La estructura jerárquica del modelo por parcela, especie y clase diamétrica permite estimaciones a alta resolución espacial y taxonómica. Se implementaron y evaluaron distintos enfoques de aprendizaje supervisado, incluyendo modelos basados en árboles, redes neuronales y técnicas probabilísticas, seleccionando finalmente la opción con mejor rendimiento general.

    \medskip

    \textbf{Palabras clave:} créditos de carbono, inteligencia artificial, forestación, reforestación, modelado predictivo, cambio climático.
\end{abstract}
