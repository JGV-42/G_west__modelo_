\section{Recomendaciones para Futuras Investigaciones}

% A partir de los resultados obtenidos y de las limitaciones identificadas durante el desarrollo de este trabajo, se proponen a continuación varias líneas de investigación que podrían contribuir a mejorar y ampliar el alcance del modelo desarrollado.

% En primer lugar, sería recomendable ampliar y diversificar la base de datos empleada. La incorporación de futuras ediciones del Inventario Forestal Nacional permitiría reforzar la dimensión temporal del conjunto de entrenamiento y evaluar con mayor detalle la estabilidad del modelo ante horizontes temporales más largos. Así mismo, la extensión del estudio a otras regiones bioclimáticas, tanto dentro como fuera del ámbito nacional, permitiría analizar la capacidad de generalización del modelo y su adaptabilidad a contextos ecológicos distintos.

% En relación con las variables explicativas, futuras investigaciones podrían explorar la inclusión de nuevas fuentes de información, como datos de teledetección de mayor resolución espacial o temporal (por ejemplo, LIDAR aéreo o satelital). Del mismo modo, la incorporación explícita de variables relacionadas con perturbaciones (incendios, plagas, siembras, talas\dots) podría mejorar la capacidad del modelo para capturar dinámicas no lineales en la acumulación de carbono.

% Desde el punto de vista metodológico, sería de interés evaluar arquitecturas de aprendizaje más avanzadas, como modelos de deep learning especializados en series temporales o enfoques híbridos que combinen modelos mecanicistas de crecimiento forestal con técnicas de aprendizaje automático. Así mismo, el análisis sistemático de la incertidumbre asociada a las predicciones, por ejemplo, mediante enfoques bayesianos o técnicas de quantile regression, permitiría proporcionar intervalos de confianza, un aspecto especialmente relevante para aplicaciones vinculadas a la certificación de créditos de carbono.

% Otra línea de trabajo prometedora consiste en profundizar en la interpretabilidad de los modelos. El uso de técnicas explicativas avanzadas podría facilitar una comprensión más detallada del papel de cada variable en la predicción final, reforzando la confianza de técnicos y gestores en el uso del modelo y favoreciendo su adopción en contextos operativos.

% Por último, desde una perspectiva aplicada, sería recomendable desarrollar herramientas que faciliten la transferencia del modelo a usuarios finales. Esto podría materializarse en una interfaz gráfica o plataforma web que permita introducir escenarios de plantación y obtener estimaciones de captura de carbono de forma directa. En este contexto, también podría explorarse la integración del modelo con sistemas de registro y trazabilidad, como tecnologías de blockchain, para apoyar la gestión y certificación de créditos de carbono de manera transparente y verificable.

% En conjunto, estas líneas de investigación futura permitirían consolidar y ampliar el impacto del modelo propuesto, reforzando su utilidad científica, técnica y aplicada en el ámbito de la gestión forestal y la mitigación del cambio climático.