% sections/06_resultados.tex
\section{Resultados}
Presentar los resultados obtenidos al aplicar el modelo a los datos de entrada. Incluir gráficos y tablas que ayuden a ilustrar el rendimiento del modelo.
% Este es el corazón de tu sección de resultados.
% Incluye:
% - Métricas de rendimiento: Presenta los valores numéricos de las métricas que definiste en la sección de Metodología (RMSE, MAE, R², etc.) para los conjuntos de entrenamiento y prueba.
% - Gráficos de rendimiento:
%   - Curvas de aprendizaje (pérdida vs. época para entrenamiento y validación).
%   - Gráficos de dispersión de valores predichos vs. valores reales.
%   - Histogramas de errores o residuos.
%   - Si aplicable, mapas de calor o visualizaciones de los resultados georreferenciados.
% - Tablas:
%   - Resumen de las métricas clave.
%   - Comparación del rendimiento de diferentes modelos si probaste varios.
% - Ejemplos de predicciones: Puedes mostrar algunos ejemplos concretos de predicciones del modelo.

% Ejemplo de inclusión de imagen:
% \begin{figure}[H]
%     \centering
%     \includegraphics[width=0.8\textwidth]{images/curva_perdida.png} % Asegúrate de la ruta correcta
%     \caption{Curva de pérdida durante el entrenamiento y la validación.}
%     \label{fig:curva_perdida}
% \end{figure}

% Ejemplo de tabla:
% \begin{table}[H]
%     \centering
%     \caption{Métricas de rendimiento del modelo final.}
%     \label{tab:performance_metrics}
%     \begin{tabular}{lcc}
%         \toprule
%         Métrica & Conjunto de Entrenamiento & Conjunto de Prueba \\
%         \midrule
%         RMSE    & \SI{0.123}{tCO2} & \SI{0.156}{tCO2} \\
%         MAE     & \SI{0.098}{tCO2} & \SI{0.112}{tCO2} \\
%         R$^2$   & 0.95            & 0.88             \\
%         \bottomrule
%     \end{tabular}
% \end{table}