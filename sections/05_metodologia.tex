\section{Metodología}

Esta sección describe el procedimiento seguido para el entrenamiento y validación de los modelos predictivos desarrollados.
La metodología se fundamenta en la identificación de los factores que determinan el crecimiento forestal y, en consecuencia, la capacidad de los ecosistemas para capturar carbono a lo largo del tiempo.
El enfoque integra información estructural, climática y espectral procedente del Inventario Forestal Nacional (IFN) y de otras fuentes ambientales, con el propósito de construir modelos robustos que permitan predecir el contenido de carbono acumulado en la biomasa viva.


El carbono fijado por los árboles se acumula progresivamente en su biomasa, en función del tamaño y vigor de los individuos, los cuales están condicionados por variables ambientales, topográficas y de competencia intraespecífica.
Las condiciones meteorológicas, como la temperatura y la precipitación, inciden directamente en la fotosíntesis y en la disponibilidad hídrica;
la orientación, la pendiente y la altitud modifican la radiación incidente y el microclima local;
mientras que la densidad de árboles por unidad de superficie determina el nivel de competencia por los recursos, variando según la especie y su tolerancia ecológica \cite{IPCC2006}.


A partir de estos fundamentos, se construyó una base de datos relacional que integra información forestal, climática y espectral a nivel de parcela, especie y clase diamétrica.
Esta estructura permite caracterizar con precisión la dinámica del bosque entre inventarios sucesivos y alimentar modelos predictivos capaces de estimar el contenido futuro de carbono a partir de las condiciones observadas en el pasado.


\subsection{Origen y estructura de los datos}
% Origen (IFN2/3/4 + clima + índices)
% Estructura relacional (parcelas, inventarios, especie, CD, estación, árbol)
% Referencia al esquema (figura) y a meta_variables


La base de datos empleada en este trabajo integra información forestal, climática y espectral estructurada en torno a la parcela como unidad básica. Cada parcela se describe mediante sus coordenadas geográficas, características edáficas y su evolución a través de distintos inventarios (IFN2, IFN3, IFN4).


Los datos forestales incluyen información por especie y clase diamétrica, como número de pies, volumen con y sin corteza, área basimétrica, carbono aéreo, radical y total. Estos valores permiten caracterizar con precisión la estructura y crecimiento de la vegetación.


A cada parcela se asocian también estadísticas climáticas agregadas por estación e inventario: temperaturas (superficie, aire y subsuelo) y precipitaciones, resumidas mediante métricas como media, máxima, mínima y desviación típica.


Finalmente, se incorporan índices espectrales derivados de imágenes satelitales (NDVI, EVI, NDII, GNDVI), que permiten cuantificar propiedades biofísicas de la vegetación:
\begin{itemize}
    \item \textbf{NDVI (Normalized Difference Vegetation Index):} estima la actividad fotosintética.
    \item \textbf{EVI (Enhanced Vegetation Index):} mejora la sensibilidad en zonas densamente vegetadas.
    \item \textbf{NDII (Normalized Difference Infrared Index):} refleja el contenido hídrico de la vegetación.
    \item \textbf{GNDVI (Green NDVI):} variante del NDVI basada en la banda verde, sensible al clorofila.
\end{itemize}


\subsubsection{Estrutura de la base de datos}
Estos datos se organizan en las siguientes entidades troncales:

\begin{itemize}
    \item \textbf{parcelas}: icontiene la información básica de localización y características edáficas de cada parcela.
    \item \textbf{parcela\_inventario}: describe el estado de cada parcela en un inventario determinado (\texttt{parcela\_id}, \texttt{inventario\_id}), incluyendo atributos edáficos y de contexto (p. ej., \texttt{nivel1\_id}, \texttt{textura\_id}).
    \item \textbf{parcela\_inventario\_especie}: detalla la presencia y condición de cada especie dentro de una parcela e inventario, incorporando descriptores de masa y tratamientos silvícolas.
    \item \textbf{parcela\_inventario\_especie\_cd}: describe las poblaciones arbóreas por parcela, especie y \emph{clase diamétrica} (\texttt{cd\_id}): n.º de pies (\texttt{npies}), área basimétrica (\texttt{abas}), volúmenes (\texttt{vcc}, \texttt{vsc}, \texttt{vle}), incrementos (\texttt{iavc}) y carbono (\texttt{ca}, \texttt{cr}).
    \item \textbf{parcela\_especie\_arbol}: caracteriza los pies mayores identificados por parcela y especie en el inventario cuarto. Recoge las caracteristicas particulares de cada pie como altura (\texttt{ht}), diámetros (\texttt{dn1} y \texttt{dn2}), ubicación respecto del centro de la parcela (\texttt{rumbo}, \texttt{distancia}), volumen (\texttt{vcc, vsc, vle}), incremento (\texttt{iavc}) y carbono (\texttt{ca, cr}).
    \item \textbf{parcela\_inventario\_estacion}: almacena agregados climático-biofísicos por estación (\texttt{estacion\_id}) en la misma granularidad parcela–inventario, incluyendo variables como precipitación (\texttt{PR}) y temperatura (\texttt{T2M, SKT, STL*}), junto a índices de vegetación (NDVI, EVI, NDII, GNDVI).
    \item \textbf{especies} y \textbf{grupos}: recogen la información taxonómica y su clasificación jerárquica, estableciendo la relación entre especies individuales y grupos funcionales.
\end{itemize}

Cada variable categórica posee una tabla de catálogo propia (\texttt{cat\_}), donde se definen los valores posibles y sus descripciones. Por ejemplo, \texttt{cat\_textura}, \texttt{cat\_nivel1}, \texttt{cat\_tratmasa} o \texttt{cat\_origen}. Todas siguen un patrón uniforme: la clave primaria es el identificador de la variable (\texttt{<variable>\_id}), y las tablas troncales referencian este mismo campo como clave foránea. Además la base de datos incluye una tabla llamada \texttt{meta\_variables} que recoge los metadatos.


La Figura~\ref{fig:GWest_BBDD} muestra el esquema general de las tablas troncales y sus principales relaciones. Este diagrama resume la estructura interna de la base de datos y su jerarquía de dependencias.

\begin{figure}[htbp]
    \centering
    \includegraphics[width=0.9\textwidth]{figuras/Estrctr_BBDD_GWest.png}
    \caption{Esquema relacional de las tablas principales de la base de datos. Tabla extraida de \cite{greenwestdb}, donde se pueden consultar más detalles sobre las variables.}
    \label{fig:GWest_BBDD}
\end{figure}

\subsubsection{Diccionario resumido de variables}
\footnotesize
\setlength{\LTcapwidth}{\textwidth}
\begin{longtable}{@{}p{2.8cm} p{6.8cm} p{2.2cm} p{2.2cm}@{}}
    \caption{Resumen de variables principales por entidad. Tabla extraida de \cite{greenwestdb}.}                                                                                                  \\
    \toprule
    \textbf{Variable}                                                      & \textbf{Descripción}                                           & \textbf{Unidad}              & \textbf{Tipo de dato} \\
    \midrule
    \endfirsthead
    \toprule
    \textbf{Variable}                                                      & \textbf{Descripción}                                           & \textbf{Unidad}              & \textbf{Tipo de dato} \\
    \midrule
    \endhead
    \midrule
    \multicolumn{4}{r}{\emph{Continúa en la siguiente página}}                                                                                                                                     \\
    \midrule
    \endfoot
    \bottomrule
    \endlastfoot

    \multicolumn{4}{l}{\textbf{parcelas}}                                                                                                                                                          \\
    \texttt{parcela\_id}                                                   & Identificador único de parcela (IFN).                          & --                           & Identificador         \\
    \texttt{latitud}, \texttt{longitud}                                    & Coordenadas geográficas (WGS84).                               & °                            & Geográfico            \\
    \texttt{coorx}, \texttt{coory}                                         & Coordenadas UTM; \texttt{huso} especifica zona.                & m (UTM)                      & Geográfico            \\
    \texttt{elevacion}                                                     & Cota sobre el nivel del mar (NASADEM).                         & m                            & Numérico              \\
    \texttt{pendiente}                                                     & Inclinación del terreno.                                       & °                            & Numérico              \\
    \texttt{orientacion}                                                   & Orientación del terreno (0–360).                               & °                            & Numérico              \\
    \texttt{presencia\_id}                                                 & Presencia en IFN $\rightarrow$ \texttt{cat\_presencia}.        & --                           & Categórico            \\
    \texttt{tipsuelo1\_id}, \texttt{tipsuelo2\_id}, \texttt{tipsuelo3\_id} & Tipos de suelo $\rightarrow$ \texttt{cat\_tipsuelo*}.          & --                           & Categórico            \\
    \texttt{rocosidad\_id}                                                 & Rocosidad $\rightarrow$ \texttt{cat\_rocosidad}.               & --                           & Categórico            \\
    \texttt{radio}, \texttt{superficie}                                    & Radio de parcela y superficie derivada.                        & m; ha                        & Numérico              \\
    \addlinespace

    \multicolumn{4}{l}{\textbf{parcela\_inventario}}                                                                                                                                               \\
    \texttt{parcela\_id}, \texttt{inventario\_id}                          & Clave compuesta (parcela-inventario).                          & --                           & Identificador         \\
    \texttt{ano}                                                           & Año de apeo.                                                   & año                          & Numérico              \\
    \texttt{nivel1\_id}, \texttt{nivel2\_id}                               & Morfoestructura. $\rightarrow$ \texttt{cat\_nivel*}.           & --                           & Categórico            \\
    \texttt{textura\_id}                                                   & Textura de suelo $\rightarrow$ \texttt{cat\_textura}.          & --                           & Categórico            \\
    \texttt{merosiva\_id}                                                  & Manifestaciones erosivas $\rightarrow$ \texttt{cat\_merosiva}. & --                           & Categórico            \\
    \texttt{matorg\_id}                                                    & Materia orgánica $\rightarrow$ \texttt{cat\_matorg}.           & --                           & Categórico            \\
    \texttt{modcomb\_id}                                                   & Modelo de combustible $\rightarrow$ \texttt{cat\_modcomb}.     & --                           & Categórico            \\
    \texttt{disesp\_id}                                                    & Distribución espacial $\rightarrow$ \texttt{cat\_disesp}.      & --                           & Categórico            \\
    \texttt{comesp\_id}                                                    & Composición específica $\rightarrow$ \texttt{cat\_comesp}.     & --                           & Categórico            \\
    \texttt{fccarb}, \texttt{fcctot}                                       & Fracción de cabida cubierta (árboles).                         & \%                           & Numérico              \\
    \addlinespace

    \multicolumn{4}{l}{\textbf{parcela\_inventario\_especie}}                                                                                                                                      \\
    \texttt{parcela\_id}, \texttt{inventario\_id}, \texttt{especie\_id}    & Clave compuesta (parcela-inventario-especie).                  & --                           & Identificador         \\
    \texttt{ocupa}                                                         & Grado de ocupación de la especie.                              & (0--10)                      & Numérico              \\
    \texttt{estado\_id}                                                    & Estado de desarrollo. $\rightarrow$ \texttt{cat\_estado}.      & --                           & Categórico            \\
    \texttt{fpmasa\_id}                                                    & Forma principal de masa $\rightarrow$ \texttt{cat\_fpmasa}.    & --                           & Categórico            \\
    \texttt{tratmasa\_id}                                                  & Tratamientos selvícolas $\rightarrow$ \texttt{cat\_tratmasa}.  & --                           & Categórico            \\
    \texttt{orgmasa1\_id}                                                  & Origen de masa (IFN3/4)$\rightarrow$ \texttt{cat\_orgmasa1}.   & --                           & Categórico            \\
    \texttt{masa\_id}                                                      & Clasificación de masa $\rightarrow$ \texttt{cat\_masa}.        & --                           & Categórico            \\
    \texttt{origen\_id}                                                    & Origen de la masa (IFN2) $\rightarrow$ \texttt{cat\_origen}.   & --                           & Categórico            \\
    \addlinespace

    \multicolumn{4}{l}{\textbf{parcela\_inventario\_especie\_cd}}                                                                                                                                  \\
    \texttt{parcela\_id}, \texttt{inventario\_id}, \texttt{especie\_id}    & Clave compuesta ( parcela-inventario-especie-cd).              & --                           & Identificador         \\
    \texttt{cd\_id}                                                        & Clase diamétrica (CD) reglamento IFN.                          & cm                           & Numérico discreto     \\
    \texttt{npies}                                                         & Número de pies.                                                & pies/ha                      & Numérico              \\
    \texttt{abas}                                                          & Área basimétrica.                                              & m$^{2}$/ha                   & Numérico              \\
    \texttt{vcc}, \texttt{vsc}, \texttt{vle}                               & Volúmenes (con/sin corteza; leñas).                            & m$^{3}$/ha                   & Numérico              \\
    \texttt{iavc}                                                          & Incremento anual del volumen con corteza.                      & m$^{3}$/ha$\cdot$año         & Numérico              \\
    \texttt{ca}, \texttt{cr}                                               & Carbono aéreo y radical.                                       & t/ha                         & Numérico              \\
    \texttt{ht}                                                            & Altura media (modelo CatBoost).                                & m                            & Numérico              \\
    \texttt{carbono\_bruto}                                                & Carbono total estimado (alometrías).                           & t                            & Numérico              \\
    \addlinespace

    \multicolumn{4}{l}{\textbf{parcela\_especie\_arbol}}                                                                                                                                           \\
    \texttt{parcela\_id}, \texttt{especie\_id}                             & Clave compuesta (parcela–especie–árbol).                       & --                           & Identificador         \\
    \texttt{arbol\_id}                                                     & Identificador del árbol dentro de parcela y especie.           & --                           & Entero                \\
    \texttt{rumbo}                                                         & Rumbo desde el centro de la parcela al árbol.                  & grados centesimales          & Numérico              \\
    \texttt{distancia}                                                     & Distancia desde el centro de la parcela al árbol.              & m                            & Numérico              \\
    \texttt{cd}                                                            & Clase diamétrica (reglamento IFN).                             & cm                           & Numérico discreto     \\
    \texttt{ht}                                                            & Altura total del árbol inventariado.                           & m                            & Numérico              \\
    \texttt{dn1}, \texttt{dn2}                                             & Diámetros normales perpendiculares.                            & mm                           & Numérico              \\
    \texttt{abas}                                                          & Área basimétrica del pie medido.                               & m$^{2}$                      & Numérico              \\
    \texttt{iavc}                                                          & Incremento anual del volumen con corteza.                      & dm$^{3}$/año                 & Numérico              \\
    \texttt{vcc}, \texttt{vsc}, \texttt{vle}                               & Volúmenes (con corteza, sin corteza, leñas).                   & dm$^{3}$                     & Numérico              \\
    \texttt{ca}, \texttt{cr}                                               & Carbono aéreo y radical del árbol.                             & t                            & Numérico              \\
    \addlinespace

    \multicolumn{4}{l}{\textbf{parcela\_inventario\_estacion}}                                                                                                                                     \\
    \texttt{parcela\_id}, \texttt{inventario\_id}, \texttt{estacion\_id}   & Clave compuesta (agregado estacional).                         & --                           & Identificador         \\
    \texttt{PR\_*}                                                         & Estadísticos de precipitación (mean, max, min, std, sum).      & mm/(m$^2\cdot$día), mm/m$^2$ & Numérico              \\
    \texttt{T2M\_*}, \texttt{SKT\_*}                                       & Aire 2\,m y temperatura superficial (mean, max, min, std).     & °C                           & Numérico              \\
    \texttt{STL1\_*}--\texttt{STL4\_*}                                     & Temperatura del suelo por niveles (mean, max, min, std).       & °C                           & Numérico              \\
    \texttt{NDVI\_*}, \texttt{EVI\_*}, \texttt{NDII\_*}, \texttt{GNDVI\_*} & Índices de vegetación (max, mean, median, min, std).           & adimensional                 & Numérico              \\
    \addlinespace

    \multicolumn{4}{l}{\textbf{especies} y \textbf{grupos}}                                                                                                                                        \\
    \texttt{especie\_id}                                                   & Identificador de especie IFN.                                  & --                           & Identificador         \\
    \texttt{nombre}, \texttt{sinonimia}                                    & Denominación IFN y sinónimos.                                  & --                           & Texto                 \\
    \texttt{tipo\_especie}                                                 & 0\,= conífera; 1\,= frondosa.                                  & --                           & Categórico            \\
    \texttt{grupo\_id}                                                     & Grupo funcional $\rightarrow$ \texttt{grupos}.                 & --                           & Identificador         \\
    \texttt{grupos.nombregrupo}                                            & Nombre del grupo.                                              & --                           & Texto                 \\
\end{longtable}
\normalsize

\subsubsection{Cardinalidad y completitud}

El volumen de entradas por tabla es:
\begin{center}
    \begin{tabular}{l r}
        \toprule
        \textbf{Tabla}                            & \textbf{Número de registros} \\
        \midrule
        \texttt{parcelas}                         & 52{,}298                     \\
        \texttt{parcela\_inventario}              & 147{,}995                    \\
        \texttt{parcela\_inventario\_especie}     & 417{,}119                    \\
        \texttt{parcela\_inventario\_especie\_cd} & 1{,}191{,}070                \\
        \texttt{parcela\_especie\_arbol}          & 855{,}860                    \\
        \texttt{parcela\_inventario\_estacion}    & 470{,}056                    \\
        \texttt{especies}                         & 195                          \\
        \texttt{grupos}                           & 33                           \\
        \bottomrule
    \end{tabular}
\end{center}

\subsection{Variables objetivo}

El objetivo del modelo es estimar el \textbf{carbono total} que una parcela forestal puede capturar en un horizonte temporal de 20--30 años, a partir de las condiciones observadas en inventarios previos.
Para ello se definieron dos variables de respuesta complementarias, ambas derivadas de los datos del Inventario Forestal Nacional (IFN), que permiten analizar el contenido de carbono desde perspectivas distintas: una normalizada por superficie y otra en términos absolutos.


\begin{enumerate}
    \item \textbf{\texttt{c}} (tC/ha): representa el \textbf{carbono total contenido en la biomasa viva aérea y subterránea} por unidad de superficie, expresado en \emph{toneladas de carbono por hectárea}.
          Su cálculo se basa en la suma de las estimaciones de carbono aéreo (\texttt{ca}) y radical (\texttt{cr}) reportadas por el IFN.
          En los casos con valores faltantes, se completó la información mediante un modelo de \emph{Random Forest Regressor} ajustado sobre variables dendrométricas observadas (Especie, CD, VSC, NPies, ABas, IAVC, VCC y VLE), alcanzando un rendimiento satisfactorio (\(R^2_{test} > 0.90\)).
          Esta variable es coherente con los formatos internacionales de reporte de inventarios forestales y permite comparar el contenido de carbono entre parcelas o especies.


    \item \textbf{\texttt{carbono\_bruto}} (tC): corresponde al \textbf{carbono total capturado por parcela y especie}, expresado en \emph{toneladas de carbono totales}.
          Su estimación se realiza de forma trazable y físicamente interpretable a partir de variables medidas directamente en campo: número de pies (\texttt{npies}), altura media (\texttt{ht}), tipo de especie (\texttt{clase\_especie}) y clase diamétrica (\texttt{cd\_id}).
          El cálculo sigue un modelo alométrico adaptado de \cite{chave2014} y las directrices del IPCC~\cite{ipcc2006}, incorporando tanto la biomasa aérea como la biomasa radical mediante la relación Parte Radical:Parte Aérea ($R$).
          El resultado se expresa en toneladas de carbono totales por parcela, sin normalizar por superficie, lo que facilita la trazabilidad del proceso y la comparación entre inventarios sin depender de factores de expansión específicos del IFN.
          En coherencia con los criterios de proyectos de forestación y reforestación, las observaciones correspondientes a brinzales o plantones se consideran con valor de carbono nulo, dado que las fases tempranas de desarrollo no se contabilizan oficialmente como carbono capturado.
\end{enumerate}


Estas dos variables resumen el contenido de carbono forestal desde enfoques complementarios:
\texttt{c} (tC/ha) permite la comparación espacial y temporal entre masas forestales, mientras que \texttt{carbono\_bruto} (tC) ofrece una medida absoluta y directamente derivada de las observaciones de campo.
Ambas constituyen los objetivos principales del modelado predictivo, orientado a estimar el carbono acumulado en el \textbf{IFN4} a partir de las condiciones registradas en los inventarios anteriores (\textbf{IFN2} e \textbf{IFN3}).

\subsection{Supuestos de elegibilidad y verificación externa}
% Intervención humana, permanencia 30a (extrapolación cauta), superficie>=1ha,
% fccarb>=20% (filtro), altura>=3m en madurez (decisión de diseño).

Para que un proyecto forestal sea elegible en programas de \emph{créditos de carbono}, debe cumplir requisitos técnicos establecidos por marcos regulatorios internacionales \cite{IPCC2006, miteco_guia_co2}. A continuación se resume cada criterio y la forma en que se aborda en este estudio:

\begin{itemize}
    \item \textbf{Intervención humana directa.} El incremento de carbono debe proceder de actuaciones planificadas (reforestación, restauración o manejo sostenible). En nuestro caso, el modelo se entrena sobre datos observacionales (IFN2--IFN3--IFN4); por tanto, la \emph{verificación de intervención} no se deduce del modelo, sino que se contempla como \emph{condición externa} de elegibilidad del proyecto a evaluar.

    \item \textbf{Permanencia mínima de 30 años.} Para caracterizar el crecimiento de las parcelas forestales en los datos que alimentan el modelo, es necesario disponer de dos mediciones sucesivas de cada parcela, separadas por un intervalo temporal conocido. Estas mediciones permiten cuantificar la evolución de las variables forestales y, por tanto, estimar el incremento de carbono asociado al crecimiento del arbolado durante dicho periodo.

          En este trabajo, el objetivo es predecir el contenido de carbono correspondiente al \textbf{IFN4}, utilizando como información explicativa las variables observadas en inventarios anteriores. Dado que los inventarios tercero y cuarto comparten una estructura homogénea y un conjunto de variables comparable la elección más directa para el entrenamiento del modelo sería emplear exclusivamente estos dos inventarios. Esta estrategia aprovecha la coherencia estructural de los inventarios más recientes, que incluyen un mayor número de variables y una caracterización más detallada del terreno.


          No obstante, este planteamiento se enfrenta a la limitación impuesta por la \textbf{permanencia mínima de 30 años}, requisito fundamental en el contexto de los proyectos de compensación. El intervalo de tiempo entre los inventarios \textbf{IFN3} e \textbf{IFN4} es relativamente corto: no supera los 18 años.


          La Figura~\ref{fig:periodo34} muestra la distribución de la diferencia de años entre las mediciones del IFN3 y el IFN4. Como puede observarse, la mayoría de las parcelas presentan intervalos comprendidos entre 6 y 17 años, un rango demasiado estrecho para evaluar la estabilidad del modelo en horizontes más amplios.

          TODO: Actualizar figura y quitar título de la propia imagen
          \begin{figure}[h!]
              \centering
              \includegraphics[width=0.9\textwidth]{figuras/periodo34.png}
              \caption{Distribución de la diferencia de años entre los inventarios IFN3 e IFN4.}
              \label{fig:periodo34}
          \end{figure}

          Para ampliar la cobertura temporal y mejorar la capacidad de generalización del modelo, se optó por unificar la información de los inventarios \textbf{IFN2} e \textbf{IFN3} como base explicativa para la predicción del \textbf{IFN4}. Esta integración permite disponer de pares de mediciones de parcelas separadas por intervalos que oscilan entre 6 y 29 años, lo que constituye un rango mucho más representativo del horizonte de 20--30 años establecido como referencia.

          TODO: Actualizar figura y quitar título de la propia imagen
          \begin{figure}[h!]
              \centering
              \includegraphics[width=0.9\textwidth]{figuras/periodo234.png}
              \caption{Distribución de la diferencia de años entre los inventarios IFN2--IFN3 e IFN3--IFN4.}
              \label{fig:periodo234}
          \end{figure}


          De esta forma, el modelo se entrena y valida sobre un conjunto de datos más diverso y equilibrado, tanto en estructura como en amplitud temporal, manteniendo la coherencia metodológica y la trazabilidad de las estimaciones. Este enfoque no sólo mejora la robustez del aprendizaje, sino que también refuerza la capacidad del modelo para proyectar la captura de carbono en escenarios compatibles con los requisitos de permanencia de los proyectos de compensación.

    \item \textbf{Superficie mínima de 1 ha.} Este criterio se considera \emph{externo} al alcance del modelo predictivo, ya que el aprendizaje se realiza a nivel de parcela e inventario y no sobre polígonos de superficie total. En la práctica, la verificación de la superficie se realiza \emph{ex ante}, sobre la geometría declarada del proyecto forestal. En los terrenos forestales generados a partir de intervención humana directa —como plantaciones o repoblaciones—, la extensión suele presentar una estructura homogénea, con una especie dominante, edades coetáneas y densidades estandarizadas. Bajo estas condiciones, el carbono total es proporcional a la superficie: duplicar el área de una masa forestal homogénea implica aproximadamente duplicar su carbono almacenado. Por tanto, la variable de superficie no afecta al ajuste interno del modelo y su cumplimiento puede evaluarse fácilmente a nivel de proyecto, sin comprometer la validez de las predicciones.

    \item \textbf{Fracción mínima de cabida cubierta del 20\%.} La base de datos dispone de \texttt{fccarb} (arbórea) y \texttt{fcctot} (total). Este umbral se aplica como \emph{filtro de elegibilidad} previo o posterior al modelado, sin modificar la arquitectura del modelo (\texttt{fccarb}$>20$).

    \item \textbf{Altura mínima de 3 m en la madurez.} Este requisito se refiere a la altura que alcanzan los árboles en su fase de pleno desarrollo, y no a la altura inicial de los plantones. Por tanto, las mediciones realizadas durante las etapas tempranas de crecimiento no determinan la elegibilidad del proyecto, siempre que las especies seleccionadas sean capaces de superar los 3 metros en la madurez. En nuestro conjunto de datos, la altura no se registra explícitamente, por lo que este criterio se evalúa de forma \emph{externa} al modelo, mediante la selección de especies forestales adecuadas y la verificación con fuentes auxiliares (catálogos silvícolas o tipologías de masa). En la práctica, el cumplimiento del requisito depende de una decisión de diseño del proyecto —\emph{no plantar especies cuyo tamaño adulto sea inferior a 3 metros}— más que del ajuste predictivo del modelo. Por ello, la altura no interviene directamente en el entrenamiento, aunque sí condiciona la elegibilidad final del proyecto forestal.
\end{itemize}

\subsection{Preparación y tratamiento de los datos}
% Filtros (fccarb>=20%, crecimiento positivo, etc.)
% Agregaciones (parcela-especie, compresión CD -> npies_{cd}, etc.)
% Cálculo de variables derivadas (carbono\_bruto)
% Codificación y escalado

Como ya se ha introducido el entrenamiento se realiza en dos líneas según la variable objetivo: \texttt{c} de \textbf{IFN4} o \texttt{carbono\_bruto} de \textbf{IFN4}; y según la información que se usa como explicativa: \textbf{IFN3} o \textbf{IFN3} e \textbf{IFN2}. Se plantea la preparación y filtrado de los datos en términos generales (variable objetivo por \texttt{c} o \texttt{carbono\_bruto} y primera inventariación/ inventariación explicativa por \textbf{IFN3} o la unión de \textbf{IFN2} e \textbf{IFN3}).

\subsubsection{Filtrado de registros}\label{subsec:filtrado_registros}

Se descartan todas aquellas parcelas en las que el valor de carbono total (variable objetivo) en la segunda inventariación es inferior a la primera. Estos casos suelen deberse a episodios de deforestación, incendios u otras perturbaciones, y no representan un crecimiento forestal neto.


El conjunto de datos se restringe únicamente a las parcelas que presentan una \texttt{fccarb} (fracción de cabida cubierta arbórea) igual o superior al 20\,\% en el \textbf{IFN3}. Este umbral define la proporción mínima de superficie ocupada por copas de árboles respecto al área total de la parcela, y constituye una de las condiciones esenciales para considerar una superficie como terreno forestal. La exclusión de parcelas con \texttt{fccarb} inferior al 20\,\% permite asegurar que las estimaciones de carbono se realicen sobre masas forestales consolidadas, evitando sesgos asociados a áreas agrícolas o matorrales. A los datos del \textbf{IFN2} no se les aplica dicho filtro porque no disponen de la variable \texttt{fccarb}.


\subsubsection{Cálculo y agregación de variables}

Cada registro de entrada se genera a nivel de combinación parcela--especie, incorporando las variables correspondientes de la primera medición y la variable objetivo (carbono) de la segunda medición (IFN4). Las variables de \texttt{parcela} y \texttt{parcela\_inventario} se desdoblan para cada especie. Las entradas de la tabla \texttt{parcela\_inventario\_especie\_cd} se agrupan por parcela y especie y se comprimen en una única entrada creando un conjunto de variables para cada clase diamétrica.


La Tabla~\ref{tab:entrada_modelo} resume las variables empleadas como entrada al modelo, integradas desde las distintas tablas que conforman la base de datos relacional.

\begin{table}[htbp]
    \renewcommand{\arraystretch}{1.2}
    \setlength{\tabcolsep}{3pt}
    \centering
    \footnotesize
    \begin{tabular}{|p{3.2cm}|p{1.8cm}|p{6.5cm}|p{2.2cm}|}
        \hline
        \multicolumn{4}{|c|}{\textbf{Resumen de Datos de Entrada del Modelo}} \\
        \hline
        \textbf{Variable} & \textbf{Tipo} & \textbf{Descripción} & \textbf{Anexo} \\
        \hline
        \textcolor{ForestGreen}{\texttt{parcela\_id}} & varchar & Identificador único de parcela. & -- \\
        \hline
        \textcolor{ForestGreen}{\texttt{especie\_id}, \texttt{tipo\_especie}, \texttt{grupo\_id}} & int (CF) & Especie, tipo y grupo taxonómico. & Anexos \ref{sec:especies}, \ref{sec:gruposespecies} \\
        \hline
        \textcolor{ForestGreen}{\texttt{ocupa}} & int & Grado de ocupación (0--10). & -- \\
        \hline
        \textcolor{ForestGreen}{\texttt{estado\_id}}, \texttt{fpmasa\_id}, \texttt{tratmasa\_id}, \texttt{orgmasa\_1\_id} & int (CF) & Estado, forma de masa, tratamiento, organización. & Anexos \ref{sec:EstadoIFN34}, \ref{sec:FPMasa}, \ref{sec:tratmasa}, \ref{sec:OrgMasa} \\
        \hline
        \textcolor{ForestGreen}{\texttt{tipsuelo1-3\_id}} & int (CF) & Tipos de suelo. & Anexo \ref{sec:TipSuelo} \\
        \hline
        \textcolor{ForestGreen}{\texttt{rocosidad\_id}}, \texttt{textura\_id}, \texttt{matorg\_id}, \texttt{modcomb\_id}, \texttt{disesp\_id}, \texttt{comesp\_id}, \texttt{merosiva\_id} & int (CF) & Variables edáficas y estructurales. & Anexos varios \\
        \hline
        \textcolor{ForestGreen}{\texttt{radio}, \texttt{orientacion}, \texttt{elevacion}, \texttt{pendiente}} & float & Topografía y geometría de parcela. & -- \\
        \hline
        \texttt{nivel1\_id}, \texttt{nivel2\_id}, \texttt{fccarb}, \texttt{fcctot} & int/float & Niveles jerárquicos y cabida cubierta. & Anexos \ref{sec:nivel1}, \ref{sec:nivel2} \\
        \hline
        \textcolor{ForestGreen}{\texttt{npies\_\{CD\}}} & float & N.º de pies por clase diamétrica. & -- \\
        \hline
        \textcolor{ForestGreen}{\texttt{periodo}} & int & Años entre inventarios. & -- \\
        \hline
        \textcolor{ForestGreen}{\texttt{evi, gndvi, ndii, ndvi\_\{stat\}\_\{est\}}} & float & Índices de vegetación por estación. & -- \\
        \hline
        \textcolor{ForestGreen}{\texttt{pr, skt, stl1-4, t2m\_\{stat\}\_\{est\}}} & float & Variables climáticas por estación. & -- \\
        \hline
        \textcolor{ForestGreen}{\texttt{c4}, \texttt{carbono\_bruto4}} & float & Carbono IFN4 (t/ha y t). & -- \\
        \hline
    \end{tabular}
    \caption{Variables de entrada del modelo. Las variables en \textcolor{ForestGreen}{verde} están disponibles en IFN2 e IFN3; el resto solo en IFN3.}
    \label{tab:entrada_modelo}
\end{table}

\subsubsection{Codificación y normalización}

Las variables categóricas se codifican mediante \textit{one-hot encoding}, generando variables binarias para cada clase. Las variables numéricas se escalan (normalización estándar o min-max, según el modelo) para asegurar que todas las magnitudes tengan el mismo orden de importancia durante el entrenamiento.

\subsubsection{Reclasificación de las variables \texttt{pendiente} y \texttt{orientacion}}

Las variables topográficas originales \texttt{pendiente} (en grados) y \texttt{orientacion} (acimut en grados) se registran de forma continua en las parcelas del IFN. Sin embargo, desde el punto de vista ecológico su efecto sobre la acumulación de carbono suele ser no lineal y está asociado a clases discretas (e.g.\ laderas suaves frente a escarpadas, exposición norte frente a sur), por lo que resulta más adecuado tratarlas como factores categóricos.

A partir de la distribución empírica y de criterios habituales en estudios de fisiografía forestal, se definió una variable categórica \texttt{pendiente\_cat} mediante cortes en grados:

\begin{itemize}
    \item \(\textless 5^\circ\): \textit{muy suave},
    \item \(5{-}10^\circ\): \textit{suave},
    \item \(10{-}15^\circ\): \textit{moderada},
    \item \(15{-}20^\circ\): \textit{fuerte},
    \item \(20{-}30^\circ\): \textit{muy fuerte},
    \item \(30{-}50^\circ\): \textit{escarpada},
    \item \(>50^\circ\): \textit{extrema}.
\end{itemize}

Esta reclasificación permite capturar diferencias funcionales relevantes (accesibilidad, estabilidad del suelo, escorrentía, profundidad efectiva del suelo) sin asumir una relación lineal entre la pendiente y el carbono almacenado.

De forma análoga, la variable \texttt{orientacion} se reclasificó en ocho sectores cardinales equiángulos: \texttt{N}, \texttt{NE}, \texttt{E}, \texttt{SE}, \texttt{S}, \texttt{SO}, \texttt{O} y \texttt{NO}. La nueva variable \texttt{orientacion\_cat} agrupa orientaciones con condiciones de insolación y balance hídrico similares, lo que facilita la interpretación ecológica y reduce el ruido asociado a pequeñas variaciones angulares.


\subsection{Partición y validación}

Para obtener una estimación imparcial del rendimiento y evitar \emph{fugas de información} debidas a la correlación espacial dentro de cada parcela, la partición del conjunto de datos se realiza \textbf{por identificador de parcela} (\texttt{parcela\_id}). Todas las observaciones asociadas a una misma parcela se asignan \emph{íntegramente} a un único subconjunto, de modo que ninguna parcela aparece simultáneamente en entrenamiento y evaluación.
\vspace{0.25em}
\noindent\textbf{Validación interna y control de sesgo temporal.} Sobre el subconjunto de entrenamiento (80\,\%) se aplica \emph{validación cruzada por grupos} utilizando como agrupador los \emph{años transcurridos entre inventarios} (p.\,ej., 15, 16, 17, \dots). Esta estrategia comprueba la \emph{estabilidad} del modelo frente a cambios en el horizonte temporal y reduce el riesgo de sobreajuste específico de un periodo. La selección de hiperparámetros se realiza exclusivamente dentro de esta validación interna; el conjunto de evaluación (20\,\%) permanece \emph{sellado} para la prueba final.

\vspace{0.25em}
\noindent\textbf{Métricas de evaluación.} El rendimiento se informa con un conjunto de medidas complementarias:

\begin{itemize}
    \item \textbf{RMSE (Root Mean Squared Error):} raíz del error cuadrático medio entre valores observados y predichos; se expresa en las mismas unidades que la variable objetivo y penaliza con mayor peso los errores grandes. Valores más bajos indican mejor ajuste.

    \item \boldmath\textbf{$R^2$}\unboldmath{} (coeficiente de determinación): proporción de la varianza observada explicada por el modelo (idealmente en $[0,1]$). Valores cercanos a 1 denotan alta capacidad explicativa; puede ser negativo si el modelo es peor que la predicción constante.

    \item \textbf{MAE (Mean Absolute Error):} media aritmética del error absoluto, que cuantifica la desviación media entre las predicciones y los valores observados. Penaliza todos los errores de forma lineal y es más interpretable que el RMSE. Valores más bajos indican mejor ajuste.

    \item \textbf{Moda del Error}: valor más frecuente del error absoluto a nivel de parcela o individuo, útil para identificar el error típico en la predicción y detectar patrones dominantes en el comportamiento del modelo.
\end{itemize}

%\vspace{0.25em}
%\noindent\textbf{Protocolo de reporte.} Para cada modelo se reportan: (i) el rendimiento medio y la dispersión en la validación cruzada por grupos (entrenamiento), y (ii) el desempeño final en el conjunto de evaluación independiente (20\,\%). Este protocolo garantiza comparabilidad entre modelos, control del sesgo espacial por parcela y verificación explícita de la robustez temporal.


\subsection{Selección de variables explicativas}

La selección de predictores se abordó mediante cuatro estrategias complementarias: (1) selección automática mediante \textit{Featurewiz}, (2) selección basada en el criterio de mínima redundancia y máxima relevancia (\textit{mRMR}), (3) selección manual basada en criterios estadísticos y conceptuales, y (4) un procedimiento secuencial supervisado fundamentado en el rendimiento predictivo (SSSRP). El objetivo común de estas aproximaciones fue identificar un subconjunto parsimonioso de variables que maximizara la capacidad predictiva del modelo, redujera la colinealidad y mantuviera la coherencia ecológica de las relaciones.

\subsubsection{Selección automática mediante Featurewiz}

El algoritmo \textit{Featurewiz} aplica un enfoque híbrido orientado a la relevancia predictiva. Primero ejecuta un filtrado por correlación, eliminando predictores altamente colineales (umbral \( |r| > 0.70 \)), y posteriormente refina el conjunto mediante modelos de \textit{Gradient Boosting} para estimar la importancia relativa de cada variable. El resultado es un subconjunto compacto de predictores con contribución significativa al rendimiento del modelo.

\subsubsection{Selección mediante mRMR}

El método \textit{mRMR} (minimum Redundancy–maximum Relevance) selecciona las variables que mejor explican la variabilidad del objetivo a la vez que minimizan la redundancia informativa entre ellas. Para ello emplea información mutua, permitiendo capturar relaciones potencialmente no lineales. Este enfoque prioriza predictores que aportan información complementaria sobre el proceso ecológico modelado, evitando duplicidades entre atributos altamente correlacionados.

\subsubsection{Selección manual basada en criterios estadísticos y conceptuales}

La selección manual integró criterios estadísticos (correlaciones, ANOVA y análisis de redundancia) con criterios ecológicos y de interpretabilidad. Se descartaron predictores sin asociación significativa con la variable objetivo y se redujo la colinealidad reteniendo un único representante por cada grupo altamente correlacionado. Asimismo, se garantizaron variables que describieran dimensiones esenciales del sistema (estructura del arbolado, topografía, suelo, clima e índices espectrales), asegurando un equilibrio entre precisión predictiva y coherencia biogeográfica.

\subsubsection{Selección Secuencial Supervisada basada en Rendimiento Predictivo (SSSRP)}

El método SSSRP complementó las estrategias anteriores mediante un enfoque explícitamente orientado al rendimiento predictivo. Se partió de un \emph{bloque base} de variables estructurales y se evaluó el impacto marginal de cada candidato añadiéndolo individualmente y comparando el cambio en \(R^{2}\) y RMSE mediante un modelo CatBoost con validación holdout estratificada por parcela. A continuación, se aplicó una estrategia de \textit{forward selection} codiciosa, incorporando en cada iteración la variable que proporcionaba la mayor mejora y deteniendo el proceso cuando la ganancia resultaba inferior a un umbral predefinido (\(\Delta R^{2} > 10^{-5}\)). Este procedimiento produjo un conjunto final de predictores reducido, no redundante y específicamente optimizado para maximizar el rendimiento del modelo.


\subsection{Modelos evaluados}

A continuación se describe el procedimiento seguido para la selección, optimización y combinación de modelos. El objetivo es construir un conjunto de predictores base sólidos y posteriormente integrarlos en un \textit{stack-ensemble} capaz de mejorar la capacidad de generalización.

\subsubsection{Modelos ensemble}

Se utilizaron diversos métodos de \textit{ensemble learning} con el fin de aumentar precisión y robustez del sistema predictivo. El principio fundamental consiste en combinar predicciones de múltiples modelos, aprovechando su diversidad para reducir varianza, sesgo o ambos.

\paragraph{Técnicas empleadas:}
\begin{itemize}
    \item \textbf{Bagging:} entrena modelos independientes sobre subconjuntos generados mediante muestreo bootstrap. Reduce varianza y mejora estabilidad.
    \item \textbf{Boosting:} construye modelos secuenciales donde cada uno corrige los errores del anterior. Tiende a reducir el sesgo y producir modelos altamente precisos.
    \item \textbf{Stacking:} integra múltiples modelos base mediante un meta-modelo entrenado sobre sus predicciones. Permite capturar relaciones no lineales entre las salidas de los modelos base.
\end{itemize}

\subsubsection{Boosting y aprendizaje secuencial}

El conjunto de modelos de boosting evaluados incluye:

\begin{itemize}
    \item \textbf{XGBoost:} implementación avanzada del \textit{gradient boosting}, que incorpora regularización L1/L2, optimización mediante segundo orden y manejo interno de valores faltantes.
    \item \textbf{LightGBM:} algoritmo especialmente eficiente, basado en crecimiento \textit{leaf-wise}, capaz de manejar grandes volúmenes de datos y con soporte nativo para variables categóricas.
    \item \textbf{CatBoost:} optimizado para variables categóricas y robusto frente a ruido mediante técnicas como \textit{ordered boosting}.
    \item \textbf{Gradient Boosting Decision Trees (GBDT):} implementación clásica del algoritmo basado en descenso por gradiente sobre residuos.
    \item \textbf{AdaBoost:} técnica que ajusta modelos simples (stumps) secuencialmente, asignando más peso a observaciones difíciles.
\end{itemize}

\subsubsection{Bagging}

Los modelos basados en bootstrap empleados fueron:

\begin{itemize}
    \item \textbf{Random Forest:} conjunto de árboles de decisión que introduce aleatoriedad tanto en datos como en características. Suele ser robusto y relativamente estable.
    \item \textbf{Bagged Decision Trees (BaggedDT):} árboles no podados entrenados sobre muestras bootstrap, cuyas predicciones se promedian para reducir varianza.
\end{itemize}

\subsubsection{Otros modelos evaluados}

Además de los métodos ensemble, se evaluaron modelos representativos de paradigmas adicionales:

\begin{itemize}
    \item \textbf{Support Vector Regression (SVR):} modelo de márgenes para regresión, evaluado con kernel lineal.
    \item \textbf{K-Nearest Neighbors (KNN):} modelo basado en vecinos más próximos; útil como referencia no paramétrica, aunque sensible a la escala.
    \item \textbf{Multi-Layer Perceptron (MLP):} red neuronal densa capaz de capturar relaciones no lineales.
    \item \textbf{Bayesian Neural Network (BayesianNN):} aproximación probabilística que permite cuantificar incertidumbre a través de regularización bayesiana.
\end{itemize}

\vspace{0.2cm}

\subsubsection{Configuración del \textit{stacking}}

Tras evaluar todos los modelos anteriores, se construyeron diferentes configuraciones de modelos base (\textit{base learners}) que se combinan mediante un meta-modelo. Las configuraciones empleadas son:

\begin{verbatim}
['CatBoost', 'LightGBM', 'XGBoost', 'Random Forest', 'GBDT', 'BaggedDT'],
['CatBoost', 'LightGBM', 'Random Forest', 'GBDT'],
['LightGBM', 'XGBoost', 'GBDT'],
['CatBoost', 'Random Forest', 'GBDT'],
['LightGBM', 'Random Forest']
\end{verbatim}

Estas combinaciones se diseñaron con dos criterios principales:

\begin{enumerate}
    \item \textbf{Diversidad estructural:} mezclar métodos de boosting y bagging, así como variantes de boosting con distintas estrategias de crecimiento y regularización.
    \item \textbf{Rendimiento individual:} incluir preferentemente los modelos con mayor $R^2$ y menor error (RMSE, MAE) en las pruebas individuales.
\end{enumerate}


Los meta-modelos utilizados para integrar las predicciones fueron:

\begin{itemize}
    \item \textbf{Modelos lineales:} Regresión Lineal, Ridge.
    \item \textbf{Modelos basados en árboles:} Random Forest, Gradient Boosting Regressor.
    \item \textbf{Modelos kernel:} SVR lineal.
    \item \textbf{Red neuronal:} MLP con una capa oculta.
\end{itemize}

Esta selección permite comparar desde combinadores lineales simples hasta integradores no lineales capaces de capturar interacciones complejas entre predicciones.

\subsubsection{Comparación y justificación de modelos}

La evaluación exhaustiva de múltiples algoritmos permite identificar no solo el modelo individual con mejor rendimiento, sino también combinaciones sinérgicas para el \textit{stacking}. La Tabla~\ref{tab:modelos} resume los modelos finalmente entrenados y evaluados.


\begin{table}[htbp]
    \centering
    \footnotesize
    \begin{tabular}{|p{2.5cm}|p{2.2cm}|p{4.5cm}|p{4cm}|}
        \hline
        \textbf{Modelo} & \textbf{Tipo} & \textbf{Características} & \textbf{Observaciones} \\
        \hline
        Random Forest & Bagging & Bootstrap con selección aleatoria de atributos & Robusto y estable \\
        \hline
        BaggedDT & Bagging & Árboles sin poda sobre muestras bootstrap & Mejora por agregación \\
        \hline
        XGBoost & Boosting & Regularización L1/L2, segundo orden & Muy preciso; sensible a tuning \\
        \hline
        LightGBM & Boosting & Crecimiento leaf-wise, muy eficiente & Rápido; riesgo de sobreajuste \\
        \hline
        CatBoost & Boosting & Codificación ordenada; robusto al ruido & Excelente sin gran tuning \\
        \hline
        GBDT & Boosting & Árboles secuenciales ajustados a residuos & Buen rendimiento \\
        \hline
        AdaBoost & Boosting & Aumenta peso de obs. mal predichas & Menos robusto \\
        \hline
        KNN & Instancia & Predicción por proximidad & Sensible a escala y ruido \\
        \hline
        MLP & Red neuronal & Captura relaciones no lineales & Requiere normalización \\
        \hline
        SVR & Márgenes & Kernel lineal, gran margen & Robusto al sobreajuste \\
        \hline
        BayesianNN & Probabilístico & Cuantifica incertidumbre & Reduce sobreajuste \\
        \hline
    \end{tabular}
    \caption{Resumen de los modelos de aprendizaje supervisado evaluados.}
    \label{tab:modelos}
\end{table}




