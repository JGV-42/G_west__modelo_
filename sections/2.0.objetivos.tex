\section{Objetivos y Justificación}

El presente estudio tiene como objetivo principal desarrollar un modelo de inteligencia artificial capaz de predecir con precisión la capacidad de absorción de dióxido de carbono (\(CO_2\)) en cultivos forestales españoles. Este modelo se basa en variables que describen la especie arbórea, las características del terreno y las condiciones climáticas. A partir de este objetivo general se derivan varias metas específicas, que en conjunto justifican la relevancia y aplicabilidad del proyecto.

\subsection*{Objetivos específicos}

\begin{itemize}
    \item \textbf{Desarrollar un modelo predictivo robusto:} Construir un modelo de aprendizaje automático que estime la cantidad de \(CO_2\) que será capturado a lo largo del tiempo por un cultivo forestal, a partir de datos como especie, tipo de suelo, clase diamétrica, clima y otras variables relevantes.

    \item \textbf{Optimizar la captura de carbono:} Utilizar el modelo para identificar combinaciones óptimas de especies y terrenos que maximicen la fijación de carbono, contribuyendo a la planificación eficiente de proyectos de forestación y reforestación.

    \item \textbf{Asegurar la compatibilidad con las normativas internacionales:} Garantizar que las predicciones y salidas del modelo sean compatibles con los marcos normativos definidos por la \textit{Convención Marco de las Naciones Unidas sobre el Cambio Climático} (CMNUCC) y el \textit{Protocolo de Kioto}, cumpliendo así los criterios necesarios para la validación de créditos de carbono \cite{cmnucc1992, kioto1997}.

    \item \textbf{Analizar los factores determinantes del desarrollo forestal:} Estudiar la influencia de variables climáticas (como la temperatura y la precipitación) y edáficas (como el tipo de suelo o la pendiente) sobre el crecimiento forestal y su capacidad de capturar carbono.

    \item \textbf{Apoyar la toma de decisiones ambientales y empresariales:} Proporcionar una herramienta práctica y validada que permita a técnicos, gestores y empresas seleccionar las especies más adecuadas y planificar actuaciones de forestación con la mayor eficiencia posible en términos de secuestro de carbono.
\end{itemize}

\subsection*{Justificación}

La necesidad de contar con herramientas predictivas para estimar la captura de \(CO_2\) se ha intensificado ante el crecimiento del mercado voluntario de créditos de carbono, y las obligaciones adquiridas en el marco de la CMNUCC y el Protocolo de Kioto. Según estos acuerdos, cada país debe reportar sus emisiones y absorciones de gases de efecto invernadero, y puede utilizar actividades de forestación y reforestación como mecanismos de compensación \cite{cmnucc1992, kioto1997}.

Para que estos proyectos sean elegibles, deben cumplir criterios específicos como intervención humana directa, permanencia de al menos 30 años, cobertura mínima del 20\%, superficie mínima de una hectárea y una altura mínima de los árboles maduros de 3 metros. Estos criterios hacen imprescindible disponer de modelos que no solo estimen el carbono actual, sino que sean capaces de prever su evolución a futuro con base en condiciones iniciales y variables predictoras.

Este trabajo busca cubrir ese vacío mediante el uso de inteligencia artificial aplicada a datos reales y multifuente. Integrar su manejo dentro del sistema de créditos de carbono puede representar una importante oportunidad para la economía local y para la mitigación del cambio climático.

