\section{Metodología}

Esta sección describe el procedimiento seguido para el entrenamiento y validación de los modelos predictivos desarrollados. 
La metodología se fundamenta en la identificación de los factores que determinan el crecimiento forestal y, en consecuencia, la capacidad de los ecosistemas para capturar carbono a lo largo del tiempo. 
El enfoque integra información estructural, climática y espectral procedente del Inventario Forestal Nacional (IFN) y de otras fuentes ambientales, con el propósito de construir modelos robustos que permitan predecir el contenido de carbono acumulado en la biomasa viva.

\medskip

El carbono fijado por los árboles se acumula progresivamente en su biomasa, en función del tamaño y vigor de los individuos, los cuales están condicionados por variables ambientales, topográficas y de competencia intraespecífica. 
Las condiciones meteorológicas, como la temperatura y la precipitación, inciden directamente en la fotosíntesis y en la disponibilidad hídrica; 
la orientación, la pendiente y la altitud modifican la radiación incidente y el microclima local; 
mientras que la densidad de árboles por unidad de superficie determina el nivel de competencia por los recursos, variando según la especie y su tolerancia ecológica \cite{IPCC2006, Buchholz2014}. 

\medskip

A partir de estos fundamentos, se construyó una base de datos relacional que integra información forestal, climática y espectral a nivel de parcela, especie y clase diamétrica. 
Esta estructura permite caracterizar con precisión la dinámica del bosque entre inventarios sucesivos y alimentar modelos predictivos capaces de estimar el contenido futuro de carbono a partir de las condiciones observadas en el pasado.


\subsection{Origen y estructura de los datos}
% Origen (IFN2/3/4 + clima + índices)
% Estructura relacional (parcelas, inventarios, especie, CD, estación, árbol)
% Referencia al esquema (figura) y a meta_variables


La base de datos empleada en este trabajo integra información forestal, climática y espectral estructurada en torno a la parcela como unidad básica. Cada parcela se describe mediante sus coordenadas geográficas, características edáficas y su evolución a través de distintos inventarios (IFN2, IFN3, IFN4).

\medskip

Los datos forestales incluyen información por especie y clase diamétrica, como número de pies, volumen con y sin corteza, área basimétrica, carbono aéreo, radical y total. Estos valores permiten caracterizar con precisión la estructura y crecimiento de la vegetación.

\medskip

A cada parcela se asocian también estadísticas climáticas agregadas por estación e inventario: temperaturas (superficie, aire y subsuelo) y precipitaciones, resumidas mediante métricas como media, máxima, mínima y desviación típica.

\medskip

Finalmente, se incorporan índices espectrales derivados de imágenes satelitales (NDVI, EVI, NDII, GNDVI), que permiten cuantificar propiedades biofísicas de la vegetación:
\begin{itemize}
    \item \textbf{NDVI (Normalized Difference Vegetation Index):} estima la actividad fotosintética.
    \item \textbf{EVI (Enhanced Vegetation Index):} mejora la sensibilidad en zonas densamente vegetadas.
    \item \textbf{NDII (Normalized Difference Infrared Index):} refleja el contenido hídrico de la vegetación.
    \item \textbf{GNDVI (Green NDVI):} variante del NDVI basada en la banda verde, sensible al clorofila.
\end{itemize}

\medskip

\subsubsection*{Estrutura de la base de datos}
Estos datos se organizan en las siguientes entidades troncales:

\begin{itemize}
  \item \textbf{parcelas}: icontiene la información básica de localización y características edáficas de cada parcela.
  \item \textbf{parcela\_inventario}: describe el estado de cada parcela en un inventario determinado (\texttt{parcela\_id}, \texttt{inventario\_id}), incluyendo atributos edáficos y de contexto (p. ej., \texttt{nivel1\_id}, \texttt{textura\_id}).
  \item \textbf{parcela\_inventario\_especie}: detalla la presencia y condición de cada especie dentro de una parcela e inventario, incorporando descriptores de masa y tratamientos silvícolas.
  \item \textbf{parcela\_inventario\_especie\_cd}: describe las poblaciones arbóreas por parcela, especie y \emph{clase diamétrica} (\texttt{cd\_id}): n.º de pies (\texttt{npies}), área basimétrica (\texttt{abas}), volúmenes (\texttt{vcc}, \texttt{vsc}, \texttt{vle}), incrementos (\texttt{iavc}) y carbono (\texttt{ca}, \texttt{cr}).
  \item \textbf{parcela\_especie\_arbol}: caracteriza los pies mayores identificados por parcela y especie en el inventario cuarto. Recoge las caracteristicas particulares de cada pie como altura (\texttt{ht}), diámetros (\texttt{dn1} y \texttt{dn2}), ubicación respecto del centro de la parcela (\texttt{rumbo}, \texttt{distancia}), volumen (\texttt{vcc, vsc, vle}), incremento (\texttt{iavc}) y carbono (\texttt{ca, cr}).  
  \item \textbf{parcela\_inventario\_estacion}: almacena agregados climático-biofísicos por estación (\texttt{estacion\_id}) en la misma granularidad parcela–inventario, incluyendo variables como precipitación (\texttt{PR}) y temperatura (\texttt{T2M, SKT, STL*}), junto a índices de vegetación (NDVI, EVI, NDII, GNDVI).
  \item \textbf{especies} y \textbf{grupos}: recogen la información taxonómica y su clasificación jerárquica, estableciendo la relación entre especies individuales y grupos funcionales.
\end{itemize}

Cada variable categórica posee una tabla de catálogo propia (\texttt{cat\_}), donde se definen los valores posibles y sus descripciones. Por ejemplo, \texttt{cat\_textura}, \texttt{cat\_nivel1}, \texttt{cat\_tratmasa} o \texttt{cat\_origen}. Todas siguen un patrón uniforme: la clave primaria es el identificador de la variable (\texttt{<variable>\_id}), y las tablas troncales referencian este mismo campo como clave foránea. Además la base de datos incluye una tabla llamada \texttt{meta\_variables} que recoge los metadatos.

\medskip

La Figura~\ref{fig:GWest_BBDD} muestra el esquema general de las tablas troncales y sus principales relaciones. Este diagrama resume la estructura interna de la base de datos y su jerarquía de dependencias.

\begin{figure}[H]
  \centering
  \includegraphics[width=0.9\textwidth]{figuras/Estrctr_BBDD_GWest.png}
  \caption{Esquema relacional de las tablas principales de la base de datos. Tabla extraida de \cite{greenwestdb}, donde se pueden consultar más detalles sobre las variables.}
  \label{fig:GWest_BBDD}
\end{figure}

\subsubsection*{Diccionario resumido de variables}
\small
\setlength{\LTcapwidth}{\textwidth}
\begin{longtable}{p{3.2cm} p{7.6cm} p{2.4cm} p{2.4cm}}
\caption{Resumen de variables principales por entidad. Tabla extraida de \cite{greenwestdb}.}\\
\toprule
\textbf{Variable} & \textbf{Descripción} & \textbf{Unidad} & \textbf{Tipo de dato} \\
\midrule
\endfirsthead
\toprule
\textbf{Variable} & \textbf{Descripción} & \textbf{Unidad} & \textbf{Tipo de dato} \\
\midrule
\endhead
\midrule
\multicolumn{4}{r}{\emph{Continúa en la siguiente página}} \\
\midrule
\endfoot
\bottomrule
\endlastfoot

\multicolumn{4}{l}{\textbf{parcelas}} \\
\texttt{parcela\_id} & Identificador único de parcela (IFN). & -- & Identificador \\
\texttt{latitud}, \texttt{longitud} & Coordenadas geográficas (WGS84). & ° & Geográfico \\
\texttt{coorx}, \texttt{coory} & Coordenadas UTM; \texttt{huso} especifica zona. & m (UTM) & Geográfico \\
\texttt{elevacion} & Cota sobre el nivel del mar (NASADEM). & m & Numérico \\
\texttt{pendiente} & Inclinación del terreno. & ° & Numérico \\
\texttt{orientacion} & Orientación del terreno (0–360). & ° & Numérico \\
\texttt{presencia\_id} & Presencia en IFN $\rightarrow$ \texttt{cat\_presencia}. & -- & Categórico \\
\texttt{tipsuelo1\_id}, \texttt{tipsuelo2\_id}, \texttt{tipsuelo3\_id} & Tipos de suelo $\rightarrow$ \texttt{cat\_tipsuelo*}. & -- & Categórico \\
\texttt{rocosidad\_id} & Rocosidad $\rightarrow$ \texttt{cat\_rocosidad}. & -- & Categórico \\
\texttt{radio}, \texttt{superficie} & Radio de parcela y superficie derivada. & m; ha & Numérico \\
\addlinespace

\multicolumn{4}{l}{\textbf{parcela\_inventario}} \\
\texttt{parcela\_id}, \texttt{inventario\_id} & Clave compuesta (parcela-inventario). & -- & Identificador \\
\texttt{ano} & Año de apeo. & año & Numérico \\
\texttt{nivel1\_id}, \texttt{nivel2\_id} & Morfoestructura. $\rightarrow$ \texttt{cat\_nivel*}. & -- & Categórico \\
\texttt{textura\_id} & Textura de suelo $\rightarrow$ \texttt{cat\_textura}. & -- & Categórico \\
\texttt{merosiva\_id} & Manifestaciones erosivas $\rightarrow$ \texttt{cat\_merosiva}. & -- & Categórico \\
\texttt{matorg\_id} & Materia orgánica $\rightarrow$ \texttt{cat\_matorg}. & -- & Categórico \\
\texttt{modcomb\_id} & Modelo de combustible $\rightarrow$ \texttt{cat\_modcomb}. & -- & Categórico \\
\texttt{disesp\_id} & Distribución espacial $\rightarrow$ \texttt{cat\_disesp}. & -- & Categórico \\
\texttt{comesp\_id} & Composición específica $\rightarrow$ \texttt{cat\_comesp}. & -- & Categórico \\
\texttt{fccarb}, \texttt{fcctot} & Fracción de cabida cubierta (árboles). & \% & Numérico \\
\addlinespace

\multicolumn{4}{l}{\textbf{parcela\_inventario\_especie}} \\
\texttt{parcela\_id}, \texttt{inventario\_id}, \texttt{especie\_id} & Clave compuesta (parcela-inventario-especie). & -- & Identificador \\
\texttt{ocupa} & Grado de ocupación de la especie. & (0--10) & Numérico \\
\texttt{estado\_id} & Estado de desarrollo. $\rightarrow$ \texttt{cat\_estado}. & -- & Categórico \\
\texttt{fpmasa\_id} & Forma principal de masa $\rightarrow$ \texttt{cat\_fpmasa}. & -- & Categórico \\
\texttt{tratmasa\_id} & Tratamientos selvícolas $\rightarrow$ \texttt{cat\_tratmasa}. & -- & Categórico \\
\texttt{orgmasa1\_id} & Origen de masa (IFN3/4)$\rightarrow$ \texttt{cat\_orgmasa1}. & -- & Categórico \\
\texttt{masa\_id} & Clasificación de masa $\rightarrow$ \texttt{cat\_masa}. & -- & Categórico \\
\texttt{origen\_id} & Origen de la masa (IFN2) $\rightarrow$ \texttt{cat\_origen}. & -- & Categórico \\
\addlinespace

\multicolumn{4}{l}{\textbf{parcela\_inventario\_especie\_cd}} \\
\texttt{parcela\_id}, \texttt{inventario\_id}, \texttt{especie\_id} & Clave compuesta ( parcela-inventario-especie-cd). & -- & Identificador \\
\texttt{cd\_id} & Clase diamétrica (CD) reglamento IFN. & cm & Numérico discreto \\
\texttt{npies} & Número de pies. & pies/ha & Numérico \\
\texttt{abas} & Área basimétrica. & m$^{2}$/ha & Numérico \\
\texttt{vcc}, \texttt{vsc}, \texttt{vle} & Volúmenes (con/sin corteza; leñas). & m$^{3}$/ha & Numérico \\
\texttt{iavc} & Incremento anual del volumen con corteza. & m$^{3}$/ha$\cdot$año & Numérico \\
\texttt{ca}, \texttt{cr} & Carbono aéreo y radical. & t/ha & Numérico \\
\texttt{ht} & Altura media (modelo CatBoost). & m & Numérico \\
\texttt{carbono\_bruto} & Carbono total estimado (alometrías). & t & Numérico \\
\addlinespace

\multicolumn{4}{l}{\textbf{parcela\_especie\_arbol}} \\
\texttt{parcela\_id}, \texttt{especie\_id} & Clave compuesta (parcela–especie–árbol). & -- & Identificador \\ 
\texttt{arbol\_id} & Identificador del árbol dentro de parcela y especie. & -- & Entero \\ 
\texttt{rumbo} & Rumbo desde el centro de la parcela al árbol. & grados centesimales & Numérico \\ 
\texttt{distancia} & Distancia desde el centro de la parcela al árbol. & m & Numérico \\ 
\texttt{cd} & Clase diamétrica (reglamento IFN). & cm & Numérico discreto \\ 
\texttt{ht} & Altura total del árbol inventariado. & m & Numérico \\ 
\texttt{dn1}, \texttt{dn2} & Diámetros normales perpendiculares. & mm & Numérico \\ 
\texttt{abas} & Área basimétrica del pie medido. & m$^{2}$ & Numérico \\ 
\texttt{iavc} & Incremento anual del volumen con corteza. & dm$^{3}$/año & Numérico \\ 
\texttt{vcc}, \texttt{vsc}, \texttt{vle} & Volúmenes (con corteza, sin corteza, leñas). & dm$^{3}$ & Numérico \\ 
\texttt{ca}, \texttt{cr} & Carbono aéreo y radical del árbol. & t & Numérico \\
\addlinespace

\multicolumn{4}{l}{\textbf{parcela\_inventario\_estacion}} \\
\texttt{parcela\_id}, \texttt{inventario\_id}, \texttt{estacion\_id} & Clave compuesta (agregado estacional). & -- & Identificador \\
\texttt{PR\_*} & Estadísticos de precipitación (mean, max, min, std, sum). & mm/(m$^2\cdot$día), mm/m$^2$ & Numérico \\
\texttt{T2M\_*}, \texttt{SKT\_*} & Aire 2\,m y temperatura superficial (mean, max, min, std). & °C & Numérico \\
\texttt{STL1\_*}--\texttt{STL4\_*} & Temperatura del suelo por niveles (mean, max, min, std). & °C & Numérico \\
\texttt{NDVI\_*}, \texttt{EVI\_*}, \texttt{NDII\_*}, \texttt{GNDVI\_*} & Índices de vegetación (max, mean, median, min, std). & adimensional & Numérico \\
\addlinespace

\multicolumn{4}{l}{\textbf{especies} y \textbf{grupos}} \\
\texttt{especie\_id} & Identificador de especie IFN. & -- & Identificador \\
\texttt{nombre}, \texttt{sinonimia} & Denominación IFN y sinónimos. & -- & Texto \\
\texttt{tipo\_especie} & 0\,= conífera; 1\,= frondosa. & -- & Categórico \\
\texttt{grupo\_id} & Grupo funcional $\rightarrow$ \texttt{grupos}. & -- & Identificador \\
\texttt{grupos.nombregrupo} & Nombre del grupo. & -- & Texto \\
\end{longtable}
\normalsize

\subsubsection*{Cardinalidad y completitud}

El volumen de entradas por tabla es:
\begin{center}
\begin{tabular}{l r}
\toprule
\textbf{Tabla} & \textbf{Número de registros} \\
\midrule
\texttt{parcelas} & 52{,}298 \\
\texttt{parcela\_inventario} & 147{,}995 \\
\texttt{parcela\_inventario\_especie} & 417{,}119 \\
\texttt{parcela\_inventario\_especie\_cd} & 1{,}191{,}070 \\
\texttt{parcela\_especie\_arbol} & 855{,}860 \\
\texttt{parcela\_inventario\_estacion} & 470{,}056 \\
\texttt{especies} & 195 \\
\texttt{grupos} & 33 \\
\bottomrule
\end{tabular}
\end{center}

\subsection{Variables objetivo}

El objetivo del modelo es estimar el \textbf{carbono total} que una parcela forestal puede capturar en un horizonte temporal de 20--30 años, a partir de las condiciones observadas en inventarios previos. 
Para ello se definieron dos variables de respuesta complementarias, ambas derivadas de los datos del Inventario Forestal Nacional (IFN), que permiten analizar el contenido de carbono desde perspectivas distintas: una normalizada por superficie y otra en términos absolutos.

\medskip

\begin{enumerate}
    \item \textbf{\texttt{c}} (tC/ha): representa el \textbf{carbono total contenido en la biomasa viva aérea y subterránea} por unidad de superficie, expresado en \emph{toneladas de carbono por hectárea}. 
    Su cálculo se basa en la suma de las estimaciones de carbono aéreo (\texttt{ca}) y radical (\texttt{cr}) reportadas por el IFN. 
    En los casos con valores faltantes, se completó la información mediante un modelo de \emph{Random Forest Regressor} ajustado sobre variables dendrométricas observadas (Especie, CD, VSC, NPies, ABas, IAVC, VCC y VLE), alcanzando un rendimiento satisfactorio (\(R^2_{test} > 0.90\)). 
    Esta variable es coherente con los formatos internacionales de reporte de inventarios forestales y permite comparar el contenido de carbono entre parcelas o especies.

    \medskip

    \item \textbf{\texttt{carbono\_bruto}} (tC): corresponde al \textbf{carbono total capturado por parcela y especie}, expresado en \emph{toneladas de carbono totales}. 
    Su estimación se realiza de forma trazable y físicamente interpretable a partir de variables medidas directamente en campo: número de pies (\texttt{npies}), altura media (\texttt{ht}), tipo de especie (\texttt{clase\_especie}) y clase diamétrica (\texttt{cd\_id}). 
    El cálculo sigue un modelo alométrico adaptado de \cite{chave2014} y las directrices del IPCC~\cite{ipcc2006}, incorporando tanto la biomasa aérea como la biomasa radical mediante la relación Parte Radical:Parte Aérea ($R$). 
    El resultado se expresa en toneladas de carbono totales por parcela, sin normalizar por superficie, lo que facilita la trazabilidad del proceso y la comparación entre inventarios sin depender de factores de expansión específicos del IFN. 
    En coherencia con los criterios de proyectos de forestación y reforestación, las observaciones correspondientes a brinzales o plantones se consideran con valor de carbono nulo, dado que las fases tempranas de desarrollo no se contabilizan oficialmente como carbono capturado.
\end{enumerate}

\medskip

Estas dos variables resumen el contenido de carbono forestal desde enfoques complementarios: 
\texttt{c} (tC/ha) permite la comparación espacial y temporal entre masas forestales, mientras que \texttt{carbono\_bruto} (tC) ofrece una medida absoluta y directamente derivada de las observaciones de campo. 
Ambas constituyen los objetivos principales del modelado predictivo, orientado a estimar el carbono acumulado en el \textbf{IFN4} a partir de las condiciones registradas en los inventarios anteriores (\textbf{IFN2} e \textbf{IFN3}).

\subsection{Supuestos de elegibilidad y verificación externa}
% Intervención humana, permanencia 30a (extrapolación cauta), superficie>=1ha,
% fccarb>=20% (filtro), altura>=3m en madurez (decisión de diseño).

Para que un proyecto forestal sea elegible en programas de \emph{créditos de carbono}, debe cumplir requisitos técnicos establecidos por marcos regulatorios internacionales \cite{IPCC2006, miteco_guia_co2}. A continuación se resume cada criterio y la forma en que se aborda en este estudio:

\begin{itemize}
    \item \textbf{Intervención humana directa.} El incremento de carbono debe proceder de actuaciones planificadas (reforestación, restauración o manejo sostenible). En nuestro caso, el modelo se entrena sobre datos observacionales (IFN2--IFN3--IFN4); por tanto, la \emph{verificación de intervención} no se deduce del modelo, sino que se contempla como \emph{condición externa} de elegibilidad del proyecto a evaluar.

    \item \textbf{Permanencia mínima de 30 años.} Para caracterizar el crecimiento de las parcelas forestales en los datos que alimentan el modelo, es necesario disponer de dos mediciones sucesivas de cada parcela, separadas por un intervalo temporal conocido. Estas mediciones permiten cuantificar la evolución de las variables forestales y, por tanto, estimar el incremento de carbono asociado al crecimiento del arbolado durante dicho periodo. 

    En este trabajo, el objetivo es predecir el contenido de carbono correspondiente al \textbf{IFN4}, utilizando como información explicativa las variables observadas en inventarios anteriores. Dado que los inventarios tercero y cuarto comparten una estructura homogénea y un conjunto de variables comparable la elección más directa para el entrenamiento del modelo sería emplear exclusivamente estos dos inventarios. Esta estrategia aprovecha la coherencia estructural de los inventarios más recientes, que incluyen un mayor número de variables y una caracterización más detallada del terreno. 

    \medskip

    No obstante, este planteamiento se enfrenta a la limitación impuesta por la \textbf{permanencia mínima de 30 años}, requisito fundamental en el contexto de los proyectos de compensación. El intervalo de tiempo entre los inventarios \textbf{IFN3} e \textbf{IFN4} es relativamente corto: no supera los 18 años.   
    
    \medskip

    La Figura~\ref{fig:periodo34} muestra la distribución de la diferencia de años entre las mediciones del IFN3 y el IFN4. Como puede observarse, la mayoría de las parcelas presentan intervalos comprendidos entre 6 y 17 años, un rango demasiado estrecho para evaluar la estabilidad del modelo en horizontes más amplios.

    \begin{figure}[h!]
        \centering
        \includegraphics[width=0.9\textwidth]{figuras/periodo34.png}
        \caption{Distribución de la diferencia de años entre los inventarios IFN3 e IFN4.}
        \label{fig:periodo34}
    \end{figure}

    Para ampliar la cobertura temporal y mejorar la capacidad de generalización del modelo, se optó por unificar la información de los inventarios \textbf{IFN2} e \textbf{IFN3} como base explicativa para la predicción del \textbf{IFN4}. Esta integración permite disponer de pares de mediciones de parcelas separadas por intervalos que oscilan entre 6 y 29 años, lo que constituye un rango mucho más representativo del horizonte de 20--30 años establecido como referencia.

    \begin{figure}[h!]
        \centering
        \includegraphics[width=0.9\textwidth]{figuras/periodo234.png}
        \caption{Distribución de la diferencia de años entre los inventarios IFN2--IFN3 e IFN3--IFN4.}
        \label{fig:periodo234}
    \end{figure}

    \medskip

    De esta forma, el modelo se entrena y valida sobre un conjunto de datos más diverso y equilibrado, tanto en estructura como en amplitud temporal, manteniendo la coherencia metodológica y la trazabilidad de las estimaciones. Este enfoque no sólo mejora la robustez del aprendizaje, sino que también refuerza la capacidad del modelo para proyectar la captura de carbono en escenarios compatibles con los requisitos de permanencia de los proyectos de compensación.

    \item \textbf{Superficie mínima de 1 ha.} Este criterio se considera \emph{externo} al alcance del modelo predictivo, ya que el aprendizaje se realiza a nivel de parcela e inventario y no sobre polígonos de superficie total. En la práctica, la verificación de la superficie se realiza \emph{ex ante}, sobre la geometría declarada del proyecto forestal. En los terrenos forestales generados a partir de intervención humana directa —como plantaciones o repoblaciones—, la extensión suele presentar una estructura homogénea, con una especie dominante, edades coetáneas y densidades estandarizadas. Bajo estas condiciones, el carbono total es proporcional a la superficie: duplicar el área de una masa forestal homogénea implica aproximadamente duplicar su carbono almacenado. Por tanto, la variable de superficie no afecta al ajuste interno del modelo y su cumplimiento puede evaluarse fácilmente a nivel de proyecto, sin comprometer la validez de las predicciones.

    \item \textbf{Fracción mínima de cabida cubierta del 20\%.} La base de datos dispone de \texttt{fccarb} (arbórea) y \texttt{fcctot} (total). Este umbral se aplica como \emph{filtro de elegibilidad} previo o posterior al modelado, sin modificar la arquitectura del modelo (\texttt{fccarb}$>20$).

    \item \textbf{Altura mínima de 3 m en la madurez.} Este requisito se refiere a la altura que alcanzan los árboles en su fase de pleno desarrollo, y no a la altura inicial de los plantones. Por tanto, las mediciones realizadas durante las etapas tempranas de crecimiento no determinan la elegibilidad del proyecto, siempre que las especies seleccionadas sean capaces de superar los 3 metros en la madurez. En nuestro conjunto de datos, la altura no se registra explícitamente, por lo que este criterio se evalúa de forma \emph{externa} al modelo, mediante la selección de especies forestales adecuadas y la verificación con fuentes auxiliares (catálogos silvícolas o tipologías de masa). En la práctica, el cumplimiento del requisito depende de una decisión de diseño del proyecto —\emph{no plantar especies cuyo tamaño adulto sea inferior a 3 metros}— más que del ajuste predictivo del modelo. Por ello, la altura no interviene directamente en el entrenamiento, aunque sí condiciona la elegibilidad final del proyecto forestal. 
\end{itemize}

\subsection{Preparación y tratamiento de los datos}
% Filtros (fccarb>=20%, crecimiento positivo, etc.)
% Agregaciones (parcela-especie, compresión CD -> npies_{cd}, etc.)
% Cálculo de variables derivadas (carbono\_bruto)
% Codificación y escalado

Como ya se ha introducido el entrenamiento se realiza en dos líneas según la variable objetivo: \texttt{c} de \textbf{IFN4} o \texttt{carbono\_bruto} de \textbf{IFN4}; y según la información que se usa como explicativa: \textbf{IFN3} o \textbf{IFN3} e \textbf{IFN2}. Se plantea la preparación y filtrado de los datos en términos generales (variable objetivo por \texttt{c} o \texttt{carbono\_bruto} y primera inventariación/ inventariación explicativa por \textbf{IFN3} o la unión de \textbf{IFN2} e \textbf{IFN3}).

\subsubsection*{Filtrado de registros}

Se descartan todas aquellas parcelas en las que el valor de carbono total (variable objetivo) en la segunda inventariación es inferior a la primera. Estos casos suelen deberse a episodios de deforestación, incendios u otras perturbaciones, y no representan un crecimiento forestal neto.

\medskip

El conjunto de datos se restringe únicamente a las parcelas que presentan una \texttt{fccarb} (fracción de cabida cubierta arbórea) igual o superior al 20\,\% en el \textbf{IFN3}. Este umbral define la proporción mínima de superficie ocupada por copas de árboles respecto al área total de la parcela, y constituye una de las condiciones esenciales para considerar una superficie como terreno forestal. La exclusión de parcelas con \texttt{fccarb} inferior al 20\,\% permite asegurar que las estimaciones de carbono se realicen sobre masas forestales consolidadas, evitando sesgos asociados a áreas agrícolas o matorrales. A los datos del \textbf{IFN2} no se les aplica dicho filtro porque no disponen de la variable \texttt{fccarb}.

\medskip

Los conteos de observaciones por inventario y condición se resumen a continuación:

\begin{itemize}
    \item \textbf{IFN2:} Total de parcelas = \textbf{88.696}
    \begin{itemize}
        \item Casos con $c4 > c$: \textbf{31.428}
        \item Casos con $carbono\_bruto4 > carbono\_bruto$: \textbf{32.403}
    \end{itemize}

    \item \textbf{IFN3:} Total de parcelas = \textbf{171.157}
    \begin{itemize}
        \item Casos con $fccarb > 20$: \textbf{158.434}
        \item Casos con $fccarb > 20$ y $c4 > c$: \textbf{57.401}
        \item Casos con $fccarb > 20$ y $carbono\_bruto4 > carbono\_bruto$: \textbf{76.617}
    \end{itemize}
\end{itemize}

\subsubsection*{Cálculo y agregación de variables}

Cada registro de entrada se genera a nivel de combinación parcela--especie, incorporando las variables correspondientes de la primera medición y la variable objetivo (carbono) de la segunda medición (IFN4). Las variables de \texttt{parcela} y \texttt{parcela\_inventario} se desdoblan para cada especie. Las entradas de la tabla \texttt{parcela\_inventario\_especie\_cd} se agrupan por parcela y especie y se comprimen en una única entrada creando un conjunto de variables para cada clase diamétrica.

\medskip

La Tabla~\ref{tab:entrada_modelo} resume las variables empleadas como entrada al modelo, integradas desde las distintas tablas que conforman la base de datos relacional.

\begin{table}[H]
\renewcommand{\arraystretch}{1.2}
\setlength{\tabcolsep}{3pt}
\centering
\small
\resizebox{\textwidth}{!}{%
\begin{tabular}{|p{3.6cm}|p{2.2cm}|p{7.6cm}|p{2.2cm}|}
\hline
\multicolumn{4}{|c|}{\textbf{Resumen de Datos de Entrada del Modelo}} \\
\hline
\textbf{Variable} & \textbf{Tipo} & \textbf{Descripción} & \textbf{Anexo} \\
\hline
\textcolor{ForestGreen}{\texttt{parcela\_id}} & varchar & Identificador \emph{único} de parcela. & -- \\
\hline
\textcolor{ForestGreen}{\texttt{especie\_id}, \texttt{tipo\_especie}, \texttt{grupo\_id}} & int (CF) & Especie (código), tipo de especie y grupo taxonómico. & Anexos \ref{sec:especies} y \ref{sec:gruposespecies} \\
\hline
\textcolor{ForestGreen}{\texttt{ocupa}} & int & Grado de ocupación/presencia de la especie en la parcela (0--10). & -- \\
\hline
\textcolor{ForestGreen}{\texttt{estado\_id}}, \texttt{fpmasa\_id}, \texttt{tratmasa\_id}, \texttt{orgmasa\_1\_id} & int (CF) & Estado/fase de desarrollo, forma principal de masa, tratamiento de masa, organización de masa. & Anexos \ref{sec:EstadoIFN34}, \ref{sec:FPMasa}, \ref{sec:tratmasa} y \ref{sec:OrgMasa} \\
\hline
\textcolor{ForestGreen}{\texttt{tipsuelo1\_id}, \texttt{tipsuelo2\_id}, \texttt{tipsuelo3\_id}} & int (CF) & Tipos de suelo de la parcela (niveles jerárquicos). & Anexo \ref{sec:TipSuelo} \\
\hline
\textcolor{ForestGreen}{\texttt{rocosidad\_id}}, \texttt{textura\_id}, \texttt{matorg\_id}, \texttt{modcomb\_id}, \textcolor{ForestGreen}{\texttt{disesp\_id}, \texttt{comesp\_id}}, \texttt{merosiva\_id} & int (CF) & Rocosidad, textura, materia orgánica, modelo de combustibilidad, distribución/composición específica, manifestaciones erosivas. & Anexos \ref{sec:Rocosid}, \ref{sec:textura}, \ref{sec:MatOrg}, \ref{sec:modComb}, \ref{sec:disEsp}, \ref{anexo:compesp} y \ref{sec:ManERo}. \\
\hline
\textcolor{ForestGreen}{\texttt{radio}} & float & Radio de la parcela (m). & -- \\
\hline
\textcolor{ForestGreen}{\texttt{orientacion}, \texttt{elevacion}, \texttt{pendiente}} & float & Orientación (grados), elevación (m s.n.m.), pendiente (\%). & -- \\
\hline
\texttt{nivel1\_id}, \texttt{nivel2\_id} & int (CF) & Niveles jerárquicos/estratos de inventario. & Anexos \ref{sec:nivel1} y \ref{sec:nivel2}. \\
\hline
\texttt{fccarb}, \texttt{fcctot} & float & Fracción de cabida cubierta arbórea y total. & -- \\
\hline
\texttt{\textcolor{ForestGreen}{npies}\_\{1,2,5,\textcolor{ForestGreen}{10,15, 20,25,30,35,40,45, 50,55,60,65,70}\}} & float & Número de pies por clase diamétrica CD (cm). Cada campo corresponde a la CD indicada. & -- \\
\hline
\textcolor{ForestGreen}{\texttt{periodo}} & int & Años transcurridos entre inventarios considerados en el modelo. & -- \\
\hline
\textcolor{ForestGreen}{\texttt{evi\_\{stat\}\_\{est\}}} & float & Índice EVI por estación; \texttt{stat} $\in$ \{max, mean, median, min, std\}, \texttt{est} $\in$ \{invierno, oto\~no, primavera, verano\}. & -- \\
\hline
\textcolor{ForestGreen}{\texttt{gndvi\_\{stat\}\_\{est\}}} & float & Índice GNDVI por estación (misma convención de \texttt{stat} y \texttt{est}). & -- \\
\hline
\textcolor{ForestGreen}{\texttt{ndii\_\{stat\}\_\{est\}}} & float & Índice NDII por estación (misma convención). & -- \\
\hline
\textcolor{ForestGreen}{\texttt{ndvi\_\{stat\}\_\{est\}}} & float & Índice NDVI por estación (misma convención). & -- \\
\hline
\textcolor{ForestGreen}{\texttt{pr\_\{stat\}\_\{est\}}} & float & Precipitación: \texttt{stat} $\in$ \{max, mean, min, std, sum\} por estación \texttt{est}. & -- \\
\hline
\textcolor{ForestGreen}{\texttt{skt\_\{stat\}\_\{est\}}} & float & Temperatura de superficie (skin temperature): \texttt{stat} $\in$ \{max, mean, min, std\}. & -- \\
\hline
\textcolor{ForestGreen}{\texttt{stl1-4\_\{stat\}\_\{est\}}} & float & Temperatura de suelo por capa (1--4): \texttt{stat} $\in$ \{max, mean, min, std\}. & -- \\
\hline
\textcolor{ForestGreen}{\texttt{t2m\_\{stat\}\_\{est\}}} & float & Temperatura del aire a 2 m: \texttt{stat} $\in$ \{max, mean, min, std\}. & -- \\
\hline
\textcolor{ForestGreen}{\texttt{c4}} & float & Carbono capturado en el cultivo en el IFN4 en t/ha. & -- \\
\hline
\textcolor{ForestGreen}{\texttt{carbono\_bruto4}} & float & Carbono capturado en el cultivo en el IFN4 en t. & -- \\
\hline
\end{tabular}%
}
\caption{\tiny Resumen de variables de entrada del modelo. Para variables estacionales se usa la notación \texttt{variable\_\{stat\}\_\{est\}}, con estadísticas \texttt{stat} y estaciones \texttt{est} en \{invierno, oto\~no, primavera, verano\}. Las variables \texttt{npies\_\{CD\}} se repiten para cada clase diamétrica indicada. Las variables destacadas en \textcolor{ForestGreen}{verde} se encuentran recogidas tanto para el IFN2 como para el IFN3, las no destacadas solo se recogen para el IFN3.}
\label{tab:entrada_modelo}
\end{table}

\subsubsection*{Codificación y normalización}

Las variables categóricas se codifican mediante \textit{one-hot encoding}, generando variables binarias para cada clase. Las variables numéricas se escalan (normalización estándar o min-max, según el modelo) para asegurar que todas las magnitudes tengan el mismo orden de importancia durante el entrenamiento.



\section{Selección de variables explicativas}

La selección de variables explicativas se abordó mediante tres estrategias complementarias: (1) selección automática basada en el algoritmo \textit{Featurewiz}, (2) selección mediante el criterio de relevancia y mínima redundancia (\textit{mRMR}), y (3) una selección manual fundamentada en criterios estadísticos, ecológicos y de interpretabilidad. El objetivo común fue identificar un subconjunto óptimo de predictores que maximice la capacidad explicativa del modelo sobre la variable dependiente \( c4 \) (carbono estimado), evitando colinealidad y preservando el sentido físico de las relaciones.

\subsection{Selección automática mediante Featurewiz}
El método \textit{Featurewiz} se basa en un enfoque de selección de características guiado por importancia predictiva. El procedimiento combina dos etapas principales: (i) un filtrado inicial por correlación, en el que se eliminan variables altamente colineales (en este caso, con un umbral de \( |r| > 0.70 \)); y (ii) un refinamiento mediante modelos de \textit{Gradient Boosting} que estiman la importancia relativa de cada variable en la predicción del objetivo.  
De esta manera, \textit{Featurewiz} conserva únicamente aquellas variables con una contribución significativa a la mejora del rendimiento predictivo, proporcionando un conjunto compacto y eficiente de predictores.

\subsection{Selección mediante mRMR}
El enfoque \textit{mRMR} (minimum Redundancy - maximum Relevance) selecciona las variables que maximizan su relevancia estadística respecto a la variable objetivo, minimizando al mismo tiempo la redundancia entre ellas. Este método utiliza medidas de información mutua para cuantificar la dependencia no lineal entre las variables.  
En la práctica, el algoritmo mRMR prioriza aquellas variables que aportan información nueva y no redundante sobre el fenómeno modelado (en este caso, la acumulación de carbono), favoreciendo la diversidad informativa frente a la mera fuerza de correlación. Este enfoque permite obtener un conjunto equilibrado de predictores que explican diferentes dimensiones del sistema ecológico.


\subsection{Selección manual basada en criterios estadísticos y conceptuales}
La selección manual de variables se realizó de forma guiada por criterios tanto estadísticos como conceptuales. 
En primer lugar, se evaluó la significancia de cada variable mediante pruebas univariantes (ANOVA y correlaciones), eliminando aquellas sin influencia estadísticamente significativa sobre la variable objetivo. 
Posteriormente, se analizaron las correlaciones entre predictores para reducir la colinealidad, manteniendo únicamente una variable representativa de cada grupo altamente correlacionado. 
Además, se consideraron criterios ecológicos y de interpretación física, asegurando que las variables retenidas representasen aspectos estructurales, edáficos, topográficos, climáticos y espectrales relevantes para el proceso de acumulación de carbono. 
El objetivo fue equilibrar la robustez estadística con la coherencia ecológica, obteniendo un conjunto final de predictores que mantuviera un compromiso entre precisión, interpretabilidad y sentido biogeográfico.


\subsection{Partición y validación}

Para obtener una estimación imparcial del rendimiento y evitar \emph{fugas de información} debidas a la correlación espacial dentro de cada parcela, la partición del conjunto de datos se realiza \textbf{por identificador de parcela} (\texttt{parcela\_id}). Todas las observaciones asociadas a una misma parcela se asignan \emph{íntegramente} a un único subconjunto, de modo que ninguna parcela aparece simultáneamente en entrenamiento y evaluación.
\vspace{0.25em}
\noindent\textbf{Validación interna y control de sesgo temporal.} Sobre el subconjunto de entrenamiento (80\,\%) se aplica \emph{validación cruzada por grupos} utilizando como agrupador los \emph{años transcurridos entre inventarios} (p.\,ej., 15, 16, 17, \dots). Esta estrategia comprueba la \emph{estabilidad} del modelo frente a cambios en el horizonte temporal y reduce el riesgo de sobreajuste específico de un periodo. La selección de hiperparámetros se realiza exclusivamente dentro de esta validación interna; el conjunto de evaluación (20\,\%) permanece \emph{sellado} para la prueba final.

\vspace{0.25em}
\noindent\textbf{Métricas de evaluación.} El rendimiento se informa con dos medidas complementarias:
\begin{itemize}
    \item \textbf{RMSE (Root Mean Squared Error):} raíz del error cuadrático medio entre valores observados y predichos; se expresa en las mismas unidades que la variable objetivo y penaliza con mayor peso los errores grandes. Valores más bajos indican mejor ajuste.
    \item \boldmath\textbf{$R^2$}\unboldmath{} (coeficiente de determinación): proporción de la varianza observada explicada por el modelo (idealmente en $[0,1]$). Valores cercanos a 1 denotan alta capacidad explicativa; puede ser negativo si el modelo es peor que la predicción constante.
\end{itemize}

\vspace{0.25em}
\noindent\textbf{Protocolo de reporte.} Para cada modelo se reportan: (i) el rendimiento medio y la dispersión en la validación cruzada por grupos (entrenamiento), y (ii) el desempeño final en el conjunto de evaluación independiente (20\,\%). Este protocolo garantiza comparabilidad entre modelos, control del sesgo espacial por parcela y verificación explícita de la robustez temporal.




\subsection{Modelos evaluados}
% Familias, tuning (RandomizedSearchCV), por qué ensembles, etc.


A continuación, se detalla el diseño general y las estrategias empleadas para la selección y optimización de modelos.

\subsubsection*{Entrenamiento y optimización}

Se aplicó \textit{RandomizedSearchCV}, una técnica de búsqueda aleatoria de hiperparámetros que evalúa distintas combinaciones utilizando validación cruzada. Este procedimiento permite optimizar el rendimiento de cada modelo sin incurrir en un coste computacional tan elevado como el de una búsqueda exhaustiva.

\medskip

\paragraph{Validación cruzada por grupos.} 


En algunas configuraciones probadas, se emplea la validación cruzada por grupos (\textit{Group k-Fold Cross Validation}). Este método divide el área de estudio en $k$ bloques, basados en una característica común de los datos, asegurando que los datos dentro de cada bloque estén relacionados. El modelo se entrena con $k-1$ bloques y se valida con el bloque restante, repitiendo este proceso $k$ veces.

\medskip

Este enfoque es útil cuando los datos tienen agrupaciones naturales, como por ejemplo, diferentes parcelas o periodos de tiempo. Al mantener los datos relacionados en un mismo bloque, se evita la filtración de información entre los conjuntos de entrenamiento y validación, lo que permite una mejor evaluación de la capacidad de generalización del modelo. Así, se asegura que el modelo no se sobreajuste y sea robusto al ser evaluado en contextos no vistos previamente.


\subsubsection*{Modelos ensemble}

Para mejorar la precisión y robustez, se emplearon diversos métodos de \textit{ensemble learning}, que combinan múltiples modelos base (\textit{base learners}) para generar una predicción agregada. Esta estrategia se inspira en la teoría de la sabiduría colectiva: la combinación de estimaciones independientes tiende a superar a cualquier estimador individual.

\paragraph{Técnicas utilizadas:}
\begin{itemize}
    \item \textbf{Voting y Averaging:} combinan modelos ya entrenados mediante votación mayoritaria o promedio.
    \item \textbf{Bagging y Boosting:} construyen modelos desde cero y los combinan. Bagging reduce la varianza al entrenar modelos en subconjuntos aleatorios; Boosting mejora el sesgo al entrenar secuencialmente, corrigiendo errores anteriores.
    \item \textbf{Stacking:} combina modelos optimizados usando un metamodelo que aprende a integrar sus predicciones.
\end{itemize}

\subsubsection*{Boosting y aprendizaje gradual}

El \textit{boosting} se basa en el aprendizaje secuencial, donde cada nuevo modelo intenta corregir los errores residuales del anterior. Esta técnica permite construir modelos fuertes a partir de modelos débiles, alcanzando gran precisión. Sin embargo, requiere una cuidadosa configuración de hiperparámetros para evitar el sobreajuste.

Entre las implementaciones destacadas se incluyen:

\begin{itemize}
    \item \textbf{XGBoost:} modelo GBM que optimiza rendimiento con gradientes de primer y segundo orden, regularización L1/L2, manejo automático de valores faltantes, y técnicas de generalización como \textit{shrinkage} y \textit{column subsampling}.
    \item \textbf{LightGBM:} algoritmo eficiente para grandes volúmenes de datos, con crecimiento \textit{leaf-wise} y soporte nativo para variables categóricas.
    \item \textbf{AdaBoost:} ajusta modelos simples secuencialmente, enfocando el aprendizaje en observaciones mal clasificadas.
    \item \textbf{CatBoost:} especializado en variables categóricas y robusto frente a datos ruidosos, usando codificación por orden aleatorio.
    \item \textbf{Gradient Boosting Decision Trees (GBDT):} construye árboles secuenciales ajustados a residuos, optimizando mediante descenso por gradiente.
\end{itemize}

\subsubsection*{Bagging}

El \textit{bagging} (Bootstrap Aggregating) entrena múltiples modelos independientes sobre subconjuntos de datos generados por muestreo con reemplazo. Las predicciones se combinan por promedio o votación. Esta técnica reduce la varianza y mejora la estabilidad de modelos inestables.

\begin{itemize}
    \item \textbf{Random Forest:} combina árboles de decisión (CART) con selección aleatoria de características en cada división. Es escalable, robusto a datos faltantes, y menos propenso al sobreajuste.
    \item \textbf{Bagged Decision Trees (BaggedDT):} genera árboles sin poda entrenados en muestras bootstrap. Promedia sus predicciones para reducir la varianza.
\end{itemize}

\subsubsection*{Otros modelos utilizados}

Además de los métodos ensemble, se evaluaron modelos representativos de distintos paradigmas de aprendizaje supervisado:

\begin{itemize}
    \item \textbf{K-Nearest Neighbors (KNN):} modelo basado en instancia que predice a partir de los vecinos más cercanos. Sensible a la escala y a \textit{outliers}.
    \item \textbf{Multi-Layer Perceptron (MLP):} red neuronal con una o más capas ocultas, capaz de modelar relaciones no lineales complejas.
    \item \textbf{Support Vector Regression (SVR):} modelo de márgenes para regresión, con soporte para kernels no lineales.
    \item \textbf{SVM con kernel:} modelo poderoso para clasificación y regresión no lineal, aunque costoso y sensible a hiperparámetros.
    \item \textbf{Bayesian Neural Network:} enfoque probabilístico que estima incertidumbre en las predicciones. Incluye variantes como la \textit{Bayesian Ridge Regression}.
    \item \textbf{Naive Bayes:} clasificador probabilístico rápido y simple, útil en texto y alta dimensionalidad. Se evaluaron variantes:
        \begin{itemize}
            \item \textit{Gaussian Naive Bayes:} para datos continuos.
            \item \textit{Multinomial Naive Bayes:} para conteos y texto.
            \item \textit{Bernoulli Naive Bayes:} para variables binarias.
        \end{itemize}
\end{itemize}

\subsubsection*{Comparación y justificación de modelos}

La evaluación de múltiples modelos responde a la necesidad de identificar no solo el de mejor rendimiento, sino también el más adecuado según la naturaleza del problema y los datos disponibles. Se compararon algoritmos lineales, no lineales, basados en vecinos, redes neuronales, modelos probabilísticos y diferentes técnicas de \textit{ensemble}.  Vemos un resumen de los modelos aplicados en la tabla \ref{tab:modelos}.


\begin{table}[H]\small
\centering
\resizebox{\textwidth}{!}{%
\begin{tabular}{|p{3.2cm}|p{2.8cm}|p{5.2cm}|p{4.2cm}|}
\hline
\textbf{Modelo} & \textbf{Tipo / Técnica} & \textbf{Características destacadas} & \textbf{Observaciones} \\
\hline
\textbf{Random Forest} & Bagging (Árboles) & Uso de bootstrap, selección aleatoria de atributos, reducción de varianza & Robusto y escalable; menor interpretabilidad \\
\hline
\textbf{Bagged Decision Trees (BaggedDT)} & Bagging & Árboles sin poda, entrenados en paralelo sobre muestras con reemplazo & Preciso pero costoso computacionalmente \\
\hline
\textbf{XGBoost} & Boosting (GBM) & Regularización L1/L2, manejo de valores faltantes, poda anticipada & Alto rendimiento, sensible a hiperparámetros \\
\hline
\textbf{LightGBM} & Boosting (Leaf-wise) & Crecimiento hoja a hoja, eficiente en grandes volúmenes & Rápido y preciso; riesgo de sobreajuste \\
\hline
\textbf{AdaBoost} & Boosting (Stumps) & Aumenta peso de errores, pondera modelos por precisión & Sencillo y efectivo con datos limpios \\
\hline
\textbf{CatBoost} & Boosting especializado & Codificación avanzada de variables categóricas, robustez a ruido & Ideal para datos heterogéneos \\
\hline
\textbf{Gradient Boosting Decision Trees (GBDT)} & Boosting & Árboles secuenciales ajustados a residuos & Buen rendimiento; mayor coste de entrenamiento \\
\hline
\textbf{K-Nearest Neighbors (KNN)} & Basado en instancia & No requiere entrenamiento, predice por proximidad & Sensible a escala y outliers \\
\hline
\textbf{Multi-Layer Perceptron (MLP)} & Red neuronal & Modela relaciones no lineales complejas & Requiere normalización y regularización \\
\hline
\textbf{Support Vector Regression (SVR)} & Kernel y márgenes & Predicción dentro de tolerancia $\varepsilon$, uso de kernels no lineales & Robusto; elevado coste computacional \\
\hline
\textbf{SVM con kernel} & SVM no lineal & Maximiza margen, admite distintos kernels (RBF, polinomial, etc.) & Alta precisión; sensible a hiperparámetros \\
\hline
\textbf{Bayesian Neural Network / Ridge Regression} & Probabilístico / Bayesiano & Predicción con incertidumbre, estimación automática de hiperparámetros & Útil para inferencia y regularización \\
\hline
\textbf{Naive Bayes (Gaussian, Multinomial, Bernoulli)} & Probabilístico & Asume independencia condicional, rápido y simple & Eficaz en texto y alta dimensionalidad \\
\hline
\end{tabular}
}
\caption{Resumen de modelos de aprendizaje supervisado aplicados}
\label{tab:modelos}
\end{table}


\medskip


