\section{Metodología}

Esta sección describe el procedimiento seguido para entrenar y validar los modelos predictivos desarrollados. La metodología se fundamenta en la identificación de los factores que condicionan el crecimiento forestal y, por tanto, la capacidad de captura de carbono en un periodo determinado. Para ello, se ha trabajado sobre una base de datos específicamente estructurada para facilitar su uso en tareas de modelización ecológica y análisis predictivo.

\medskip

El carbono absorbido por los árboles se acumula en su biomasa a lo largo del tiempo. La cantidad almacenada depende del tamaño de los individuos, el cual está influenciado por múltiples variables ambientales y estructurales. Así, la estimación del carbono capturado entre dos momentos se realiza midiendo el crecimiento de los árboles y aplicando factores de conversión, considerando que aproximadamente el 50\% de la biomasa seca corresponde a carbono \cite{MITECO2020}.

\medskip

La base de datos empleada recoge características forestales, climáticas y espectrales a nivel de parcela, especie y clase diamétrica. Esta estructura relacional permite capturar la evolución del bosque entre inventarios y alimentar modelos capaces de predecir el contenido futuro de carbono a partir de observaciones pasadas. El conjunto de variables integradas satisface los principales requisitos ecológicos y estadísticos del problema, garantizando una representación suficiente del entorno forestal.

\medskip

El crecimiento de las especies arbóreas está condicionado por factores edáficos, climáticos, topográficos y de competencia intraespecífica. Las condiciones meteorológicas, como la temperatura o la precipitación, afectan a la fotosíntesis y la disponibilidad hídrica. La orientación, la pendiente y la altitud modifican la radiación recibida y la capacidad de retención de agua. A su vez, la densidad de árboles por superficie determina el nivel de competencia por los recursos, lo cual varía según la especie y su tolerancia ecológica \cite{IPCC2006, Buchholz2014}.

\medskip

A partir de estos principios, la base de datos recoge un conjunto diverso de variables que incluyen:

\begin{itemize}
    \item \textbf{Características del terreno:} textura, tipo de suelo, orientación, pendiente y elevación, que condicionan el entorno físico del crecimiento.
    
    \item \textbf{Estructura y composición forestal:} cobertura total y arbórea, ocupación específica por especie, forma y estado de la masa, y distribución espacial del arbolado.
    
    \item \textbf{Variables dendrométricas:} número de pies, área basimétrica y volumen por clase diamétrica, como indicadores directos de biomasa y densidad forestal.
    
    \item \textbf{Condiciones climáticas:} medias y extremos de temperatura y precipitación por estación, obtenidas a partir de registros satelitales y reanálisis.
    
    \item \textbf{Índices de vegetación:} métricas como NDVI, NDII, GNDVI y EVI, que capturan el vigor, humedad y densidad de la cubierta vegetal (véase Sección~\ref{sec:indices}).
\end{itemize}

\subsection{Objetivo y alcance}
% Unifica aquí todo el objetivo (una sola vez)
% Qué se predice (carbono total), horizonte 20--30 años, dos variables (t y t/ha)


El objetivo principal de este trabajo es estimar el \textbf{carbono total} que una parcela forestal será capaz de capturar en un horizonte temporal de aproximadamente 20 a 30 años. Este valor constituye un indicador clave del potencial de mitigación climática de los ecosistemas forestales y resulta esencial para la planificación de proyectos de compensación de emisiones. 

\medskip

El modelo se entrena para predecir dos variables de respuesta relacionadas con la captura de carbono: \texttt{c} y \texttt{carbono\_bruto}. La primera (\texttt{c}) representa el carbono total contenido en la biomasa aérea y subterránea por especie, parcela e inventario, expresado en toneladas de carbono por hectárea (tC/ha). Su origen se encuentra en las variables del Inventario Forestal Nacional (IFN), donde se reportan estimaciones del contenido de carbono aéreo (\texttt{ca}), radical (\texttt{cr}) y total (\texttt{c}). Sin embargo, la documentación del IFN no detalla el método de cálculo empleado, por lo que se asume que siguen las directrices de la \textit{Guía del MITECO para la estimación de absorciones de \(\text{CO}_2\)} \cite{miteco_guia_co2}. Estas variables presentan vacíos notables en el IFN4 y ausencia completa en los inventarios previos (IFN2 e IFN3), por lo que al crear la base de datos se recurrió a un modelo \emph{Random Forest Regressor} para imputar los valores faltantes a partir de variables dendrométricas observadas (Especie, CD, VSC, NPies, ABas, IAVC, VCC y VLE). Aunque este enfoque mostró un rendimiento satisfactorio (\(R^2_{test} > 0.90\)), sus estimaciones derivan de un modelo estadístico y no de mediciones directas.

\medskip

La variable \texttt{carbono\_bruto} (tC) se desarrolló como alternativa trazable y físicamente interpretable, calculada directamente a partir de variables medidas en campo: número de pies (\texttt{npies}), altura media (\texttt{ht}), tipo de especie (\texttt{clase\_especie}) y clase diamétrica (\texttt{cd\_id}). Su estimación sigue un modelo alométrico adaptado de \cite{chave2014} y las directrices del IPCC~\cite{IPCC2006}. La biomasa aérea individual se obtiene mediante ecuaciones específicas por tipo de especie:
\[
\text{Biomasa}_{\text{coníferas}} = 0.05 \times (CD^2 \times Ht)^{0.9}, \quad
\text{Biomasa}_{\text{frondosas}} = 0.067 \times (CD^2 \times Ht)^{0.9}.
\]
A partir de esta biomasa se calcula el carbono por árbol aplicando el factor de conversión recomendado por el IPCC:
\[
\text{Carbono}_{\text{árbol}} = \frac{\text{Biomasa} \times 0.5}{1000}.
\]
Finalmente, el \texttt{carbono\_bruto} total se obtiene multiplicando el carbono por árbol por el número de pies de cada clase diamétrica:
\[
\text{Carbono\_bruto} = \text{Carbono}_{\text{árbol}} \times \texttt{npies}.
\]
El resultado se expresa en toneladas de carbono (t) por parcela y especie, sin normalizar por superficie, lo que facilita la trazabilidad y la comparación entre inventarios. Esta aproximación evita dependencias con factores de expansión específicos del IFN y proporciona una medida más transparente y replicable del carbono capturado a partir de datos observados. En coherencia con los criterios de proyectos de reforestación y forestación, las entradas correspondientes a brinzales o plantones se consideran con valor de carbono cero, dado que el carbono de las fases tempranas de desarrollo no se contabiliza como carbono capturado en los registros oficiales.

\medskip

Para que un proyecto forestal sea elegible dentro de los programas de \emph{créditos de carbono}, debe cumplir una serie de requisitos técnicos establecidos en los marcos regulatorios internacionales \cite{IPCC2006, miteco_guia_co2}. Estos criterios aseguran la validez ambiental y la permanencia del secuestro de carbono, y entre los principales se encuentran:

\begin{itemize}
    \item \textbf{Intervención humana directa:} el incremento del carbono debe ser resultado de una acción planificada (reforestación, restauración o manejo forestal sostenible).
    \item \textbf{Permanencia mínima de 30 años:} el carbono capturado debe mantenerse almacenado durante al menos tres décadas.
    \item \textbf{Superficie mínima de 1 hectárea:} el proyecto debe tener una extensión suficiente para generar beneficios climáticos verificables.
    \item \textbf{Fracción mínima de cabida cubierta del 20\%:} el área arbolada debe superar este umbral para ser considerada masa forestal.
    \item \textbf{Altura mínima de los árboles de 3 metros en la madurez:} requisito que garantiza la representatividad estructural del bosque.
\end{itemize}

\medskip

El objetivo principal de este trabajo es estimar el \textbf{carbono total} que una parcela forestal será capaz de capturar en un horizonte temporal de aproximadamente 20--30 años. El conjunto de datos incluye dos variables de carbono: una en \emph{toneladas totales} y otra en \emph{toneladas por hectárea}, lo que permite analizar el contenido de carbono en términos absolutos y normalizados por superficie.

\medskip


\subsection{Origen y estructura de los datos}
% Origen (IFN2/3/4 + clima + índices)
% Estructura relacional (parcelas, inventarios, especie, CD, estación, árbol)
% Referencia al esquema (figura) y a meta_variables


La base de datos empleada en este trabajo integra información forestal, climática y espectral estructurada en torno a la parcela como unidad básica. Cada parcela se describe mediante sus coordenadas geográficas, características edáficas y su evolución a través de distintos inventarios (IFN2, IFN3, IFN4).

\medskip

Los datos forestales incluyen información por especie y clase diamétrica, como número de pies, volumen con y sin corteza, área basimétrica, carbono aéreo, radical y total. Estos valores permiten caracterizar con precisión la estructura y crecimiento de la vegetación.

\medskip

A cada parcela se asocian también estadísticas climáticas agregadas por estación e inventario: temperaturas (superficie, aire y subsuelo) y precipitaciones, resumidas mediante métricas como media, máxima, mínima y desviación típica.

\medskip

Finalmente, se incorporan índices espectrales derivados de imágenes satelitales (NDVI, EVI, NDII, GNDVI), que permiten cuantificar propiedades biofísicas de la vegetación:
\begin{itemize}
    \item \textbf{NDVI (Normalized Difference Vegetation Index):} estima la actividad fotosintética.
    \item \textbf{EVI (Enhanced Vegetation Index):} mejora la sensibilidad en zonas densamente vegetadas.
    \item \textbf{NDII (Normalized Difference Infrared Index):} refleja el contenido hídrico de la vegetación.
    \item \textbf{GNDVI (Green NDVI):} variante del NDVI basada en la banda verde, sensible al clorofila.
\end{itemize}

\medskip

\subsubsection*{Estrutura de la base de datos}
Estos datos se organizan en las siguientes entidades troncales:

\begin{itemize}
  \item \textbf{parcelas}: icontiene la información básica de localización y características edáficas de cada parcela.
  \item \textbf{parcela\_inventario}: describe el estado de cada parcela en un inventario determinado (\texttt{parcela\_id}, \texttt{inventario\_id}), incluyendo atributos edáficos y de contexto (p. ej., \texttt{nivel1\_id}, \texttt{textura\_id}).
  \item \textbf{parcela\_inventario\_especie}: detalla la presencia y condición de cada especie dentro de una parcela e inventario, incorporando descriptores de masa y tratamientos silvícolas.
  \item \textbf{parcela\_inventario\_especie\_cd}: describe las poblaciones arbóreas por parcela, especie y \emph{clase diamétrica} (\texttt{cd\_id}): n.º de pies (\texttt{npies}), área basimétrica (\texttt{abas}), volúmenes (\texttt{vcc}, \texttt{vsc}, \texttt{vle}), incrementos (\texttt{iavc}) y carbono (\texttt{ca}, \texttt{cr}).
  \item \textbf{parcela\_especie\_arbol}: caracteriza los pies mayores identificados por parcela y especie en el inventario cuarto. Recoge las caracteristicas particulares de cada pie como altura (\texttt{ht}), diámetros (\texttt{dn1} y \texttt{dn2}), ubicación respecto del centro de la parcela (\texttt{rumbo}, \texttt{distancia}), volumen (\texttt{vcc, vsc, vle}), incremento (\texttt{iavc}) y carbono (\texttt{ca, cr}).  
  \item \textbf{parcela\_inventario\_estacion}: almacena agregados climático-biofísicos por estación (\texttt{estacion\_id}) en la misma granularidad parcela–inventario, incluyendo variables como precipitación (\texttt{PR}) y temperatura (\texttt{T2M, SKT, STL*}), junto a índices de vegetación (NDVI, EVI, NDII, GNDVI).
  \item \textbf{especies} y \textbf{grupos}: recogen la información taxonómica y su clasificación jerárquica, estableciendo la relación entre especies individuales y grupos funcionales.
\end{itemize}

Cada variable categórica posee una tabla de catálogo propia (\texttt{cat\_}), donde se definen los valores posibles y sus descripciones. Por ejemplo, \texttt{cat\_textura}, \texttt{cat\_nivel1}, \texttt{cat\_tratmasa} o \texttt{cat\_origen}. Todas siguen un patrón uniforme: la clave primaria es el identificador de la variable (\texttt{<variable>\_id}), y las tablas troncales referencian este mismo campo como clave foránea. Además la base de daros incluye una tabla llamada \texttt{meta\_variables} que recoge los metadatos.

\medskip

La Figura~\ref{fig:GWest_BBDD} muestra el esquema general de las tablas troncales y sus principales relaciones. Este diagrama resume la estructura interna de la base de datos y su jerarquía de dependencias.

\begin{figure}[H]
  \centering
  \includegraphics[width=0.9\textwidth]{figuras/Estrctr_BBDD_GWest.png}
  \caption{Esquema relacional de las tablas principales de la base de datos. Tabla extraida de \cite{greenwestdb}, donde se pueden consultar más detalles sobre las variables.}
  \label{fig:GWest_BBDD}
\end{figure}

\subsubsection*{Diccionario resumido de variables}
\small
\setlength{\LTcapwidth}{\textwidth}
\begin{longtable}{p{3.2cm} p{7.6cm} p{2.4cm} p{2.4cm}}
\caption{Resumen de variables principales por entidad. Tabla extraida de \cite{greenwestdb}.}\\
\toprule
\textbf{Variable} & \textbf{Descripción} & \textbf{Unidad} & \textbf{Tipo de dato} \\
\midrule
\endfirsthead
\toprule
\textbf{Variable} & \textbf{Descripción} & \textbf{Unidad} & \textbf{Tipo de dato} \\
\midrule
\endhead
\midrule
\multicolumn{4}{r}{\emph{Continúa en la siguiente página}} \\
\midrule
\endfoot
\bottomrule
\endlastfoot

\multicolumn{4}{l}{\textbf{parcelas}} \\
\texttt{parcela\_id} & Identificador único de parcela (IFN). & -- & Identificador \\
\texttt{latitud}, \texttt{longitud} & Coordenadas geográficas (WGS84). & ° & Geográfico \\
\texttt{coorx}, \texttt{coory} & Coordenadas UTM; \texttt{huso} especifica zona. & m (UTM) & Geográfico \\
\texttt{elevacion} & Cota sobre el nivel del mar (NASADEM). & m & Numérico \\
\texttt{pendiente} & Inclinación del terreno. & ° & Numérico \\
\texttt{orientacion} & Orientación del terreno (0–360). & ° & Numérico \\
\texttt{presencia\_id} & Presencia en IFN $\rightarrow$ \texttt{cat\_presencia}. & -- & Categórico \\
\texttt{tipsuelo1\_id}, \texttt{tipsuelo2\_id}, \texttt{tipsuelo3\_id} & Tipos de suelo $\rightarrow$ \texttt{cat\_tipsuelo*}. & -- & Categórico \\
\texttt{rocosidad\_id} & Rocosidad $\rightarrow$ \texttt{cat\_rocosidad}. & -- & Categórico \\
\texttt{radio}, \texttt{superficie} & Radio de parcela y superficie derivada. & m; ha & Numérico \\
\addlinespace

\multicolumn{4}{l}{\textbf{parcela\_inventario}} \\
\texttt{parcela\_id}, \texttt{inventario\_id} & Clave compuesta (parcela-inventario). & -- & Identificador \\
\texttt{ano} & Año de apeo. & año & Numérico \\
\texttt{nivel1\_id}, \texttt{nivel2\_id} & Morfoestructura. $\rightarrow$ \texttt{cat\_nivel*}. & -- & Categórico \\
\texttt{textura\_id} & Textura de suelo $\rightarrow$ \texttt{cat\_textura}. & -- & Categórico \\
\texttt{merosiva\_id} & Manifestaciones erosivas $\rightarrow$ \texttt{cat\_merosiva}. & -- & Categórico \\
\texttt{matorg\_id} & Materia orgánica $\rightarrow$ \texttt{cat\_matorg}. & -- & Categórico \\
\texttt{modcomb\_id} & Modelo de combustible $\rightarrow$ \texttt{cat\_modcomb}. & -- & Categórico \\
\texttt{disesp\_id} & Distribución espacial $\rightarrow$ \texttt{cat\_disesp}. & -- & Categórico \\
\texttt{comesp\_id} & Composición específica $\rightarrow$ \texttt{cat\_comesp}. & -- & Categórico \\
\texttt{fccarb}, \texttt{fcctot} & Fracción de cabida cubierta (árboles). & \% & Numérico \\
\addlinespace

\multicolumn{4}{l}{\textbf{parcela\_inventario\_especie}} \\
\texttt{parcela\_id}, \texttt{inventario\_id}, \texttt{especie\_id} & Clave compuesta (parcela-inventario-especie). & -- & Identificador \\
\texttt{ocupa} & Grado de ocupación de la especie. & (0--10) & Numérico \\
\texttt{estado\_id} & Estado de desarrollo. $\rightarrow$ \texttt{cat\_estado}. & -- & Categórico \\
\texttt{fpmasa\_id} & Forma principal de masa $\rightarrow$ \texttt{cat\_fpmasa}. & -- & Categórico \\
\texttt{tratmasa\_id} & Tratamientos selvícolas $\rightarrow$ \texttt{cat\_tratmasa}. & -- & Categórico \\
\texttt{orgmasa1\_id} & Origen de masa (IFN3/4)$\rightarrow$ \texttt{cat\_orgmasa1}. & -- & Categórico \\
\texttt{masa\_id} & Clasificación de masa $\rightarrow$ \texttt{cat\_masa}. & -- & Categórico \\
\texttt{origen\_id} & Origen de la masa (IFN2) $\rightarrow$ \texttt{cat\_origen}. & -- & Categórico \\
\addlinespace

\multicolumn{4}{l}{\textbf{parcela\_inventario\_especie\_cd}} \\
\texttt{parcela\_id}, \texttt{inventario\_id}, \texttt{especie\_id} & Clave compuesta ( parcela-inventario-especie-cd). & -- & Identificador \\
\texttt{cd\_id} & Clase diamétrica (CD) reglamento IFN. & cm & Numérico discreto \\
\texttt{npies} & Número de pies. & pies/ha & Numérico \\
\texttt{abas} & Área basimétrica. & m$^{2}$/ha & Numérico \\
\texttt{vcc}, \texttt{vsc}, \texttt{vle} & Volúmenes (con/sin corteza; leñas). & m$^{3}$/ha & Numérico \\
\texttt{iavc} & Incremento anual del volumen con corteza. & m$^{3}$/ha$\cdot$año & Numérico \\
\texttt{ca}, \texttt{cr} & Carbono aéreo y radical. & t/ha & Numérico \\
\texttt{ht} & Altura media (modelo CatBoost). & m & Numérico \\
\texttt{carbono\_bruto} & Carbono total estimado (alometrías). & t & Numérico \\
\addlinespace

\multicolumn{4}{l}{\textbf{parcela\_especie\_arbol}} \\
\texttt{parcela\_id}, \texttt{especie\_id} & Clave compuesta (parcela–especie–árbol). & -- & Identificador \\ 
\texttt{arbol\_id} & Identificador del árbol dentro de parcela y especie. & -- & Entero \\ 
\texttt{rumbo} & Rumbo desde el centro de la parcela al árbol. & grados centesimales & Numérico \\ 
\texttt{distancia} & Distancia desde el centro de la parcela al árbol. & m & Numérico \\ 
\texttt{cd} & Clase diamétrica (reglamento IFN). & cm & Numérico discreto \\ 
\texttt{ht} & Altura total del árbol inventariado. & m & Numérico \\ 
\texttt{dn1}, \texttt{dn2} & Diámetros normales perpendiculares. & mm & Numérico \\ 
\texttt{abas} & Área basimétrica del pie medido. & m$^{2}$ & Numérico \\ 
\texttt{iavc} & Incremento anual del volumen con corteza. & dm$^{3}$/año & Numérico \\ 
\texttt{vcc}, \texttt{vsc}, \texttt{vle} & Volúmenes (con corteza, sin corteza, leñas). & dm$^{3}$ & Numérico \\ 
\texttt{ca}, \texttt{cr} & Carbono aéreo y radical del árbol. & t & Numérico \\
\addlinespace

\multicolumn{4}{l}{\textbf{parcela\_inventario\_estacion}} \\
\texttt{parcela\_id}, \texttt{inventario\_id}, \texttt{estacion\_id} & Clave compuesta (agregado estacional). & -- & Identificador \\
\texttt{PR\_*} & Estadísticos de precipitación (mean, max, min, std, sum). & mm/(m$^2\cdot$día), mm/m$^2$ & Numérico \\
\texttt{T2M\_*}, \texttt{SKT\_*} & Aire 2\,m y temperatura superficial (mean, max, min, std). & °C & Numérico \\
\texttt{STL1\_*}--\texttt{STL4\_*} & Temperatura del suelo por niveles (mean, max, min, std). & °C & Numérico \\
\texttt{NDVI\_*}, \texttt{EVI\_*}, \texttt{NDII\_*}, \texttt{GNDVI\_*} & Índices de vegetación (max, mean, median, min, std). & adimensional & Numérico \\
\addlinespace

\multicolumn{4}{l}{\textbf{especies} y \textbf{grupos}} \\
\texttt{especie\_id} & Identificador de especie IFN. & -- & Identificador \\
\texttt{nombre}, \texttt{sinonimia} & Denominación IFN y sinónimos. & -- & Texto \\
\texttt{tipo\_especie} & 0\,= conífera; 1\,= frondosa. & -- & Categórico \\
\texttt{grupo\_id} & Grupo funcional $\rightarrow$ \texttt{grupos}. & -- & Identificador \\
\texttt{grupos.nombregrupo} & Nombre del grupo. & -- & Texto \\
\end{longtable}
\normalsize

\subsubsection*{Cardinalidad y completitud}

El volumen de entradas por tabla es:
\begin{center}
\begin{tabular}{l r}
\toprule
\textbf{Tabla} & \textbf{Número de registros} \\
\midrule
\texttt{parcelas} & 52{,}298 \\
\texttt{parcela\_inventario} & 147{,}995 \\
\texttt{parcela\_inventario\_especie} & 417{,}119 \\
\texttt{parcela\_inventario\_especie\_cd} & 1{,}191{,}070 \\
\texttt{parcela\_especie\_arbol} & 855{,}860 \\
\texttt{parcela\_inventario\_estacion} & 470{,}056 \\
\texttt{especies} & 195 \\
\texttt{grupos} & 33 \\
\bottomrule
\end{tabular}
\end{center}


\subsection{Preparación y tratamiento de los datos}
% Filtros (fccarb>=20%, crecimiento positivo, etc.)
% Agregaciones (parcela-especie, compresión CD -> npies_{cd}, etc.)
% Cálculo de variables derivadas (carbono_bruto)
% Codificación y escalado


Para caracterizar el crecimiento de los árboles en los datos que se alimentan al modelo es necesario seleccionar dos inventarios distintos, permitiendo calcular la evolución de las variables forestales en el periodo transcurrido entre ambos. Dado que los inventarios IFN3 e IFN4 tienen una estructura similar y caracterizan mejor el terreno (más variables) la primera elección natural es emplear estos inventarios. 

\subsubsection*{Filtrado de registros}

Se descartan todas aquellas parcelas en las que el valor de carbono total (\texttt{C} o \texttt{carbono\_bruto}) en la segunda inventariación es inferior a la primera. Estos casos suelen deberse a episodios de deforestación, incendios u otras perturbaciones, y no representan un crecimiento forestal neto.

\medskip

El conjunto de datos empleado se restringe únicamente a las parcelas que presentan una \texttt{fccarb} (fracción de cabida cubierta arbórea) igual o superior al 20\,\%. Este umbral define la proporción mínima de superficie ocupada por copas de árboles respecto al área total de la parcela, y constituye una de las condiciones esenciales para considerar una superficie como terreno forestal. La exclusión de parcelas con \texttt{fccarb} inferior al 20\,\% permite asegurar que las estimaciones de carbono se realicen sobre masas forestales consolidadas, evitando sesgos asociados a áreas agrícolas o matorrales.

\medskip

Tras aplicar los criterios de inclusión definidos —esto es, \texttt{fccarb} $>$ 20 y la no reducción del carbono total entre inventarios (\texttt{c3} $\geq$ \texttt{c4} o \texttt{carbono\_bruto3} $\geq$ \texttt{carbono\_bruto4})—, se obtuvo la siguiente distribución de registros:

\begin{itemize}
  \item \textbf{Caso C (c3 $\geq$ c4):} de un total de 171\,170 parcelas, 104\,558 (61.1\,\%) cumplen únicamente la condición de \texttt{fccarb} $>$ 20, mientras que 3\,660 (2.1\,\%) cumplen solo la condición de no reducción del carbono total. El número de parcelas que satisfacen ambas condiciones simultáneamente —y, por tanto, se consideran válidas para el entrenamiento de modelos— asciende a 53\,888 (31.5\,\% del total). En conjunto, el 94.7\,\% de las parcelas cumplen al menos una de las dos condiciones de inclusión.

  \item \textbf{Caso carbono\_bruto (carbono\_bruto3 $\geq$ carbono\_bruto4):} igualmente, sobre las mismas 171\,170 parcelas iniciales, 123\,774 (72.3\,\%) presentan una \texttt{fccarb} $>$ 20, y 1\,738 (1.0\,\%) muestran un incremento neto de carbono bruto sin cumplir la primera condición. Las parcelas que cumplen ambas simultáneamente suman 34\,672 (20.3\,\%), lo que representa el conjunto final de datos empleados para entrenamiento en este caso. El 93.6\,\% de las parcelas cumplen al menos una de las condiciones de inclusión.
\end{itemize}

\medskip

En la Tabla~\ref{tab:filtrado_fccarb} se resume la relación entre ambas condiciones de filtrado en cada caso, evidenciando que la fracción de cabida cubierta es el criterio más restrictivo. 

\begin{table}[h!]
\centering
\caption{Resumen del filtrado de parcelas según las condiciones de inclusión.}
\label{tab:filtrado_fccarb}
\begin{tabular}{lcccc}
\toprule
\textbf{Caso} & \textbf{Total} & \textbf{F1 (\texttt{fccarb}>20)} & \textbf{F2} & \textbf{F1 $\wedge$ F2 (entrenamiento)} \\
\midrule
C (c3$\geq$c4) & 171\,170 & 104\,558 & 3\,660 & 53\,888 \\
Carbono bruto (cb3$\geq$cb4) & 171\,170 & 123\,774 & 1\,738 & 34\,672 \\
\bottomrule
\end{tabular}
\end{table}

\subsubsection*{Cálculo y agregación de variables}

Cada registro de entrada se genera a nivel de combinación parcela--especie, incorporando las variables correspondientes de la primera medición (IFN3 en la configuración empleada) y la variable objetivo (carbono) de la segunda medición (IFN4). Las variables de \texttt{parcela} y \texttt{parcela\_inventario} se desdoblan para cada especie. Las entradas de la tabla \texttt{parcela\_inventario\_especie\_cd} se agrupan por parcela y especie y se comprimen en una única entrada creando un conjunto de variables para cada clase diamétrica.

\medskip

La Tabla~\ref{tab:entrada_modelo} resume las variables empleadas como entrada al modelo, integradas desde las distintas tablas que conforman la base de datos relacional.

\begin{table}[H]
\renewcommand{\arraystretch}{1.4}
\setlength{\tabcolsep}{3pt}
\centering
\small
\resizebox{\textwidth}{!}{%
\begin{tabular}{|p{3.6cm}|p{2.2cm}|p{7.6cm}|p{2.2cm}|}
\hline
\multicolumn{4}{|c|}{\textbf{Resumen de Datos de Entrada del Modelo}} \\
\hline
\textbf{Variable} & \textbf{Tipo} & \textbf{Descripción} & \textbf{Anexo} \\
\hline
\texttt{parcela\_id} & varchar & Identificador \emph{único} de parcela. & -- \\
\hline
\texttt{especie\_id}, \texttt{tipo\_especie}, \texttt{grupo\_id} & int (CF) & Especie (código), tipo de especie y grupo taxonómico. & Anexo \ref{sec:especies} \\
\hline
\texttt{ocupa} & int & Grado de ocupación/presencia de la especie en la parcela (0--10). & -- \\
\hline
\texttt{estado\_id}, \texttt{fpmasa\_id}, \texttt{tratmasa\_id}, \texttt{orgmasa\_1\_id} & int (CF) & Estado/fase de desarrollo, forma principal de masa, tratamiento de masa, organización de masa. & -- \\
\hline
\texttt{tipsuelo1\_id}, \texttt{tipsuelo2\_id}, \texttt{tipsuelo3\_id} & int (CF) & Tipos de suelo de la parcela (niveles jerárquicos). & Anexo \ref{sec:TipSuelo} \\
\hline
\texttt{rocosidad\_id}, \texttt{textura\_id}, \texttt{matorg\_id}, \texttt{modcomb\_id}, \texttt{disesp\_id}, \texttt{comesp\_id}, \texttt{merosiva\_id} & int (CF) & Rocosidad, textura, materia orgánica, modo de combustibilidad, distribución/composición específica, manifestaciones erosivas. & Anexo \ref{sec:Rocosid},\ref{sec:ManERo}, Anexo \ref{sec:disEsp}, \ref{anexo:compesp}, \ref{sec:textura}, \ref{sec:MatOrg}  \\
\hline
\texttt{radio} & float & Radio de la parcela (m). & -- \\
\hline
\texttt{orientacion}, \texttt{elevacion}, \texttt{pendiente} & float & Orientación (grados), elevación (m s.n.m.), pendiente (\%). & -- \\
\hline
\texttt{nivel1\_id}, \texttt{nivel2\_id} & int (CF) & Niveles jerárquicos/estratos de inventario. & -- \\
\hline
\texttt{fccarb}, \texttt{fcctot} & float & Fracción de cabida cubierta arbórea y total. & -- \\
\hline
\texttt{npies\_\{1,2,5,10,15, 20,25,30,35,40,45, 50,55,60,65,70\}} & float & Número de pies por clase diamétrica CD (cm). Cada campo corresponde a la CD indicada. & -- \\
\hline
\texttt{periodo} & int & Años transcurridos entre inventarios considerados en el modelo. & -- \\
\hline
\texttt{evi\_\{stat\}\_\{est\}} & float & Índice EVI por estación; \texttt{stat} $\in$ \{max, mean, median, min, std\}, \texttt{est} $\in$ \{invierno, oto\~no, primavera, verano\}. & -- \\
\hline
\texttt{gndvi\_\{stat\}\_\{est\}} & float & Índice GNDVI por estación (misma convención de \texttt{stat} y \texttt{est}). & -- \\
\hline
\texttt{ndii\_\{stat\}\_\{est\}} & float & Índice NDII por estación (misma convención). & -- \\
\hline
\texttt{ndvi\_\{stat\}\_\{est\}} & float & Índice NDVI por estación (misma convención). & -- \\
\hline
\texttt{pr\_\{stat\}\_\{est\}} & float & Precipitación: \texttt{stat} $\in$ \{max, mean, min, std, sum\} por estación \texttt{est}. & -- \\
\hline
\texttt{skt\_\{stat\}\_\{est\}} & float & Temperatura de superficie (skin temperature): \texttt{stat} $\in$ \{max, mean, min, std\}. & -- \\
\hline
\texttt{stl1-4\_\{stat\}\_\{est\}} & float & Temperatura de suelo por capa (1--4): \texttt{stat} $\in$ \{max, mean, min, std\}. & -- \\
\hline
\texttt{t2m\_\{stat\}\_\{est\}} & float & Temperatura del aire a 2 m: \texttt{stat} $\in$ \{max, mean, min, std\}. & -- \\
\hline
\end{tabular}%
}
\caption{\tiny Resumen de variables de entrada del modelo. Para variables estacionales se usa la notación \texttt{variable\_\{stat\}\_\{est\}}, con estadísticas \texttt{stat} y estaciones \texttt{est} en \{invierno, oto\~no, primavera, verano\}. Las variables \texttt{npies\_\{CD\}} se repiten para cada clase diamétrica indicada.}
\label{tab:entrada_modelo}
\end{table}

\subsubsection*{Codificación y normalización}

Las variables categóricas se codifican mediante \textit{one-hot encoding}, generando variables binarias para cada clase. Las variables numéricas se escalan (normalización estándar o min-max, según el modelo) para asegurar que todas las magnitudes tengan el mismo orden de importancia durante el entrenamiento.


\subsection{Limitación temporal y mitigación}
% Explica la restricción 18 años IFN3->IFN4
% Inserta figura de periodos (barras) y conclusiones modales 15--17

De todos los criterios, la \textbf{permanencia de 30 años} es el único que impacta directamente en el entrenamiento: el intervalo IFN3--IFN4 es, en la mayoría de los casos, claramente inferior a dos décadas. La Figura~\ref{fig:periodo_ifn} muestra la distribución de los años transcurridos entre inventarios. 

El conteo de valores de la variable \texttt{periodo} confirma esta tendencia: los años más frecuentes son 15 - 17. En contraste, los valores superiores a 20 años son menos comunes, tal como se aprecia en el gráfico.  

Para reducir el riesgo de sesgo temporal, se emplea validación cruzada por \emph{grupos de periodo} y se evalúa la robustez del modelo ante cambios en el horizonte temporal.

\begin{figure}[H]
\centering
\includegraphics[width=0.78\textwidth]{figuras/periodo_barras.png}
\caption{Distribución del número de años transcurridos entre la inventariación tercera y cuarta de cada parcela (considerando únicamente las parcelas inventariadas en ambas ocasiones). Se aprecia un intervalo modal entre 15 y 17 años, con valores superiores a 20 muy poco frecuentes.}
\label{fig:periodo_ifn}
\end{figure}

\subsection{Partición y validación}

Para obtener una estimación imparcial del rendimiento y evitar \emph{fugas de información} debidas a la correlación espacial dentro de cada parcela, la partición del conjunto de datos se realiza \textbf{por identificador de parcela} (\texttt{parcela\_id}). Todas las observaciones asociadas a una misma parcela se asignan \emph{íntegramente} a un único subconjunto, de modo que ninguna parcela aparece simultáneamente en entrenamiento y evaluación. Tras aplicar los filtros de preprocesamiento, se dispone de \textbf{XXX} registros correspondientes a \textbf{XXX} parcelas, que se dividen en un \textbf{80\,\%} para entrenamiento y un \textbf{20\,\%} para evaluación.

\vspace{0.25em}
\noindent\textbf{Validación interna y control de sesgo temporal.} Sobre el subconjunto de entrenamiento (80\,\%) se aplica \emph{validación cruzada por grupos} utilizando como agrupador los \emph{años transcurridos entre inventarios} (p.\,ej., 15, 16, 17, \dots). Esta estrategia comprueba la \emph{estabilidad} del modelo frente a cambios en el horizonte temporal y reduce el riesgo de sobreajuste específico de un periodo. La selección de hiperparámetros se realiza exclusivamente dentro de esta validación interna; el conjunto de evaluación (20\,\%) permanece \emph{sellado} para la prueba final.

\vspace{0.25em}
\noindent\textbf{Métricas de evaluación.} El rendimiento se informa con dos medidas complementarias:
\begin{itemize}
    \item \textbf{RMSE (Root Mean Squared Error):} raíz del error cuadrático medio entre valores observados y predichos; se expresa en las mismas unidades que la variable objetivo y penaliza con mayor peso los errores grandes. Valores más bajos indican mejor ajuste.
    \item \boldmath\textbf{$R^2$}\unboldmath{} (coeficiente de determinación): proporción de la varianza observada explicada por el modelo (idealmente en $[0,1]$). Valores cercanos a 1 denotan alta capacidad explicativa; puede ser negativo si el modelo es peor que la predicción constante.
\end{itemize}

\vspace{0.25em}
\noindent\textbf{Protocolo de reporte.} Para cada modelo se reportan: (i) el rendimiento medio y la dispersión en la validación cruzada por grupos (entrenamiento), y (ii) el desempeño final en el conjunto de evaluación independiente (20\,\%). Este protocolo garantiza comparabilidad entre modelos, control del sesgo espacial por parcela y verificación explícita de la robustez temporal.

\subsection{Modelos evaluados}
% Familias, tuning (RandomizedSearchCV), por qué ensembles, etc.


A continuación, se detalla el diseño general y las estrategias empleadas para la selección y optimización de modelos.

\subsubsection*{Entrenamiento y optimización}

Se aplicó \textit{RandomizedSearchCV}, una técnica de búsqueda aleatoria de hiperparámetros que evalúa distintas combinaciones utilizando validación cruzada. Este procedimiento permite optimizar el rendimiento de cada modelo sin incurrir en un coste computacional tan elevado como el de una búsqueda exhaustiva.

\medskip

\paragraph{Validación cruzada por grupos.} 


En algunas configuraciones probadas, se emplea la validación cruzada por grupos (\textit{Group k-Fold Cross Validation}). Este método divide el área de estudio en $k$ bloques, basados en una característica común de los datos, asegurando que los datos dentro de cada bloque estén relacionados. El modelo se entrena con $k-1$ bloques y se valida con el bloque restante, repitiendo este proceso $k$ veces.

\medskip

Este enfoque es útil cuando los datos tienen agrupaciones naturales, como por ejemplo, diferentes parcelas o periodos de tiempo. Al mantener los datos relacionados en un mismo bloque, se evita la filtración de información entre los conjuntos de entrenamiento y validación, lo que permite una mejor evaluación de la capacidad de generalización del modelo. Así, se asegura que el modelo no se sobreajuste y sea robusto al ser evaluado en contextos no vistos previamente.


\subsubsection*{Modelos ensemble}

Para mejorar la precisión y robustez, se emplearon diversos métodos de \textit{ensemble learning}, que combinan múltiples modelos base (\textit{base learners}) para generar una predicción agregada. Esta estrategia se inspira en la teoría de la sabiduría colectiva: la combinación de estimaciones independientes tiende a superar a cualquier estimador individual.

\paragraph{Técnicas utilizadas:}
\begin{itemize}
    \item \textbf{Voting y Averaging:} combinan modelos ya entrenados mediante votación mayoritaria o promedio.
    \item \textbf{Bagging y Boosting:} construyen modelos desde cero y los combinan. Bagging reduce la varianza al entrenar modelos en subconjuntos aleatorios; Boosting mejora el sesgo al entrenar secuencialmente, corrigiendo errores anteriores.
    \item \textbf{Stacking:} combina modelos optimizados usando un metamodelo que aprende a integrar sus predicciones.
\end{itemize}

\subsubsection*{Boosting y aprendizaje gradual}

El \textit{boosting} se basa en el aprendizaje secuencial, donde cada nuevo modelo intenta corregir los errores residuales del anterior. Esta técnica permite construir modelos fuertes a partir de modelos débiles, alcanzando gran precisión. Sin embargo, requiere una cuidadosa configuración de hiperparámetros para evitar el sobreajuste.

Entre las implementaciones destacadas se incluyen:

\begin{itemize}
    \item \textbf{XGBoost:} modelo GBM que optimiza rendimiento con gradientes de primer y segundo orden, regularización L1/L2, manejo automático de valores faltantes, y técnicas de generalización como \textit{shrinkage} y \textit{column subsampling}.
    \item \textbf{LightGBM:} algoritmo eficiente para grandes volúmenes de datos, con crecimiento \textit{leaf-wise} y soporte nativo para variables categóricas.
    \item \textbf{AdaBoost:} ajusta modelos simples secuencialmente, enfocando el aprendizaje en observaciones mal clasificadas.
    \item \textbf{CatBoost:} especializado en variables categóricas y robusto frente a datos ruidosos, usando codificación por orden aleatorio.
    \item \textbf{Gradient Boosting Decision Trees (GBDT):} construye árboles secuenciales ajustados a residuos, optimizando mediante descenso por gradiente.
\end{itemize}

\subsubsection*{Bagging}

El \textit{bagging} (Bootstrap Aggregating) entrena múltiples modelos independientes sobre subconjuntos de datos generados por muestreo con reemplazo. Las predicciones se combinan por promedio o votación. Esta técnica reduce la varianza y mejora la estabilidad de modelos inestables.

\begin{itemize}
    \item \textbf{Random Forest:} combina árboles de decisión (CART) con selección aleatoria de características en cada división. Es escalable, robusto a datos faltantes, y menos propenso al sobreajuste.
    \item \textbf{Bagged Decision Trees (BaggedDT):} genera árboles sin poda entrenados en muestras bootstrap. Promedia sus predicciones para reducir la varianza.
\end{itemize}

\subsubsection*{Otros modelos utilizados}

Además de los métodos ensemble, se evaluaron modelos representativos de distintos paradigmas de aprendizaje supervisado:

\begin{itemize}
    \item \textbf{K-Nearest Neighbors (KNN):} modelo basado en instancia que predice a partir de los vecinos más cercanos. Sensible a la escala y a \textit{outliers}.
    \item \textbf{Multi-Layer Perceptron (MLP):} red neuronal con una o más capas ocultas, capaz de modelar relaciones no lineales complejas.
    \item \textbf{Support Vector Regression (SVR):} modelo de márgenes para regresión, con soporte para kernels no lineales.
    \item \textbf{SVM con kernel:} modelo poderoso para clasificación y regresión no lineal, aunque costoso y sensible a hiperparámetros.
    \item \textbf{Bayesian Neural Network:} enfoque probabilístico que estima incertidumbre en las predicciones. Incluye variantes como la \textit{Bayesian Ridge Regression}.
    \item \textbf{Naive Bayes:} clasificador probabilístico rápido y simple, útil en texto y alta dimensionalidad. Se evaluaron variantes:
        \begin{itemize}
            \item \textit{Gaussian Naive Bayes:} para datos continuos.
            \item \textit{Multinomial Naive Bayes:} para conteos y texto.
            \item \textit{Bernoulli Naive Bayes:} para variables binarias.
        \end{itemize}
\end{itemize}

\subsubsection*{Comparación y justificación de modelos}

La evaluación de múltiples modelos responde a la necesidad de identificar no solo el de mejor rendimiento, sino también el más adecuado según la naturaleza del problema y los datos disponibles. Se compararon algoritmos lineales, no lineales, basados en vecinos, redes neuronales, modelos probabilísticos y diferentes técnicas de \textit{ensemble}.  Vemos un resumen de los modelos aplicados en la tabla \ref{tab:modelos}.


\begin{table}[H]\small
\centering
\begin{tabular}{|p{3.2cm}|p{2.8cm}|p{5.2cm}|p{4.2cm}|}
\hline
\textbf{Modelo} & \textbf{Tipo / Técnica} & \textbf{Características destacadas} & \textbf{Observaciones} \\
\hline
\textbf{Random Forest} & Bagging (Árboles) & Uso de bootstrap, selección aleatoria de atributos, reducción de varianza & Robusto y escalable; menor interpretabilidad \\
\hline
\textbf{Bagged Decision Trees (BaggedDT)} & Bagging & Árboles sin poda, entrenados en paralelo sobre muestras con reemplazo & Preciso pero costoso computacionalmente \\
\hline
\textbf{XGBoost} & Boosting (GBM) & Regularización L1/L2, manejo de valores faltantes, poda anticipada & Alto rendimiento, sensible a hiperparámetros \\
\hline
\textbf{LightGBM} & Boosting (Leaf-wise) & Crecimiento hoja a hoja, eficiente en grandes volúmenes & Rápido y preciso; riesgo de sobreajuste \\
\hline
\textbf{AdaBoost} & Boosting (Stumps) & Aumenta peso de errores, pondera modelos por precisión & Sencillo y efectivo con datos limpios \\
\hline
\textbf{CatBoost} & Boosting especializado & Codificación avanzada de variables categóricas, robustez a ruido & Ideal para datos heterogéneos \\
\hline
\textbf{Gradient Boosting Decision Trees (GBDT)} & Boosting & Árboles secuenciales ajustados a residuos & Buen rendimiento; mayor coste de entrenamiento \\
\hline
\textbf{K-Nearest Neighbors (KNN)} & Basado en instancia & No requiere entrenamiento, predice por proximidad & Sensible a escala y outliers \\
\hline
\textbf{Multi-Layer Perceptron (MLP)} & Red neuronal & Modela relaciones no lineales complejas & Requiere normalización y regularización \\
\hline
\textbf{Support Vector Regression (SVR)} & Kernel y márgenes & Predicción dentro de tolerancia $\varepsilon$, uso de kernels no lineales & Robusto; elevado coste computacional \\
\hline
\textbf{SVM con kernel} & SVM no lineal & Maximiza margen, admite distintos kernels (RBF, polinomial, etc.) & Alta precisión; sensible a hiperparámetros \\
\hline
\textbf{Bayesian Neural Network / Ridge Regression} & Probabilístico / Bayesiano & Predicción con incertidumbre, estimación automática de hiperparámetros & Útil para inferencia y regularización \\
\hline
\textbf{Naive Bayes (Gaussian, Multinomial, Bernoulli)} & Probabilístico & Asume independencia condicional, rápido y simple & Eficaz en texto y alta dimensionalidad \\
\hline
\end{tabular}
\caption{Resumen de modelos de aprendizaje supervisado aplicados}
\label{tab:modelos}
\end{table}


\subsection{Supuestos de elegibilidad y verificación externa}
% Intervención humana, permanencia 30a (extrapolación cauta), superficie>=1ha,
% fccarb>=20% (filtro), altura>=3m en madurez (decisión de diseño).

Para que un proyecto forestal sea elegible en programas de \emph{créditos de carbono}, debe cumplir requisitos técnicos establecidos por marcos regulatorios internacionales \cite{IPCC2006, miteco_guia_co2}. A continuación se resume cada criterio y la forma en que se aborda en este estudio:

\begin{itemize}
    \item \textbf{Intervención humana directa.} El incremento de carbono debe proceder de actuaciones planificadas (reforestación, restauración o manejo sostenible). En nuestro caso, el modelo se entrena sobre datos observacionales (IFN3--IFN4); por tanto, la \emph{verificación de intervención} no se deduce del modelo, sino que se contempla como \emph{condición externa} de elegibilidad del proyecto a evaluar.

    \item \textbf{Permanencia mínima de 30 años.} Este es el criterio más restrictivo en relación con los datos disponibles, dado que el intervalo entre inventarios IFN3 e IFN4 no supera los 18 años. En consecuencia, el modelo se entrena y valida en horizontes de 15--18 años, evaluando su estabilidad temporal mediante validación cruzada estratificada por periodo. 

        Aunque los datos no cubren los 30 años requeridos por los estándares de proyectos de carbono, los resultados obtenidos permiten plantear una \emph{extrapolación cautelosa} siempre que se cumplan ciertas condiciones. En particular, un modelo que mantenga un desempeño estable y coherente dentro del rango observado puede extenderse a horizontes mayores si:

        \begin{enumerate}
            \item se \textbf{reporta la incertidumbre} asociada a las predicciones, mediante intervalos de confianza o bandas de predicción;
            \item se realiza un \textbf{análisis de sensibilidad temporal}, explorando escenarios conservadores y pesimistas;
            \item y se verifica externamente la \textbf{validez de las condiciones del proyecto} —superficie, altura mínima e intervención humana— conforme a los marcos regulatorios aplicables.
        \end{enumerate}

        Este enfoque combina una base empírica sólida (validación cruzada en horizontes observados) con una verificación externa de elegibilidad, permitiendo aplicar el modelo a contextos de mayor duración sin comprometer la trazabilidad ni la consistencia metodológica.
    
    \item \textbf{Superficie mínima de 1 ha.} Este criterio se considera \emph{externo} al alcance del modelo predictivo, ya que el aprendizaje se realiza a nivel de parcela e inventario y no sobre polígonos de superficie total. En la práctica, la verificación de la superficie se realiza \emph{ex ante}, sobre la geometría declarada del proyecto forestal. En los terrenos forestales generados a partir de intervención humana directa —como plantaciones o repoblaciones—, la extensión suele presentar una estructura homogénea, con una especie dominante, edades coetáneas y densidades estandarizadas. Bajo estas condiciones, el carbono total es proporcional a la superficie: duplicar el área de una masa forestal homogénea implica aproximadamente duplicar su carbono almacenado. Por tanto, la variable de superficie no afecta al ajuste interno del modelo y su cumplimiento puede evaluarse fácilmente a nivel de proyecto, sin comprometer la validez de las predicciones.

    \item \textbf{Fracción mínima de cabida cubierta del 20\%.} La base de datos dispone de \texttt{fccarb} (arbórea) y \texttt{fcctot} (total). Este umbral se aplica como \emph{filtro de elegibilidad} previo o posterior al modelado, sin modificar la arquitectura del modelo.

    \item \textbf{Altura mínima de 3 m en la madurez.} Este requisito se refiere a la altura que alcanzan los árboles en su fase de pleno desarrollo, y no a la altura inicial de los plantones. Por tanto, las mediciones realizadas durante las etapas tempranas de crecimiento no determinan la elegibilidad del proyecto, siempre que las especies seleccionadas sean capaces de superar los 3 metros en la madurez. En nuestro conjunto de datos, la altura no se registra explícitamente, por lo que este criterio se evalúa de forma \emph{externa} al modelo, mediante la selección de especies forestales adecuadas y la verificación con fuentes auxiliares (catálogos silvícolas o tipologías de masa). En la práctica, el cumplimiento del requisito depende de una decisión de diseño del proyecto —\emph{no plantar especies cuyo tamaño adulto sea inferior a 3 metros}— más que del ajuste predictivo del modelo. Por ello, la altura no interviene directamente en el entrenamiento, aunque sí condiciona la elegibilidad final del proyecto forestal. 
\end{itemize}

\medskip


