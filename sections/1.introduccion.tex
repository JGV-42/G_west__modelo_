\section{Introducción}

El cambio climático es uno de los mayores desafíos globales de la actualidad, y su impacto negativo se refleja principalmente en el aumento de las concentraciones de dióxido de carbono (\(CO_2\)) en la atmósfera. Este aumento contribuye a fenómenos críticos como el deshielo de los polos, el incremento de fenómenos climáticos extremos y el deterioro de los ecosistemas naturales \cite{IPCC2007}. Los sumideros de carbono naturales, como los bosques, juegan un papel crucial en mitigar estos efectos, ya que mediante la fotosíntesis, los árboles capturan \(CO_2\) y lo almacenan en su biomasa, contribuyendo significativamente a la reducción de las concentraciones de este gas en la atmósfera.

\medskip 

A lo largo de las últimas décadas, instrumentos internacionales como el \textit{Protocolo de Kioto} \cite{UNFCCC1997} y la \textit{Convención Marco de las Naciones Unidas sobre el Cambio Climático (CMNUCC)} \cite{UNFCCC2015} han establecido mecanismos para reducir las emisiones de gases de efecto invernadero. En este marco, las actividades de forestación y reforestación han sido identificadas como fundamentales para la captura de \(CO_2\). Los \textit{créditos de carbono}, que representan la cantidad de \(CO_2\) en toneladas evitada o secuestrada por actividades como la forestación y reforestación de bosques, se han convertido en una herramienta clave para cumplir con los compromisos internacionales de reducción de emisiones.

\medskip 

Sin embargo, para que un proyecto de forestación o reforestación sea considerado válido para la obtención de créditos de carbono, existen diversas limitaciones legales que deben cumplirse. Según los requisitos establecidos por el \textit{Protocolo de Kioto} y las normativas nacionales, las siguientes condiciones deben ser satisfechas:

\begin{itemize}
    \item \textbf{Intervención humana directa:} Los árboles deben provenir de actividades de intervención humana, como la plantación, siembra o fomento de semilleros naturales. Esto significa que los cultivos forestales naturales no son elegibles para la contabilización de carbono.
    \item \textbf{Período mínimo de 30 años:} Para que un proyecto sea válido, debe garantizarse que los árboles permanezcan en el terreno durante un período mínimo de tiempo, generalmente 30 años, lo que excluye la absorción de carbono de cultivos estacionales, cuyo carbono es liberado nuevamente al ser cosechados.
    \item \textbf{Superficie mínima de 1 hectárea:} El proyecto debe abarcar al menos 1 hectárea de terreno para ser considerado.
    \item \textbf{Fracción mínima de cabida cubierta del 20\%:} Para que un área sea considerada como bosque, debe cubrir al menos el 20\% del área con especies arbóreas.
    \item \textbf{Altura mínima de los árboles maduros de 3 metros:} Los árboles deben alcanzar una altura mínima de 3 metros en su madurez, aunque no es necesario que alcancen esta altura al inicio de la plantación.
\end{itemize}

\medskip 

Satisfacer estas limitaciones legales es imprescindible para la correcta generación de créditos de carbono, y han sido tomadas en cuenta a lo largo del desarrollo del modelo predictivo del proyecto \textit{GreenWest}. 

\medskip 

El proyecto \textit{GreenWest} tiene como objetivo principal predecir la capacidad de absorción de \(CO_2\) en los cultivos forestales españoles, mediante el uso de modelos de inteligencia artificial. Este enfoque innovador tiene el potencial de transformar la gestión de proyectos de forestación y reforestación, optimizando las prácticas de plantación y maximizando la cantidad de carbono que se puede capturar en estos ecosistemas. 

\medskip 

Para lograrlo, el proyecto desarrollará un modelo predictivo que analizará datos sobre las características del terreno, las especies de árboles y las condiciones climáticas para estimar con precisión la cantidad de carbono que podría ser absorbido por un cultivo forestal en un periodo determinado. Este modelo no solo mejorará la comprensión del comportamiento de los sumideros de carbono, sino que también proporcionará herramientas útiles para la toma de decisiones estratégicas tanto en el ámbito empresarial como en el ambiental.

\medskip 

De esta forma, el proyecto \textit{GreenWest} contribuye a la transición hacia una economía baja en carbono, alineándose con los objetivos globales de sostenibilidad establecidos en el marco de la CMNUCC y el \textit{Protocolo de Kioto}, y promoviendo la creación de un mercado de créditos de carbono más eficiente y accesible para los actores económicos involucrados en la gestión de los recursos naturales.


