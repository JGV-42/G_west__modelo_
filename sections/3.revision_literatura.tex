% sections/03_revision_literatura.tex
\section{Revisión de la Literatura}

El secuestro de carbono en ecosistemas forestales ha cobrado una importancia creciente en la literatura científica, impulsada tanto por los compromisos internacionales en materia de cambio climático \cite{UNFCCC1997, UNFCCC2015} como por el auge de los mercados de créditos de carbono. Esto ha motivado el desarrollo de modelos orientados a cuantificar la biomasa forestal y estimar el contenido de carbono, aprovechando avances recientes en sensores remotos y técnicas de inteligencia artificial (IA).

Una de las estrategias más consolidadas es la estimación del carbono almacenado en un momento dado a partir de datos de teledetección. Goetz et al. (2009) \cite{goetz2009remote} revisan el uso de imágenes satelitales (MODIS, Landsat) en modelos empíricos de biomasa aérea, destacando su eficacia a escala regional en zonas boreales. Este tipo de estimaciones suele realizarse mediante regresiones lineales o algoritmos de mínimos cuadrados generalizados, con coeficientes de determinación (\(R^2\)) típicamente entre 0.6 y 0.8 según la resolución de entrada y la heterogeneidad del ecosistema.

La aplicación de aprendizaje profundo ha permitido mejorar sustancialmente la precisión y resolución espacial de estas estimaciones. Por ejemplo, Zhang et al. (2022) \cite{zhang2022carbon} integran imágenes Sentinel-2 con redes neuronales convolucionales, alcanzando un \(R^2\) de 0.84 para estimar el carbono en bosques subtropicales. Del mismo modo, Yang et al. (2023) \cite{yang2023forestcarbonai} desarrollan el modelo \textit{ForestCarbonAI}, entrenado con datos multiespectrales y LIDAR, con el que generan mapas de carbono forestal de alta resolución (10 m), reportando errores medios absolutos (MAE) inferiores a 3.5 tC/ha en zonas templadas. Otros trabajos recientes, como Reiersen et al. (2022) \cite{reiersen2022reforestree} o Dong et al. (2023) \cite{dong2023forest}, también demuestran la eficacia del deep learning para estimaciones estáticas, aunque se centran en contextos tropicales y no consideran el componente temporal.

Frente a estos enfoques descriptivos, algunas iniciativas han intentado proyectar la evolución del carbono a futuro. En el ámbito nacional, el Ministerio para la Transición Ecológica (MITECO) ha implementado herramientas como la calculadora ex ante de absorciones \cite{miteco_abexante_2025}, que permite obtener estimaciones simplificadas del carbono que puede fijarse en una plantación forestal en función de la especie y la zona agroclimática. No obstante, este instrumento se basa en coeficientes tabulados y no incorpora variables edafoclimáticas reales ni técnicas de modelización basadas en datos, lo que limita su precisión y capacidad de adaptación a contextos específicos.

En este escenario, el presente trabajo propone una metodología innovadora centrada en la predicción dinámica de carbono a largo plazo. A diferencia de los modelos anteriores, que estiman el carbono ya almacenado, este estudio se enfoca en anticipar cuánto carbono capturará un cultivo forestal en un horizonte temporal de 20 a 30 años. Para ello, se estudian diversos modelos de aprendizaje supervisado entrenados con datos históricos del Inventario Forestal Nacional (IFN2, IFN3 e IFN4), variables climáticas de Copernicus, características edáficas y métricas espectrales derivadas de imágenes Landsat \cite{landsat5_data, copuernicus_temps, miteco_guia_co2}. Los detalles sobre la arquitectura del modelo, las variables utilizadas, los algoritmos implementados y las métricas de evaluación se desarrollan en la siguiente sección.





