\section{Conclusiones}
% Este es un resumen conciso de tus hallazgos clave.
% - Reafirma el objetivo principal y si se logró.
% - Destaca los resultados más significativos.
% - Reitera la contribución principal de tu trabajo.
% - Enfatiza la importancia y el impacto potencial de tu modelo.
% - Evita introducir nueva información aquí.

Cosas que comentar (informal, hay que ponerlo a limpio):
\begin{itemize}
    \item Las métricas son bastante buenas, sobre todo para los modelos que se han entrenado solo con el IFN3 y sobre todo para la predicción del carbono total.
    \item Igualmente, los datos son muy poco uniformes. Por ejemplo, hay muchos datos de un año, muy pocos de otros. Esto es de esperar por la naturaleza de los datos: cubren una extensión muy grande con mucho detalle en cada parcela (sobre todo el IFN3).
    \item Esta es la principal limitación del estudio, porque queda claro que los modelos son capaces de entender el comportamiento de los sistemas.
    \item En el paper \cite{fasihi2024assessing} hacen una cosa similar a la que hacemos nosotros aquí. Además es en una región muy cercana, del norte de Italia. Aunque no hacen exactamente lo mismo que nosotros, ya que ellos predicen el carbono presente en el mismo momento de la medición mediante observaciones del bosque ``desde fuera'', es decir, sin utilizar como entrada datos medidos en el propio bosque (datos meteorológicos, del terreno y LiDAR), los resultados que obtienen tienen un error notablemente mayor que los nuestros. Sobre todo para predicción más temprana que hacemos (a 5 años), que es lo más parecido a ellos.
    \item Los modelos stacking sacan métricas muy ligeramente mejores que los mejores modelos base. Por otro lado, el esfuerzo necesario para obtenerlos y usarlos es notablemente mayor, así que puede que no esté justificado el uso de modelos tan complejos en un hipotético sistema en producción. No obstante, esta bién haber podido subir un poco las métricas. 
    \item Para su uso en producción como sistema de recomendación de proyectos de carbono habría que hacer algunas comprobaciones. Esto es, los datos están cogidos de los bosques. Hay que comprobar cómo se parece un bosque a una plantación de un proyecto de carbono (por tema de densidades principalmente). Por otro lado, el ``recomendador de especies'' (la idea que teníamos de tener un sitio para plantar y que el sistema recomedase una u otra en función de qué podría absorber más carbono) debería poder usarse igualmente bien.
    \item Como idea para un futuro trabajo, podría estar muy interesante usar medidas LiDAR. En otros estudios las usan y obtienen muy buenos resultados.
\end{itemize}


% Inicio de parte de Maider
% El objetivo principal de este trabajo es desarrollar un modelo de inteligencia artificial capaz de predecir de forma precisa la capacidad de captura de dióxido de carbono en cultivos forestales españoles, a partir de información estructural, edáfica, climática y espectral disponible en los Inventarios Forestales Nacionales y en fuentes de datos auxiliares. Los resultados obtenidos permiten afirmar que dicho objetivo se ha cumplido de manera satisfactoria.

% En primer lugar, se ha demostrado que es posible construir modelos predictivos robustos y generalizables para estimar el carbono forestal a medio y largo plazo. Entre las distintas configuraciones evaluadas, el mejor rendimiento global se alcanzó mediante un esquema de stacking entrenado con datos del IFN2 e IFN3 como variables explicativas y del IFN4 como variable objetivo, utilizando como modelos base CatBoost, LightGBM, XGBoost, Random Forest, GBDT y BaggedDT, y una red neuronal multicapa (MLP) como metamodelo. Esta configuración permite predecir el carbono total en toneladas para horizontes temporales comprendidos entre 5 y 30 años, alcanzando un coeficiente de determinación en test de aproximadamente $R^2=0.85$ y un error absoluto medio del orden de 6.4 toneladas de carbono. Estos valores indican una elevada capacidad explicativa y una precisión compatible con aplicaciones prácticas en planificación forestal y estimación de créditos de carbono.

% Así mismo, el análisis comparativo entre modelos individuales y esquemas de stacking ha puesto de manifiesto que los algoritmos basados en árboles de decisión son los que presentan un mejor comportamiento de forma consistente, destacando especialmente CatBoost y LightGBM. Aunque el stacking no produce incrementos sustanciales en el coeficiente de determinación, sí aporta mejoras sistemáticas en el MAE, lo que resulta especialmente relevante desde un punto de vista operativo, al reducir el error medio en las estimaciones finales.

% Un segundo resultado relevante del estudio es la identificación de los factores que condicionan en mayor medida la capacidad de captura de carbono. El proceso de selección manual de variables permitió reducir un conjunto inicial de 445 predictores a un subconjunto compacto de 44 variables, manteniendo una representación equilibrada de todos los ámbitos ecológicos implicados. Los resultados muestran que la mayor parte de la capacidad predictiva del modelo se explica por variables estructurales y de composición de la masa forestal, en particular el número de pies por clase diamétrica y las variables asociadas a la especie y su estado. Las variables edáficas, topográficas y de manejo aportan información adicional relevante, mientras que las variables climáticas y los índices de vegetación actúan como moduladores del crecimiento y la acumulación de carbono, refinando las predicciones especialmente a escala estacional.

% En conjunto, este trabajo contribuye con una metodología reproducible y basada en datos reales para la predicción de la captura de carbono forestal a futuro. La integración de información multifuente, el uso de técnicas avanzadas de aprendizaje automático y la validación rigurosa del rendimiento permiten disponer de una herramienta con potencial aplicación en la planificación de proyectos de forestación, la optimización del secuestro de carbono y la evaluación técnica de iniciativas vinculadas al mercado de créditos de carbono. Estos resultados refuerzan el valor de la inteligencia artificial como apoyo a la toma de decisiones ambientales y abren la puerta a futuras extensiones del modelo, como su adaptación a otros contextos geográficos o su integración en sistemas operativos de gestión forestal.

% El objetivo de este trabajo es la obtención de un modelo de Inteligencia Artificial capaz de predecir el carbono que una cierta parcela de terreno forestada o reforestada capturará en un cierto periodo de tiempo. Para ello se han recogido datos de tierra (Inventario Forestal Nacional \cite{ifn}), datos meteorológicos \cite{era5land} e imágenes satelitales \cite{landsat5_data} con los que se han entrenado varios modelos para intentar predecir el carbono capturado por las parcelas presentes en las iteraciones $2$ y $3$ del Inventario Forestal Nacional, comparando el resultado con la última de las iteraciones, la $4$. Las predicciones se hicieron para dos configuraciones distintas: usando como datos explicativos únicamente los del inventario $2$ y usando como datos explicativos los de los inventarios $3$ y $4$. A su vez, para cada caso se realizó la predicción de dos variables objetivo: la predicción del carbono en toneladas por hectárea (tC/ha) y la predicción del carbono en toneladas (tC). Los resultados para los mejores modelos en cada caso están recogidos en la Tabla \ref{poner_tabla}.


% Con estos resultados podemos afirmar que disponemos de datos suficientes y de suficiente calidad para entrenar modelos capaces de predecir el carbono capturado con un error aceptable. 