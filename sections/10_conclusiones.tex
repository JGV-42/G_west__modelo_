\section{Conclusiones}
% Este es un resumen conciso de tus hallazgos clave.
% - Reafirma el objetivo principal y si se logró.
% - Destaca los resultados más significativos.
% - Reitera la contribución principal de tu trabajo.
% - Enfatiza la importancia y el impacto potencial de tu modelo.
% - Evita introducir nueva información aquí.

El objetivo de este trabajo es la obtención de un modelo de Inteligencia Artificial capaz de predecir el carbono que una cierta parcela de terreno forestada o reforestada capturará en un cierto periodo de tiempo. Para ello se han recogido datos de tierra (Inventario Forestal Nacional \cite{ifn}), datos meteorológicos \cite{era5land} e imágenes satelitales \cite{landsat5_data} con los que se han entrenado varios modelos para intentar predecir el carbono capturado por las parcelas presenten en las iteraciones $2$ y $3$ del Inventario Forestal Nacional, comparando el resultado con la última de las iteraciones, la $4$. Las predicciones se hicieron para dos configuraciones distintas: usando como datos explicativos únicamente los del inventario $2$ y usando como datos explicativos los de los inventarios $3$ y $4$. A su vez, para cada caso se realizó la predicción de dos variables objetivo: la predicción del carbono en toneladas por hectárea (tC/ha) y la predicción del carbono en toneladas (tC). Los resultados para los mejores modelo en cada caso están recogidos en la Tabla \ref{poner_tabla}.


Con estos resultados podemos afirmar que disponemos de datos suficientes y de suficiente calidad para entrenar modelos capaces de predecir el carbono capturado con un error aceptable. 