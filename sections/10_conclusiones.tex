\section{Conclusiones}
% Este es un resumen conciso de tus hallazgos clave.
% - Reafirma el objetivo principal y si se logró.
% - Destaca los resultados más significativos.
% - Reitera la contribución principal de tu trabajo.
% - Enfatiza la importancia y el impacto potencial de tu modelo.
% - Evita introducir nueva información aquí.

El objetivo principal de este trabajo es desarrollar un modelo de inteligencia artificial capaz de predecir de forma precisa la capacidad de captura de dióxido de carbono en cultivos forestales españoles, a partir de información estructural, edáfica, climática y espectral disponible en los Inventarios Forestales Nacionales y en fuentes de datos auxiliares. Los resultados obtenidos permiten afirmar que dicho objetivo se ha cumplido de manera satisfactoria.

En primer lugar, se ha demostrado que es posible construir modelos predictivos robustos y generalizables para estimar el carbono forestal a medio y largo plazo. Entre las distintas configuraciones evaluadas, el mejor rendimiento global se alcanzó mediante un esquema de stacking entrenado con datos del IFN2 e IFN3 como variables explicativas y del IFN4 como variable objetivo, utilizando como modelos base CatBoost, LightGBM, XGBoost, Random Forest, GBDT y BaggedDT, y una red neuronal multicapa (MLP) como metamodelo. Esta configuración permite predecir el carbono total en toneladas para horizontes temporales comprendidos entre 5 y 30 años, alcanzando un coeficiente de determinación en test de aproximadamente $R^2=0.85$ y un error absoluto medio del orden de 6.4 toneladas de carbono. Estos valores indican una elevada capacidad explicativa y una precisión compatible con aplicaciones prácticas en planificación forestal y estimación de créditos de carbono.

Así mismo, el análisis comparativo entre modelos individuales y esquemas de stacking ha puesto de manifiesto que los algoritmos basados en árboles de decisión son los que presentan un mejor comportamiento de forma consistente, destacando especialmente CatBoost y LightGBM. Aunque el stacking no produce incrementos sustanciales en el coeficiente de determinación, sí aporta mejoras sistemáticas en el MAE, lo que resulta especialmente relevante desde un punto de vista operativo, al reducir el error medio en las estimaciones finales.

Un segundo resultado relevante del estudio es la identificación de los factores que condicionan en mayor medida la capacidad de captura de carbono. El proceso de selección manual de variables permitió reducir un conjunto inicial de 445 predictores a un subconjunto compacto de 44 variables, manteniendo una representación equilibrada de todos los ámbitos ecológicos implicados. Los resultados muestran que la mayor parte de la capacidad predictiva del modelo se explica por variables estructurales y de composición de la masa forestal, en particular el número de pies por clase diamétrica y las variables asociadas a la especie y su estado. Las variables edáficas, topográficas y de manejo aportan información adicional relevante, mientras que las variables climáticas y los índices de vegetación actúan como moduladores del crecimiento y la acumulación de carbono, refinando las predicciones especialmente a escala estacional.

En conjunto, este trabajo contribuye con una metodología reproducible y basada en datos reales para la predicción de la captura de carbono forestal a futuro. La integración de información multifuente, el uso de técnicas avanzadas de aprendizaje automático y la validación rigurosa del rendimiento permiten disponer de una herramienta con potencial aplicación en la planificación de proyectos de forestación, la optimización del secuestro de carbono y la evaluación técnica de iniciativas vinculadas al mercado de créditos de carbono. Estos resultados refuerzan el valor de la inteligencia artificial como apoyo a la toma de decisiones ambientales y abren la puerta a futuras extensiones del modelo, como su adaptación a otros contextos geográficos o su integración en sistemas operativos de gestión forestal.

