\section{Introducción}

El cambio climático es uno de los mayores desafíos globales y su manifestación más directa es el aumento de las concentraciones atmosféricas de dióxido de carbono (\(CO_2\)), con impactos sobre criosfera, extremos climáticos y ecosistemas \cite{IPCC2007}. Los bosques actúan como sumideros naturales al fijar \(CO_2\) en biomasa vía fotosíntesis, por lo que su gestión resulta clave para la mitigación. 


A lo largo de las últimas décadas, instrumentos internacionales como la \textit{Convención Marco de las Naciones Unidas sobre el Cambio Climático (CMNUCC)} y el \textit{Protocolo de Kioto} \cite{UNFCCC1997,UNFCCC2015} han establecido los marcos regulatorios para reducir las emisiones de gases de efecto invernadero mediante mecanismos basados en el mercado. En este contexto surgen los \emph{créditos de carbono}, unidades que representan la cantidad de dióxido de carbono (\(CO_2\)), habitualmente una tonelada, que ha sido capturada o cuya emisión ha sido evitada a través de proyectos certificados de mitigación. 


Entre las actividades elegibles, la forestación y reforestación destacan por su capacidad de actuar como sumideros naturales de carbono, fijando \(CO_2\) en la biomasa y el suelo. No obstante, para que estas actuaciones puedan generar créditos de carbono válidos, deben cumplir una serie de criterios técnicos y legales establecidos en la normativa internacional sobre cambio climático y en su aplicación a nivel nacional. En particular, estos requisitos derivan de las reglas de contabilidad de sumideros forestales adoptadas en el marco de la Convención Marco de las Naciones Unidas sobre el Cambio Climático (CMNUCC) y del Protocolo de Kioto, concretadas en los Acuerdos de Marrakech, así como de la definición nacional de bosque comunicada por España para estos fines \cite{UNFCCCMarrakech,UNFCCCSpainReview}. Dichos criterios incluyen:

\begin{itemize}
\item \textbf{Intervención humana directa:}
Los árboles deben ser el resultado de actividades de intervención humana directa, tales como la plantación, la siembra o el fomento deliberado de la regeneración natural. Este requisito se deriva de la definición de \emph{forestación} y \emph{reforestación} establecida en el Protocolo de Kioto, que excluye expresamente la regeneración natural no inducida por la acción humana \cite{UNFCCCMarrakech}.

\item \textbf{Período mínimo de permanencia:}
El proyecto debe garantizar la permanencia del sumidero de carbono durante un período prolongado (habitualmente del orden de 20-30 años), con el fin de asegurar que el carbono capturado no sea liberado de forma prematura a la atmósfera. Este criterio responde al principio de permanencia exigido en la contabilidad de sumideros forestales del régimen LULUCF y en los marcos de aplicación nacionales y europeos, lo que excluye cultivos de corta rotación cuyo carbono se libera tras la cosecha \cite{UNFCCCMarrakech,EULULUCF2018}.

\item \textbf{Superficie mínima de 1 hectárea:}
El área objeto del proyecto debe tener una extensión mínima de 1 hectárea. Este umbral procede de la definición nacional de bosque adoptada por España dentro de los rangos permitidos por los Acuerdos de Marrakech (0,05–1 ha), comunicada oficialmente a la CMNUCC \cite{UNFCCCSpainReview}.

\item \textbf{Fracción mínima de cabida cubierta del 20\%:}
Para que un terreno sea considerado bosque, la cobertura de copas de las especies arbóreas debe alcanzar al menos el 20\% de la superficie. Este valor corresponde a la elección nacional realizada por España para la definición de bosque a efectos de contabilidad climática \cite{UNFCCCSpainReview}.

\item \textbf{Altura mínima de los árboles maduros de 3 metros:}
Las especies arbóreas deben ser capaces de alcanzar una altura mínima de 3 metros en su madurez. No es necesario que dicha altura se alcance en el momento inicial del proyecto, pero sí que sea alcanzable bajo condiciones normales de crecimiento. Este criterio forma igualmente parte de la definición nacional de bosque comunicada por España conforme a las decisiones adoptadas bajo la CMNUCC \cite{UNFCCCSpainReview}.
\end{itemize}

Este trabajo presenta \textbf{GreenWest}, un modelo de inteligencia artificial para estimar la cantidad de carbono que capturará un cultivo forestal en España a partir de variables de vegetación, clima y terreno en un período de 20 a 30 años. Este enfoque innovador tiene el potencial de transformar la gestión de proyectos de forestación y reforestación, optimizando las prácticas de plantación y maximizando la cantidad de carbono que se puede capturar en estos ecosistemas. 

La pregunta operativa es: \emph{dadas las características iniciales de una plantación, ¿cuánto \(CO_2\) contendrá tras \(t\) años? con t número natural.} Para responderla, se integran datos del \textbf{Inventario Forestal Nacional} (IFN2–IFN4, MITECO) \cite{ifn}, reanálisis \textbf{ERA5-Land} \cite{era5land} e \textbf{índices espectrales Landsat} (Collection~2, L2) \cite{landsatc2} en una base de datos relacional jerárquica descrita en un trabajo complementario \cite{greenwestdb}.

Este modelo no solo mejorará la comprensión del comportamiento de los sumideros de carbono, sino que también proporcionará herramientas útiles para la toma de decisiones estratégicas tanto en el ámbito empresarial como en el ambiental. De esta forma, el proyecto \textit{GreenWest} contribuye a la transición hacia una economía baja en carbono, alineándose con los objetivos globales de sostenibilidad establecidos en el marco de la CMNUCC y el \textit{Protocolo de Kioto}, y promoviendo la creación de un mercado de créditos de carbono más eficiente y accesible para los actores económicos involucrados en la gestión de los recursos naturales.

