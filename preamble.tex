%%%%%%%%%%%%%%%%%%%%%%%%%%%%%%%%%%%%%%%%%
% Preámbulo para artículo científico
% Compatible con elsarticle y fácilmente adaptable
%%%%%%%%%%%%%%%%%%%%%%%%%%%%%%%%%%%%%%%%%

%----------------------------------------------------------------------------------------
% IDIOMA Y CODIFICACIÓN
%----------------------------------------------------------------------------------------
\usepackage[spanish,es-tabla]{babel}
\usepackage[utf8]{inputenc}
\usepackage[T1]{fontenc}

%----------------------------------------------------------------------------------------
% BIBLIOGRAFÍA (biblatex + biber)
%----------------------------------------------------------------------------------------
\usepackage[
    backend=biber,
    style=authoryear,    % Estilo autor-año (compatible con elsarticle authoryear)
    sorting=nyt,         % Ordenar por Nombre, Año, Título
    natbib=true,         % Permite usar \citep y \citet
    maxcitenames=2,
    maxbibnames=99
]{biblatex}
\addbibresource{referencias.bib}

%----------------------------------------------------------------------------------------
% PAQUETES ESENCIALES
%----------------------------------------------------------------------------------------
\usepackage{amsmath,amsfonts,amssymb,amsthm}  % Matemáticas
\usepackage{graphicx}                          % Imágenes
\graphicspath{{figuras/}{./}}
\usepackage{booktabs}                          % Tablas profesionales
\usepackage{array}
\usepackage{tabularx}
\usepackage{longtable}
\usepackage{multirow}
\usepackage{float}
\usepackage{subcaption}

%----------------------------------------------------------------------------------------
% COLORES Y ENLACES
%----------------------------------------------------------------------------------------
\usepackage[table,xcdraw,dvipsnames]{xcolor}
\usepackage{hyperref}
\hypersetup{
    colorlinks=true,
    linkcolor=blue!70!black,
    citecolor=blue!70!black,
    urlcolor=blue!70!black
}

%----------------------------------------------------------------------------------------
% CONFIGURACIÓN DE TABLAS Y FIGURAS
%----------------------------------------------------------------------------------------
\usepackage{caption}
\captionsetup{
    font=small,
    labelfont=bf,
    labelsep=period,
    justification=justified,
    skip=6pt
}

% Espaciado uniforme en tablas
\renewcommand{\arraystretch}{1.15}

%----------------------------------------------------------------------------------------
% LISTAS
%----------------------------------------------------------------------------------------
\usepackage{enumitem}
\setlist{noitemsep,topsep=3pt}

%----------------------------------------------------------------------------------------
% CÓDIGO Y TEXTO MONOESPACIADO
%----------------------------------------------------------------------------------------
\usepackage{listings}
\lstset{
    basicstyle=\ttfamily\small,
    breaklines=true,
    frame=single,
    backgroundcolor=\color{gray!10}
}

% Permitir que \texttt{} rompa líneas en guiones bajos
\usepackage{underscore}

%----------------------------------------------------------------------------------------
% OTROS PAQUETES ÚTILES
%----------------------------------------------------------------------------------------
\usepackage{siunitx}        % Unidades SI
\usepackage{tikz}           % Diagramas
\usetikzlibrary{shapes.geometric,arrows}
\usepackage{pdflscape}      % Páginas horizontales
\usepackage{etoolbox}       % Hooks

%----------------------------------------------------------------------------------------
% NUMERACIÓN
%----------------------------------------------------------------------------------------
% Numerar figuras, tablas y ecuaciones por sección
\numberwithin{equation}{section}
\numberwithin{figure}{section}
\numberwithin{table}{section}

%----------------------------------------------------------------------------------------
% COMANDOS PERSONALIZADOS
%----------------------------------------------------------------------------------------
% Comando para código inline con mejor formato
\newcommand{\code}[1]{\texttt{#1}}
