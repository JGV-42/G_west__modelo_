\documentclass[12pt]{article}

% Incluir preámbulo (paquetes y comandos personalizados)
%%%%%%%%%%%%%%%%%%%%%%%%%%%%%%%%%%%%%%%%%
% Preámbulo para artículo científico
% Compatible con elsarticle y fácilmente adaptable
%%%%%%%%%%%%%%%%%%%%%%%%%%%%%%%%%%%%%%%%%

%----------------------------------------------------------------------------------------
% IDIOMA Y CODIFICACIÓN
%----------------------------------------------------------------------------------------
\usepackage[spanish,es-tabla]{babel}
\usepackage[utf8]{inputenc}
\usepackage[T1]{fontenc}

%----------------------------------------------------------------------------------------
% BIBLIOGRAFÍA (biblatex + biber)
%----------------------------------------------------------------------------------------
\usepackage[
    backend=biber,
    style=authoryear,    % Estilo autor-año (compatible con elsarticle authoryear)
    sorting=nyt,         % Ordenar por Nombre, Año, Título
    natbib=true,         % Permite usar \citep y \citet
    maxcitenames=2,
    maxbibnames=99
]{biblatex}
\addbibresource{referencias.bib}

%----------------------------------------------------------------------------------------
% PAQUETES ESENCIALES
%----------------------------------------------------------------------------------------
\usepackage{amsmath,amsfonts,amssymb,amsthm}  % Matemáticas
\usepackage{graphicx}                          % Imágenes
\graphicspath{{figuras/}{./}}
\usepackage{booktabs}                          % Tablas profesionales
\usepackage{array}
\usepackage{tabularx}
\usepackage{longtable}
\usepackage{multirow}
\usepackage{float}
\usepackage{subcaption}

%----------------------------------------------------------------------------------------
% COLORES Y ENLACES
%----------------------------------------------------------------------------------------
\usepackage[table,xcdraw,dvipsnames]{xcolor}
\usepackage{hyperref}
\hypersetup{
    colorlinks=true,
    linkcolor=blue!70!black,
    citecolor=blue!70!black,
    urlcolor=blue!70!black
}

%----------------------------------------------------------------------------------------
% CONFIGURACIÓN DE TABLAS Y FIGURAS
%----------------------------------------------------------------------------------------
\usepackage{caption}
\captionsetup{
    font=small,
    labelfont=bf,
    labelsep=period,
    justification=justified,
    skip=6pt
}

% Espaciado uniforme en tablas
\renewcommand{\arraystretch}{1.15}

%----------------------------------------------------------------------------------------
% LISTAS
%----------------------------------------------------------------------------------------
\usepackage{enumitem}
\setlist{noitemsep,topsep=3pt}

%----------------------------------------------------------------------------------------
% CÓDIGO Y TEXTO MONOESPACIADO
%----------------------------------------------------------------------------------------
\usepackage{listings}
\lstset{
    basicstyle=\ttfamily\small,
    breaklines=true,
    frame=single,
    backgroundcolor=\color{gray!10}
}

% Permitir que \texttt{} rompa líneas en guiones bajos
\usepackage{underscore}

%----------------------------------------------------------------------------------------
% OTROS PAQUETES ÚTILES
%----------------------------------------------------------------------------------------
\usepackage{siunitx}        % Unidades SI
\usepackage{tikz}           % Diagramas
\usetikzlibrary{shapes.geometric,arrows}
\usepackage{pdflscape}      % Páginas horizontales
\usepackage{etoolbox}       % Hooks

%----------------------------------------------------------------------------------------
% NUMERACIÓN
%----------------------------------------------------------------------------------------
% Numerar figuras, tablas y ecuaciones por sección
\numberwithin{equation}{section}
\numberwithin{figure}{section}
\numberwithin{table}{section}

%----------------------------------------------------------------------------------------
% COMANDOS PERSONALIZADOS
%----------------------------------------------------------------------------------------
% Comando para código inline con mejor formato
\newcommand{\code}[1]{\texttt{#1}}


% ---------------------------------------------------------------------------
% Datos del artículo
% ---------------------------------------------------------------------------

% --- En tu main ---
\title{\textbf{GreenWest: inteligencia artificial para la predicción de créditos de carbono en proyectos de (re)forestación en España}\\
\small Correspondiente al grupo B1 (GreenWest) de las líneas de investigación de DemIA}

\author[1]{Maider Araceli Urbón Jiménez\thanks{Autora de correspondencia: murbon001@usal.es}}
\author[1]{Jaime Gabriel Vegas}
\author[1]{Ana de Luis Reboredo}
\author[1]{Belén Pérez Lancho}
\author[1]{Ana-Belén Gil-González}

\affil[1]{Grupo B1, Equipo de investigación BISITE, Universidad de Salamanca, Facultad de Ciencias, Salamanca, Castilla y León, España\\

\texttt{\{murbon001, JaimeGabrielVegas, adeluis, lancho, abg\}@usal.es}}


% ---------------------------------------------------------------------------

\begin{document}
\maketitle


% Abstract y Palabras Clave
\begin{abstract}
Este trabajo presenta \textbf{GreenWest}, un modelo de inteligencia artificial diseñado para predecir la cantidad de carbono capturado en proyectos de forestación y reforestación en España. El modelo se entrena con datos multifuente: registros del \textbf{Inventario Forestal Nacional} (\textbf{IFN3–IFN4}, MITECO)~\cite{ifn}, variables climáticas derivadas de \textbf{Copernicus/ERA5-Land}~\cite{era5land} e índices espectrales procedentes de \textbf{imágenes Landsat} (Collection~2, Level~2, USGS)~\cite{landsatc2}. Estos datos se integran en una base de datos relacional jerárquica descrita en un trabajo complementario~\cite{greenwestdb}, que organiza la información por parcela, especie y clase diamétrica, manteniendo trazabilidad y coherencia estructural entre inventarios.

\medskip

El modelo desarrollado responde a la pregunta: \textit{Dado un cultivo forestal con características concretas de vegetación, clima y terreno, ¿cuánto CO$_2$ contendrá pasados unos años?} Esta capacidad predictiva permite su integración en marcos de optimización forestal, abordando cuestiones como la selección de especies o la asignación óptima de terrenos para maximizar la fijación de carbono.

\medskip

Se evaluaron múltiples enfoques de aprendizaje supervisado, destacando \textbf{CatBoost}~\cite{catboost} como el modelo con mejor rendimiento (\(R^2>0.80\), RMSE<15), con alta capacidad de generalización temporal mediante validación cruzada por grupos. Los resultados demuestran el potencial del enfoque para estimar la absorción futura de CO$_2$ y optimizar decisiones de gestión forestal sostenible, contribuyendo a la transición hacia una economía baja en emisiones~\cite{cmnucc1992,UNFCCC1997}.

\medskip

\textbf{Palabras clave:} créditos de carbono, inteligencia artificial, forestación, reforestación, modelado predictivo, cambio climático.
\end{abstract}


% --- Contenido principal ---
% Cada sección se incluye desde su archivo correspondiente
\newpage
\section{Introducción}

El cambio climático es uno de los mayores desafíos globales de la actualidad, y su impacto negativo se refleja principalmente en el aumento de las concentraciones de dióxido de carbono (\(CO_2\)) en la atmósfera. Este aumento contribuye a fenómenos críticos como el deshielo de los polos, el incremento de fenómenos climáticos extremos y el deterioro de los ecosistemas naturales \cite{IPCC2007}. Los sumideros de carbono naturales, como los bosques, juegan un papel crucial en mitigar estos efectos, ya que mediante la fotosíntesis, los árboles capturan \(CO_2\) y lo almacenan en su biomasa, contribuyendo significativamente a la reducción de las concentraciones de este gas en la atmósfera.

\medskip 

A lo largo de las últimas décadas, instrumentos internacionales como el \textit{Protocolo de Kioto} \cite{UNFCCC1997} y la \textit{Convención Marco de las Naciones Unidas sobre el Cambio Climático (CMNUCC)} \cite{UNFCCC2015} han establecido mecanismos para reducir las emisiones de gases de efecto invernadero. En este marco, las actividades de forestación y reforestación han sido identificadas como fundamentales para la captura de \(CO_2\). Los \textit{créditos de carbono}, que representan la cantidad de \(CO_2\) en toneladas evitada o secuestrada por actividades como la forestación y reforestación de bosques, se han convertido en una herramienta clave para cumplir con los compromisos internacionales de reducción de emisiones.

\medskip 

Sin embargo, para que un proyecto de forestación o reforestación sea considerado válido para la obtención de créditos de carbono, existen diversas limitaciones legales que deben cumplirse. Según los requisitos establecidos por el \textit{Protocolo de Kioto} y las normativas nacionales, las siguientes condiciones deben ser satisfechas:

\begin{itemize}
    \item \textbf{Intervención humana directa:} Los árboles deben provenir de actividades de intervención humana, como la plantación, siembra o fomento de semilleros naturales. Esto significa que los cultivos forestales naturales no son elegibles para la contabilización de carbono.
    \item \textbf{Período mínimo de 30 años:} Para que un proyecto sea válido, debe garantizarse que los árboles permanezcan en el terreno durante un período mínimo de tiempo, generalmente 30 años, lo que excluye la absorción de carbono de cultivos estacionales, cuyo carbono es liberado nuevamente al ser cosechados.
    \item \textbf{Superficie mínima de 1 hectárea:} El proyecto debe abarcar al menos 1 hectárea de terreno para ser considerado.
    \item \textbf{Fracción mínima de cabida cubierta del 20\%:} Para que un área sea considerada como bosque, debe cubrir al menos el 20\% del área con especies arbóreas.
    \item \textbf{Altura mínima de los árboles maduros de 3 metros:} Los árboles deben alcanzar una altura mínima de 3 metros en su madurez, aunque no es necesario que alcancen esta altura al inicio de la plantación.
\end{itemize}

\medskip 

Satisfacer estas limitaciones legales es imprescindible para la correcta generación de créditos de carbono, y han sido tomadas en cuenta a lo largo del desarrollo del modelo predictivo del proyecto \textit{GreenWest}. 

\medskip 

El proyecto \textit{GreenWest} tiene como objetivo principal predecir la capacidad de absorción de \(CO_2\) en los cultivos forestales españoles, mediante el uso de modelos de inteligencia artificial. Este enfoque innovador tiene el potencial de transformar la gestión de proyectos de forestación y reforestación, optimizando las prácticas de plantación y maximizando la cantidad de carbono que se puede capturar en estos ecosistemas. 

\medskip 

Para lograrlo, el proyecto desarrollará un modelo predictivo que analizará datos sobre las características del terreno, las especies de árboles y las condiciones climáticas para estimar con precisión la cantidad de carbono que podría ser absorbido por un cultivo forestal en un periodo determinado. Este modelo no solo mejorará la comprensión del comportamiento de los sumideros de carbono, sino que también proporcionará herramientas útiles para la toma de decisiones estratégicas tanto en el ámbito empresarial como en el ambiental.

\medskip 

De esta forma, el proyecto \textit{GreenWest} contribuye a la transición hacia una economía baja en carbono, alineándose con los objetivos globales de sostenibilidad establecidos en el marco de la CMNUCC y el \textit{Protocolo de Kioto}, y promoviendo la creación de un mercado de créditos de carbono más eficiente y accesible para los actores económicos involucrados en la gestión de los recursos naturales.



\newpage
\section{Objetivos y Justificación}

El presente estudio tiene como objetivo principal desarrollar un modelo de inteligencia artificial capaz de predecir con precisión la capacidad de absorción de dióxido de carbono (\(CO_2\)) en cultivos forestales españoles. Este modelo se basa en variables que describen la especie arbórea, las características del terreno y las condiciones climáticas. A partir de este objetivo general se derivan varias metas específicas, que en conjunto justifican la relevancia y aplicabilidad del proyecto.

\subsection*{Objetivos específicos}

\begin{itemize}
    \item \textbf{Desarrollar un modelo predictivo robusto:} Construir un modelo de aprendizaje automático que estime la cantidad de \(CO_2\) que será capturado a lo largo del tiempo por un cultivo forestal, a partir de datos como especie, tipo de suelo, clase diamétrica, clima y otras variables relevantes.

    \item \textbf{Optimizar la captura de carbono:} Utilizar el modelo para identificar combinaciones óptimas de especies y terrenos que maximicen la fijación de carbono, contribuyendo a la planificación eficiente de proyectos de forestación y reforestación.

    \item \textbf{Asegurar la compatibilidad con las normativas internacionales:} Garantizar que las predicciones y salidas del modelo sean compatibles con los marcos normativos definidos por la \textit{Convención Marco de las Naciones Unidas sobre el Cambio Climático} (CMNUCC) y el \textit{Protocolo de Kioto}, cumpliendo así los criterios necesarios para la validación de créditos de carbono \cite{cmnucc1992, kioto1997}.

    \item \textbf{Analizar los factores determinantes del desarrollo forestal:} Estudiar la influencia de variables climáticas (como la temperatura y la precipitación) y edáficas (como el tipo de suelo o la pendiente) sobre el crecimiento forestal y su capacidad de capturar carbono.

    \item \textbf{Apoyar la toma de decisiones ambientales y empresariales:} Proporcionar una herramienta práctica y validada que permita a técnicos, gestores y empresas seleccionar las especies más adecuadas y planificar actuaciones de forestación con la mayor eficiencia posible en términos de secuestro de carbono.
\end{itemize}

\subsection*{Justificación}

La necesidad de contar con herramientas predictivas para estimar la captura de \(CO_2\) se ha intensificado ante el crecimiento del mercado voluntario de créditos de carbono, y las obligaciones adquiridas en el marco de la CMNUCC y el Protocolo de Kioto. Según estos acuerdos, cada país debe reportar sus emisiones y absorciones de gases de efecto invernadero, y puede utilizar actividades de forestación y reforestación como mecanismos de compensación \cite{cmnucc1992, kioto1997}.

Para que estos proyectos sean elegibles, deben cumplir criterios específicos como intervención humana directa, permanencia de al menos 30 años, cobertura mínima del 20\%, superficie mínima de una hectárea y una altura mínima de los árboles maduros de 3 metros. Estos criterios hacen imprescindible disponer de modelos que no solo estimen el carbono actual, sino que sean capaces de prever su evolución a futuro con base en condiciones iniciales y variables predictoras.

Este trabajo busca cubrir ese vacío mediante el uso de inteligencia artificial aplicada a datos reales y multifuente. Integrar su manejo dentro del sistema de créditos de carbono puede representar una importante oportunidad para la economía local y para la mitigación del cambio climático.


\newpage
% sections/03_revision_literatura.tex
\section{Revisión de la Literatura}

El secuestro de carbono en ecosistemas forestales ha cobrado una importancia creciente en la literatura científica, impulsada tanto por los compromisos internacionales en materia de cambio climático \cite{UNFCCC1997, UNFCCC2015} como por el auge de los mercados de créditos de carbono. Esto ha motivado el desarrollo de modelos orientados a cuantificar la biomasa forestal y estimar el contenido de carbono, aprovechando avances recientes en sensores remotos y técnicas de inteligencia artificial (IA).

Una de las estrategias más consolidadas es la estimación del carbono almacenado en un momento dado a partir de datos de teledetección. Goetz et al. (2009) \cite{goetz2009remote} revisan el uso de imágenes satelitales (MODIS, Landsat) en modelos empíricos de biomasa aérea, destacando su eficacia a escala regional en zonas boreales. Este tipo de estimaciones suele realizarse mediante regresiones lineales o algoritmos de mínimos cuadrados generalizados, con coeficientes de determinación (\(R^2\)) típicamente entre 0.6 y 0.8 según la resolución de entrada y la heterogeneidad del ecosistema.

La aplicación de aprendizaje profundo ha permitido mejorar sustancialmente la precisión y resolución espacial de estas estimaciones. Por ejemplo, Zhang et al. (2022) \cite{zhang2022carbon} integran imágenes Sentinel-2 con redes neuronales convolucionales, alcanzando un \(R^2\) de 0.84 para estimar el carbono en bosques subtropicales. Del mismo modo, Yang et al. (2023) \cite{yang2023forestcarbonai} desarrollan el modelo \textit{ForestCarbonAI}, entrenado con datos multiespectrales y LIDAR, con el que generan mapas de carbono forestal de alta resolución (10 m), reportando errores medios absolutos (MAE) inferiores a 3.5 tC/ha en zonas templadas. Otros trabajos recientes, como Reiersen et al. (2022) \cite{reiersen2022reforestree} o Dong et al. (2023) \cite{dong2023forest}, también demuestran la eficacia del deep learning para estimaciones estáticas, aunque se centran en contextos tropicales y no consideran el componente temporal.

Frente a estos enfoques descriptivos, algunas iniciativas han intentado proyectar la evolución del carbono a futuro. En el ámbito nacional, el Ministerio para la Transición Ecológica (MITECO) ha implementado herramientas como la calculadora ex ante de absorciones \cite{miteco_abexante_2025}, que permite obtener estimaciones simplificadas del carbono que puede fijarse en una plantación forestal en función de la especie y la zona agroclimática. No obstante, este instrumento se basa en coeficientes tabulados y no incorpora variables edafoclimáticas reales ni técnicas de modelización basadas en datos, lo que limita su precisión y capacidad de adaptación a contextos específicos.

En este escenario, el presente trabajo propone una metodología innovadora centrada en la predicción dinámica de carbono a largo plazo. A diferencia de los modelos anteriores, que estiman el carbono ya almacenado, este estudio se enfoca en anticipar cuánto carbono capturará un cultivo forestal en un horizonte temporal de 20 a 30 años. Para ello, se estudian diversos modelos de aprendizaje supervisado entrenados con datos históricos del Inventario Forestal Nacional (IFN2, IFN3 e IFN4), variables climáticas de Copernicus, características edáficas y métricas espectrales derivadas de imágenes Landsat \cite{landsat5_data, copuernicus_temps, miteco_guia_co2}. Los detalles sobre la arquitectura del modelo, las variables utilizadas, los algoritmos implementados y las métricas de evaluación se desarrollan en la siguiente sección.






\newpage
\section{Metodología}

Esta sección describe el procedimiento seguido para el entrenamiento y validación de los modelos predictivos desarrollados. 
La metodología se fundamenta en la identificación de los factores que determinan el crecimiento forestal y, en consecuencia, la capacidad de los ecosistemas para capturar carbono a lo largo del tiempo. 
El enfoque integra información estructural, climática y espectral procedente del Inventario Forestal Nacional (IFN) y de otras fuentes ambientales, con el propósito de construir modelos robustos que permitan predecir el contenido de carbono acumulado en la biomasa viva.

\medskip

El carbono fijado por los árboles se acumula progresivamente en su biomasa, en función del tamaño y vigor de los individuos, los cuales están condicionados por variables ambientales, topográficas y de competencia intraespecífica. 
Las condiciones meteorológicas, como la temperatura y la precipitación, inciden directamente en la fotosíntesis y en la disponibilidad hídrica; 
la orientación, la pendiente y la altitud modifican la radiación incidente y el microclima local; 
mientras que la densidad de árboles por unidad de superficie determina el nivel de competencia por los recursos, variando según la especie y su tolerancia ecológica \cite{IPCC2006, Buchholz2014}. 

\medskip

A partir de estos fundamentos, se construyó una base de datos relacional que integra información forestal, climática y espectral a nivel de parcela, especie y clase diamétrica. 
Esta estructura permite caracterizar con precisión la dinámica del bosque entre inventarios sucesivos y alimentar modelos predictivos capaces de estimar el contenido futuro de carbono a partir de las condiciones observadas en el pasado.


\subsection{Origen y estructura de los datos}
% Origen (IFN2/3/4 + clima + índices)
% Estructura relacional (parcelas, inventarios, especie, CD, estación, árbol)
% Referencia al esquema (figura) y a meta_variables


La base de datos empleada en este trabajo integra información forestal, climática y espectral estructurada en torno a la parcela como unidad básica. Cada parcela se describe mediante sus coordenadas geográficas, características edáficas y su evolución a través de distintos inventarios (IFN2, IFN3, IFN4).

\medskip

Los datos forestales incluyen información por especie y clase diamétrica, como número de pies, volumen con y sin corteza, área basimétrica, carbono aéreo, radical y total. Estos valores permiten caracterizar con precisión la estructura y crecimiento de la vegetación.

\medskip

A cada parcela se asocian también estadísticas climáticas agregadas por estación e inventario: temperaturas (superficie, aire y subsuelo) y precipitaciones, resumidas mediante métricas como media, máxima, mínima y desviación típica.

\medskip

Finalmente, se incorporan índices espectrales derivados de imágenes satelitales (NDVI, EVI, NDII, GNDVI), que permiten cuantificar propiedades biofísicas de la vegetación:
\begin{itemize}
    \item \textbf{NDVI (Normalized Difference Vegetation Index):} estima la actividad fotosintética.
    \item \textbf{EVI (Enhanced Vegetation Index):} mejora la sensibilidad en zonas densamente vegetadas.
    \item \textbf{NDII (Normalized Difference Infrared Index):} refleja el contenido hídrico de la vegetación.
    \item \textbf{GNDVI (Green NDVI):} variante del NDVI basada en la banda verde, sensible al clorofila.
\end{itemize}

\medskip

\subsubsection*{Estrutura de la base de datos}
Estos datos se organizan en las siguientes entidades troncales:

\begin{itemize}
  \item \textbf{parcelas}: icontiene la información básica de localización y características edáficas de cada parcela.
  \item \textbf{parcela\_inventario}: describe el estado de cada parcela en un inventario determinado (\texttt{parcela\_id}, \texttt{inventario\_id}), incluyendo atributos edáficos y de contexto (p. ej., \texttt{nivel1\_id}, \texttt{textura\_id}).
  \item \textbf{parcela\_inventario\_especie}: detalla la presencia y condición de cada especie dentro de una parcela e inventario, incorporando descriptores de masa y tratamientos silvícolas.
  \item \textbf{parcela\_inventario\_especie\_cd}: describe las poblaciones arbóreas por parcela, especie y \emph{clase diamétrica} (\texttt{cd\_id}): n.º de pies (\texttt{npies}), área basimétrica (\texttt{abas}), volúmenes (\texttt{vcc}, \texttt{vsc}, \texttt{vle}), incrementos (\texttt{iavc}) y carbono (\texttt{ca}, \texttt{cr}).
  \item \textbf{parcela\_especie\_arbol}: caracteriza los pies mayores identificados por parcela y especie en el inventario cuarto. Recoge las caracteristicas particulares de cada pie como altura (\texttt{ht}), diámetros (\texttt{dn1} y \texttt{dn2}), ubicación respecto del centro de la parcela (\texttt{rumbo}, \texttt{distancia}), volumen (\texttt{vcc, vsc, vle}), incremento (\texttt{iavc}) y carbono (\texttt{ca, cr}).  
  \item \textbf{parcela\_inventario\_estacion}: almacena agregados climático-biofísicos por estación (\texttt{estacion\_id}) en la misma granularidad parcela–inventario, incluyendo variables como precipitación (\texttt{PR}) y temperatura (\texttt{T2M, SKT, STL*}), junto a índices de vegetación (NDVI, EVI, NDII, GNDVI).
  \item \textbf{especies} y \textbf{grupos}: recogen la información taxonómica y su clasificación jerárquica, estableciendo la relación entre especies individuales y grupos funcionales.
\end{itemize}

Cada variable categórica posee una tabla de catálogo propia (\texttt{cat\_}), donde se definen los valores posibles y sus descripciones. Por ejemplo, \texttt{cat\_textura}, \texttt{cat\_nivel1}, \texttt{cat\_tratmasa} o \texttt{cat\_origen}. Todas siguen un patrón uniforme: la clave primaria es el identificador de la variable (\texttt{<variable>\_id}), y las tablas troncales referencian este mismo campo como clave foránea. Además la base de datos incluye una tabla llamada \texttt{meta\_variables} que recoge los metadatos.

\medskip

La Figura~\ref{fig:GWest_BBDD} muestra el esquema general de las tablas troncales y sus principales relaciones. Este diagrama resume la estructura interna de la base de datos y su jerarquía de dependencias.

\begin{figure}[H]
  \centering
  \includegraphics[width=0.9\textwidth]{figuras/Estrctr_BBDD_GWest.png}
  \caption{Esquema relacional de las tablas principales de la base de datos. Tabla extraida de \cite{greenwestdb}, donde se pueden consultar más detalles sobre las variables.}
  \label{fig:GWest_BBDD}
\end{figure}

\subsubsection*{Diccionario resumido de variables}
\small
\setlength{\LTcapwidth}{\textwidth}
\begin{longtable}{p{3.2cm} p{7.6cm} p{2.4cm} p{2.4cm}}
\caption{Resumen de variables principales por entidad. Tabla extraida de \cite{greenwestdb}.}\\
\toprule
\textbf{Variable} & \textbf{Descripción} & \textbf{Unidad} & \textbf{Tipo de dato} \\
\midrule
\endfirsthead
\toprule
\textbf{Variable} & \textbf{Descripción} & \textbf{Unidad} & \textbf{Tipo de dato} \\
\midrule
\endhead
\midrule
\multicolumn{4}{r}{\emph{Continúa en la siguiente página}} \\
\midrule
\endfoot
\bottomrule
\endlastfoot

\multicolumn{4}{l}{\textbf{parcelas}} \\
\texttt{parcela\_id} & Identificador único de parcela (IFN). & -- & Identificador \\
\texttt{latitud}, \texttt{longitud} & Coordenadas geográficas (WGS84). & ° & Geográfico \\
\texttt{coorx}, \texttt{coory} & Coordenadas UTM; \texttt{huso} especifica zona. & m (UTM) & Geográfico \\
\texttt{elevacion} & Cota sobre el nivel del mar (NASADEM). & m & Numérico \\
\texttt{pendiente} & Inclinación del terreno. & ° & Numérico \\
\texttt{orientacion} & Orientación del terreno (0–360). & ° & Numérico \\
\texttt{presencia\_id} & Presencia en IFN $\rightarrow$ \texttt{cat\_presencia}. & -- & Categórico \\
\texttt{tipsuelo1\_id}, \texttt{tipsuelo2\_id}, \texttt{tipsuelo3\_id} & Tipos de suelo $\rightarrow$ \texttt{cat\_tipsuelo*}. & -- & Categórico \\
\texttt{rocosidad\_id} & Rocosidad $\rightarrow$ \texttt{cat\_rocosidad}. & -- & Categórico \\
\texttt{radio}, \texttt{superficie} & Radio de parcela y superficie derivada. & m; ha & Numérico \\
\addlinespace

\multicolumn{4}{l}{\textbf{parcela\_inventario}} \\
\texttt{parcela\_id}, \texttt{inventario\_id} & Clave compuesta (parcela-inventario). & -- & Identificador \\
\texttt{ano} & Año de apeo. & año & Numérico \\
\texttt{nivel1\_id}, \texttt{nivel2\_id} & Morfoestructura. $\rightarrow$ \texttt{cat\_nivel*}. & -- & Categórico \\
\texttt{textura\_id} & Textura de suelo $\rightarrow$ \texttt{cat\_textura}. & -- & Categórico \\
\texttt{merosiva\_id} & Manifestaciones erosivas $\rightarrow$ \texttt{cat\_merosiva}. & -- & Categórico \\
\texttt{matorg\_id} & Materia orgánica $\rightarrow$ \texttt{cat\_matorg}. & -- & Categórico \\
\texttt{modcomb\_id} & Modelo de combustible $\rightarrow$ \texttt{cat\_modcomb}. & -- & Categórico \\
\texttt{disesp\_id} & Distribución espacial $\rightarrow$ \texttt{cat\_disesp}. & -- & Categórico \\
\texttt{comesp\_id} & Composición específica $\rightarrow$ \texttt{cat\_comesp}. & -- & Categórico \\
\texttt{fccarb}, \texttt{fcctot} & Fracción de cabida cubierta (árboles). & \% & Numérico \\
\addlinespace

\multicolumn{4}{l}{\textbf{parcela\_inventario\_especie}} \\
\texttt{parcela\_id}, \texttt{inventario\_id}, \texttt{especie\_id} & Clave compuesta (parcela-inventario-especie). & -- & Identificador \\
\texttt{ocupa} & Grado de ocupación de la especie. & (0--10) & Numérico \\
\texttt{estado\_id} & Estado de desarrollo. $\rightarrow$ \texttt{cat\_estado}. & -- & Categórico \\
\texttt{fpmasa\_id} & Forma principal de masa $\rightarrow$ \texttt{cat\_fpmasa}. & -- & Categórico \\
\texttt{tratmasa\_id} & Tratamientos selvícolas $\rightarrow$ \texttt{cat\_tratmasa}. & -- & Categórico \\
\texttt{orgmasa1\_id} & Origen de masa (IFN3/4)$\rightarrow$ \texttt{cat\_orgmasa1}. & -- & Categórico \\
\texttt{masa\_id} & Clasificación de masa $\rightarrow$ \texttt{cat\_masa}. & -- & Categórico \\
\texttt{origen\_id} & Origen de la masa (IFN2) $\rightarrow$ \texttt{cat\_origen}. & -- & Categórico \\
\addlinespace

\multicolumn{4}{l}{\textbf{parcela\_inventario\_especie\_cd}} \\
\texttt{parcela\_id}, \texttt{inventario\_id}, \texttt{especie\_id} & Clave compuesta ( parcela-inventario-especie-cd). & -- & Identificador \\
\texttt{cd\_id} & Clase diamétrica (CD) reglamento IFN. & cm & Numérico discreto \\
\texttt{npies} & Número de pies. & pies/ha & Numérico \\
\texttt{abas} & Área basimétrica. & m$^{2}$/ha & Numérico \\
\texttt{vcc}, \texttt{vsc}, \texttt{vle} & Volúmenes (con/sin corteza; leñas). & m$^{3}$/ha & Numérico \\
\texttt{iavc} & Incremento anual del volumen con corteza. & m$^{3}$/ha$\cdot$año & Numérico \\
\texttt{ca}, \texttt{cr} & Carbono aéreo y radical. & t/ha & Numérico \\
\texttt{ht} & Altura media (modelo CatBoost). & m & Numérico \\
\texttt{carbono\_bruto} & Carbono total estimado (alometrías). & t & Numérico \\
\addlinespace

\multicolumn{4}{l}{\textbf{parcela\_especie\_arbol}} \\
\texttt{parcela\_id}, \texttt{especie\_id} & Clave compuesta (parcela–especie–árbol). & -- & Identificador \\ 
\texttt{arbol\_id} & Identificador del árbol dentro de parcela y especie. & -- & Entero \\ 
\texttt{rumbo} & Rumbo desde el centro de la parcela al árbol. & grados centesimales & Numérico \\ 
\texttt{distancia} & Distancia desde el centro de la parcela al árbol. & m & Numérico \\ 
\texttt{cd} & Clase diamétrica (reglamento IFN). & cm & Numérico discreto \\ 
\texttt{ht} & Altura total del árbol inventariado. & m & Numérico \\ 
\texttt{dn1}, \texttt{dn2} & Diámetros normales perpendiculares. & mm & Numérico \\ 
\texttt{abas} & Área basimétrica del pie medido. & m$^{2}$ & Numérico \\ 
\texttt{iavc} & Incremento anual del volumen con corteza. & dm$^{3}$/año & Numérico \\ 
\texttt{vcc}, \texttt{vsc}, \texttt{vle} & Volúmenes (con corteza, sin corteza, leñas). & dm$^{3}$ & Numérico \\ 
\texttt{ca}, \texttt{cr} & Carbono aéreo y radical del árbol. & t & Numérico \\
\addlinespace

\multicolumn{4}{l}{\textbf{parcela\_inventario\_estacion}} \\
\texttt{parcela\_id}, \texttt{inventario\_id}, \texttt{estacion\_id} & Clave compuesta (agregado estacional). & -- & Identificador \\
\texttt{PR\_*} & Estadísticos de precipitación (mean, max, min, std, sum). & mm/(m$^2\cdot$día), mm/m$^2$ & Numérico \\
\texttt{T2M\_*}, \texttt{SKT\_*} & Aire 2\,m y temperatura superficial (mean, max, min, std). & °C & Numérico \\
\texttt{STL1\_*}--\texttt{STL4\_*} & Temperatura del suelo por niveles (mean, max, min, std). & °C & Numérico \\
\texttt{NDVI\_*}, \texttt{EVI\_*}, \texttt{NDII\_*}, \texttt{GNDVI\_*} & Índices de vegetación (max, mean, median, min, std). & adimensional & Numérico \\
\addlinespace

\multicolumn{4}{l}{\textbf{especies} y \textbf{grupos}} \\
\texttt{especie\_id} & Identificador de especie IFN. & -- & Identificador \\
\texttt{nombre}, \texttt{sinonimia} & Denominación IFN y sinónimos. & -- & Texto \\
\texttt{tipo\_especie} & 0\,= conífera; 1\,= frondosa. & -- & Categórico \\
\texttt{grupo\_id} & Grupo funcional $\rightarrow$ \texttt{grupos}. & -- & Identificador \\
\texttt{grupos.nombregrupo} & Nombre del grupo. & -- & Texto \\
\end{longtable}
\normalsize

\subsubsection*{Cardinalidad y completitud}

El volumen de entradas por tabla es:
\begin{center}
\begin{tabular}{l r}
\toprule
\textbf{Tabla} & \textbf{Número de registros} \\
\midrule
\texttt{parcelas} & 52{,}298 \\
\texttt{parcela\_inventario} & 147{,}995 \\
\texttt{parcela\_inventario\_especie} & 417{,}119 \\
\texttt{parcela\_inventario\_especie\_cd} & 1{,}191{,}070 \\
\texttt{parcela\_especie\_arbol} & 855{,}860 \\
\texttt{parcela\_inventario\_estacion} & 470{,}056 \\
\texttt{especies} & 195 \\
\texttt{grupos} & 33 \\
\bottomrule
\end{tabular}
\end{center}

\subsection{Variables objetivo}

El objetivo del modelo es estimar el \textbf{carbono total} que una parcela forestal puede capturar en un horizonte temporal de 20--30 años, a partir de las condiciones observadas en inventarios previos. 
Para ello se definieron dos variables de respuesta complementarias, ambas derivadas de los datos del Inventario Forestal Nacional (IFN), que permiten analizar el contenido de carbono desde perspectivas distintas: una normalizada por superficie y otra en términos absolutos.

\medskip

\begin{enumerate}
    \item \textbf{\texttt{c}} (tC/ha): representa el \textbf{carbono total contenido en la biomasa viva aérea y subterránea} por unidad de superficie, expresado en \emph{toneladas de carbono por hectárea}. 
    Su cálculo se basa en la suma de las estimaciones de carbono aéreo (\texttt{ca}) y radical (\texttt{cr}) reportadas por el IFN. 
    En los casos con valores faltantes, se completó la información mediante un modelo de \emph{Random Forest Regressor} ajustado sobre variables dendrométricas observadas (Especie, CD, VSC, NPies, ABas, IAVC, VCC y VLE), alcanzando un rendimiento satisfactorio (\(R^2_{test} > 0.90\)). 
    Esta variable es coherente con los formatos internacionales de reporte de inventarios forestales y permite comparar el contenido de carbono entre parcelas o especies.

    \medskip

    \item \textbf{\texttt{carbono\_bruto}} (tC): corresponde al \textbf{carbono total capturado por parcela y especie}, expresado en \emph{toneladas de carbono totales}. 
    Su estimación se realiza de forma trazable y físicamente interpretable a partir de variables medidas directamente en campo: número de pies (\texttt{npies}), altura media (\texttt{ht}), tipo de especie (\texttt{clase\_especie}) y clase diamétrica (\texttt{cd\_id}). 
    El cálculo sigue un modelo alométrico adaptado de \cite{chave2014} y las directrices del IPCC~\cite{ipcc2006}, incorporando tanto la biomasa aérea como la biomasa radical mediante la relación Parte Radical:Parte Aérea ($R$). 
    El resultado se expresa en toneladas de carbono totales por parcela, sin normalizar por superficie, lo que facilita la trazabilidad del proceso y la comparación entre inventarios sin depender de factores de expansión específicos del IFN. 
    En coherencia con los criterios de proyectos de forestación y reforestación, las observaciones correspondientes a brinzales o plantones se consideran con valor de carbono nulo, dado que las fases tempranas de desarrollo no se contabilizan oficialmente como carbono capturado.
\end{enumerate}

\medskip

Estas dos variables resumen el contenido de carbono forestal desde enfoques complementarios: 
\texttt{c} (tC/ha) permite la comparación espacial y temporal entre masas forestales, mientras que \texttt{carbono\_bruto} (tC) ofrece una medida absoluta y directamente derivada de las observaciones de campo. 
Ambas constituyen los objetivos principales del modelado predictivo, orientado a estimar el carbono acumulado en el \textbf{IFN4} a partir de las condiciones registradas en los inventarios anteriores (\textbf{IFN2} e \textbf{IFN3}).

\subsection{Supuestos de elegibilidad y verificación externa}
% Intervención humana, permanencia 30a (extrapolación cauta), superficie>=1ha,
% fccarb>=20% (filtro), altura>=3m en madurez (decisión de diseño).

Para que un proyecto forestal sea elegible en programas de \emph{créditos de carbono}, debe cumplir requisitos técnicos establecidos por marcos regulatorios internacionales \cite{IPCC2006, miteco_guia_co2}. A continuación se resume cada criterio y la forma en que se aborda en este estudio:

\begin{itemize}
    \item \textbf{Intervención humana directa.} El incremento de carbono debe proceder de actuaciones planificadas (reforestación, restauración o manejo sostenible). En nuestro caso, el modelo se entrena sobre datos observacionales (IFN2--IFN3--IFN4); por tanto, la \emph{verificación de intervención} no se deduce del modelo, sino que se contempla como \emph{condición externa} de elegibilidad del proyecto a evaluar.

    \item \textbf{Permanencia mínima de 30 años.} Para caracterizar el crecimiento de las parcelas forestales en los datos que alimentan el modelo, es necesario disponer de dos mediciones sucesivas de cada parcela, separadas por un intervalo temporal conocido. Estas mediciones permiten cuantificar la evolución de las variables forestales y, por tanto, estimar el incremento de carbono asociado al crecimiento del arbolado durante dicho periodo. 

    En este trabajo, el objetivo es predecir el contenido de carbono correspondiente al \textbf{IFN4}, utilizando como información explicativa las variables observadas en inventarios anteriores. Dado que los inventarios tercero y cuarto comparten una estructura homogénea y un conjunto de variables comparable la elección más directa para el entrenamiento del modelo sería emplear exclusivamente estos dos inventarios. Esta estrategia aprovecha la coherencia estructural de los inventarios más recientes, que incluyen un mayor número de variables y una caracterización más detallada del terreno. 

    \medskip

    No obstante, este planteamiento se enfrenta a la limitación impuesta por la \textbf{permanencia mínima de 30 años}, requisito fundamental en el contexto de los proyectos de compensación. El intervalo de tiempo entre los inventarios \textbf{IFN3} e \textbf{IFN4} es relativamente corto: no supera los 18 años.   
    
    \medskip

    La Figura~\ref{fig:periodo34} muestra la distribución de la diferencia de años entre las mediciones del IFN3 y el IFN4. Como puede observarse, la mayoría de las parcelas presentan intervalos comprendidos entre 6 y 17 años, un rango demasiado estrecho para evaluar la estabilidad del modelo en horizontes más amplios.

    \begin{figure}[h!]
        \centering
        \includegraphics[width=0.9\textwidth]{figuras/periodo34.png}
        \caption{Distribución de la diferencia de años entre los inventarios IFN3 e IFN4.}
        \label{fig:periodo34}
    \end{figure}

    Para ampliar la cobertura temporal y mejorar la capacidad de generalización del modelo, se optó por unificar la información de los inventarios \textbf{IFN2} e \textbf{IFN3} como base explicativa para la predicción del \textbf{IFN4}. Esta integración permite disponer de pares de mediciones de parcelas separadas por intervalos que oscilan entre 6 y 29 años, lo que constituye un rango mucho más representativo del horizonte de 20--30 años establecido como referencia.

    \begin{figure}[h!]
        \centering
        \includegraphics[width=0.9\textwidth]{figuras/periodo234.png}
        \caption{Distribución de la diferencia de años entre los inventarios IFN2--IFN3 e IFN3--IFN4.}
        \label{fig:periodo234}
    \end{figure}

    \medskip

    De esta forma, el modelo se entrena y valida sobre un conjunto de datos más diverso y equilibrado, tanto en estructura como en amplitud temporal, manteniendo la coherencia metodológica y la trazabilidad de las estimaciones. Este enfoque no sólo mejora la robustez del aprendizaje, sino que también refuerza la capacidad del modelo para proyectar la captura de carbono en escenarios compatibles con los requisitos de permanencia de los proyectos de compensación.

    \item \textbf{Superficie mínima de 1 ha.} Este criterio se considera \emph{externo} al alcance del modelo predictivo, ya que el aprendizaje se realiza a nivel de parcela e inventario y no sobre polígonos de superficie total. En la práctica, la verificación de la superficie se realiza \emph{ex ante}, sobre la geometría declarada del proyecto forestal. En los terrenos forestales generados a partir de intervención humana directa —como plantaciones o repoblaciones—, la extensión suele presentar una estructura homogénea, con una especie dominante, edades coetáneas y densidades estandarizadas. Bajo estas condiciones, el carbono total es proporcional a la superficie: duplicar el área de una masa forestal homogénea implica aproximadamente duplicar su carbono almacenado. Por tanto, la variable de superficie no afecta al ajuste interno del modelo y su cumplimiento puede evaluarse fácilmente a nivel de proyecto, sin comprometer la validez de las predicciones.

    \item \textbf{Fracción mínima de cabida cubierta del 20\%.} La base de datos dispone de \texttt{fccarb} (arbórea) y \texttt{fcctot} (total). Este umbral se aplica como \emph{filtro de elegibilidad} previo o posterior al modelado, sin modificar la arquitectura del modelo (\texttt{fccarb}$>20$).

    \item \textbf{Altura mínima de 3 m en la madurez.} Este requisito se refiere a la altura que alcanzan los árboles en su fase de pleno desarrollo, y no a la altura inicial de los plantones. Por tanto, las mediciones realizadas durante las etapas tempranas de crecimiento no determinan la elegibilidad del proyecto, siempre que las especies seleccionadas sean capaces de superar los 3 metros en la madurez. En nuestro conjunto de datos, la altura no se registra explícitamente, por lo que este criterio se evalúa de forma \emph{externa} al modelo, mediante la selección de especies forestales adecuadas y la verificación con fuentes auxiliares (catálogos silvícolas o tipologías de masa). En la práctica, el cumplimiento del requisito depende de una decisión de diseño del proyecto —\emph{no plantar especies cuyo tamaño adulto sea inferior a 3 metros}— más que del ajuste predictivo del modelo. Por ello, la altura no interviene directamente en el entrenamiento, aunque sí condiciona la elegibilidad final del proyecto forestal. 
\end{itemize}

\subsection{Preparación y tratamiento de los datos}
% Filtros (fccarb>=20%, crecimiento positivo, etc.)
% Agregaciones (parcela-especie, compresión CD -> npies_{cd}, etc.)
% Cálculo de variables derivadas (carbono\_bruto)
% Codificación y escalado

Como ya se ha introducido el entrenamiento se realiza en dos líneas según la variable objetivo: \texttt{c} de \textbf{IFN4} o \texttt{carbono\_bruto} de \textbf{IFN4}; y según la información que se usa como explicativa: \textbf{IFN3} o \textbf{IFN3} e \textbf{IFN2}. Se plantea la preparación y filtrado de los datos en términos generales (variable objetivo por \texttt{c} o \texttt{carbono\_bruto} y primera inventariación/ inventariación explicativa por \textbf{IFN3} o la unión de \textbf{IFN2} e \textbf{IFN3}).

\subsubsection*{Filtrado de registros}

Se descartan todas aquellas parcelas en las que el valor de carbono total (variable objetivo) en la segunda inventariación es inferior a la primera. Estos casos suelen deberse a episodios de deforestación, incendios u otras perturbaciones, y no representan un crecimiento forestal neto.

\medskip

El conjunto de datos se restringe únicamente a las parcelas que presentan una \texttt{fccarb} (fracción de cabida cubierta arbórea) igual o superior al 20\,\% en el \textbf{IFN3}. Este umbral define la proporción mínima de superficie ocupada por copas de árboles respecto al área total de la parcela, y constituye una de las condiciones esenciales para considerar una superficie como terreno forestal. La exclusión de parcelas con \texttt{fccarb} inferior al 20\,\% permite asegurar que las estimaciones de carbono se realicen sobre masas forestales consolidadas, evitando sesgos asociados a áreas agrícolas o matorrales. A los datos del \textbf{IFN2} no se les aplica dicho filtro porque no disponen de la variable \texttt{fccarb}.

\medskip

Los conteos de observaciones por inventario y condición se resumen a continuación:

\begin{itemize}
    \item \textbf{IFN2:} Total de parcelas = \textbf{88.696}
    \begin{itemize}
        \item Casos con $c4 > c$: \textbf{31.428}
        \item Casos con $carbono\_bruto4 > carbono\_bruto$: \textbf{32.403}
    \end{itemize}

    \item \textbf{IFN3:} Total de parcelas = \textbf{171.157}
    \begin{itemize}
        \item Casos con $fccarb > 20$: \textbf{158.434}
        \item Casos con $fccarb > 20$ y $c4 > c$: \textbf{57.401}
        \item Casos con $fccarb > 20$ y $carbono\_bruto4 > carbono\_bruto$: \textbf{76.617}
    \end{itemize}
\end{itemize}

\subsubsection*{Cálculo y agregación de variables}

Cada registro de entrada se genera a nivel de combinación parcela--especie, incorporando las variables correspondientes de la primera medición y la variable objetivo (carbono) de la segunda medición (IFN4). Las variables de \texttt{parcela} y \texttt{parcela\_inventario} se desdoblan para cada especie. Las entradas de la tabla \texttt{parcela\_inventario\_especie\_cd} se agrupan por parcela y especie y se comprimen en una única entrada creando un conjunto de variables para cada clase diamétrica.

\medskip

La Tabla~\ref{tab:entrada_modelo} resume las variables empleadas como entrada al modelo, integradas desde las distintas tablas que conforman la base de datos relacional.

\begin{table}[H]
\renewcommand{\arraystretch}{1.2}
\setlength{\tabcolsep}{3pt}
\centering
\small
\resizebox{\textwidth}{!}{%
\begin{tabular}{|p{3.6cm}|p{2.2cm}|p{7.6cm}|p{2.2cm}|}
\hline
\multicolumn{4}{|c|}{\textbf{Resumen de Datos de Entrada del Modelo}} \\
\hline
\textbf{Variable} & \textbf{Tipo} & \textbf{Descripción} & \textbf{Anexo} \\
\hline
\textcolor{ForestGreen}{\texttt{parcela\_id}} & varchar & Identificador \emph{único} de parcela. & -- \\
\hline
\textcolor{ForestGreen}{\texttt{especie\_id}, \texttt{tipo\_especie}, \texttt{grupo\_id}} & int (CF) & Especie (código), tipo de especie y grupo taxonómico. & Anexos \ref{sec:especies} y \ref{sec:gruposespecies} \\
\hline
\textcolor{ForestGreen}{\texttt{ocupa}} & int & Grado de ocupación/presencia de la especie en la parcela (0--10). & -- \\
\hline
\textcolor{ForestGreen}{\texttt{estado\_id}}, \texttt{fpmasa\_id}, \texttt{tratmasa\_id}, \texttt{orgmasa\_1\_id} & int (CF) & Estado/fase de desarrollo, forma principal de masa, tratamiento de masa, organización de masa. & Anexos \ref{sec:EstadoIFN34}, \ref{sec:FPMasa}, \ref{sec:tratmasa} y \ref{sec:OrgMasa} \\
\hline
\textcolor{ForestGreen}{\texttt{tipsuelo1\_id}, \texttt{tipsuelo2\_id}, \texttt{tipsuelo3\_id}} & int (CF) & Tipos de suelo de la parcela (niveles jerárquicos). & Anexo \ref{sec:TipSuelo} \\
\hline
\textcolor{ForestGreen}{\texttt{rocosidad\_id}}, \texttt{textura\_id}, \texttt{matorg\_id}, \texttt{modcomb\_id}, \textcolor{ForestGreen}{\texttt{disesp\_id}, \texttt{comesp\_id}}, \texttt{merosiva\_id} & int (CF) & Rocosidad, textura, materia orgánica, modelo de combustibilidad, distribución/composición específica, manifestaciones erosivas. & Anexos \ref{sec:Rocosid}, \ref{sec:textura}, \ref{sec:MatOrg}, \ref{sec:modComb}, \ref{sec:disEsp}, \ref{anexo:compesp} y \ref{sec:ManERo}. \\
\hline
\textcolor{ForestGreen}{\texttt{radio}} & float & Radio de la parcela (m). & -- \\
\hline
\textcolor{ForestGreen}{\texttt{orientacion}, \texttt{elevacion}, \texttt{pendiente}} & float & Orientación (grados), elevación (m s.n.m.), pendiente (\%). & -- \\
\hline
\texttt{nivel1\_id}, \texttt{nivel2\_id} & int (CF) & Niveles jerárquicos/estratos de inventario. & Anexos \ref{sec:nivel1} y \ref{sec:nivel2}. \\
\hline
\texttt{fccarb}, \texttt{fcctot} & float & Fracción de cabida cubierta arbórea y total. & -- \\
\hline
\texttt{\textcolor{ForestGreen}{npies}\_\{1,2,5,\textcolor{ForestGreen}{10,15, 20,25,30,35,40,45, 50,55,60,65,70}\}} & float & Número de pies por clase diamétrica CD (cm). Cada campo corresponde a la CD indicada. & -- \\
\hline
\textcolor{ForestGreen}{\texttt{periodo}} & int & Años transcurridos entre inventarios considerados en el modelo. & -- \\
\hline
\textcolor{ForestGreen}{\texttt{evi\_\{stat\}\_\{est\}}} & float & Índice EVI por estación; \texttt{stat} $\in$ \{max, mean, median, min, std\}, \texttt{est} $\in$ \{invierno, oto\~no, primavera, verano\}. & -- \\
\hline
\textcolor{ForestGreen}{\texttt{gndvi\_\{stat\}\_\{est\}}} & float & Índice GNDVI por estación (misma convención de \texttt{stat} y \texttt{est}). & -- \\
\hline
\textcolor{ForestGreen}{\texttt{ndii\_\{stat\}\_\{est\}}} & float & Índice NDII por estación (misma convención). & -- \\
\hline
\textcolor{ForestGreen}{\texttt{ndvi\_\{stat\}\_\{est\}}} & float & Índice NDVI por estación (misma convención). & -- \\
\hline
\textcolor{ForestGreen}{\texttt{pr\_\{stat\}\_\{est\}}} & float & Precipitación: \texttt{stat} $\in$ \{max, mean, min, std, sum\} por estación \texttt{est}. & -- \\
\hline
\textcolor{ForestGreen}{\texttt{skt\_\{stat\}\_\{est\}}} & float & Temperatura de superficie (skin temperature): \texttt{stat} $\in$ \{max, mean, min, std\}. & -- \\
\hline
\textcolor{ForestGreen}{\texttt{stl1-4\_\{stat\}\_\{est\}}} & float & Temperatura de suelo por capa (1--4): \texttt{stat} $\in$ \{max, mean, min, std\}. & -- \\
\hline
\textcolor{ForestGreen}{\texttt{t2m\_\{stat\}\_\{est\}}} & float & Temperatura del aire a 2 m: \texttt{stat} $\in$ \{max, mean, min, std\}. & -- \\
\hline
\textcolor{ForestGreen}{\texttt{c4}} & float & Carbono capturado en el cultivo en el IFN4 en t/ha. & -- \\
\hline
\textcolor{ForestGreen}{\texttt{carbono\_bruto4}} & float & Carbono capturado en el cultivo en el IFN4 en t. & -- \\
\hline
\end{tabular}%
}
\caption{\tiny Resumen de variables de entrada del modelo. Para variables estacionales se usa la notación \texttt{variable\_\{stat\}\_\{est\}}, con estadísticas \texttt{stat} y estaciones \texttt{est} en \{invierno, oto\~no, primavera, verano\}. Las variables \texttt{npies\_\{CD\}} se repiten para cada clase diamétrica indicada. Las variables destacadas en \textcolor{ForestGreen}{verde} se encuentran recogidas tanto para el IFN2 como para el IFN3, las no destacadas solo se recogen para el IFN3.}
\label{tab:entrada_modelo}
\end{table}

\subsubsection*{Codificación y normalización}

Las variables categóricas se codifican mediante \textit{one-hot encoding}, generando variables binarias para cada clase. Las variables numéricas se escalan (normalización estándar o min-max, según el modelo) para asegurar que todas las magnitudes tengan el mismo orden de importancia durante el entrenamiento.



\section{Selección de variables explicativas}

La selección de variables explicativas se abordó mediante tres estrategias complementarias: (1) selección automática basada en el algoritmo \textit{Featurewiz}, (2) selección mediante el criterio de relevancia y mínima redundancia (\textit{mRMR}), y (3) una selección manual fundamentada en criterios estadísticos, ecológicos y de interpretabilidad. El objetivo común fue identificar un subconjunto óptimo de predictores que maximice la capacidad explicativa del modelo sobre la variable dependiente \( c4 \) (carbono estimado), evitando colinealidad y preservando el sentido físico de las relaciones.

\subsection{Selección automática mediante Featurewiz}
El método \textit{Featurewiz} se basa en un enfoque de selección de características guiado por importancia predictiva. El procedimiento combina dos etapas principales: (i) un filtrado inicial por correlación, en el que se eliminan variables altamente colineales (en este caso, con un umbral de \( |r| > 0.70 \)); y (ii) un refinamiento mediante modelos de \textit{Gradient Boosting} que estiman la importancia relativa de cada variable en la predicción del objetivo.  
De esta manera, \textit{Featurewiz} conserva únicamente aquellas variables con una contribución significativa a la mejora del rendimiento predictivo, proporcionando un conjunto compacto y eficiente de predictores.

\subsection{Selección mediante mRMR}
El enfoque \textit{mRMR} (minimum Redundancy - maximum Relevance) selecciona las variables que maximizan su relevancia estadística respecto a la variable objetivo, minimizando al mismo tiempo la redundancia entre ellas. Este método utiliza medidas de información mutua para cuantificar la dependencia no lineal entre las variables.  
En la práctica, el algoritmo mRMR prioriza aquellas variables que aportan información nueva y no redundante sobre el fenómeno modelado (en este caso, la acumulación de carbono), favoreciendo la diversidad informativa frente a la mera fuerza de correlación. Este enfoque permite obtener un conjunto equilibrado de predictores que explican diferentes dimensiones del sistema ecológico.


\subsection{Selección manual basada en criterios estadísticos y conceptuales}
La selección manual de variables se realizó de forma guiada por criterios tanto estadísticos como conceptuales. 
En primer lugar, se evaluó la significancia de cada variable mediante pruebas univariantes (ANOVA y correlaciones), eliminando aquellas sin influencia estadísticamente significativa sobre la variable objetivo. 
Posteriormente, se analizaron las correlaciones entre predictores para reducir la colinealidad, manteniendo únicamente una variable representativa de cada grupo altamente correlacionado. 
Además, se consideraron criterios ecológicos y de interpretación física, asegurando que las variables retenidas representasen aspectos estructurales, edáficos, topográficos, climáticos y espectrales relevantes para el proceso de acumulación de carbono. 
El objetivo fue equilibrar la robustez estadística con la coherencia ecológica, obteniendo un conjunto final de predictores que mantuviera un compromiso entre precisión, interpretabilidad y sentido biogeográfico.


\subsection{Partición y validación}

Para obtener una estimación imparcial del rendimiento y evitar \emph{fugas de información} debidas a la correlación espacial dentro de cada parcela, la partición del conjunto de datos se realiza \textbf{por identificador de parcela} (\texttt{parcela\_id}). Todas las observaciones asociadas a una misma parcela se asignan \emph{íntegramente} a un único subconjunto, de modo que ninguna parcela aparece simultáneamente en entrenamiento y evaluación.
\vspace{0.25em}
\noindent\textbf{Validación interna y control de sesgo temporal.} Sobre el subconjunto de entrenamiento (80\,\%) se aplica \emph{validación cruzada por grupos} utilizando como agrupador los \emph{años transcurridos entre inventarios} (p.\,ej., 15, 16, 17, \dots). Esta estrategia comprueba la \emph{estabilidad} del modelo frente a cambios en el horizonte temporal y reduce el riesgo de sobreajuste específico de un periodo. La selección de hiperparámetros se realiza exclusivamente dentro de esta validación interna; el conjunto de evaluación (20\,\%) permanece \emph{sellado} para la prueba final.

\vspace{0.25em}
\noindent\textbf{Métricas de evaluación.} El rendimiento se informa con dos medidas complementarias:
\begin{itemize}
    \item \textbf{RMSE (Root Mean Squared Error):} raíz del error cuadrático medio entre valores observados y predichos; se expresa en las mismas unidades que la variable objetivo y penaliza con mayor peso los errores grandes. Valores más bajos indican mejor ajuste.
    \item \boldmath\textbf{$R^2$}\unboldmath{} (coeficiente de determinación): proporción de la varianza observada explicada por el modelo (idealmente en $[0,1]$). Valores cercanos a 1 denotan alta capacidad explicativa; puede ser negativo si el modelo es peor que la predicción constante.
\end{itemize}

\vspace{0.25em}
\noindent\textbf{Protocolo de reporte.} Para cada modelo se reportan: (i) el rendimiento medio y la dispersión en la validación cruzada por grupos (entrenamiento), y (ii) el desempeño final en el conjunto de evaluación independiente (20\,\%). Este protocolo garantiza comparabilidad entre modelos, control del sesgo espacial por parcela y verificación explícita de la robustez temporal.




\subsection{Modelos evaluados}
% Familias, tuning (RandomizedSearchCV), por qué ensembles, etc.


A continuación, se detalla el diseño general y las estrategias empleadas para la selección y optimización de modelos.

\subsubsection*{Entrenamiento y optimización}

Se aplicó \textit{RandomizedSearchCV}, una técnica de búsqueda aleatoria de hiperparámetros que evalúa distintas combinaciones utilizando validación cruzada. Este procedimiento permite optimizar el rendimiento de cada modelo sin incurrir en un coste computacional tan elevado como el de una búsqueda exhaustiva.

\medskip

\paragraph{Validación cruzada por grupos.} 


En algunas configuraciones probadas, se emplea la validación cruzada por grupos (\textit{Group k-Fold Cross Validation}). Este método divide el área de estudio en $k$ bloques, basados en una característica común de los datos, asegurando que los datos dentro de cada bloque estén relacionados. El modelo se entrena con $k-1$ bloques y se valida con el bloque restante, repitiendo este proceso $k$ veces.

\medskip

Este enfoque es útil cuando los datos tienen agrupaciones naturales, como por ejemplo, diferentes parcelas o periodos de tiempo. Al mantener los datos relacionados en un mismo bloque, se evita la filtración de información entre los conjuntos de entrenamiento y validación, lo que permite una mejor evaluación de la capacidad de generalización del modelo. Así, se asegura que el modelo no se sobreajuste y sea robusto al ser evaluado en contextos no vistos previamente.


\subsubsection*{Modelos ensemble}

Para mejorar la precisión y robustez, se emplearon diversos métodos de \textit{ensemble learning}, que combinan múltiples modelos base (\textit{base learners}) para generar una predicción agregada. Esta estrategia se inspira en la teoría de la sabiduría colectiva: la combinación de estimaciones independientes tiende a superar a cualquier estimador individual.

\paragraph{Técnicas utilizadas:}
\begin{itemize}
    \item \textbf{Voting y Averaging:} combinan modelos ya entrenados mediante votación mayoritaria o promedio.
    \item \textbf{Bagging y Boosting:} construyen modelos desde cero y los combinan. Bagging reduce la varianza al entrenar modelos en subconjuntos aleatorios; Boosting mejora el sesgo al entrenar secuencialmente, corrigiendo errores anteriores.
    \item \textbf{Stacking:} combina modelos optimizados usando un metamodelo que aprende a integrar sus predicciones.
\end{itemize}

\subsubsection*{Boosting y aprendizaje gradual}

El \textit{boosting} se basa en el aprendizaje secuencial, donde cada nuevo modelo intenta corregir los errores residuales del anterior. Esta técnica permite construir modelos fuertes a partir de modelos débiles, alcanzando gran precisión. Sin embargo, requiere una cuidadosa configuración de hiperparámetros para evitar el sobreajuste.

Entre las implementaciones destacadas se incluyen:

\begin{itemize}
    \item \textbf{XGBoost:} modelo GBM que optimiza rendimiento con gradientes de primer y segundo orden, regularización L1/L2, manejo automático de valores faltantes, y técnicas de generalización como \textit{shrinkage} y \textit{column subsampling}.
    \item \textbf{LightGBM:} algoritmo eficiente para grandes volúmenes de datos, con crecimiento \textit{leaf-wise} y soporte nativo para variables categóricas.
    \item \textbf{AdaBoost:} ajusta modelos simples secuencialmente, enfocando el aprendizaje en observaciones mal clasificadas.
    \item \textbf{CatBoost:} especializado en variables categóricas y robusto frente a datos ruidosos, usando codificación por orden aleatorio.
    \item \textbf{Gradient Boosting Decision Trees (GBDT):} construye árboles secuenciales ajustados a residuos, optimizando mediante descenso por gradiente.
\end{itemize}

\subsubsection*{Bagging}

El \textit{bagging} (Bootstrap Aggregating) entrena múltiples modelos independientes sobre subconjuntos de datos generados por muestreo con reemplazo. Las predicciones se combinan por promedio o votación. Esta técnica reduce la varianza y mejora la estabilidad de modelos inestables.

\begin{itemize}
    \item \textbf{Random Forest:} combina árboles de decisión (CART) con selección aleatoria de características en cada división. Es escalable, robusto a datos faltantes, y menos propenso al sobreajuste.
    \item \textbf{Bagged Decision Trees (BaggedDT):} genera árboles sin poda entrenados en muestras bootstrap. Promedia sus predicciones para reducir la varianza.
\end{itemize}

\subsubsection*{Otros modelos utilizados}

Además de los métodos ensemble, se evaluaron modelos representativos de distintos paradigmas de aprendizaje supervisado:

\begin{itemize}
    \item \textbf{K-Nearest Neighbors (KNN):} modelo basado en instancia que predice a partir de los vecinos más cercanos. Sensible a la escala y a \textit{outliers}.
    \item \textbf{Multi-Layer Perceptron (MLP):} red neuronal con una o más capas ocultas, capaz de modelar relaciones no lineales complejas.
    \item \textbf{Support Vector Regression (SVR):} modelo de márgenes para regresión, con soporte para kernels no lineales.
    \item \textbf{SVM con kernel:} modelo poderoso para clasificación y regresión no lineal, aunque costoso y sensible a hiperparámetros.
    \item \textbf{Bayesian Neural Network:} enfoque probabilístico que estima incertidumbre en las predicciones. Incluye variantes como la \textit{Bayesian Ridge Regression}.
    \item \textbf{Naive Bayes:} clasificador probabilístico rápido y simple, útil en texto y alta dimensionalidad. Se evaluaron variantes:
        \begin{itemize}
            \item \textit{Gaussian Naive Bayes:} para datos continuos.
            \item \textit{Multinomial Naive Bayes:} para conteos y texto.
            \item \textit{Bernoulli Naive Bayes:} para variables binarias.
        \end{itemize}
\end{itemize}

\subsubsection*{Comparación y justificación de modelos}

La evaluación de múltiples modelos responde a la necesidad de identificar no solo el de mejor rendimiento, sino también el más adecuado según la naturaleza del problema y los datos disponibles. Se compararon algoritmos lineales, no lineales, basados en vecinos, redes neuronales, modelos probabilísticos y diferentes técnicas de \textit{ensemble}.  Vemos un resumen de los modelos aplicados en la tabla \ref{tab:modelos}.


\begin{table}[H]\small
\centering
\resizebox{\textwidth}{!}{%
\begin{tabular}{|p{3.2cm}|p{2.8cm}|p{5.2cm}|p{4.2cm}|}
\hline
\textbf{Modelo} & \textbf{Tipo / Técnica} & \textbf{Características destacadas} & \textbf{Observaciones} \\
\hline
\textbf{Random Forest} & Bagging (Árboles) & Uso de bootstrap, selección aleatoria de atributos, reducción de varianza & Robusto y escalable; menor interpretabilidad \\
\hline
\textbf{Bagged Decision Trees (BaggedDT)} & Bagging & Árboles sin poda, entrenados en paralelo sobre muestras con reemplazo & Preciso pero costoso computacionalmente \\
\hline
\textbf{XGBoost} & Boosting (GBM) & Regularización L1/L2, manejo de valores faltantes, poda anticipada & Alto rendimiento, sensible a hiperparámetros \\
\hline
\textbf{LightGBM} & Boosting (Leaf-wise) & Crecimiento hoja a hoja, eficiente en grandes volúmenes & Rápido y preciso; riesgo de sobreajuste \\
\hline
\textbf{AdaBoost} & Boosting (Stumps) & Aumenta peso de errores, pondera modelos por precisión & Sencillo y efectivo con datos limpios \\
\hline
\textbf{CatBoost} & Boosting especializado & Codificación avanzada de variables categóricas, robustez a ruido & Ideal para datos heterogéneos \\
\hline
\textbf{Gradient Boosting Decision Trees (GBDT)} & Boosting & Árboles secuenciales ajustados a residuos & Buen rendimiento; mayor coste de entrenamiento \\
\hline
\textbf{K-Nearest Neighbors (KNN)} & Basado en instancia & No requiere entrenamiento, predice por proximidad & Sensible a escala y outliers \\
\hline
\textbf{Multi-Layer Perceptron (MLP)} & Red neuronal & Modela relaciones no lineales complejas & Requiere normalización y regularización \\
\hline
\textbf{Support Vector Regression (SVR)} & Kernel y márgenes & Predicción dentro de tolerancia $\varepsilon$, uso de kernels no lineales & Robusto; elevado coste computacional \\
\hline
\textbf{SVM con kernel} & SVM no lineal & Maximiza margen, admite distintos kernels (RBF, polinomial, etc.) & Alta precisión; sensible a hiperparámetros \\
\hline
\textbf{Bayesian Neural Network / Ridge Regression} & Probabilístico / Bayesiano & Predicción con incertidumbre, estimación automática de hiperparámetros & Útil para inferencia y regularización \\
\hline
\textbf{Naive Bayes (Gaussian, Multinomial, Bernoulli)} & Probabilístico & Asume independencia condicional, rápido y simple & Eficaz en texto y alta dimensionalidad \\
\hline
\end{tabular}
}
\caption{Resumen de modelos de aprendizaje supervisado aplicados}
\label{tab:modelos}
\end{table}


\medskip



\newpage
% sections/05_desarrollo_modelo.tex
\section{Implementación de los modelos}

% Este apartado es donde profundizas en la parte práctica de tu IA.
% Podrías incluir:
% - Preprocesamiento de datos avanzado: escalado, codificación de variables categóricas, ingeniería de características.
% - División del dataset: explicar cómo se dividieron los datos en conjuntos de entrenamiento, validación y prueba.
% - Arquitectura final del modelo: si es una red neuronal, un diagrama o una descripción detallada de las capas, neuronas y funciones de activación. Si es un modelo de árbol, cómo se configuró (número de estimadores, profundidad máxima, etc.).
% - Proceso de entrenamiento: cómo se entrenó el modelo, cuánto tiempo tomó, hardware utilizado (GPU/CPU).
% - Optimización de hiperparámetros: si utilizaste Grid Search, Random Search, o librerías como Optuna/Hyperopt.
% - Manejo de sobreajuste (overfitting) o subajuste (underfitting).
% - Ejemplos de código o pseudocódigo (si son relevantes y no demasiado extensos para el documento).

% - Herramientas y librerías utilizadas (Python, scikit-learn, XGBoost, etc.).
% - Mención opcional del entorno de ejecución (entorno local, servidor, etc.).
% - Tiempo de entrenamiento aproximado, tamaño del dataset.

El desarrollo y evaluación de los modelos predictivos se realizó íntegramente en \textbf{Python} y el entorno de ejecución fue local, en un equipo con procesador Intel Core i7 y 32 GB de RAM, lo que permitió realizar experimentos de forma eficiente con un conjunto de datos de tamaño considerable (\textasciitilde{}80.000 muestras).

\medskip

El preprocesamiento de datos se llevó a cabo mediante la librería \texttt{scikit-learn}, utilizando \texttt{Pipeline} y \texttt{ColumnTransformer} para combinar transformaciones numéricas y categóricas. En particular, las variables numéricas se imputaron con la mediana y se escalaron con \texttt{StandardScaler}, mientras que las variables categóricas se trataron mediante imputación por moda y codificación \textit{one-hot}. La función objetivo a predecir fue el carbono total (\texttt{CZ}) acumulado en cada parcela.

\medskip

Se implementaron y optimizaron diversos modelos de regresión supervisada, incluyendo:

\begin{itemize}
    \item \textbf{Modelos basados en árboles:} \texttt{RandomForestRegressor}, \texttt{XGBoost}, \texttt{LightGBM}, \texttt{CatBoost}, \texttt{GradientBoosting}, \texttt{AdaBoost} y \texttt{Bagging}.
    \item \textbf{Modelos basados en instancias:} \texttt{KNeighborsRegressor}.
    \item \textbf{Modelos de redes neuronales:} \texttt{MLPRegressor}.
    \item \textbf{Modelos de soporte vectorial:} \texttt{SVR}.
    \item \textbf{Modelos probabilísticos:} \texttt{BayesianRidge}.
\end{itemize}

La optimización de hiperparámetros se realizó mediante \texttt{RandomizedSearchCV}, con validación cruzada de 5 particiones y búsqueda en espacios definidos manualmente para cada modelo. Los modelos fueron evaluados en términos de \textbf{\(R^2\)} y \textbf{RMSE}, tanto en el conjunto de entrenamiento como en el de prueba.

\medskip

Con el objetivo de mejorar el rendimiento predictivo, se evaluaron además varias configuraciones de \texttt{StackingRegressor}, combinando distintos subconjuntos de modelos base (previamente entrenados) con diversos meta-modelos (\texttt{LinearRegression}, \texttt{Ridge}, \texttt{GradientBoosting}, \texttt{SVR}, \texttt{MLP}, entre otros). Estas combinaciones permitieron comparar sinergias entre modelos complementarios.

\medskip

El tiempo de entrenamiento varió según el modelo y la configuración de hiperparámetros. Gracias al uso de \texttt{n\_jobs=-1} se aprovechó el paralelismo multinúcleo para acelerar la optimización.

\newpage
% sections/06_resultados.tex
\section{Resultados}
Presentar los resultados obtenidos al aplicar el modelo a los datos de entrada. Incluir gráficos y tablas que ayuden a ilustrar el rendimiento del modelo.
% Este es el corazón de tu sección de resultados.
% Incluye:
% - Métricas de rendimiento: Presenta los valores numéricos de las métricas que definiste en la sección de Metodología (RMSE, MAE, R², etc.) para los conjuntos de entrenamiento y prueba.
% - Gráficos de rendimiento:
%   - Curvas de aprendizaje (pérdida vs. época para entrenamiento y validación).
%   - Gráficos de dispersión de valores predichos vs. valores reales.
%   - Histogramas de errores o residuos.
%   - Si aplicable, mapas de calor o visualizaciones de los resultados georreferenciados.
% - Tablas:
%   - Resumen de las métricas clave.
%   - Comparación del rendimiento de diferentes modelos si probaste varios.
% - Ejemplos de predicciones: Puedes mostrar algunos ejemplos concretos de predicciones del modelo.

% Ejemplo de inclusión de imagen:
% \begin{figure}[H]
%     \centering
%     \includegraphics[width=0.8\textwidth]{images/curva_perdida.png} % Asegúrate de la ruta correcta
%     \caption{Curva de pérdida durante el entrenamiento y la validación.}
%     \label{fig:curva_perdida}
% \end{figure}

% Ejemplo de tabla:
% \begin{table}[H]
%     \centering
%     \caption{Métricas de rendimiento del modelo final.}
%     \label{tab:performance_metrics}
%     \begin{tabular}{lcc}
%         \toprule
%         Métrica & Conjunto de Entrenamiento & Conjunto de Prueba \\
%         \midrule
%         RMSE    & \SI{0.123}{tCO2} & \SI{0.156}{tCO2} \\
%         MAE     & \SI{0.098}{tCO2} & \SI{0.112}{tCO2} \\
%         R$^2$   & 0.95            & 0.88             \\
%         \bottomrule
%     \end{tabular}
% \end{table}


\section*{Proceso de entrenamiento y validación del modelo}

El proceso de entrenamiento se estructuró en varias fases orientadas a optimizar tanto la selección de variables predictoras como la robustez del modelo final. En primer lugar, se llevó a cabo una etapa de \textbf{selección de variables}, en la que se evaluaron distintos subconjuntos de características definidos por bloques temáticos con significado ecológico y funcional. Para esta tarea se adoptó un enfoque sistemático basado en la comparación del desempeño predictivo de las distintas combinaciones mediante el algoritmo \texttt{CatBoost}, seleccionado tras pruebas preliminares que mostraron su alta capacidad de ajuste y estabilidad frente a la heterogeneidad de los datos. En todas las configuraciones se mantuvo constante la variable objetivo (carbono capturado) y los parámetros del modelo, de modo que las variaciones en el coeficiente de determinación ($R^2$) y el error cuadrático medio (RMSE) reflejaran exclusivamente la contribución informativa de cada bloque. 

Las configuraciones analizadas incorporaron progresivamente variables relacionadas con las características de la especie, las propiedades edáficas, el terreno, las condiciones climáticas y los índices de vegetación. A partir de los resultados obtenidos, se identificaron los bloques con mayor aporte marginal al rendimiento del modelo, priorizando aquellos cuya inclusión mejoró consistentemente el $R^2$ sin aumentar de forma significativa la complejidad o redundancia del conjunto de predictores.

En una segunda fase, se procedió al \textbf{entrenamiento comparativo de modelos}, implementando un conjunto de algoritmos de aprendizaje supervisado con el fin de contrastar su capacidad predictiva. Entre los estimadores evaluados se incluyeron \texttt{LightGBM}, \texttt{Random Forest}, \texttt{XGBoost}, \texttt{CatBoost}, \texttt{Gradient Boosting}, \texttt{Bagging Regressor}, \texttt{AdaBoost}, \texttt{KNN}, \texttt{MLP}, \texttt{SVR} y \texttt{Bayesian Ridge}. Cada modelo fue entrenado bajo las mismas condiciones experimentales, utilizando las configuraciones de variables seleccionadas en la fase anterior. Esta comparación permitió identificar los algoritmos con mejor ajuste global y menor error de predicción, destacando de nuevo el desempeño de \texttt{CatBoost}.

Posteriormente, se implementó una estrategia de \textbf{stacking}, combinando las predicciones de los modelos individuales mediante un metamodelo de segundo nivel, con el objetivo de aprovechar la complementariedad entre los distintos enfoques y mejorar la capacidad de generalización.

Finalmente, el modelo seleccionado se \textbf{reentrenó con validación cruzada estratificada por grupos}, definidos según el periodo temporal de la observación. Este esquema de validación cruzada por grupos permitió evaluar la estabilidad del modelo frente a periodos no observados durante el entrenamiento, garantizando así su capacidad de generalización temporal y la fiabilidad de las predicciones en escenarios futuros.
 

\subsection{Elección de variables}

Para la selección de variables se adoptó un enfoque sistemático basado en la evaluación del desempeño predictivo de distintos subconjuntos de características, definidos por bloques temáticos con significado ecológico y funcional. Cada combinación de variables se entrenó mediante el algoritmo \texttt{CatBoost}, manteniendo constantes la variable objetivo (carbono capturado) y los parámetros de modelado, de forma que las variaciones en el coeficiente de determinación ($R^2$) y el error cuadrático medio (RMSE) reflejaran exclusivamente la contribución informativa de cada bloque. Las configuraciones incluyeron progresivamente grupos de variables relacionadas con las características de la especie, las propiedades edáficas, el terreno, las condiciones climáticas (temperatura y precipitación) y los índices de vegetación. A partir de la comparación de los resultados, se identificaron los bloques con mayor aporte marginal al rendimiento del modelo, priorizando aquellos cuya incorporación mejoró consistentemente el $R^2$ sin incrementar de forma significativa la complejidad o redundancia del conjunto de predictores.

El análisis comparativo de configuraciones de variables mediante el modelo \texttt{CatBoost} permitió estimar la contribución marginal de cada bloque al poder predictivo del modelo (medido a través de $R^2$ y RMSE). A partir de los resultados, se establecieron las siguientes conclusiones:

\begin{enumerate}
    \item \textbf{Bloque especies.} Incluye variables sobre estado, forma, tratamiento, origen, distribución, composición y fracción de cabida cubierta. Es el bloque con mayor impacto, con incrementos de hasta $+0{,}0054$ en $R^2$. Su efecto es consistente en todas las combinaciones, por lo que se considera esencial.

    \item \textbf{Bloque temperaturas y precipitaciones.} Aporta una mejora sistemática y estable, especialmente en presencia de variables edáficas o de terreno. Los incrementos típicos oscilan entre $+0{,}002$ y $+0{,}005$ en $R^2$. Se recomienda su inclusión por su bajo coste computacional y su valor informativo adicional.

    \item \textbf{Bloque terreno.} Comprende tipo de suelo, rocosidad, orientación, elevación y pendiente. Su efecto es moderado (variaciones de $\pm0{,}001$ en $R^2$) y, aunque aporta cierta estabilidad al modelo, su relevancia es secundaria. Puede incorporarse cuando la disponibilidad de datos lo permita.

    \item \textbf{Bloque índices de vegetación.} (NDII, GNDVI) Produce una ganancia leve y no siempre significativa ($\Delta R^2 \approx 0{,}002$--$0{,}004$), pero tiende a mejorar ligeramente el ajuste medio. Su inclusión es recomendable si los datos están disponibles sin aumentar el coste del pipeline.

    \item \textbf{Bloque soil.} (erosividad, textura, materia orgánica, combustibilidad) Presenta la menor rentabilidad predictiva. Aunque puede mejorar en combinación con variables climáticas, su efecto es inferior al del bloque especies. Se recomienda mantenerlo solo por motivos interpretativos o contextuales.
\end{enumerate}

Aunque el bloque \textit{terreno} no mostró inicialmente una contribución destacada al rendimiento global del modelo, se realizó un análisis adicional para evaluar el efecto individual de cada una de sus variables. Los resultados indicaron que la variable \textit{elevación} presenta una influencia positiva sobre la capacidad predictiva del modelo, mejorando ligeramente las métricas de ajuste. En concreto, la inclusión de \textit{elevación} elevó el coeficiente de determinación a $R^2 = 0{,}8772$ y redujo el error cuadrático medio a $\text{RMSE} = 14{,}36$, frente a los valores obtenidos sin dicha variable ($R^2 = 0{,}8766$, $\text{RMSE} = 14{,}40$). Este incremento, aunque moderado, resulta relevante al tratarse de una única variable adicional (41 predictores frente a 40), lo que sugiere que la altitud puede captar gradientes ambientales asociados a la variabilidad en la captura de carbono que no están plenamente representados por los demás bloques.


\noindent \textbf{Recomendación práctica:} la configuración óptima corresponde a fijas + especies + temperaturas/precipitaciones + índices, con $R^2=0{,}8766$ y 40 variables. Como alternativa ligera, fijas + especies + terreno, $R^2=0{,}8745$ y fijas + especies + índices, $R^2=0{,}8744$ ofrecen un equilibrio adecuado entre rendimiento y complejidad. En términos de prioridad, los bloques deben incluirse en el siguiente orden: \textbf{especies} $\gg$ \textbf{temperaturas y precipitaciones} $>$ \textbf{terreno} $\gtrsim$ \textbf{índices} $>$ \textbf{soil}.

\noindent En conjunto, las variables fijas explican la mayor parte de la varianza, mientras que los bloques adicionales permiten afinar la estimación del carbono capturado. Dado el alto rendimiento base del modelo \texttt{CatBoost}, las mejoras marginales son pequeñas pero consistentes, destacando el valor de las variables de especie y climáticas como componentes clave en la predicción.

Finalmente, se evaluó la sustitución del bloque de variables climáticas (temperaturas y precipitaciones) por el Índice de Martonne, una métrica sintética que integra de forma conjunta la información térmica e hídrica en un solo parámetro. Esta modificación permitió reducir significativamente la dimensionalidad del bloque climático (de ocho variables a una), manteniendo un desempeño prácticamente equivalente. El modelo resultante (\texttt{CatBoost}, 33 variables) alcanzó un $R^2 = 0{,}8721$ y un RMSE de $14{,}66$, valores muy similares a los obtenidos con las variables climáticas explícitas ($R^2 = 0{,8766}$, RMSE $=14{,}40$). Esta equivalencia, junto con la reducción en complejidad y carga computacional, respalda el uso del Índice de Martonne como sustituto eficiente del conjunto de variables de temperatura y precipitación en la modelización de la captura de carbono. 
\newpage
% sections/07_discusion.tex
\section{Discusión}
Interpretar los resultados obtenidos, comparándolos con investigaciones previas. Discutir las limitaciones del modelo y las posibles áreas de mejora.
% Aquí es donde analizas lo que significan tus resultados.
% - Interpretación de los resultados: ¿Qué significan los valores de tus métricas? ¿El modelo es bueno, aceptable, o tiene problemas?
% - Comparación con la literatura: ¿Cómo se comparan tus resultados con estudios similares en el campo? ¿Superas, igualas o quedas por debajo de lo esperado? ¿Por qué?
% - Implicaciones: ¿Qué implicaciones tienen tus hallazgos para la predicción de créditos de carbono, la gestión de dehesas, o la toma de decisiones empresariales?
% - Limitaciones del estudio: ¿Qué aspectos no pudo abordar tu modelo o tu metodología? (ej. tamaño del dataset, calidad de los datos, alcance geográfico, factores no considerados).
% - Áreas de mejora: ¿Cómo se podría mejorar el modelo o la investigación en el futuro? (ej. más datos, diferentes modelos de IA, considerar otras variables, integrar datos de otras fuentes).
% - Relevancia práctica: ¿Cómo puede ser utilizado este modelo en el mundo real?
\newpage
% sections/08_conclusiones.tex
\section{Conclusiones}
Recapitular las conclusiones más importantes del estudio. Resaltar la relevancia del modelo desarrollado y su aplicación en proyectos de forestación y reforestación.
% Este es un resumen conciso de tus hallazgos clave.
% - Reafirma el objetivo principal y si se logró.
% - Destaca los resultados más significativos.
% - Reitera la contribución principal de tu trabajo.
% - Enfatiza la importancia y el impacto potencial de tu modelo.
% - Evita introducir nueva información aquí.
\newpage
% sections/09_recomendaciones.tex
\section{Recomendaciones para Futuras Investigaciones}
Sugerir áreas que podrían beneficiarse de estudios adicionales o mejoras en la metodología.
% Aquí puedes ser más específico que en la discusión.
% - Sugerir nuevas variables a incluir.
% - Probar otros modelos de IA o arquitecturas.
% - Expandir la base de datos o el área de estudio.
% - Investigar la incertidumbre o robustez del modelo.
% - Desarrollar una interfaz de usuario para el modelo.
% - Explorar la integración con otras tecnologías (ej. blockchain para el registro de créditos).
\newpage
% sections/10_agradecimientos.tex
\section*{Agradecimientos} % Usa \section* para que no numere esta sección

Investigación financiada por la subvención \textbf{TSI-100933-2023-1} de la \textbf{Convocatoria de Cátedras Universidad-Empresa (Cátedras ENIA 2022), Ministerio de Transformación Digital y Función Pública de España}, y el \textbf{Plan de Recuperación y Resiliencia de la UE} (\textit{NextGenerationEU/PRTR}).
\newpage
\section{Anexos}
\small

\subsection*{Anexo: Origen y cálculo de las variables \textit{CA} y \textit{CR}}\label{sec:Carbono}

Las variables \textbf{CA} (carbono arbóreo) y \textbf{CR} (carbono radical) incluidas en la base de datos del \textit{Inventario Forestal Nacional} (IFN4) derivan de las ecuaciones alométricas de biomasa desarrolladas por el \textit{Instituto Nacional de Investigación y Tecnología Agraria y Alimentaria} (INIA), en particular por \textit{Gregorio Montero} y \textit{Ricardo Ruiz-Peinado} \cite{montero2009, ruizpeinado2011}. Estas ecuaciones fueron elaboradas a partir de datos de campo obtenidos mediante talas y pesadas directas de árboles de distintas especies representativas de la flora forestal española.

\medskip

Cada ecuación estima la biomasa seca (en kilogramos) de los diferentes componentes del árbol en función del diámetro normal (\textit{D}, en cm, medido a 1,3 m del suelo) y la altura total (\textit{H}, en m). Para cada especie o grupo de especies similares se dispone de ecuaciones específicas de la forma:

\[
W_i = a_i \cdot D^{b_i} \cdot H^{c_i}
\]

donde $W_i$ representa la biomasa del componente $i$ (fuste, corteza, ramas, hojas, raíces, etc.), y $a_i$, $b_i$ y $c_i$ son coeficientes empíricos obtenidos mediante regresión no lineal. En los casos en que una especie no dispone de ecuación propia, se utiliza la de otra especie considerada análoga por similitud morfológica o ecológica.

\medskip

Los componentes de biomasa definidos en el IFN4 incluyen \cite{miteco_ifn4_manual}:

\begin{itemize}
    \item $W_s$: biomasa del fuste (kg),
    \item $W_c$: biomasa de la corteza del fuste (kg),
    \item $W_{b7}$: biomasa de ramas mayores de 7 cm de diámetro (kg),
    \item $W_{b2-7}$: biomasa de ramas entre 2 y 7 cm de diámetro (kg),
    \item $W_{b0.5-2}$: biomasa de ramas entre 0,5 y 2 cm de diámetro (kg),
    \item $W_t$: biomasa de ramas menores de 0,5 cm de diámetro (kg),
    \item $W_h$: biomasa de hojas (kg),
    \item $W_{db}$: biomasa de ramas muertas (kg),
    \item $W_T = W_s + W_c + W_{b7} + W_{b2-7} + W_{b0.5-2} + W_t + W_h$: biomasa aérea total (kg),
    \item $W_r$: biomasa radical (raíces, kg).
\end{itemize}

\medskip

\noindent
A partir de estas ecuaciones, el cálculo de biomasa y carbono en el IFN4 se realiza de la siguiente forma:

\begin{enumerate}
    \item \textbf{Biomasa por árbol (kg):} en la tabla \texttt{Mayores\_exs} se incluyen las medidas de diámetro y altura de cada pie. Aplicando las ecuaciones alométricas correspondientes se obtiene la biomasa aérea ($W_T$) y radical ($W_r$) para cada árbol.

    \item \textbf{Conversión a carbono (kg):} se aplica un factor de conversión estándar de 0.5, según las directrices del IPCC \cite{ipcc2006}, de forma que:
    \[
    \text{CA} = 0.5 \times W_T, \quad \text{CR} = 0.5 \times W_r
    \]

    \item \textbf{Expansión a valores por hectárea (t/ha):} los valores por árbol se convierten a toneladas por hectárea mediante un \textit{factor de expansión} (\textit{Fac}), que refleja la densidad de árboles por unidad de superficie dentro de cada clase diamétrica y especie. Este factor se calcula en función del número de pies inventariados y la superficie de muestreo, permitiendo expresar los resultados en términos comparables de biomasa o carbono por hectárea.

    \item \textbf{Agregación por clases diamétricas y especie:} finalmente, en la tabla \texttt{Parcelas\_exs} se agrupan los valores por parcela, especie y clase diamétrica (CD), sumando las contribuciones individuales ya expandidas. El resultado son los valores medios de biomasa y carbono por hectárea (\textit{t/ha}) para cada combinación de parcela y especie.
\end{enumerate}

\medskip

El mismo procedimiento se aplica tanto a la biomasa aérea (para obtener \textit{CA}) como a la biomasa radical (para \textit{CR}). De esta forma, \textbf{CA y CR representan el carbono almacenado en la biomasa viva, aérea y subterránea respectivamente, expresado en toneladas de carbono por hectárea (t/ha)}.

\medskip

Este enfoque metodológico se ajusta a las recomendaciones del \textit{IPCC Guidelines for National Greenhouse Gas Inventories} \cite{ipcc2006}, garantizando la coherencia con los métodos de reporte de carbono a nivel internacional y facilitando la comparación de los resultados con otros estudios y marcos regulatorios.




\subsection*{Anexo: Códigos de provincias de España}\label{sec:provincias}

\begin{table}[H]
\centering
\renewcommand{\arraystretch}{1.3}
\begin{tabular}{|c|l||c|l|}
\hline
\textbf{Código} & \textbf{Provincia} & \textbf{Código} & \textbf{Provincia} \\
\hline
01 & Álava             & 27 & Lugo \\
02 & Albacete          & 28 & Madrid \\
03 & Alicante          & 29 & Málaga \\
04 & Almería           & 30 & Murcia \\
05 & Ávila             & 31 & Navarra \\
06 & Badajoz           & 32 & Ourense \\
07 & Baleares (Illes)  & 33 & Asturias \\
08 & Barcelona         & 34 & Palencia \\
09 & Burgos            & 35 & Palmas (Las) \\
10 & Cáceres           & 36 & Pontevedra \\
11 & Cádiz             & 37 & Salamanca \\
12 & Castellón         & 38 & Santa Cruz de Tenerife \\
13 & Ciudad Real       & 39 & Cantabria \\
14 & Córdoba           & 40 & Segovia \\
15 & Coruña (A)        & 41 & Sevilla \\
16 & Cuenca            & 42 & Soria \\
17 & Girona            & 43 & Tarragona \\
18 & Granada           & 44 & Teruel \\
19 & Guadalajara       & 45 & Toledo \\
20 & Guipúzcoa         & 46 & Valencia \\
21 & Huelva            & 47 & Valladolid \\
22 & Huesca            & 48 & Vizcaya \\
23 & Jaén              & 49 & Zamora \\
24 & León              & 50 & Zaragoza \\
25 & Lleida            & 51 & Ceuta \\
26 & La Rioja          & 52 & Melilla \\
\hline
\end{tabular}
\caption{Relación de códigos numéricos utilizados para las provincias españolas.}
\label{anexo:provincias}
\end{table}


\subsection*{Anexo: Rocosidad}\label{sec:Rocosid}
Se considerará el conjunto de la parcela clasificando la rocosidad según la siguiente codificación:

\begin{enumerate}
    \item \textbf{Sin pedregosidad}: la superficie de la parcela está completamente cubierta de vegetación.
    \item \textbf{Poco pedregoso}: cuando la superficie de la parcela cubierta por rocas coherentes es menor del 25\,\%.
    \item \textbf{Pedregoso}: cuando la superficie rocosa está comprendida entre el 25\,\% y el 50\,\%.
    \item \textbf{Muy pedregoso}: cuando la superficie rocosa se sitúa entre el 50\,\% y el 75\,\%.
    \item \textbf{Roquedo}: cuando la superficie de rocas es mayor del 75\,\%. En este caso, no se tomará ningún dato adicional correspondiente a suelos.
\end{enumerate}





\subsection*{Anexo: Tipo de Suelo}\label{sec:TipSuelo}

Se utilizará la siguiente codificación para el tipo de suelo, diferenciando tres variables:

\vspace{1em}
\noindent
\textbf{Tipo de suelo (I):} \textbf{Presencia de sales, yesos o hidromorfía}

\begin{enumerate}
    \item \textbf{No se observan sales, yesos ni procesos de fidromorfía.}
    \item \textbf{Suelo salino.} Si presenta al menos dos de las siguientes características:
    \begin{itemize}
        \item Presencia de eflorescencias en la superficie o a distintas profundidades.
        \item Existencia de plantas halófitas.
        \item Zonas llanas o endorreicas con climas secos que provocan gran evaporación.
    \end{itemize}
    
    
    \item \textbf{Suelo yesífero.} Si presenta alguna de las siguientes características:
    \begin{itemize}
        \item Presencia de materia yesífera en superficie o a distintas profundidades.
        \item Existencia de plantas gipsófilas.
    \end{itemize}
    
    
    \item \textbf{Suelo hidromorfo.} Si el suelo presenta síntomas de hidromorfía acusada, cumpliendo al menos dos de las siguientes:
    \begin{itemize}
        \item Zona encharcada permanente o casi permanentemente de forma natural.
        \item Zona llana o endorreica con climas húmedos.
        \item Grietas en verano si no hay encharcamiento.
        \item Presencia de vegetación indicadora de hidromorfismo.
    \end{itemize}
\end{enumerate}

Identificandose las siguientes:
\begin{itemize}
    \item Formaciones vegetales indicadoras de hidromorfía:
    \begin{itemize}
        \item Ribereñas: \textit{saucedas, mimbreras, alisedas}.
        \item Brezales con \textit{Erica ciliaris, Erica tetralix}.
        \item Turberas arboladas (excepto Cornisa Cantábrica y Pirineos).
        \item Turberas de montaña con \textit{Sphagnum, Erica tetralix}.
        \item Cervunales con \textit{Nardus stricta}.
        \item Carrizales y espadañares (\textit{Phragmites, Tipha, Cladium}).
        \item Juncales (\textit{Scirpus, Juncus}).
        \item Pastizales con cárices (\textit{Carex spp.}).
        \item Marismas.
    \end{itemize}
    \item Formaciones vegetales gipsófilas:
    \begin{itemize}
        \item Aznallar: matorral de \textit{Ononis tridentata}.
        \item Tomillares gipsófilos con:
        \begin{itemize}
            \item \textit{Lepidium subulatum}
            \item \textit{Gypsophila spp.}
            \item \textit{Matthiola fruticulosa}
        \end{itemize}
    \end{itemize}
    \item   Formaciones vegetales indicadoras de suelos salinos:
    \begin{itemize}
        \item Salicorniales: matas leñosas crasas (Salicornia, Arthrocnemum, Halozylon).
        \item Bosques halófitos del género \textit{Tamarix}.
        \item Saladar o sosar: predominio de \textit{Suaeda vera}.
        \item Saladar blanco: predominio de \textit{Atriplex halimus}.
    \end{itemize}
\end{itemize}

    
\vspace{1em}
\noindent
\textbf{Tipo de suelo (II y III):} \textbf{Composición del suelo (calizo o silíceo)}

\begin{enumerate}
    \item \textbf{Suelo calizo.} Más del 50\,\% de la vertical del perfil da efervescencia con ácido clorhídrico.
    
    \begin{itemize}
        \item \textbf{Moderadamente básico:} pH en superficie ≤ 8.5.
        \item \textbf{Fuertemente básico:} pH en superficie > 8.5.
    \end{itemize}
    
    \item \textbf{Suelo silíceo.} Menos del 50\,\% de la vertical del perfil da efervescencia.
    
    \begin{itemize}
        \item \textbf{Moderadamente ácido:} pH ≥ 5.5.
        \item \textbf{Fuertemente ácido:} pH < 5.5.
    \end{itemize}
\end{enumerate}

\subsection*{Anexo: Manifestaciones Erosivas}\label{sec:ManERo}

Se observará la parcela y sus alrededores hasta una distancia de 60 metros desde el centro, y se codificará la existencia de manifestaciones erosivas según la siguiente clave:

\begin{enumerate}
    \item \textbf{No hay ninguna manifestación.}
    
    \item \textbf{Cuellos de raíces al descubierto:} los cuellos de las raíces están visibles, con acumulación de residuos aguas arriba de los tallos y obstáculos, así como abundancia superficial de piedras.
    
    \item \textbf{Presencia de regueros:} canales paralelos de erosión con una profundidad máxima de un palmo (aproximadamente 20 cm).
    
    \item \textbf{Cárcavas y barrancos en V:} erosión lineal más profunda que los regueros, con forma de ``V''.
    
    \item \textbf{Cárcavas y barrancos en U:} erosión avanzada con formas suavizadas y amplias en ``U''.
    
    \item \textbf{Deslizamientos del terreno:} desplazamientos de masas de tierra, ladera o materiales del suelo.
\end{enumerate}

\subsection*{Anexo: Distribución Espacial}\label{sec:disEsp}

La disposición de la vegetación en el espacio se clasificará según la siguiente codificación:

\begin{enumerate}
    \item \textbf{Uniforme.} Cuando el estrato arbóreo presenta continuidad en el espacio.

    \item \textbf{Diseminada en bosquetes aislados.} Cuando la masa arbórea se encuentra dividida en porciones que tienen una superficie inferior a 0,5 ha.

    \item \textbf{Diseminada en individuos aislados.} Cuando se trata de dehesas.

    \item[9.] \textbf{Otras o no se sabe.} En caso diferente a los anteriores o si se desconoce el dato exacto.
\end{enumerate}


\subsection*{Anexo: Composición Específica}\label{anexo:compesp}

En función de las especies presentes:

\begin{enumerate}
    \item \textbf{Masas homogéneas o puras}. Masas monoespecíficas con una única especie arbórea. La normativa española precisa que una masa es monoespecífica o pura cuando al menos el 90\% de los pies pertenecen a la misma especie.
    
    \item \textbf{Masas heterogéneas o mezcladas pie a pie}. Masas de diferentes especies que se juntan o bien se entremezclan por golpes o grupos, siempre que tengan una altura similar.
    
    \item \textbf{Masas heterogéneas o mezcladas con subpiso}. Las dos o más especies mezcladas, cuando alcancen el estado adulto y la estabilidad, presentarán alturas diferentes.
    
    \item[9.] \textbf{Otras o no se sabe}. En caso diferente a los anteriores o desconocer el dato exacto.
\end{enumerate}



\subsection*{Anexo: Textura del Suelo}\label{sec:textura}

Se clasificará en función de la siguiente codificación:

\begin{enumerate}
    \item \textbf{Suelo arenoso.} Si los cilindros se deshacen sin apenas formarse.
    \item \textbf{Suelo franco.} Es posible hacer cilindros gruesos pero no delgados.
    \item \textbf{Suelo arcilloso.} Se consiguen cilindros de unos 5 mm de diámetro.
\end{enumerate}

\subsection*{Anexo: Nivel de usos del suelo}\label{sec:nivel1}

\begin{enumerate}
    \item \textbf{Monte.} Toda superficie en la que vegetan especies arbóreas, arbustivas, de matorral o herbáceas, ya sea espontáneamente o procedan de siembra o plantación, siempre que no sean características de cultivo agrícola o fueran objeto del mismo.
    \item \textbf{Agrícola.} Territorio o ecosistema poblado con siembras o plantaciones de herbáceas y/o leñosas, anuales o plurianuales que se laborea con una fuerte intervención humana, puede estar poblado por especies forestales de fruto (flor, hojas o en el futuro biomasa) siempre que la intervención humana sea importante. Incluye las dehesas, montes huecos o montes adehesados de base cultivo, siempre que la fracción de cabida cubierta de los árboles sea inferior al 5\%.
    \item \textbf{Artificial.} Territorio o ecosistemas dominado por edificios, parques urbanos (aunque estén poblados de árboles), viveros fuera de los montes (aunque sean de especies forestales), carreteras (salvo las vías de servicio de los montes) u otras construcciones humanas que tengan superficies continuas.
    \item \textbf{Humedal.} Lo constituyen las lagunas, charcas, zonas húmedas, marismas y corrientes discontinuas de agua en las que, al menos durante 6 meses del año, esté presente dicho líquido.
    \item \textbf{Agua.} Es la parte de la tierra constituida por ríos, lagos, embalses, canales o estanques con superficies continuas de más de 0.26 ha y con agua prácticamente todo el año.
\end{enumerate}


\subsection*{Anexo: Nivel morfoestructural}\label{sec:nivel2}
Para el nivel de usos del suelo Monte se definirán los siguientes niveles morfoestructurales.

\begin{enumerate}
    \item \textbf{Monte arbolado.} Territorio o ecosistema con especies forestales arbóreas como manifestación vegetal de estructura vertical dominante y con una fracción de cabida cubierta igual o superior al 20\%; incluye dehesas con base cultivo o pastizal con labores siempre que la fracción arbolada supere el 20\%, y excluye terrenos con fuerte intervención humana para obtener frutos, hojas, flores o varas.
    
    \item \textbf{Monte arbolado ralo.} Terreno de uso forestal con especies arbóreas forestales dominantes y fracción de cabida cubierta entre el 10\% y 20\% (incluido el 10\%, excluido el 20\%); también aplica a terrenos con matorral o pastizal natural como dominantes, pero con presencia importante de árboles forestales, incluyendo dehesas de base de cultivo.
    
    \item \textbf{Monte temporalmente desarbolado.} Terreno que fue monte arbolado recientemente y que casi con seguridad volverá a estar cubierto de árboles en un futuro próximo.
    
    \item \textbf{Monte desarbolado.} Terreno con matorral y/o pastizal natural o débil intervención humana como cobertura dominante, con fracción de cabida cubierta por árboles forestales inferior al 5\%.
    
    \item \textbf{Monte sin vegetación superior.} Terreno de uso forestal que no está poblado por vegetales superiores debido a condiciones actuales de suelo, clima o topografía, aunque podría estarlo en otras circunstancias.
    
    \item \textbf{Árboles fuera del monte.} Incluye riberas arboladas no estructuradas con los montes, bosquetes de menos de 2.500 m\textsuperscript{2}, alineaciones de especies arbóreas o arbustivas de menos de 25 m de anchura, y árboles sueltos en terreno forestal.
    
    \item \textbf{Monte arbolado disperso.} Terreno forestal con especies arbóreas dominantes y fracción de cabida cubierta entre el 5\% y el 10\% (incluido el 5\%, excluido el 10\%); también terrenos con matorral o pastizal como cobertura dominante pero con presencia significativa de árboles forestales, incluyendo dehesas de base cultivo.
\end{enumerate}

\subsection*{Anexo: Contenido en Materia Orgánica}\label{sec:MatOrg}


Según la siguiente clasificación:

\begin{enumerate}
    \item \textbf{Suelo muy humífero.} Cuando a 15 cm la pureza es menor de 4, o cuando la capa de broza sea de espesor mayor de 5 cm y a 15 cm de profundidad la pureza sea menor de 6.
    \item \textbf{Suelo moderadamente humífero.} Cuando a 15 cm la pureza sea menor de 6 con capa de broza nula o de escaso espesor, o cuando dicha capa tenga espesor mayor de 5 cm y a 15 cm de profundidad la pureza sea igual o mayor de 6.
    \item \textbf{Suelo poco humífero.} En los restantes casos.
\end{enumerate}



\subsection*{Anexo: Reacción del Suelo (pH)}\label{sec:ph}

En función del pH, el suelo se clasifica según la siguiente codificación:

\begin{table}[H]
\centering
\renewcommand{\arraystretch}{1.4}
\begin{tabular}{|c|l|c|}
\hline
\textbf{Valores del pH de la solución del suelo} & \textbf{Clasificación del suelo} & \textbf{Codificación} \\
\hline
$1$     & Suelo extremadamente ácido       & $1$ \\
$2$     & Suelo muy fuertemente ácido      & $2$ \\
$3–4$   & Suelo fuertemente ácido          & $3$ \\
$5–6$   & Suelo moderadamente ácido        & $4$ \\
$7 $    & Suelo neutro                     & $5$ \\
$8 $    & Suelo moderadamente básico       & $6$ \\
$9 $    & Suelo fuertemente básico         & $7$ \\
$10 $   & Suelo extremadamente básico      & $8$ \\
\hline
\end{tabular}
\caption{Clasificación del suelo según el pH de la solución del suelo.}
\label{anexo:ph}
\end{table}


\subsection*{Anexo: Espesor de la Capa Muerta, Césped, Musgo y Líquenes}\label{sec:EspMue}

Se anotará con la siguiente codificación:

\begin{itemize}
    \item Espesor menor de $0,5$ cm \hfill \textbf{00}
    \item Espesor de $0,5$ a $1,4$ cm \hfill \textbf{01}
    \item Espesor de $1,5$ a $2,4$ cm \hfill \textbf{02}
    \item Espesor de $2,5$ a $3,4$ cm \hfill \textbf{03}
    \item \textit{Y así sucesivamente.}
\end{itemize}

\vspace{1em}
\noindent
Si en la parcela hay zonas con diferentes espesores de capa muerta, se anotará el valor medio estimado.

\subsection{Anexo: Modelo de Combustible }\label{sec:modComb}
Se determinará la clase de combustible que es más probable que propague el fuego si hubiese un incendio en la zona, hasta un máximo de 60m: pasto, matorral, hojarasca de bosque o deshechos o restos de corta. Se determinará el modelo de combustible a partir de la siguiente clave:

\begin{table}[H]
\renewcommand{\arraystretch}{2.2}
\centering
\resizebox{\textwidth}{!}{%
\begin{tabular}{|m{8cm}|m{2cm}|m{14cm}|}
\hline
\cellcolor[HTML]{D9EAD3}{\color[HTML]{000000}\textbf{GRUPO}} &
\cellcolor[HTML]{D9EAD3}{\color[HTML]{000000}\textbf{MOD. COMBUSTIBLE}} &
\cellcolor[HTML]{D9EAD3}{\color[HTML]{000000}\textbf{DESCRIPCIÓN DEL MODELO}} \\
\hline

\multirow{3}{*}{PASTOS} 
& 1 & Pasto fino, seco y bajo, que recubre completamente el suelo. Puede aparecer algunas plantas leñosas dispersas ocupando menos de 1/3 de la superficie. \\
\cline{2-3}
& 2 & Pasto fino, seco y bajo, que recubre completamente el suelo. Las plantas leñosas dispersas cubren de 1/3 a 2/3 de la superficie; pero la propagación del fuego se realiza por el pasto. \\
\cline{2-3}
& 3 & Pasto grueso, denso, seco y alto (> 1 m). Puede haber algunas plantas leñosas dispersas. Los campos de cereales son representativos de este modelo. \\
\hline

\multirow{4}{*}{MATORRAL} 
& 4 & Matorral o plantación joven muy densa; de más de 2 m de altura; con ramas muertas en su interior. Propagación del fuego por las copas de las plantas. \\
\cline{2-3}
& 5 & Matorral disperso, denso y verde, de menos de 1 m de altura. Propagación del fuego por la hojarasca, el pasto, las ramillas y el matorral. \\
\cline{2-3}
& 6 & Parecido al modelo 5, pero con especies más inflamables, de mayor talla, pudiéndose encontrar ramas gruesas en el suelo. Propagación del fuego con vientos moderados a fuertes. \\
\cline{2-3}
& 7 & Matorral de especies muy inflamables; de 0.5 a 2 m de altura, situado como sotobosque en masas de coníferas. \\
\hline

\multirow{3}{*}{HOJARASCA\\BAJO ARBOLADO} 
& 8 & Bosque denso, sin matorral. Propagación del fuego por la hojarasca muy compacta, formada por acículas cortas (5 cm o menos) o por hojas planas no muy grandes. \\
\cline{2-3}
& 9 & Parecido al modelo 8, pero con hojarasca menos compacta, formada por acículas largas y rígidas (P. pinaster) o follaje de frondosas de hoja grande, caducas (castaño o robles). \\
\cline{2-3}
& 10 & Bosque con gran cantidad de leña y árboles caídos, como consecuencia de vendavales, plagas intensas, etc. \\
\hline

\multirow{3}{*}{RESTOS DE CORTA Y \\ OPERACIONES SELVÍCOLAS} 
& 11 & Bosque claro y fuertemente aclarado. Restos de poda o aclareo ligeros (diámetro < 7.5 cm). \\
\cline{2-3}
& 12 & Predominio de los restos sobre el arbolado. La hojarasca y el matorral presente ayudarán a la propagación del fuego. \\
\cline{2-3}
& 13 & Grandes acumulaciones de restos gruesos y pesados, cubriendo todo el suelo. \\
\hline

\end{tabular}%
}
\caption{Descripción de los modelos de combustible del Inventario Forestal Nacional, clasificados por grupo funcional.}
\label{tab:modelos_combustible}
\end{table}


\subsection*{Anexo: Categoría de Desarrollo y Densidad}\label{sec:CatDesDensidad}

La categoría de desarrollo se identifica en función de la altura y el diámetro de los pies de las diferentes especies. Cuando el 85\% de los ejemplares pertenecen a una determinada categoría, se considerarán todos dentro de la misma.

\begin{enumerate}
    \item Pies con altura inferior a 30 cm.
    \item Pies con altura comprendida entre 30 y 130 cm.
    \item Pies con altura superior a 130 cm y diámetro normal menor de 2,5 cm.
    \item Pies con altura superior a 130 cm y diámetro normal comprendido entre 2,5 y 7,5 cm. Corresponde a los pies menores del IFN-2.
\end{enumerate}
\vspace{0.5em}

La densidad se cuantifica según la categoría de desarrollo:

\textbf{Para las categorías 1, 2 y 3} (radio de parcela = 5 m):
\begin{enumerate}
    \item \textbf{Escasa:} De 1 a 4 pies en la parcela.
    \item \textbf{Normal:} De 5 a 15 pies en la parcela.
    \item \textbf{Abundante:} Más de 15 pies en la parcela.

\end{enumerate}

\textbf{Para la categoría 4:}
\begin{itemize}
    \item Se cuenta el número exacto de pies por especie en la subparcela de 5 m de radio. Se registra en la casilla ``N'' y se calcula aproximadamente la altura media total de cada grupo.
\end{itemize}

\vspace{0.5em}

\textit{Nota:} Si aparecen más de 40 pies en las categorías 1, 2 o 3, el conteo puede ser estimado. Los pies menores muertos no se contabilizan. Para brotes de cepa, cada uno se considera como una planta.




\subsection*{Anexo: Tipo de Regeneración}\label{sec:TipoReg}

Se identifica el origen de los pies con la siguiente clave:

\begin{enumerate}
    \item \textbf{Siembra o semilla.}
    \item \textbf{Plantación.}
    \item \textbf{Brote de cepa o raíz.}
    \item \textbf{Desconocido.}
    \item \textbf{Dudoso.}
    \item \textbf{Mixto.}
\end{enumerate}


\subsection*{Anexo: Estado de las Poblaciones (IFN3 e IFN4)}\label{sec:EstadoIFN34}

Se determinará las fases de desarrollo de las \textit{poblaciones} codificándose de la siguiente forma:

\begin{enumerate}
    \item \textbf{Repoblado}. Conjunto de pies que desde el estrato herbáceo llega hasta el subarbustivo y los pies inician la tangencia de copas.
    \item \textbf{Monte bravo}. Comprende desde el estrato y clase de edad anterior hasta el momento en que por efecto del crecimiento, los pies empiezan a perder las ramas inferiores; es decir que en esta clase de edad, las ramas se encuentran a lo largo de todo el fuste.
    \item \textbf{Latizal}. Comprende desde la clase anterior hasta que los pies tienen 20 cm de diámetro normal; es decir, el diámetro de su fuste, medido a la altura de 1,30 m del suelo.
    \item \textbf{Fustal}. Se caracteriza esta clase de edad, porque sus pies tienen diámetros normales superiores a 20 cm.
\end{enumerate}


\subsection*{Anexo: Forma Principal de Masa (IFN3 e IFN4)}\label{sec:FPMasa}

\begin{enumerate}
    \item \textbf{Coetánea}. Cuando al menos el 90\% de sus pies tienen la misma edad individual. Ejemplo típico: las repoblaciones.
    \item \textbf{Regular}. Cuando al menos el 90\% de sus pies pertenecen a la misma clase artificial de edad o misma clase diamétrica en su defecto.
    \item \textbf{Semirregular}. Cuando al menos el 90\% de sus pies pertenecen a dos clases artificiales de edad cíclicamente contiguas o dos clases diamétricas contiguas en su defecto.
    \item \textbf{Irregular}. Cuando no se cumplen las condiciones anteriores, es decir, cuando en cualquier parte de la masa existen pies más o menos mezclados, de todas las clases de edad que tiene la masa o de varias clases diamétricas en su defecto.
\end{enumerate}

\subsection*{Anexo: Tratamiento de la Masa (IFN3 e IFN4)}\label{sec:tratmasa}

\begin{enumerate}
    \item \textbf{Monte alto}. Cuando todos los pies proceden de semilla.
    \item \textbf{Monte medio}. Cuando coexisten pies de la misma especie, unos procedentes de semilla (brinzales) y otros de brote (chirpiales).
    \item \textbf{Monte bajo}. Cuando todos los pies proceden de brote de cepa o de raíz.
\end{enumerate}


\subsection*{Anexo: Origen de la Masa (IFN3 e IFN4)}\label{sec:OrgMasa}

\begin{enumerate}
    \item \textbf{Natural}. Bosque desarrollado espontáneamente, sin intervención humana directa.
    \item \textbf{Artificial}. Plantado intencionadamente por el ser humano.
    \item \textbf{Naturalizado}. Bosque originalmente plantado pero que ha evolucionado hacia una estructura más similar a un bosque natural.
\end{enumerate}

\subsection*{Anexo: MASA (IFN2)}\label{sec:masIFN2}
\begin{itemize}
    \item 1. Artificial.
    \item 2. Natural regular.
    \item 3. Natural irregular.
    \item 9. Dudoso.
\end{itemize}

\subsection*{Anexo: Estado (IFN2)}\label{sec:EstadoIFN2}
\begin{itemize}
    \item 0. Repoblado. 
    \item 1. Monte bravo-repoblado. 
    \item 2. Monte bravo. 
    \item 3. Latizal-monte bravo. 
    \item 4. Latizal.
    \item 5. Fustal-latizal. 
    \item 6. Fustal.
\end{itemize}

\subsection*{Anexo: Origen (IFN2)}\label{sec:OrigenIFN2}
\begin{enumerate}
    \item Siembra o semilla. 
    \item Plantación.
    \item Brote de cepa o raíz. 
    \item Desconocido.
    \item Dudoso. 
    \item Mixto.
\end{enumerate}






























\newpage
\subsection{Anexo: Código de las especies}\label{sec:especies}

\begin{landscape}
\begin{longtable}{|c|p{4cm}|p{4cm}|p{4cm}|p{4cm}|}
\caption{Relación de especies con claves IFN3 e IFN4. Esta tabla se construye cruzando información de las tablas \texttt{CambioEspecie} de la base de datos de Sig de IFN3 e IFN4 y la tabla \texttt{ESPECIES ARBÓREAS Y ARBUSTIVAS} incluida tanto en el documentador de Sig del IFN4 como en el documentador de Sig del IFN3.} \\
\hline
\textbf{Clave} & \textbf{Nombre especie} & \textbf{Sinonimias} & \textbf{Claves IFN3} & \textbf{Claves IFN4} \\
\hline
\endfirsthead
\hline \textbf{Clave} & \textbf{Nombre especie} & \textbf{Sinonimias} & \textbf{Claves IFN3} & \textbf{Claves IFN4} \\
\hline
\endhead
1 & Heberdenia bahamensis & Heberdenia excelsa &  &  \\
\hline
2 & Amelanchier ovalis & Guillomo &  &  \\
\hline
3 & Frangula alnus & Rhamnus frangula &  &  \\
\hline
4 & Rhamnus alaternus & Aladierno &  &  \\
\hline
5 & Euonymus europaeus &  &  &  \\
\hline
6 & Myrtus communis &  &  &  \\
\hline
7 & Acacia spp. &  & [7, 92, 207, 307] & [7, 90, 99, 207, 307] \\
\hline
8 & Phillyrea latifolia &  &  & [8, 90, 99, 999] \\
\hline
9 & Cornus sanguinea &  &  &  \\
\hline
10 & Sin asignar & Sin asignar &  &  \\
\hline
11 & Ailanthus altissima & Ailanthus glandulosa &  &  \\
\hline
12 & Malus sylvestris &  &  & [12, 70, 99, 999] \\
\hline
13 & Celtis australis &  &  & [13, 99] \\
\hline
14 & Taxus baccata &  &  & [14, 19, 21, 22] \\
\hline
15 & Crataegus spp. &  & [4, 15, 95, 215, 295, 315] & [15, 99, 215] \\
\hline
16 & Pyrus spp. &  &  & [16, 70, 99, 395, 999] \\
\hline
17 & Cedrus atlantica &  &  &  \\
\hline
18 & Chamaecyparis lawsoniana &  & [18, 319] & [18, 19, 28, 34, 35] \\
\hline
19 & Otras coníferas &  & [14, 17, 18, 19, 21, 22, 23, 24, 25, 26, 28, 31, 32, 33, 34, 35, 36, 37, 38, 39, 217, 219, 235, 236, 237, 238, 239, 319, 336, 337, 436, 926] &  \\
\hline
20 & Pinos &  &  & [20, 26] \\
\hline
21 & Pinus sylvestris &  & [14, 17, 19, 21, 22, 25, 31, 32, 36, 37, 237] & [19, 21, 26] \\
\hline
22 & Pinus uncinata & Pinus montana, Pinus mugo & [14, 22] & [19, 21, 22] \\
\hline
23 & Pinus pinea &  & [14, 17, 20, 23, 24, 25, 26, 36, 37, 38, 39, 219, 236, 237, 317, 319, 436] & [19, 23, 24, 26] \\
\hline
24 & Pinus halepensis &  & [17, 20, 21, 23, 24, 25, 26, 36, 37, 38, 39, 236, 237, 336] & [19, 24] \\
\hline
25 & Pinus nigra & Pinus laricio, Pinus clusiana & [14, 17, 19, 21, 25, 26, 28, 34] & [19, 21, 25, 26] \\
\hline
26 & Pinus pinaster & Pinus maritima & [14, 17, 18, 21, 23, 24, 25, 26, 27, 28, 29, 36, 37, 38, 39, 217, 219, 236, 237, 239, 317, 319, 336, 436, 826, 926] & [19, 24, 26, 28, 526, 626, 726, 826, 926] \\
\hline
27 & Pinus canariensis &  & [27] &  \\
\hline
28 & Pinus radiata & Pinus insignis & [17, 19, 21, 23, 24, 25, 26, 27, 28, 33, 34, 35, 36, 236, 337, 436] & [19, 28] \\
\hline
29 & Otros pinos &  &  &  \\
\hline
30 & Mezcla de coníferas & Coníferas, excepto pinos &  &  \\
\hline
31 & Abies alba & Abies pectinata & [31, 33] & [19, 31] \\
\hline
32 & Abies pinsapo &  &  &  \\
\hline
33 & Picea abies & Picea excelsa &  & [19, 28, 33, 34, 35] \\
\hline
34 & Pseudotsuga menziesii & Pseudotsuga douglasii & [19, 28, 31, 33, 34, 35] & [19, 28, 34, 35] \\
\hline
35 & Larix spp. &  & [33, 34, 35, 235, 335] & [19, 35] \\
\hline
36 & Cupressus sempervirens &  &  & [19, 21, 24, 26, 36] \\
\hline
37 & Juniperus communis &  &  & [21, 37, 237] \\
\hline
38 & Juniperus thurifera &  & [37, 38, 39, 237, 239] & [26, 38] \\
\hline
39 & Juniperus phoenicea &  & [36, 37, 38, 39, 237] & [26, 39, 237] \\
\hline
40 & Quercus &  & [41, 42, 43, 44, 45, 46, 48, 243] & [40, 41, 44, 999] \\
\hline
41 & Quercus robur & Quercus pedunculata & [41, 42, 43, 44, 46, 48, 243] & [41, 999] \\
\hline
42 & Quercus petraea & Quercus sessiliflora & [41, 42, 48] & [41, 42, 43, 999] \\
\hline
43 & Quercus pyrenaica & Quercus toza & [41, 42, 43, 44, 45, 46, 47, 48, 243, 746, 846, 946] & [43, 99, 999] \\
\hline
44 & Quercus faginea & Quercus lusitanica var. faginea & [40, 42, 43, 44, 46, 47, 49, 243, 946] & [43, 44, 45, 999] \\
\hline
45 & Quercus ilex ssp. ballota & Quercus rotundifolia & [2, 4, 40, 44, 45, 46, 47, 49, 66, 67, 68, 75, 91, 93, 95, 215, 269, 276, 295, 378, 476] & [41, 45, 99, 245] \\
\hline
46 & Quercus suber &  & [43, 44, 45, 46, 47, 144, 846, 946] & [43, 45, 46, 99, 646, 746, 846, 946] \\
\hline
47 & Quercus canariensis & Quercus lusitanica var. baetica & [44, 47, 49] & [47] \\
\hline
48 & Quercus rubra & Quercus borealis &  & [41, 42, 43, 48, 99, 999] \\
\hline
49 & Otros quercus &  &  & [44, 999] \\
\hline
50 & Mezcla de árboles de ribera & Árboles ripícolas & [2, 3, 4, 6, 7, 9, 51, 52, 53, 54, 55, 56, 57, 58, 59, 61, 62, 63, 69, 74, 79, 92, 94, 97, 253, 255, 256, 257, 258, 297, 299, 307, 355, 357, 392, 457, 657, 757, 857, 957] &  \\
\hline
51 & Populus alba &  & [51, 53, 55, 57, 58, 62, 255, 257, 258, 357, 657, 757, 857, 957] & [50, 51, 58, 99, 258, 999] \\
\hline
52 & Populus tremula &  &  & [50, 51, 52, 58, 99, 258, 999] \\
\hline
53 & Tamarix spp. &  &  & [50, 51, 53, 999] \\
\hline
54 & Alnus glutinosa &  & [7, 52, 53, 54] & [54, 99, 999] \\
\hline
55 & Fraxinus angustifolia &  & [55, 255] & [50, 55, 99, 255, 999] \\
\hline
56 & Ulmus minor & Ulmus campestris &  & [50, 56, 70, 99, 256, 999] \\
\hline
57 & Salix spp. &  & [3, 4, 7, 9, 51, 52, 53, 54, 55, 57, 58, 79, 92, 97, 207, 257, 258, 297, 357, 457, 557, 657, 757, 857, 957] & [57, 99, 357, 657, 999] \\
\hline
58 & Populus nigra &  &  & [50, 58, 99, 258, 999] \\
\hline
59 & Otros árboles ripícolas &  &  & [54, 59] \\
\hline
60 & Mezcla de eucaliptos & Eucaliptos & [61, 62, 63, 64] &  \\
\hline
61 & Eucalyptus globulus &  & [60, 61, 62, 63, 64, 264, 364] & [61, 99] \\
\hline
62 & Eucalyptus camaldulensis & Eucalyptus rostrata & [60, 61, 62, 63, 64, 364] & [61, 62, 64, 99, 264] \\
\hline
63 & Otros eucaliptos &  &  & [61, 63, 64, 99, 264] \\
\hline
64 & Eucalyptus nitens &  & [62, 64] & [61, 64, 99, 264] \\
\hline
65 & Ilex aquifolium &  &  & [65, 99, 999] \\
\hline
66 & Olea europaea & Olea oleaster & [66, 67] & [45, 66, 99, 999] \\
\hline
67 & Ceratonia siliqua &  &  & [45, 67] \\
\hline
68 & Arbutus unedo &  & [2, 3, 4, 5, 8, 9, 12, 13, 15, 16, 56, 65, 66, 68, 72, 76, 93, 95, 97, 99, 215, 256, 275, 276, 295, 297, 299, 315, 369, 376, 395, 399, 578] & [45, 68, 99, 999] \\
\hline
69 & Phoenix spp. &  &  &  \\
\hline
70 & Mezcla de frondosas de gran porte & Frondosas de gran porte (H.t. > 10 m) & [11, 12, 13, 16, 42, 56, 66, 71, 72, 73, 74, 75, 76, 77, 78, 256, 273, 275, 276, 277, 278, 356, 373, 376, 377, 378, 476, 478, 576, 578, 676, 678] &  \\
\hline
71 & Fagus sylvatica &  &  & [71, 99] \\
\hline
72 & Castanea sativa & Castanea vesca & [11, 16, 42, 46, 66, 72, 75, 92, 99, 299, 399, 858] & [72, 99, 999] \\
\hline
73 & Betula spp. &  & [73, 273, 373] & [73, 99, 273, 999] \\
\hline
74 & Corylus avellana &  &  & [74, 99, 999] \\
\hline
75 & Juglans regia &  &  & [70, 75, 99, 256, 999] \\
\hline
76 & Acer campestre &  & [76, 276, 376, 476, 576, 676] & [70, 76, 99, 576, 999] \\
\hline
77 & Tilia spp. &  & [77, 277, 377] & [70, 77, 377, 999] \\
\hline
78 & Sorbus spp. &  & [12, 16, 78, 278, 378, 478, 578, 678] & [50, 70, 78, 99, 278, 378, 999] \\
\hline
79 & Platanus hispanica & Platanus hybrida &  & [50, 58, 79, 99, 258, 999] \\
\hline
80 & Laurisilva &  &  &  \\
\hline
81 & Myrica faya &  &  &  \\
\hline
82 & Ilex canariensis &  & [82, 282] &  \\
\hline
83 & Erica arborea &  & [83, 283] &  \\
\hline
84 & Persea indica &  &  &  \\
\hline
85 & Sideroxylon marmulano &  &  &  \\
\hline
86 & Picconia excelsa & Notelaea excelsa &  &  \\
\hline
87 & Ocotea phoetens &  &  &  \\
\hline
88 & Apollonias barbujana & Apollonias canariensis &  &  \\
\hline
89 & Otras laurisilvas &  & [1, 86, 87, 88, 89, 95, 389, 489, 495] &  \\
\hline
90 & Mezcla de pequeñas frondosas & Frondosas de pequeño porte (H.t. $\leq$ 10 m) & [1, 2, 3, 4, 5, 6, 7, 8, 9, 11, 12, 13, 15, 16, 65, 66, 67, 68, 69, 74, 78, 91, 92, 93, 94, 95, 96, 97, 99, 215, 269, 278, 295, 297, 299, 307, 369, 378, 395, 399, 478, 495, 499, 578, 595, 599] &  \\
\hline
91 & Buxus sempervirens &  &  &  \\
\hline
92 & Robinia pseudoacacia & Acacia robinia & [7, 56, 61, 64, 72, 73, 75, 77, 79, 92, 256, 273, 373, 377] & [50, 58, 92, 99, 207, 258, 999] \\
\hline
93 & Pistacia terebinthus & Cornicabra &  &  \\
\hline
94 & Laurus nobilis & Laurel &  & [90, 94, 99] \\
\hline
95 & Prunus spp. & Prunus & [4, 5, 9, 95, 295] & [45, 90, 95, 99, 395, 999] \\
\hline
96 & Rhus coriaria & Zumaque &  & [50, 96] \\
\hline
97 & Sambucus nigra & Saúco negro &  & [50, 90, 97, 99, 657, 999] \\
\hline
98 & Carpinus betulus & Carpe & [43, 44, 46, 47, 51, 53, 54, 55, 57, 58, 60, 62, 63, 64, 72, 75, 144, 257, 357, 657] & [71, 273] \\
\hline
99 & Otras frondosas & Otras frondosas & [1, 2, 3, 4, 5, 6, 7, 8, 9, 10, 11, 12, 13, 14, 15, 16, 40, 41, 42, 43, 44, 45, 46, 47, 48, 49, 51, 52, 53, 54, 55, 56, 57, 58, 59, 60, 61, 62, 64, 65, 66, 67, 68, 69, 71, 72, 73, 74, 75, 76, 77, 78, 79, 82, 83, 84, 86, 87, 88, 89, 91, 92, 93, 94, 95, 96, 97, 99, 207, 215, 217, 243, 253, 255, 256, 257, 258, 264, 269, 273, 275, 276, 277, 278, 279, 291, 292, 293, 295, 297, 299, 307, 315, 355, 356, 357, 364, 369, 373, 375, 376, 377, 378, 389, 392, 395, 399, 415, 457, 469, 476, 478, 489, 492, 495, 499, 557, 569, 576, 578, 592, 595, 657, 676, 678, 757, 778, 857, 858, 957] & [90, 99, 999] \\
\hline
207 & Acacia melanoxylon & Acacia melanoxylon &  & [7, 99, 207, 307] \\
\hline
215 & Crataegus monogyna & Majuelo & [15, 215, 315, 415] & [15, 45, 99, 215] \\
\hline
217 & Cedrus deodara & Cedrus deodara &  & [17, 26, 217] \\
\hline
219 & Tetraclinis articulata & Tetraclinis articulata &  & [26, 219] \\
\hline
235 & Larix decidua & Alerce común &  & [19, 34, 35, 235] \\
\hline
236 & Cupressus arizonica & Ciprés arizónica &  & [19, 24, 26, 36, 236] \\
\hline
237 & Juniperus oxycedrus & Enebro oxicedro & [36, 37, 38, 39, 236, 237, 239] & [21, 26, 39, 237] \\
\hline
238 & Juniperus turbinata & Sabina canaria & [238, 337] & [38] \\
\hline
239 & Juniperus sabina & Sabina rastrera &  &  \\
\hline
243 & Quercus pubescens & Quercus pubescens, Quercus humilis & [42, 43, 44, 47, 243] & [43, 243, 999] \\
\hline
244 & Quercus lusitanica & Quercus fruticosa, Quejigueta &  &  \\
\hline
245 &  &  &  & [45, 245] \\
\hline
253 & Tamarix canariensis & Tarajal &  & [53] \\
\hline
255 & Fraxinus excelsior & Fresno excelsior & [55, 255, 355] & [50, 55, 99, 255, 999] \\
\hline
256 & Ulmus glabra & Ulmus montana &  & [70, 99, 256, 999] \\
\hline
257 & Salix alba & Sauce blanco &  & [57, 99, 257, 357, 657, 999] \\
\hline
258 & Populus x canadensis & Populus x euroamericana & [58, 258] & [50, 58, 99, 258, 999] \\
\hline
264 & Eucalyptus viminalis & Eucalipto viminalis &  & [64, 99, 264] \\
\hline
268 & Arbutus canariensis & Madroño canario &  & [68] \\
\hline
273 & Betula alba & Betula verrucosa, Abedul pubescens &  & [73, 99, 273, 999] \\
\hline
275 & Juglans nigra & Nogal &  & [256, 275] \\
\hline
276 & Acer monspessulanum & Arce de Montpelier &  & [45, 70, 276, 999] \\
\hline
277 & Tilia cordata & Tilo cordata &  & [277, 377, 999] \\
\hline
278 & Sorbus aria & Mostajo &  & [70, 99, 278, 378, 999] \\
\hline
279 & Platanus orientalis & Plátano oriental &  & [58, 999] \\
\hline
281 &  &  &  &  \\
\hline
282 & Ilex platyphylla & Naranjero &  & [82] \\
\hline
283 & Erica scoparia & Tejo, brezo arbóreo escopario &  & [83] \\
\hline
289 & Pleiomeris canariensis & Delfino &  & [89] \\
\hline
291 & Buxus balearica & Boj de Baleares &  &  \\
\hline
292 & Sophora japonica & Acacia sofora &  &  \\
\hline
293 & Pistacia atlantica & Cornicabra canaria &  & [93] \\
\hline
294 & Laurus azorica & Laurel canario & [68, 268, 294] & [94] \\
\hline
295 & Prunus spinosa & Espino negro &  &  \\
\hline
297 & Sambucus racemosa & Saúco racemosa &  &  \\
\hline
299 & Ficus carica & Higuera &  & [90, 99, 299, 999] \\
\hline
307 & Acacia dealbata & Acacia dealbata &  & [7, 90, 99, 207, 307] \\
\hline
315 & Crataegus laevigata & Espino majuelo &  & [215, 315] \\
\hline
317 & Cedrus libani & Cedrus libani &  &  \\
\hline
319 & Thuja spp. & Thuja &  & [19, 28, 319] \\
\hline
335 & Larix leptolepis & Larix kaempferi, Alerce leptolepis &  &  \\
\hline
336 & Cupressus lusitanica & Ciprés lambertiana &  & [26, 36] \\
\hline
337 & Juniperus cedrus & Enebro canario &  & [37] \\
\hline
344 &  &  &  &  \\
\hline
355 & Fraxinus ornus & Fresno orno &  &  \\
\hline
356 & Ulmus pumila & Olmo pumilo &  & [70, 356] \\
\hline
357 & Salix atrocinerea & Bardaguera &  & [57, 99, 357, 657, 999] \\
\hline
364 & Eucalyptus gomphocephalus & Eucalipto gonfo & [62, 63, 64, 264, 364] & [64, 264, 364] \\
\hline
369 & Chamaerops humilis & Palmito &  &  \\
\hline
373 & Betula pendula & Betula hispanica, Abedul péndula &  & [73, 273, 373, 999] \\
\hline
376 & Acer negundo & Negundo fraxinifolia, Arce negundo &  & [76, 99, 276, 376, 999] \\
\hline
377 & Tilia platyphyllos & Tilo común &  & [70, 377, 999] \\
\hline
378 & Sorbus aucuparia & Serbal de cazadores &  & [45, 50, 70, 78, 99, 278, 378, 999] \\
\hline
389 & Rhamnus glandulosa & Sanguino &  & [89] \\
\hline
392 & Gleditsia triacanthos & Acacia gleditsia &  &  \\
\hline
395 & Prunus avium & Cerezo silvestre &  & [90, 95, 99, 395, 999] \\
\hline
399 & Morus spp. & Morera &  &  \\
\hline
415 & Crataegus laciniata & Majoleto &  &  \\
\hline
435 & Larix x eurolepis & Alerce híbrido &  &  \\
\hline
436 & Cupressus macrocarpa & Ciprés americano &  & [26, 36, 436] \\
\hline
455 &  &  &  &  \\
\hline
456 &  &  &  & [70, 456] \\
\hline
457 & Salix babylonica & Sauce llorón &  & [57, 357, 457, 657] \\
\hline
464 &  &  &  & [61, 64, 99, 464] \\
\hline
469 & Phoenix canariensis & Palmera & [469, 569] & [69] \\
\hline
476 & Acer opalus & Arce ópalus &  & [45, 70, 276, 476, 576, 999] \\
\hline
478 & Sorbus domestica & Serbal común &  & [278, 478, 999] \\
\hline
489 & Visnea mocanera & Mocan &  & [89] \\
\hline
495 & Prunus lusitanica & Loro, hija &  & [95, 495] \\
\hline
499 & Morus alba & Morera &  &  \\
\hline
515 & Crataegus azarolus & Espino &  &  \\
\hline
557 & Salix cantabrica & Sauce cantábrico &  & [357, 557] \\
\hline
569 & Dracaena draco & Drago &  &  \\
\hline
576 & Acer pseudoplatanus & Arce seudoplátano &  & [70, 76, 99, 276, 576, 999] \\
\hline
578 & Sorbus torminalis & Serbal torminal &  & [70, 278, 378, 578, 999] \\
\hline
595 & Prunus padus & Prunus &  &  \\
\hline
599 & Morus nigra & Morera &  &  \\
\hline
657 & Salix caprea & Sauce cabruno &  & [57, 99, 357, 657, 999] \\
\hline
676 & Acer platanoides & Arce platanoide &  & [70, 99, 276, 576, 676, 999] \\
\hline
678 & Sorbus latifolia & Serbal de hoja ancha &  & [278, 678, 999] \\
\hline
757 & Salix elaeagnos & Sarga &  & [99, 357, 657, 757, 999] \\
\hline
776 &  &  &  &  \\
\hline
778 & Sorbus chamaemespilus & Serbal chame &  &  \\
\hline
857 & Salix fragilis & Mimbre &  & [357, 657, 857, 999] \\
\hline
858 & Salix canariensis & Sauce canario &  & [58] \\
\hline
917 & Cedrus spp. & Cedrus spp. &  &  \\
\hline
936 & Cupressus spp. & Cipres &  &  \\
\hline
937 & Juniperus spp. & Enebros y sabinas & [36, 37, 38, 39, 236, 237, 239, 336] &  \\
\hline
955 & Fraxinus spp. & Fresnos & [55, 255] &  \\
\hline
956 & Ulmus spp. & Olmo &  &  \\
\hline
957 & Salix purpurea & Mimbrera &  & [99, 357, 657, 957, 999] \\
\hline
975 & Juglans spp. &  &  &  \\
\hline
976 & Acer spp. & Arces & [71, 72, 76, 276, 278, 307, 378, 476, 478, 492, 576, 578, 676] &  \\
\hline
997 & Sambucus spp. &  &  &  \\
\hline
\end{longtable}
\label{tab:codificacion_especies}
\end{landscape}




% Referencias
\cleardoublepage % Opcional: para que las referencias empiecen en una página impar si usas book/report

\bibliographystyle{unsrt} % o plain, IEEEtran, abbrvnat, plainnat...
\bibliography{referencias} % SIN la extensión .bib y con barras /

\end{document}