%% Plantilla basada en elsarticle (Elsevier)
%% Fácilmente adaptable a otras revistas cambiando el documentclass
\documentclass[preprint,12pt,authoryear]{elsarticle}

% ---------------------------------------------------------------------------
% Preámbulo: paquetes y configuración
% ---------------------------------------------------------------------------
%%%%%%%%%%%%%%%%%%%%%%%%%%%%%%%%%%%%%%%%%
% Preámbulo para artículo científico
% Compatible con elsarticle y fácilmente adaptable
%%%%%%%%%%%%%%%%%%%%%%%%%%%%%%%%%%%%%%%%%

%----------------------------------------------------------------------------------------
% IDIOMA Y CODIFICACIÓN
%----------------------------------------------------------------------------------------
\usepackage[spanish,es-tabla]{babel}
\usepackage[utf8]{inputenc}
\usepackage[T1]{fontenc}

%----------------------------------------------------------------------------------------
% BIBLIOGRAFÍA (biblatex + biber)
%----------------------------------------------------------------------------------------
\usepackage[
    backend=biber,
    style=authoryear,    % Estilo autor-año (compatible con elsarticle authoryear)
    sorting=nyt,         % Ordenar por Nombre, Año, Título
    natbib=true,         % Permite usar \citep y \citet
    maxcitenames=2,
    maxbibnames=99
]{biblatex}
\addbibresource{referencias.bib}

%----------------------------------------------------------------------------------------
% PAQUETES ESENCIALES
%----------------------------------------------------------------------------------------
\usepackage{amsmath,amsfonts,amssymb,amsthm}  % Matemáticas
\usepackage{graphicx}                          % Imágenes
\graphicspath{{figuras/}{./}}
\usepackage{booktabs}                          % Tablas profesionales
\usepackage{array}
\usepackage{tabularx}
\usepackage{longtable}
\usepackage{multirow}
\usepackage{float}
\usepackage{subcaption}

%----------------------------------------------------------------------------------------
% COLORES Y ENLACES
%----------------------------------------------------------------------------------------
\usepackage[table,xcdraw,dvipsnames]{xcolor}
\usepackage{hyperref}
\hypersetup{
    colorlinks=true,
    linkcolor=blue!70!black,
    citecolor=blue!70!black,
    urlcolor=blue!70!black
}

%----------------------------------------------------------------------------------------
% CONFIGURACIÓN DE TABLAS Y FIGURAS
%----------------------------------------------------------------------------------------
\usepackage{caption}
\captionsetup{
    font=small,
    labelfont=bf,
    labelsep=period,
    justification=justified,
    skip=6pt
}

% Espaciado uniforme en tablas
\renewcommand{\arraystretch}{1.15}

%----------------------------------------------------------------------------------------
% LISTAS
%----------------------------------------------------------------------------------------
\usepackage{enumitem}
\setlist{noitemsep,topsep=3pt}

%----------------------------------------------------------------------------------------
% CÓDIGO Y TEXTO MONOESPACIADO
%----------------------------------------------------------------------------------------
\usepackage{listings}
\lstset{
    basicstyle=\ttfamily\small,
    breaklines=true,
    frame=single,
    backgroundcolor=\color{gray!10}
}

% Permitir que \texttt{} rompa líneas en guiones bajos
\usepackage{underscore}

%----------------------------------------------------------------------------------------
% OTROS PAQUETES ÚTILES
%----------------------------------------------------------------------------------------
\usepackage{siunitx}        % Unidades SI
\usepackage{tikz}           % Diagramas
\usetikzlibrary{shapes.geometric,arrows}
\usepackage{pdflscape}      % Páginas horizontales
\usepackage{etoolbox}       % Hooks

%----------------------------------------------------------------------------------------
% NUMERACIÓN
%----------------------------------------------------------------------------------------
% Numerar figuras, tablas y ecuaciones por sección
\numberwithin{equation}{section}
\numberwithin{figure}{section}
\numberwithin{table}{section}

%----------------------------------------------------------------------------------------
% COMANDOS PERSONALIZADOS
%----------------------------------------------------------------------------------------
% Comando para código inline con mejor formato
\newcommand{\code}[1]{\texttt{#1}}


% ---------------------------------------------------------------------------
% Metadatos del artículo
% ---------------------------------------------------------------------------
\journal{[Nombre de la Revista]} % Cambiar al enviar

\begin{document}

\begin{frontmatter}

    \title{GreenWest: inteligencia artificial para la predicción de créditos de carbono en proyectos de (re)forestación en España}

    %% Autores y afiliaciones
    \author[usal]{Maider Araceli Urbón Jiménez\corref{cor1}}
    \ead{murbon001@usal.es}
    \cortext[cor1]{Autora de correspondencia}

    \author[usal]{Jaime Gabriel Vegas}
    \ead{JaimeGabrielVegas@usal.es}

    \author[usal]{Ana de Luis Reboredo}
    \ead{adeluis@usal.es}

    \author[usal]{Belén Pérez Lancho}
    \ead{lancho@usal.es}

    \author[usal]{Ana-Belén Gil-González}
    \ead{abg@usal.es}

    \affiliation[usal]{organization={Grupo B1, Equipo de investigación BISITE, Universidad de Salamanca},
        addressline={Facultad de Ciencias},
        city={Salamanca},
        country={España}}

    %% Abstract
    \begin{abstract}
        Este trabajo presenta \textbf{GreenWest}, un modelo de inteligencia artificial diseñado para predecir la cantidad de carbono capturado en proyectos de forestación y reforestación en España. El modelo se entrena con datos multifuente: registros del \textbf{Inventario Forestal Nacional} (\textbf{IFN3--IFN4}, MITECO), variables climáticas derivadas de \textbf{Copernicus/ERA5-Land} e índices espectrales procedentes de \textbf{imágenes Landsat} (Collection~2, Level~2, USGS). Estos datos se integran en una base de datos relacional jerárquica que organiza la información por parcela, especie y clase diamétrica, manteniendo trazabilidad y coherencia estructural entre inventarios.

        El modelo desarrollado responde a la pregunta: \textit{Dado un cultivo forestal con características concretas de vegetación, clima y terreno, ¿cuánto CO$_2$ contendrá pasados unos años?} Esta capacidad predictiva permite su integración en marcos de optimización forestal, abordando cuestiones como la selección de especies o la asignación óptima de terrenos para maximizar la fijación de carbono.

        Se evaluaron múltiples enfoques de aprendizaje supervisado, destacando \textbf{CatBoost} como el modelo con mejor rendimiento ($R^2>0.80$, RMSE$<$15), con alta capacidad de generalización temporal mediante validación cruzada por grupos. Los resultados demuestran el potencial del enfoque para estimar la absorción futura de CO$_2$ y optimizar decisiones de gestión forestal sostenible, contribuyendo a la transición hacia una economía baja en emisiones.
    \end{abstract}

    %% Keywords
    \begin{keyword}
        créditos de carbono \sep inteligencia artificial \sep forestación \sep reforestación \sep modelado predictivo \sep cambio climático
    \end{keyword}

\end{frontmatter}

% ---------------------------------------------------------------------------
% Contenido principal
% ---------------------------------------------------------------------------

\section{Introducción}

El cambio climático es uno de los mayores desafíos globales y su manifestación más directa es el aumento de las concentraciones atmosféricas de dióxido de carbono (\(CO_2\)), con impactos sobre criosfera, extremos climáticos y ecosistemas \cite{IPCC2007}. Los bosques actúan como sumideros naturales al fijar \(CO_2\) en biomasa vía fotosíntesis, por lo que su gestión resulta clave para la mitigación. 


A lo largo de las últimas décadas, instrumentos internacionales como la \textit{Convención Marco de las Naciones Unidas sobre el Cambio Climático (CMNUCC)} y el \textit{Protocolo de Kioto} \cite{UNFCCC1997,UNFCCC2015} han establecido los marcos regulatorios para reducir las emisiones de gases de efecto invernadero mediante mecanismos basados en el mercado. En este contexto surgen los \emph{créditos de carbono}, unidades que representan la cantidad de dióxido de carbono (\(CO_2\)), habitualmente una tonelada, que ha sido capturada o cuya emisión ha sido evitada a través de proyectos certificados de mitigación. 


Entre las actividades elegibles, la forestación y reforestación destacan por su capacidad de actuar como sumideros naturales de carbono, fijando \(CO_2\) en la biomasa y el suelo. No obstante, para que estas actuaciones puedan generar créditos de carbono válidos, deben cumplir una serie de criterios técnicos y legales establecidos en la normativa internacional sobre cambio climático y en su aplicación a nivel nacional. En particular, estos requisitos derivan de las reglas de contabilidad de sumideros forestales adoptadas en el marco de la Convención Marco de las Naciones Unidas sobre el Cambio Climático (CMNUCC) y del Protocolo de Kioto, concretadas en los Acuerdos de Marrakech, así como de la definición nacional de bosque comunicada por España para estos fines \cite{UNFCCCMarrakech,UNFCCCSpainReview}. Dichos criterios incluyen:

\begin{itemize}
\item \textbf{Intervención humana directa:}
Los árboles deben ser el resultado de actividades de intervención humana directa, tales como la plantación, la siembra o el fomento deliberado de la regeneración natural. Este requisito se deriva de la definición de \emph{forestación} y \emph{reforestación} establecida en el Protocolo de Kioto, que excluye expresamente la regeneración natural no inducida por la acción humana \cite{UNFCCCMarrakech}.

\item \textbf{Período mínimo de permanencia:}
El proyecto debe garantizar la permanencia del sumidero de carbono durante un período prolongado (habitualmente del orden de 20-30 años), con el fin de asegurar que el carbono capturado no sea liberado de forma prematura a la atmósfera. Este criterio responde al principio de permanencia exigido en la contabilidad de sumideros forestales del régimen LULUCF y en los marcos de aplicación nacionales y europeos, lo que excluye cultivos de corta rotación cuyo carbono se libera tras la cosecha \cite{UNFCCCMarrakech,EULULUCF2018}.

\item \textbf{Superficie mínima de 1 hectárea:}
El área objeto del proyecto debe tener una extensión mínima de 1 hectárea. Este umbral procede de la definición nacional de bosque adoptada por España dentro de los rangos permitidos por los Acuerdos de Marrakech (0,05–1 ha), comunicada oficialmente a la CMNUCC \cite{UNFCCCSpainReview}.

\item \textbf{Fracción mínima de cabida cubierta del 20\%:}
Para que un terreno sea considerado bosque, la cobertura de copas de las especies arbóreas debe alcanzar al menos el 20\% de la superficie. Este valor corresponde a la elección nacional realizada por España para la definición de bosque a efectos de contabilidad climática \cite{UNFCCCSpainReview}.

\item \textbf{Altura mínima de los árboles maduros de 3 metros:}
Las especies arbóreas deben ser capaces de alcanzar una altura mínima de 3 metros en su madurez. No es necesario que dicha altura se alcance en el momento inicial del proyecto, pero sí que sea alcanzable bajo condiciones normales de crecimiento. Este criterio forma igualmente parte de la definición nacional de bosque comunicada por España conforme a las decisiones adoptadas bajo la CMNUCC \cite{UNFCCCSpainReview}.
\end{itemize}

Este trabajo presenta \textbf{GreenWest}, un modelo de inteligencia artificial para estimar la cantidad de carbono que capturará un cultivo forestal en España a partir de variables de vegetación, clima y terreno en un período de 20 a 30 años. Este enfoque innovador tiene el potencial de transformar la gestión de proyectos de forestación y reforestación, optimizando las prácticas de plantación y maximizando la cantidad de carbono que se puede capturar en estos ecosistemas. 

La pregunta operativa es: \emph{dadas las características iniciales de una plantación, ¿cuánto \(CO_2\) contendrá tras \(t\) años? con t número natural.} Para responderla, se integran datos del \textbf{Inventario Forestal Nacional} (IFN2–IFN4, MITECO) \cite{ifn}, reanálisis \textbf{ERA5-Land} \cite{era5land} e \textbf{índices espectrales Landsat} (Collection~2, L2) \cite{landsatc2} en una base de datos relacional jerárquica descrita en un trabajo complementario \cite{greenwestdb}.

Este modelo no solo mejorará la comprensión del comportamiento de los sumideros de carbono, sino que también proporcionará herramientas útiles para la toma de decisiones estratégicas tanto en el ámbito empresarial como en el ambiental. De esta forma, el proyecto \textit{GreenWest} contribuye a la transición hacia una economía baja en carbono, alineándose con los objetivos globales de sostenibilidad establecidos en el marco de la CMNUCC y el \textit{Protocolo de Kioto}, y promoviendo la creación de un mercado de créditos de carbono más eficiente y accesible para los actores económicos involucrados en la gestión de los recursos naturales.


\clearpage \thispagestyle{empty} \mbox{} \clearpage

\input{sections/02_objetivos}
\clearpage \thispagestyle{empty} \mbox{} \clearpage

% sections/03_revision_literatura.tex
\section{Revisión de la Literatura}

La cuantificación precisa de los recursos forestales ha constituido, históricamente, una de las piedras angulares de la gestión territorial y la economía de recursos naturales. La evolución de las técnicas para medir el crecimiento de los árboles y, más recientemente, para estimar su biomasa y contenido de carbono, refleja una transformación profunda en las prioridades de la sociedad humana respecto a los ecosistemas forestales. Lo que comenzó en la Europa medieval como una necesidad logística para asegurar el suministro de leña y madera estructural ante el temor de la escasez, se ha metamorfoseado en el siglo XXI en una disciplina científica de alta tecnología impulsada por la urgencia climática global \cite{brack_history_2000}.

La dendrometría tradicional, pilar de los inventarios forestales modernos, se fundamenta en el uso de relaciones alométricas para estimar la biomasa ($w$) y otros parámetros ecológicos a partir de variables de fácil medición en campo, principalmente el diámetro a la altura del pecho ($D$) y la altura total ($H$). Estas estimaciones suelen articularse mediante ecuaciones de la forma:
\begin{equation}
    w = a D^b H^c
\end{equation}
o sus transformaciones logarítmicas:
\begin{equation}
    \lg w = a + b \lg D + c \lg H
\end{equation}
donde los parámetros $a, b$ y $c$ son coeficientes de regresión empíricos \cite{shi2017methods}. En este contexto, la precisión de las mediciones primarias es crítica, ya que, dada la naturaleza potencial de estas funciones, los errores en la toma de datos de $D$ y $H$ se propagan y magnifican en el cálculo final del volumen y el contenido de carbono. Pese a la aparente sencillez del ajuste, los resultados pueden llegar a ser muy buenos, como se muestra en la Figura \ref{fig:moso_bamboo}. No obstante, existe una tensión en la literatura entre el uso de ecuaciones ``pantropicales'' o generalizadas y ecuaciones específicas de especie o sitio, ya que se pueden introducir sesgos si la arquitectura de los árboles locales difiere de la global \cite{shang2025allometric}.

\begin{figure}
    \centering
    \includegraphics[width=0.5\textwidth]{figuras/03_revision_literatura/moso_bamboo.png}
    \caption{Relación entre la biomasa total en kilogramos y el diámetro a la altura del pecho en centímetros para el bambú moso \textit{(Phyllostachys edulis)}. Visto en \cite{shi2017methods}, siendo \cite{qi2016combining} la fuente original.}
    \label{fig:moso_bamboo}
\end{figure}



La medición de la altura de los árboles ha supuesto históricamente un desafío mayor que la del diámetro. Hasta la década de 1990, predominaron hipsómetros mecánicos basados en trigonometría, que requerían medir manualmente la distancia al árbol y una línea de visión despejada \cite{nyakudanga_treemes}. Estos métodos sufrían limitaciones de precisión y ergonomía. La introducción de la electrónica marcó un punto de inflexión al utilizar ultrasonidos para medir distancias automáticamente, permitiendo trabajar en sotobosques densos y mejorando la precisión por debajo del 1\% \cite{nyakudanga_treemes}. Paralelamente, las forcípulas (aparato de medición de la distancia lineal entre dos tangentes paralelas al fuste del árbol) electrónicas modernas han digitalizado la toma de datos, integrando medición y registro de metadatos para minimizar errores de transcripción \cite{nyakudanga_treemes}.

Posteriormente, la introducción de la tecnología de escaneo láser supuso una revolución en la mensura forestal, superando las limitaciones logísticas y de precisión de los métodos tradicionales. Esta tecnología se despliega principalmente en dos modalidades: el escaneo láser terrestre (TLS, por sus siglas en inglés), que captura la estructura tridimensional del bosque desde el suelo con detalle milimétrico \cite{Kristiansen2018}, y el LiDAR aerotransportado. Este último consiste es un método que usa pulsos de luz para medir distancias. Los mapeos masivos suelen consistir en un sistema LiDAR montado en un avión, helicóptero, dron o satélite, el cuál mide la distancia a los objetos debajo de él. Una ventaja es que, al igual que la luz puede llegar al suelo a través de los huecos de las copas de los árboles, los pulsos láser del LiDAR también lo hacen, como podemos observar en la Figura \ref{fig:lidar_avion}.
\begin{figure}
    \centering
    \includegraphics[width=\textwidth]{figuras/03_revision_literatura/lidar_avion.png}
    \caption{Diagrama de el escaneo de un árbol por un sistema LiDAR aerotransportado y los diversos pulsos reflejados. Imagen obtenida de \cite{opentopography_lidar_basics}.}
    \label{fig:lidar_avion}
\end{figure}
Esto hace que mediante un solo sobrevuelo se puedan captar diversas capas a varias alturas, lo que permite realizar reconstrucciones en $3$D de las zona escaneada. El detalle que se puede lograr es tán alto que se pueden identificar los árboles individuales, como se ve en la \ref{fig:chm_example}.



\begin{figure}[htbp]
    \centering
    \begin{subfigure}[b]{0.48\textwidth}
        \centering
        \includegraphics[width=\textwidth]{figuras/03_revision_literatura/CHM_1.png}
        \caption{Visualización $3$D de un bosque escaneado con LiDAR. Cada árbol se ha identificado de un color distinto.}
        \label{fig:sub1}
    \end{subfigure}
    \hfill
    \begin{subfigure}[b]{0.48\textwidth}
        \centering
        \includegraphics[width=\textwidth]{figuras/03_revision_literatura/CHM_2.png}
        \caption{Mismo mapeo pero visto de arriba como un mapa $2$D. Cada árbol está marcado con una \textbf{x}.}
        \label{fig:sub2}
    \end{subfigure}
    \caption{Visualización de un bosque escaneado con LiDAR. Vemos las posibilidades de identificar árboles individuales en un entorno forestal denso.}
    \label{fig:chm_example}
\end{figure}

Las principales limitaciones del LiDAR aerotransportado residen en su elevado coste y complejidad logística. Estos factores dificultan la obtención de coberturas a gran escala con una periodicidad óptima, generando además una latencia significativa en la disponibilidad de datos regionales. Asimismo, el procesamiento de esta información es más exigente que en métodos previos, a lo que se suma el desafío técnico de gestionar y almacenar grandes volúmenes de datos En la Figura \ref{fig:pnoa_lidar} se muestra el plan de adquisición de datos del Tercer ciclo del proyecto PNOA-LiDAR del Instituto Geográfico Nacional de España, donde se puede apreciar la cadencia con la que se hacen los mapeos.

\begin{figure}
    \centering
    \includegraphics[width=0.5\textwidth]{figuras/03_revision_literatura/plan_lidar_mapeo.png}
    \caption{Plan de adquisición de datos del Tercer ciclo del proyecto PNOA-LiDAR del Instituto Geográfico Nacional de España.}
    \label{fig:pnoa_lidar}
\end{figure}


La integración de estos datos estructurales, junto a la teledetección satelital y la capacidad de procesado de grandes volúmenes de datos que ofrecen los algoritmos de aprendizaje automático ha inaugurado un nuevo paradigma: la capacidad de monitorear los recursos forestales a escala global con una precisión sin precedentes. Es en este contexto de ``forestería de precisión'' y observación terrestre a gran escala donde se enmarca la investigación actual, habiéndose logrado resultados muy prometedores que permiten abordar la complejidad de los ecosistemas forestales con una granularidad antes inalcanzable.


\clearpage \thispagestyle{empty} \mbox{} \clearpage

\section{Estado del Arte}
\label{sec:estado-del-arte}

El secuestro de carbono en ecosistemas forestales ha cobrado una importancia creciente en la literatura científica, impulsada tanto por los compromisos internacionales en materia de cambio climático como por el auge de los mercados de créditos de carbono. Esto ha motivado el desarrollo de modelos orientados a cuantificar la biomasa forestal y estimar el contenido de carbono, aprovechando avances recientes en sensores remotos y técnicas de inteligencia artificial (IA).


Una de las estrategias más consolidadas para la cuantificación del carbono forestal es la estimación del carbono almacenado en un momento dado a partir de datos de teledetección. Goetz et al. (2009) \cite{goetz2009remote} revisan el uso de observaciones satelitales, incluyendo sensores ópticos como MODIS y Landsat, en modelos empíricos de biomasa aérea, destacando su aplicabilidad a escala regional, especialmente en ecosistemas boreales. Este tipo de estimaciones suele basarse en regresiones lineales o modelos de mínimos cuadrados generalizados, con coeficientes de determinación habitualmente entre 0.6 y 0.8, dependiendo de la resolución espacial y la heterogeneidad del ecosistema.

La aplicación de aprendizaje profundo ha permitido mejorar sustancialmente la precisión y resolución espacial de estas estimaciones. Por ejemplo, Zhang et al. (2022) \cite{zhang2022carbon} integran imágenes Sentinel-2 con redes neuronales convolucionales, alcanzando un \(R^2\) de 0.84 para estimar el carbono en bosques subtropicales. Del mismo modo, Jiang et al. (2022) \cite{jiang2022integrating} desarrollan el modelo \textit{ForestCarbonAI}, entrenado con datos multiespectrales y LIDAR, con el que generan mapas de carbono forestal de alta resolución (10 m), reportando errores medios absolutos (MAE) inferiores a 3.5 tC/ha en zonas templadas. Otros trabajos recientes, como Reiersen et al. (2022) \cite{reiersen2022reforestree} o Dong et al. (2023) \cite{dong2023forest}, también demuestran la eficacia del \textit{deep learning} para estimaciones estáticas, aunque se centran en contextos tropicales y no consideran el componente temporal.


Frente a estos enfoques descriptivos, algunas iniciativas han intentado proyectar la evolución del carbono a futuro. En el ámbito nacional, el Ministerio para la Transición Ecológica (MITECO) ha implementado herramientas como la calculadora ex ante de absorciones \cite{miteco_abexante_2025}, que permite obtener estimaciones simplificadas del carbono que puede fijarse en una plantación forestal en función de la especie y la zona agroclimática. No obstante, este instrumento se basa en coeficientes tabulados y no incorpora variables edafoclimáticas reales ni técnicas de modelización basadas en datos, lo que limita su precisión y capacidad de adaptación a contextos específicos.

En el ámbito europeo destaca el trabajo de Fasihi et al. (2024) \cite{fasihi2024assessing}, que aplica un enfoque afín al propuesto en este estudio en la región de Friuli-Venezia Giulia (Italia). Su objetivo es estimar tanto el \textit{stock} actual de carbono como su tasa de absorción anual (secuestro), basándose en datos de dendrocronología del Inventario Forestal Nacional italiano. Estas mediciones, realizadas entre 2017 y 2019, cuantifican el crecimiento radial de los últimos cinco años para estimar la biomasa mediante ecuaciones alométricas y su conversión a carbono según directrices del IPCC. Luego, entrenan modelos predictivos utilizando variables meteorológicas, geomorfológicas, índices de vegetación y métricas derivadas de LiDAR, intentando predecir los valores obtenidos con datos de campo.

Los autores evalúan diversos modelos de conjunto (\textit{ensemble}), reportando que combinaciones de algoritmos como Random Forest, AdaBoost y CatBoost ofrecen el mejor rendimiento. En la predicción del \textit{stock} de carbono, alcanzan un $R^2$ de $0.73 \pm 0.07$ y un RMSE de $31.55 \pm 9.35$ tC/ha. Para la tasa de secuestro, los resultados son más modestos, con un $R^2$ de $0.42 \pm 0.08$ y un RMSE de $0.90 \pm 0.08$ tC/ha/año. Un hallazgo clave del estudio es que la inclusión de datos LiDAR mejora drásticamente la precisión de las estimaciones.




En este escenario, el presente trabajo propone una metodología innovadora centrada en la predicción dinámica de carbono a largo plazo. A diferencia de los modelos anteriores, que estiman el carbono ya almacenado, este estudio se enfoca en anticipar cuánto carbono capturará un cultivo forestal en un horizonte temporal concreto. Para ello, se estudian diversos modelos de aprendizaje supervisado entrenados con datos históricos del Inventario Forestal Nacional (IFN2, IFN3 e IFN4), variables climáticas de Copernicus, características edáficas y métricas espectrales derivadas de imágenes Landsat \cite{landsat5_data, copernicus_temps, miteco_guia_co2}. Los detalles sobre la arquitectura del modelo, las variables utilizadas, los algoritmos implementados y las métricas de evaluación se desarrollan en las siguientes secciones.





\clearpage \thispagestyle{empty} \mbox{} \clearpage

\section{Metodología}

Esta sección describe el procedimiento seguido para el entrenamiento y validación de los modelos predictivos desarrollados.
La metodología se fundamenta en la identificación de los factores que determinan el crecimiento forestal y, en consecuencia, la capacidad de los ecosistemas para capturar carbono a lo largo del tiempo.
El enfoque integra información estructural, climática y espectral procedente del Inventario Forestal Nacional (IFN) y de otras fuentes ambientales, con el propósito de construir modelos robustos que permitan predecir el contenido de carbono acumulado en la biomasa viva.


El carbono fijado por los árboles se acumula progresivamente en su biomasa, en función del tamaño y vigor de los individuos, los cuales están condicionados por variables ambientales, topográficas y de competencia intraespecífica.
Las condiciones meteorológicas, como la temperatura y la precipitación, inciden directamente en la fotosíntesis y en la disponibilidad hídrica;
la orientación, la pendiente y la altitud modifican la radiación incidente y el microclima local;
mientras que la densidad de árboles por unidad de superficie determina el nivel de competencia por los recursos, variando según la especie y su tolerancia ecológica \cite{IPCC2006}.


A partir de estos fundamentos, se construyó una base de datos relacional que integra información forestal, climática y espectral a nivel de parcela, especie y clase diamétrica.
Esta estructura permite caracterizar con precisión la dinámica del bosque entre inventarios sucesivos y alimentar modelos predictivos capaces de estimar el contenido futuro de carbono a partir de las condiciones observadas en el pasado.


\subsection{Origen y estructura de los datos}
% Origen (IFN2/3/4 + clima + índices)
% Estructura relacional (parcelas, inventarios, especie, CD, estación, árbol)
% Referencia al esquema (figura) y a meta_variables


La base de datos empleada en este trabajo integra información forestal, climática y espectral estructurada en torno a la parcela como unidad básica. Cada parcela se describe mediante sus coordenadas geográficas, características edáficas y su evolución a través de distintos inventarios (IFN2, IFN3, IFN4).


Los datos forestales incluyen información por especie y clase diamétrica, como número de pies, volumen con y sin corteza, área basimétrica, carbono aéreo, radical y total. Estos valores permiten caracterizar con precisión la estructura y crecimiento de la vegetación.


A cada parcela se asocian también estadísticas climáticas agregadas por estación e inventario: temperaturas (superficie, aire y subsuelo) y precipitaciones, resumidas mediante métricas como media, máxima, mínima y desviación típica.


Finalmente, se incorporan índices espectrales derivados de imágenes satelitales (NDVI, EVI, NDII, GNDVI), que permiten cuantificar propiedades biofísicas de la vegetación:
\begin{itemize}
    \item \textbf{NDVI (Normalized Difference Vegetation Index):} estima la actividad fotosintética.
    \item \textbf{EVI (Enhanced Vegetation Index):} mejora la sensibilidad en zonas densamente vegetadas.
    \item \textbf{NDII (Normalized Difference Infrared Index):} refleja el contenido hídrico de la vegetación.
    \item \textbf{GNDVI (Green NDVI):} variante del NDVI basada en la banda verde, sensible al clorofila.
\end{itemize}


\subsubsection{Estrutura de la base de datos}
Estos datos se organizan en las siguientes entidades troncales:

\begin{itemize}
    \item \textbf{parcelas}: icontiene la información básica de localización y características edáficas de cada parcela.
    \item \textbf{parcela\_inventario}: describe el estado de cada parcela en un inventario determinado (\texttt{parcela\_id}, \texttt{inventario\_id}), incluyendo atributos edáficos y de contexto (p. ej., \texttt{nivel1\_id}, \texttt{textura\_id}).
    \item \textbf{parcela\_inventario\_especie}: detalla la presencia y condición de cada especie dentro de una parcela e inventario, incorporando descriptores de masa y tratamientos silvícolas.
    \item \textbf{parcela\_inventario\_especie\_cd}: describe las poblaciones arbóreas por parcela, especie y \emph{clase diamétrica} (\texttt{cd\_id}): n.º de pies (\texttt{npies}), área basimétrica (\texttt{abas}), volúmenes (\texttt{vcc}, \texttt{vsc}, \texttt{vle}), incrementos (\texttt{iavc}) y carbono (\texttt{ca}, \texttt{cr}).
    \item \textbf{parcela\_especie\_arbol}: caracteriza los pies mayores identificados por parcela y especie en el inventario cuarto. Recoge las caracteristicas particulares de cada pie como altura (\texttt{ht}), diámetros (\texttt{dn1} y \texttt{dn2}), ubicación respecto del centro de la parcela (\texttt{rumbo}, \texttt{distancia}), volumen (\texttt{vcc, vsc, vle}), incremento (\texttt{iavc}) y carbono (\texttt{ca, cr}).
    \item \textbf{parcela\_inventario\_estacion}: almacena agregados climático-biofísicos por estación (\texttt{estacion\_id}) en la misma granularidad parcela–inventario, incluyendo variables como precipitación (\texttt{PR}) y temperatura (\texttt{T2M, SKT, STL*}), junto a índices de vegetación (NDVI, EVI, NDII, GNDVI).
    \item \textbf{especies} y \textbf{grupos}: recogen la información taxonómica y su clasificación jerárquica, estableciendo la relación entre especies individuales y grupos funcionales.
\end{itemize}

Cada variable categórica posee una tabla de catálogo propia (\texttt{cat\_}), donde se definen los valores posibles y sus descripciones. Por ejemplo, \texttt{cat\_textura}, \texttt{cat\_nivel1}, \texttt{cat\_tratmasa} o \texttt{cat\_origen}. Todas siguen un patrón uniforme: la clave primaria es el identificador de la variable (\texttt{<variable>\_id}), y las tablas troncales referencian este mismo campo como clave foránea. Además la base de datos incluye una tabla llamada \texttt{meta\_variables} que recoge los metadatos.


La Figura~\ref{fig:GWest_BBDD} muestra el esquema general de las tablas troncales y sus principales relaciones. Este diagrama resume la estructura interna de la base de datos y su jerarquía de dependencias.

\begin{figure}[htbp]
    \centering
    \includegraphics[width=0.9\textwidth]{figuras/Estrctr_BBDD_GWest.png}
    \caption{Esquema relacional de las tablas principales de la base de datos. Tabla extraida de \cite{greenwestdb}, donde se pueden consultar más detalles sobre las variables.}
    \label{fig:GWest_BBDD}
\end{figure}

\subsubsection{Diccionario resumido de variables}
\footnotesize
\setlength{\LTcapwidth}{\textwidth}
\begin{longtable}{@{}p{2.8cm} p{6.8cm} p{2.2cm} p{2.2cm}@{}}
    \caption{Resumen de variables principales por entidad. Tabla extraida de \cite{greenwestdb}.}                                                                                                  \\
    \toprule
    \textbf{Variable}                                                      & \textbf{Descripción}                                           & \textbf{Unidad}              & \textbf{Tipo de dato} \\
    \midrule
    \endfirsthead
    \toprule
    \textbf{Variable}                                                      & \textbf{Descripción}                                           & \textbf{Unidad}              & \textbf{Tipo de dato} \\
    \midrule
    \endhead
    \midrule
    \multicolumn{4}{r}{\emph{Continúa en la siguiente página}}                                                                                                                                     \\
    \midrule
    \endfoot
    \bottomrule
    \endlastfoot

    \multicolumn{4}{l}{\textbf{parcelas}}                                                                                                                                                          \\
    \texttt{parcela\_id}                                                   & Identificador único de parcela (IFN).                          & --                           & Identificador         \\
    \texttt{latitud}, \texttt{longitud}                                    & Coordenadas geográficas (WGS84).                               & °                            & Geográfico            \\
    \texttt{coorx}, \texttt{coory}                                         & Coordenadas UTM; \texttt{huso} especifica zona.                & m (UTM)                      & Geográfico            \\
    \texttt{elevacion}                                                     & Cota sobre el nivel del mar (NASADEM).                         & m                            & Numérico              \\
    \texttt{pendiente}                                                     & Inclinación del terreno.                                       & °                            & Numérico              \\
    \texttt{orientacion}                                                   & Orientación del terreno (0–360).                               & °                            & Numérico              \\
    \texttt{presencia\_id}                                                 & Presencia en IFN $\rightarrow$ \texttt{cat\_presencia}.        & --                           & Categórico            \\
    \texttt{tipsuelo1\_id}, \texttt{tipsuelo2\_id}, \texttt{tipsuelo3\_id} & Tipos de suelo $\rightarrow$ \texttt{cat\_tipsuelo*}.          & --                           & Categórico            \\
    \texttt{rocosidad\_id}                                                 & Rocosidad $\rightarrow$ \texttt{cat\_rocosidad}.               & --                           & Categórico            \\
    \texttt{radio}, \texttt{superficie}                                    & Radio de parcela y superficie derivada.                        & m; ha                        & Numérico              \\
    \addlinespace

    \multicolumn{4}{l}{\textbf{parcela\_inventario}}                                                                                                                                               \\
    \texttt{parcela\_id}, \texttt{inventario\_id}                          & Clave compuesta (parcela-inventario).                          & --                           & Identificador         \\
    \texttt{ano}                                                           & Año de apeo.                                                   & año                          & Numérico              \\
    \texttt{nivel1\_id}, \texttt{nivel2\_id}                               & Morfoestructura. $\rightarrow$ \texttt{cat\_nivel*}.           & --                           & Categórico            \\
    \texttt{textura\_id}                                                   & Textura de suelo $\rightarrow$ \texttt{cat\_textura}.          & --                           & Categórico            \\
    \texttt{merosiva\_id}                                                  & Manifestaciones erosivas $\rightarrow$ \texttt{cat\_merosiva}. & --                           & Categórico            \\
    \texttt{matorg\_id}                                                    & Materia orgánica $\rightarrow$ \texttt{cat\_matorg}.           & --                           & Categórico            \\
    \texttt{modcomb\_id}                                                   & Modelo de combustible $\rightarrow$ \texttt{cat\_modcomb}.     & --                           & Categórico            \\
    \texttt{disesp\_id}                                                    & Distribución espacial $\rightarrow$ \texttt{cat\_disesp}.      & --                           & Categórico            \\
    \texttt{comesp\_id}                                                    & Composición específica $\rightarrow$ \texttt{cat\_comesp}.     & --                           & Categórico            \\
    \texttt{fccarb}, \texttt{fcctot}                                       & Fracción de cabida cubierta (árboles).                         & \%                           & Numérico              \\
    \addlinespace

    \multicolumn{4}{l}{\textbf{parcela\_inventario\_especie}}                                                                                                                                      \\
    \texttt{parcela\_id}, \texttt{inventario\_id}, \texttt{especie\_id}    & Clave compuesta (parcela-inventario-especie).                  & --                           & Identificador         \\
    \texttt{ocupa}                                                         & Grado de ocupación de la especie.                              & (0--10)                      & Numérico              \\
    \texttt{estado\_id}                                                    & Estado de desarrollo. $\rightarrow$ \texttt{cat\_estado}.      & --                           & Categórico            \\
    \texttt{fpmasa\_id}                                                    & Forma principal de masa $\rightarrow$ \texttt{cat\_fpmasa}.    & --                           & Categórico            \\
    \texttt{tratmasa\_id}                                                  & Tratamientos selvícolas $\rightarrow$ \texttt{cat\_tratmasa}.  & --                           & Categórico            \\
    \texttt{orgmasa1\_id}                                                  & Origen de masa (IFN3/4)$\rightarrow$ \texttt{cat\_orgmasa1}.   & --                           & Categórico            \\
    \texttt{masa\_id}                                                      & Clasificación de masa $\rightarrow$ \texttt{cat\_masa}.        & --                           & Categórico            \\
    \texttt{origen\_id}                                                    & Origen de la masa (IFN2) $\rightarrow$ \texttt{cat\_origen}.   & --                           & Categórico            \\
    \addlinespace

    \multicolumn{4}{l}{\textbf{parcela\_inventario\_especie\_cd}}                                                                                                                                  \\
    \texttt{parcela\_id}, \texttt{inventario\_id}, \texttt{especie\_id}    & Clave compuesta ( parcela-inventario-especie-cd).              & --                           & Identificador         \\
    \texttt{cd\_id}                                                        & Clase diamétrica (CD) reglamento IFN.                          & cm                           & Numérico discreto     \\
    \texttt{npies}                                                         & Número de pies.                                                & pies/ha                      & Numérico              \\
    \texttt{abas}                                                          & Área basimétrica.                                              & m$^{2}$/ha                   & Numérico              \\
    \texttt{vcc}, \texttt{vsc}, \texttt{vle}                               & Volúmenes (con/sin corteza; leñas).                            & m$^{3}$/ha                   & Numérico              \\
    \texttt{iavc}                                                          & Incremento anual del volumen con corteza.                      & m$^{3}$/ha$\cdot$año         & Numérico              \\
    \texttt{ca}, \texttt{cr}                                               & Carbono aéreo y radical.                                       & t/ha                         & Numérico              \\
    \texttt{ht}                                                            & Altura media (modelo CatBoost).                                & m                            & Numérico              \\
    \texttt{carbono\_bruto}                                                & Carbono total estimado (alometrías).                           & t                            & Numérico              \\
    \addlinespace

    \multicolumn{4}{l}{\textbf{parcela\_especie\_arbol}}                                                                                                                                           \\
    \texttt{parcela\_id}, \texttt{especie\_id}                             & Clave compuesta (parcela–especie–árbol).                       & --                           & Identificador         \\
    \texttt{arbol\_id}                                                     & Identificador del árbol dentro de parcela y especie.           & --                           & Entero                \\
    \texttt{rumbo}                                                         & Rumbo desde el centro de la parcela al árbol.                  & grados centesimales          & Numérico              \\
    \texttt{distancia}                                                     & Distancia desde el centro de la parcela al árbol.              & m                            & Numérico              \\
    \texttt{cd}                                                            & Clase diamétrica (reglamento IFN).                             & cm                           & Numérico discreto     \\
    \texttt{ht}                                                            & Altura total del árbol inventariado.                           & m                            & Numérico              \\
    \texttt{dn1}, \texttt{dn2}                                             & Diámetros normales perpendiculares.                            & mm                           & Numérico              \\
    \texttt{abas}                                                          & Área basimétrica del pie medido.                               & m$^{2}$                      & Numérico              \\
    \texttt{iavc}                                                          & Incremento anual del volumen con corteza.                      & dm$^{3}$/año                 & Numérico              \\
    \texttt{vcc}, \texttt{vsc}, \texttt{vle}                               & Volúmenes (con corteza, sin corteza, leñas).                   & dm$^{3}$                     & Numérico              \\
    \texttt{ca}, \texttt{cr}                                               & Carbono aéreo y radical del árbol.                             & t                            & Numérico              \\
    \addlinespace

    \multicolumn{4}{l}{\textbf{parcela\_inventario\_estacion}}                                                                                                                                     \\
    \texttt{parcela\_id}, \texttt{inventario\_id}, \texttt{estacion\_id}   & Clave compuesta (agregado estacional).                         & --                           & Identificador         \\
    \texttt{PR\_*}                                                         & Estadísticos de precipitación (mean, max, min, std, sum).      & mm/(m$^2\cdot$día), mm/m$^2$ & Numérico              \\
    \texttt{T2M\_*}, \texttt{SKT\_*}                                       & Aire 2\,m y temperatura superficial (mean, max, min, std).     & °C                           & Numérico              \\
    \texttt{STL1\_*}--\texttt{STL4\_*}                                     & Temperatura del suelo por niveles (mean, max, min, std).       & °C                           & Numérico              \\
    \texttt{NDVI\_*}, \texttt{EVI\_*}, \texttt{NDII\_*}, \texttt{GNDVI\_*} & Índices de vegetación (max, mean, median, min, std).           & adimensional                 & Numérico              \\
    \addlinespace

    \multicolumn{4}{l}{\textbf{especies} y \textbf{grupos}}                                                                                                                                        \\
    \texttt{especie\_id}                                                   & Identificador de especie IFN.                                  & --                           & Identificador         \\
    \texttt{nombre}, \texttt{sinonimia}                                    & Denominación IFN y sinónimos.                                  & --                           & Texto                 \\
    \texttt{tipo\_especie}                                                 & 0\,= conífera; 1\,= frondosa.                                  & --                           & Categórico            \\
    \texttt{grupo\_id}                                                     & Grupo funcional $\rightarrow$ \texttt{grupos}.                 & --                           & Identificador         \\
    \texttt{grupos.nombregrupo}                                            & Nombre del grupo.                                              & --                           & Texto                 \\
\end{longtable}
\normalsize

\subsubsection{Cardinalidad y completitud}

El volumen de entradas por tabla es:
\begin{center}
    \begin{tabular}{l r}
        \toprule
        \textbf{Tabla}                            & \textbf{Número de registros} \\
        \midrule
        \texttt{parcelas}                         & 52{,}298                     \\
        \texttt{parcela\_inventario}              & 147{,}995                    \\
        \texttt{parcela\_inventario\_especie}     & 417{,}119                    \\
        \texttt{parcela\_inventario\_especie\_cd} & 1{,}191{,}070                \\
        \texttt{parcela\_especie\_arbol}          & 855{,}860                    \\
        \texttt{parcela\_inventario\_estacion}    & 470{,}056                    \\
        \texttt{especies}                         & 195                          \\
        \texttt{grupos}                           & 33                           \\
        \bottomrule
    \end{tabular}
\end{center}

\subsection{Variables objetivo}

El objetivo del modelo es estimar el \textbf{carbono total} que una parcela forestal puede capturar en un horizonte temporal de 20--30 años, a partir de las condiciones observadas en inventarios previos.
Para ello se definieron dos variables de respuesta complementarias, ambas derivadas de los datos del Inventario Forestal Nacional (IFN), que permiten analizar el contenido de carbono desde perspectivas distintas: una normalizada por superficie y otra en términos absolutos.


\begin{enumerate}
    \item \textbf{\texttt{c}} (tC/ha): representa el \textbf{carbono total contenido en la biomasa viva aérea y subterránea} por unidad de superficie, expresado en \emph{toneladas de carbono por hectárea}.
          Su cálculo se basa en la suma de las estimaciones de carbono aéreo (\texttt{ca}) y radical (\texttt{cr}) reportadas por el IFN.
          En los casos con valores faltantes, se completó la información mediante un modelo de \emph{Random Forest Regressor} ajustado sobre variables dendrométricas observadas (Especie, CD, VSC, NPies, ABas, IAVC, VCC y VLE), alcanzando un rendimiento satisfactorio (\(R^2_{test} > 0.90\)).
          Esta variable es coherente con los formatos internacionales de reporte de inventarios forestales y permite comparar el contenido de carbono entre parcelas o especies.


    \item \textbf{\texttt{carbono\_bruto}} (tC): corresponde al \textbf{carbono total capturado por parcela y especie}, expresado en \emph{toneladas de carbono totales}.
          Su estimación se realiza de forma trazable y físicamente interpretable a partir de variables medidas directamente en campo: número de pies (\texttt{npies}), altura media (\texttt{ht}), tipo de especie (\texttt{clase\_especie}) y clase diamétrica (\texttt{cd\_id}).
          El cálculo sigue un modelo alométrico adaptado de \cite{chave2014} y las directrices del IPCC~\cite{ipcc2006}, incorporando tanto la biomasa aérea como la biomasa radical mediante la relación Parte Radical:Parte Aérea ($R$).
          El resultado se expresa en toneladas de carbono totales por parcela, sin normalizar por superficie, lo que facilita la trazabilidad del proceso y la comparación entre inventarios sin depender de factores de expansión específicos del IFN.
          En coherencia con los criterios de proyectos de forestación y reforestación, las observaciones correspondientes a brinzales o plantones se consideran con valor de carbono nulo, dado que las fases tempranas de desarrollo no se contabilizan oficialmente como carbono capturado.
\end{enumerate}


Estas dos variables resumen el contenido de carbono forestal desde enfoques complementarios:
\texttt{c} (tC/ha) permite la comparación espacial y temporal entre masas forestales, mientras que \texttt{carbono\_bruto} (tC) ofrece una medida absoluta y directamente derivada de las observaciones de campo.
Ambas constituyen los objetivos principales del modelado predictivo, orientado a estimar el carbono acumulado en el \textbf{IFN4} a partir de las condiciones registradas en los inventarios anteriores (\textbf{IFN2} e \textbf{IFN3}).

\subsection{Supuestos de elegibilidad y verificación externa}
% Intervención humana, permanencia 30a (extrapolación cauta), superficie>=1ha,
% fccarb>=20% (filtro), altura>=3m en madurez (decisión de diseño).

Para que un proyecto forestal sea elegible en programas de \emph{créditos de carbono}, debe cumplir requisitos técnicos establecidos por marcos regulatorios internacionales \cite{IPCC2006, miteco_guia_co2}. A continuación se resume cada criterio y la forma en que se aborda en este estudio:

\begin{itemize}
    \item \textbf{Intervención humana directa.} El incremento de carbono debe proceder de actuaciones planificadas (reforestación, restauración o manejo sostenible). En nuestro caso, el modelo se entrena sobre datos observacionales (IFN2--IFN3--IFN4); por tanto, la \emph{verificación de intervención} no se deduce del modelo, sino que se contempla como \emph{condición externa} de elegibilidad del proyecto a evaluar.

    \item \textbf{Permanencia mínima de 30 años.} Para caracterizar el crecimiento de las parcelas forestales en los datos que alimentan el modelo, es necesario disponer de dos mediciones sucesivas de cada parcela, separadas por un intervalo temporal conocido. Estas mediciones permiten cuantificar la evolución de las variables forestales y, por tanto, estimar el incremento de carbono asociado al crecimiento del arbolado durante dicho periodo.

          En este trabajo, el objetivo es predecir el contenido de carbono correspondiente al \textbf{IFN4}, utilizando como información explicativa las variables observadas en inventarios anteriores. Dado que los inventarios tercero y cuarto comparten una estructura homogénea y un conjunto de variables comparable la elección más directa para el entrenamiento del modelo sería emplear exclusivamente estos dos inventarios. Esta estrategia aprovecha la coherencia estructural de los inventarios más recientes, que incluyen un mayor número de variables y una caracterización más detallada del terreno.


          No obstante, este planteamiento se enfrenta a la limitación impuesta por la \textbf{permanencia mínima de 30 años}, requisito fundamental en el contexto de los proyectos de compensación. El intervalo de tiempo entre los inventarios \textbf{IFN3} e \textbf{IFN4} es relativamente corto: no supera los 18 años.


          La Figura~\ref{fig:periodo34} muestra la distribución de la diferencia de años entre las mediciones del IFN3 y el IFN4. Como puede observarse, la mayoría de las parcelas presentan intervalos comprendidos entre 6 y 17 años, un rango demasiado estrecho para evaluar la estabilidad del modelo en horizontes más amplios.

          TODO: Actualizar figura y quitar título de la propia imagen
          \begin{figure}[h!]
              \centering
              \includegraphics[width=0.9\textwidth]{figuras/periodo34.png}
              \caption{Distribución de la diferencia de años entre los inventarios IFN3 e IFN4.}
              \label{fig:periodo34}
          \end{figure}

          Para ampliar la cobertura temporal y mejorar la capacidad de generalización del modelo, se optó por unificar la información de los inventarios \textbf{IFN2} e \textbf{IFN3} como base explicativa para la predicción del \textbf{IFN4}. Esta integración permite disponer de pares de mediciones de parcelas separadas por intervalos que oscilan entre 6 y 29 años, lo que constituye un rango mucho más representativo del horizonte de 20--30 años establecido como referencia.

          TODO: Actualizar figura y quitar título de la propia imagen
          \begin{figure}[h!]
              \centering
              \includegraphics[width=0.9\textwidth]{figuras/periodo234.png}
              \caption{Distribución de la diferencia de años entre los inventarios IFN2--IFN3 e IFN3--IFN4.}
              \label{fig:periodo234}
          \end{figure}


          De esta forma, el modelo se entrena y valida sobre un conjunto de datos más diverso y equilibrado, tanto en estructura como en amplitud temporal, manteniendo la coherencia metodológica y la trazabilidad de las estimaciones. Este enfoque no sólo mejora la robustez del aprendizaje, sino que también refuerza la capacidad del modelo para proyectar la captura de carbono en escenarios compatibles con los requisitos de permanencia de los proyectos de compensación.

    \item \textbf{Superficie mínima de 1 ha.} Este criterio se considera \emph{externo} al alcance del modelo predictivo, ya que el aprendizaje se realiza a nivel de parcela e inventario y no sobre polígonos de superficie total. En la práctica, la verificación de la superficie se realiza \emph{ex ante}, sobre la geometría declarada del proyecto forestal. En los terrenos forestales generados a partir de intervención humana directa —como plantaciones o repoblaciones—, la extensión suele presentar una estructura homogénea, con una especie dominante, edades coetáneas y densidades estandarizadas. Bajo estas condiciones, el carbono total es proporcional a la superficie: duplicar el área de una masa forestal homogénea implica aproximadamente duplicar su carbono almacenado. Por tanto, la variable de superficie no afecta al ajuste interno del modelo y su cumplimiento puede evaluarse fácilmente a nivel de proyecto, sin comprometer la validez de las predicciones.

    \item \textbf{Fracción mínima de cabida cubierta del 20\%.} La base de datos dispone de \texttt{fccarb} (arbórea) y \texttt{fcctot} (total). Este umbral se aplica como \emph{filtro de elegibilidad} previo o posterior al modelado, sin modificar la arquitectura del modelo (\texttt{fccarb}$>20$).

    \item \textbf{Altura mínima de 3 m en la madurez.} Este requisito se refiere a la altura que alcanzan los árboles en su fase de pleno desarrollo, y no a la altura inicial de los plantones. Por tanto, las mediciones realizadas durante las etapas tempranas de crecimiento no determinan la elegibilidad del proyecto, siempre que las especies seleccionadas sean capaces de superar los 3 metros en la madurez. En nuestro conjunto de datos, la altura no se registra explícitamente, por lo que este criterio se evalúa de forma \emph{externa} al modelo, mediante la selección de especies forestales adecuadas y la verificación con fuentes auxiliares (catálogos silvícolas o tipologías de masa). En la práctica, el cumplimiento del requisito depende de una decisión de diseño del proyecto —\emph{no plantar especies cuyo tamaño adulto sea inferior a 3 metros}— más que del ajuste predictivo del modelo. Por ello, la altura no interviene directamente en el entrenamiento, aunque sí condiciona la elegibilidad final del proyecto forestal.
\end{itemize}

\subsection{Preparación y tratamiento de los datos}
% Filtros (fccarb>=20%, crecimiento positivo, etc.)
% Agregaciones (parcela-especie, compresión CD -> npies_{cd}, etc.)
% Cálculo de variables derivadas (carbono\_bruto)
% Codificación y escalado

Como ya se ha introducido el entrenamiento se realiza en dos líneas según la variable objetivo: \texttt{c} de \textbf{IFN4} o \texttt{carbono\_bruto} de \textbf{IFN4}; y según la información que se usa como explicativa: \textbf{IFN3} o \textbf{IFN3} e \textbf{IFN2}. Se plantea la preparación y filtrado de los datos en términos generales (variable objetivo por \texttt{c} o \texttt{carbono\_bruto} y primera inventariación/ inventariación explicativa por \textbf{IFN3} o la unión de \textbf{IFN2} e \textbf{IFN3}).

\subsubsection{Filtrado de registros}\label{subsec:filtrado_registros}

Se descartan todas aquellas parcelas en las que el valor de carbono total (variable objetivo) en la segunda inventariación es inferior a la primera. Estos casos suelen deberse a episodios de deforestación, incendios u otras perturbaciones, y no representan un crecimiento forestal neto.


El conjunto de datos se restringe únicamente a las parcelas que presentan una \texttt{fccarb} (fracción de cabida cubierta arbórea) igual o superior al 20\,\% en el \textbf{IFN3}. Este umbral define la proporción mínima de superficie ocupada por copas de árboles respecto al área total de la parcela, y constituye una de las condiciones esenciales para considerar una superficie como terreno forestal. La exclusión de parcelas con \texttt{fccarb} inferior al 20\,\% permite asegurar que las estimaciones de carbono se realicen sobre masas forestales consolidadas, evitando sesgos asociados a áreas agrícolas o matorrales. A los datos del \textbf{IFN2} no se les aplica dicho filtro porque no disponen de la variable \texttt{fccarb}.


\subsubsection{Cálculo y agregación de variables}

Cada registro de entrada se genera a nivel de combinación parcela--especie, incorporando las variables correspondientes de la primera medición y la variable objetivo (carbono) de la segunda medición (IFN4). Las variables de \texttt{parcela} y \texttt{parcela\_inventario} se desdoblan para cada especie. Las entradas de la tabla \texttt{parcela\_inventario\_especie\_cd} se agrupan por parcela y especie y se comprimen en una única entrada creando un conjunto de variables para cada clase diamétrica.


La Tabla~\ref{tab:entrada_modelo} resume las variables empleadas como entrada al modelo, integradas desde las distintas tablas que conforman la base de datos relacional.

\begin{table}[htbp]
    \renewcommand{\arraystretch}{1.2}
    \setlength{\tabcolsep}{3pt}
    \centering
    \footnotesize
    \begin{tabular}{|p{3.2cm}|p{1.8cm}|p{6.5cm}|p{2.2cm}|}
        \hline
        \multicolumn{4}{|c|}{\textbf{Resumen de Datos de Entrada del Modelo}} \\
        \hline
        \textbf{Variable} & \textbf{Tipo} & \textbf{Descripción} & \textbf{Anexo} \\
        \hline
        \textcolor{ForestGreen}{\texttt{parcela\_id}} & varchar & Identificador único de parcela. & -- \\
        \hline
        \textcolor{ForestGreen}{\texttt{especie\_id}, \texttt{tipo\_especie}, \texttt{grupo\_id}} & int (CF) & Especie, tipo y grupo taxonómico. & Anexos \ref{sec:especies}, \ref{sec:gruposespecies} \\
        \hline
        \textcolor{ForestGreen}{\texttt{ocupa}} & int & Grado de ocupación (0--10). & -- \\
        \hline
        \textcolor{ForestGreen}{\texttt{estado\_id}}, \texttt{fpmasa\_id}, \texttt{tratmasa\_id}, \texttt{orgmasa\_1\_id} & int (CF) & Estado, forma de masa, tratamiento, organización. & Anexos \ref{sec:EstadoIFN34}, \ref{sec:FPMasa}, \ref{sec:tratmasa}, \ref{sec:OrgMasa} \\
        \hline
        \textcolor{ForestGreen}{\texttt{tipsuelo1-3\_id}} & int (CF) & Tipos de suelo. & Anexo \ref{sec:TipSuelo} \\
        \hline
        \textcolor{ForestGreen}{\texttt{rocosidad\_id}}, \texttt{textura\_id}, \texttt{matorg\_id}, \texttt{modcomb\_id}, \texttt{disesp\_id}, \texttt{comesp\_id}, \texttt{merosiva\_id} & int (CF) & Variables edáficas y estructurales. & Anexos varios \\
        \hline
        \textcolor{ForestGreen}{\texttt{radio}, \texttt{orientacion}, \texttt{elevacion}, \texttt{pendiente}} & float & Topografía y geometría de parcela. & -- \\
        \hline
        \texttt{nivel1\_id}, \texttt{nivel2\_id}, \texttt{fccarb}, \texttt{fcctot} & int/float & Niveles jerárquicos y cabida cubierta. & Anexos \ref{sec:nivel1}, \ref{sec:nivel2} \\
        \hline
        \textcolor{ForestGreen}{\texttt{npies\_\{CD\}}} & float & N.º de pies por clase diamétrica. & -- \\
        \hline
        \textcolor{ForestGreen}{\texttt{periodo}} & int & Años entre inventarios. & -- \\
        \hline
        \textcolor{ForestGreen}{\texttt{evi, gndvi, ndii, ndvi\_\{stat\}\_\{est\}}} & float & Índices de vegetación por estación. & -- \\
        \hline
        \textcolor{ForestGreen}{\texttt{pr, skt, stl1-4, t2m\_\{stat\}\_\{est\}}} & float & Variables climáticas por estación. & -- \\
        \hline
        \textcolor{ForestGreen}{\texttt{c4}, \texttt{carbono\_bruto4}} & float & Carbono IFN4 (t/ha y t). & -- \\
        \hline
    \end{tabular}
    \caption{Variables de entrada del modelo. Las variables en \textcolor{ForestGreen}{verde} están disponibles en IFN2 e IFN3; el resto solo en IFN3.}
    \label{tab:entrada_modelo}
\end{table}

\subsubsection{Codificación y normalización}

Las variables categóricas se codifican mediante \textit{one-hot encoding}, generando variables binarias para cada clase. Las variables numéricas se escalan (normalización estándar o min-max, según el modelo) para asegurar que todas las magnitudes tengan el mismo orden de importancia durante el entrenamiento.

\subsubsection{Reclasificación de las variables \texttt{pendiente} y \texttt{orientacion}}

Las variables topográficas originales \texttt{pendiente} (en grados) y \texttt{orientacion} (acimut en grados) se registran de forma continua en las parcelas del IFN. Sin embargo, desde el punto de vista ecológico su efecto sobre la acumulación de carbono suele ser no lineal y está asociado a clases discretas (e.g.\ laderas suaves frente a escarpadas, exposición norte frente a sur), por lo que resulta más adecuado tratarlas como factores categóricos.

A partir de la distribución empírica y de criterios habituales en estudios de fisiografía forestal, se definió una variable categórica \texttt{pendiente\_cat} mediante cortes en grados:

\begin{itemize}
    \item \(\textless 5^\circ\): \textit{muy suave},
    \item \(5{-}10^\circ\): \textit{suave},
    \item \(10{-}15^\circ\): \textit{moderada},
    \item \(15{-}20^\circ\): \textit{fuerte},
    \item \(20{-}30^\circ\): \textit{muy fuerte},
    \item \(30{-}50^\circ\): \textit{escarpada},
    \item \(>50^\circ\): \textit{extrema}.
\end{itemize}

Esta reclasificación permite capturar diferencias funcionales relevantes (accesibilidad, estabilidad del suelo, escorrentía, profundidad efectiva del suelo) sin asumir una relación lineal entre la pendiente y el carbono almacenado.

De forma análoga, la variable \texttt{orientacion} se reclasificó en ocho sectores cardinales equiángulos: \texttt{N}, \texttt{NE}, \texttt{E}, \texttt{SE}, \texttt{S}, \texttt{SO}, \texttt{O} y \texttt{NO}. La nueva variable \texttt{orientacion\_cat} agrupa orientaciones con condiciones de insolación y balance hídrico similares, lo que facilita la interpretación ecológica y reduce el ruido asociado a pequeñas variaciones angulares.


\subsection{Partición y validación}

Para obtener una estimación imparcial del rendimiento y evitar \emph{fugas de información} debidas a la correlación espacial dentro de cada parcela, la partición del conjunto de datos se realiza \textbf{por identificador de parcela} (\texttt{parcela\_id}). Todas las observaciones asociadas a una misma parcela se asignan \emph{íntegramente} a un único subconjunto, de modo que ninguna parcela aparece simultáneamente en entrenamiento y evaluación.
\vspace{0.25em}
\noindent\textbf{Validación interna y control de sesgo temporal.} Sobre el subconjunto de entrenamiento (80\,\%) se aplica \emph{validación cruzada por grupos} utilizando como agrupador los \emph{años transcurridos entre inventarios} (p.\,ej., 15, 16, 17, \dots). Esta estrategia comprueba la \emph{estabilidad} del modelo frente a cambios en el horizonte temporal y reduce el riesgo de sobreajuste específico de un periodo. La selección de hiperparámetros se realiza exclusivamente dentro de esta validación interna; el conjunto de evaluación (20\,\%) permanece \emph{sellado} para la prueba final.

\vspace{0.25em}
\noindent\textbf{Métricas de evaluación.} El rendimiento se informa con un conjunto de medidas complementarias:

\begin{itemize}
    \item \textbf{RMSE (Root Mean Squared Error):} raíz del error cuadrático medio entre valores observados y predichos; se expresa en las mismas unidades que la variable objetivo y penaliza con mayor peso los errores grandes. Valores más bajos indican mejor ajuste.

    \item \boldmath\textbf{$R^2$}\unboldmath{} (coeficiente de determinación): proporción de la varianza observada explicada por el modelo (idealmente en $[0,1]$). Valores cercanos a 1 denotan alta capacidad explicativa; puede ser negativo si el modelo es peor que la predicción constante.

    \item \textbf{MAE (Mean Absolute Error):} media aritmética del error absoluto, que cuantifica la desviación media entre las predicciones y los valores observados. Penaliza todos los errores de forma lineal y es más interpretable que el RMSE. Valores más bajos indican mejor ajuste.

    \item \textbf{Moda del Error}: valor más frecuente del error absoluto a nivel de parcela o individuo, útil para identificar el error típico en la predicción y detectar patrones dominantes en el comportamiento del modelo.
\end{itemize}

%\vspace{0.25em}
%\noindent\textbf{Protocolo de reporte.} Para cada modelo se reportan: (i) el rendimiento medio y la dispersión en la validación cruzada por grupos (entrenamiento), y (ii) el desempeño final en el conjunto de evaluación independiente (20\,\%). Este protocolo garantiza comparabilidad entre modelos, control del sesgo espacial por parcela y verificación explícita de la robustez temporal.


\subsection{Selección de variables explicativas}

La selección de predictores se abordó mediante cuatro estrategias complementarias: (1) selección automática mediante \textit{Featurewiz}, (2) selección basada en el criterio de mínima redundancia y máxima relevancia (\textit{mRMR}), (3) selección manual basada en criterios estadísticos y conceptuales, y (4) un procedimiento secuencial supervisado fundamentado en el rendimiento predictivo (SSSRP). El objetivo común de estas aproximaciones fue identificar un subconjunto parsimonioso de variables que maximizara la capacidad predictiva del modelo, redujera la colinealidad y mantuviera la coherencia ecológica de las relaciones.

\subsubsection{Selección automática mediante Featurewiz}

El algoritmo \textit{Featurewiz} aplica un enfoque híbrido orientado a la relevancia predictiva. Primero ejecuta un filtrado por correlación, eliminando predictores altamente colineales (umbral \( |r| > 0.70 \)), y posteriormente refina el conjunto mediante modelos de \textit{Gradient Boosting} para estimar la importancia relativa de cada variable. El resultado es un subconjunto compacto de predictores con contribución significativa al rendimiento del modelo.

\subsubsection{Selección mediante mRMR}

El método \textit{mRMR} (minimum Redundancy–maximum Relevance) selecciona las variables que mejor explican la variabilidad del objetivo a la vez que minimizan la redundancia informativa entre ellas. Para ello emplea información mutua, permitiendo capturar relaciones potencialmente no lineales. Este enfoque prioriza predictores que aportan información complementaria sobre el proceso ecológico modelado, evitando duplicidades entre atributos altamente correlacionados.

\subsubsection{Selección manual basada en criterios estadísticos y conceptuales}

La selección manual integró criterios estadísticos (correlaciones, ANOVA y análisis de redundancia) con criterios ecológicos y de interpretabilidad. Se descartaron predictores sin asociación significativa con la variable objetivo y se redujo la colinealidad reteniendo un único representante por cada grupo altamente correlacionado. Asimismo, se garantizaron variables que describieran dimensiones esenciales del sistema (estructura del arbolado, topografía, suelo, clima e índices espectrales), asegurando un equilibrio entre precisión predictiva y coherencia biogeográfica.

\subsubsection{Selección Secuencial Supervisada basada en Rendimiento Predictivo (SSSRP)}

El método SSSRP complementó las estrategias anteriores mediante un enfoque explícitamente orientado al rendimiento predictivo. Se partió de un \emph{bloque base} de variables estructurales y se evaluó el impacto marginal de cada candidato añadiéndolo individualmente y comparando el cambio en \(R^{2}\) y RMSE mediante un modelo CatBoost con validación holdout estratificada por parcela. A continuación, se aplicó una estrategia de \textit{forward selection} codiciosa, incorporando en cada iteración la variable que proporcionaba la mayor mejora y deteniendo el proceso cuando la ganancia resultaba inferior a un umbral predefinido (\(\Delta R^{2} > 10^{-5}\)). Este procedimiento produjo un conjunto final de predictores reducido, no redundante y específicamente optimizado para maximizar el rendimiento del modelo.


\subsection{Modelos evaluados}

A continuación se describe el procedimiento seguido para la selección, optimización y combinación de modelos. El objetivo es construir un conjunto de predictores base sólidos y posteriormente integrarlos en un \textit{stack-ensemble} capaz de mejorar la capacidad de generalización.

\subsubsection{Modelos ensemble}

Se utilizaron diversos métodos de \textit{ensemble learning} con el fin de aumentar precisión y robustez del sistema predictivo. El principio fundamental consiste en combinar predicciones de múltiples modelos, aprovechando su diversidad para reducir varianza, sesgo o ambos.

\paragraph{Técnicas empleadas:}
\begin{itemize}
    \item \textbf{Bagging:} entrena modelos independientes sobre subconjuntos generados mediante muestreo bootstrap. Reduce varianza y mejora estabilidad.
    \item \textbf{Boosting:} construye modelos secuenciales donde cada uno corrige los errores del anterior. Tiende a reducir el sesgo y producir modelos altamente precisos.
    \item \textbf{Stacking:} integra múltiples modelos base mediante un meta-modelo entrenado sobre sus predicciones. Permite capturar relaciones no lineales entre las salidas de los modelos base.
\end{itemize}

\subsubsection{Boosting y aprendizaje secuencial}

El conjunto de modelos de boosting evaluados incluye:

\begin{itemize}
    \item \textbf{XGBoost:} implementación avanzada del \textit{gradient boosting}, que incorpora regularización L1/L2, optimización mediante segundo orden y manejo interno de valores faltantes.
    \item \textbf{LightGBM:} algoritmo especialmente eficiente, basado en crecimiento \textit{leaf-wise}, capaz de manejar grandes volúmenes de datos y con soporte nativo para variables categóricas.
    \item \textbf{CatBoost:} optimizado para variables categóricas y robusto frente a ruido mediante técnicas como \textit{ordered boosting}.
    \item \textbf{Gradient Boosting Decision Trees (GBDT):} implementación clásica del algoritmo basado en descenso por gradiente sobre residuos.
    \item \textbf{AdaBoost:} técnica que ajusta modelos simples (stumps) secuencialmente, asignando más peso a observaciones difíciles.
\end{itemize}

\subsubsection{Bagging}

Los modelos basados en bootstrap empleados fueron:

\begin{itemize}
    \item \textbf{Random Forest:} conjunto de árboles de decisión que introduce aleatoriedad tanto en datos como en características. Suele ser robusto y relativamente estable.
    \item \textbf{Bagged Decision Trees (BaggedDT):} árboles no podados entrenados sobre muestras bootstrap, cuyas predicciones se promedian para reducir varianza.
\end{itemize}

\subsubsection{Otros modelos evaluados}

Además de los métodos ensemble, se evaluaron modelos representativos de paradigmas adicionales:

\begin{itemize}
    \item \textbf{Support Vector Regression (SVR):} modelo de márgenes para regresión, evaluado con kernel lineal.
    \item \textbf{K-Nearest Neighbors (KNN):} modelo basado en vecinos más próximos; útil como referencia no paramétrica, aunque sensible a la escala.
    \item \textbf{Multi-Layer Perceptron (MLP):} red neuronal densa capaz de capturar relaciones no lineales.
    \item \textbf{Bayesian Neural Network (BayesianNN):} aproximación probabilística que permite cuantificar incertidumbre a través de regularización bayesiana.
\end{itemize}

\vspace{0.2cm}

\subsubsection{Configuración del \textit{stacking}}

Tras evaluar todos los modelos anteriores, se construyeron diferentes configuraciones de modelos base (\textit{base learners}) que se combinan mediante un meta-modelo. Las configuraciones empleadas son:

\begin{verbatim}
['CatBoost', 'LightGBM', 'XGBoost', 'Random Forest', 'GBDT', 'BaggedDT'],
['CatBoost', 'LightGBM', 'Random Forest', 'GBDT'],
['LightGBM', 'XGBoost', 'GBDT'],
['CatBoost', 'Random Forest', 'GBDT'],
['LightGBM', 'Random Forest']
\end{verbatim}

Estas combinaciones se diseñaron con dos criterios principales:

\begin{enumerate}
    \item \textbf{Diversidad estructural:} mezclar métodos de boosting y bagging, así como variantes de boosting con distintas estrategias de crecimiento y regularización.
    \item \textbf{Rendimiento individual:} incluir preferentemente los modelos con mayor $R^2$ y menor error (RMSE, MAE) en las pruebas individuales.
\end{enumerate}


Los meta-modelos utilizados para integrar las predicciones fueron:

\begin{itemize}
    \item \textbf{Modelos lineales:} Regresión Lineal, Ridge.
    \item \textbf{Modelos basados en árboles:} Random Forest, Gradient Boosting Regressor.
    \item \textbf{Modelos kernel:} SVR lineal.
    \item \textbf{Red neuronal:} MLP con una capa oculta.
\end{itemize}

Esta selección permite comparar desde combinadores lineales simples hasta integradores no lineales capaces de capturar interacciones complejas entre predicciones.

\subsubsection{Comparación y justificación de modelos}

La evaluación exhaustiva de múltiples algoritmos permite identificar no solo el modelo individual con mejor rendimiento, sino también combinaciones sinérgicas para el \textit{stacking}. La Tabla~\ref{tab:modelos} resume los modelos finalmente entrenados y evaluados.


\begin{table}[htbp]
    \centering
    \footnotesize
    \begin{tabular}{|p{2.5cm}|p{2.2cm}|p{4.5cm}|p{4cm}|}
        \hline
        \textbf{Modelo} & \textbf{Tipo} & \textbf{Características} & \textbf{Observaciones} \\
        \hline
        Random Forest & Bagging & Bootstrap con selección aleatoria de atributos & Robusto y estable \\
        \hline
        BaggedDT & Bagging & Árboles sin poda sobre muestras bootstrap & Mejora por agregación \\
        \hline
        XGBoost & Boosting & Regularización L1/L2, segundo orden & Muy preciso; sensible a tuning \\
        \hline
        LightGBM & Boosting & Crecimiento leaf-wise, muy eficiente & Rápido; riesgo de sobreajuste \\
        \hline
        CatBoost & Boosting & Codificación ordenada; robusto al ruido & Excelente sin gran tuning \\
        \hline
        GBDT & Boosting & Árboles secuenciales ajustados a residuos & Buen rendimiento \\
        \hline
        AdaBoost & Boosting & Aumenta peso de obs. mal predichas & Menos robusto \\
        \hline
        KNN & Instancia & Predicción por proximidad & Sensible a escala y ruido \\
        \hline
        MLP & Red neuronal & Captura relaciones no lineales & Requiere normalización \\
        \hline
        SVR & Márgenes & Kernel lineal, gran margen & Robusto al sobreajuste \\
        \hline
        BayesianNN & Probabilístico & Cuantifica incertidumbre & Reduce sobreajuste \\
        \hline
    \end{tabular}
    \caption{Resumen de los modelos de aprendizaje supervisado evaluados.}
    \label{tab:modelos}
\end{table}





\clearpage \thispagestyle{empty} \mbox{} \clearpage

\section{Implementación del \textit{pipeline}}
El desarrollo y la evaluación de los modelos predictivos se realizaron íntegramente en \textbf{Python}, utilizando librerías como \texttt{scikit-learn}, \texttt{cuML} y \texttt{PyTorch}, junto con implementaciones específicas de \textit{gradient boosting} como \texttt{XGBoost}, \texttt{LightGBM} y \texttt{CatBoost}. El proceso de entrenamiento se llevó a cabo en dos fases diferenciadas. 

Para los modelos entrenados exclusivamente con datos del IFN3 se utilizó un equipo local equipado con un procesador Intel Core i7 y 32~GB de memoria RAM. En cambio, los modelos que empleaban conjuntamente datos del IFN2 e IFN3 se entrenaron en el sistema de computación de alto rendimiento (HPC) de la Universidad de Salamanca. Esta elección se debió a la disponibilidad de tarjetas gráficas Nvidia H100, que permiten acelerar de forma significativa el entrenamiento de aquellos modelos compatibles con ejecución en GPU gracias a su elevada capacidad de paralelización. 

No obstante, cabe señalar que el entrenamiento también podría haberse realizado en un equipo de escritorio convencional equipado con una tarjeta gráfica comercial, ya que los requisitos computacionales del problema no son especialmente elevados.

\subsection{Ingeniería práctica del entrenamiento y la validación}

Desde el punto de vista de la implementación, el proceso de entrenamiento y validación se apoyó fundamentalmente en el ecosistema de \texttt{scikit-learn}, complementado con librerías especializadas para modelos de \textit{gradient boosting}. La gestión de los datos se realizó mediante \texttt{pandas} y \texttt{numpy}, mientras que el cálculo de métricas y estadísticas adicionales del error se apoyó en \texttt{scipy} y los módulos de evaluación de \texttt{sklearn.metrics}.

A partir del conjunto de datos original, se aplicaron filtros de calidad sobre la variable objetivo utilizando operaciones vectorizadas de \texttt{pandas}, eliminando observaciones con valores nulos, inconsistentes o que no cumplían los criterios definidos en la Sección~\ref{subsec:filtradoregistros}. 

La partición de los datos en conjuntos de entrenamiento y prueba se realizó mediante \texttt{GroupShuffleSplit} del módulo \texttt{sklearn.model\_selection}, con una proporción 80/20. Este esquema garantizó que todas las observaciones asociadas a una misma parcela se asignaran íntegramente a un único subconjunto, evitando fugas de información derivadas de la correlación espacial intra-parcela. Sobre el conjunto de entrenamiento se definió una validación cruzada de cinco pliegues utilizando \texttt{GroupKFold}.

El preprocesado de las variables y el ajuste de los modelos se integraron en un único objeto \texttt{Pipeline}, combinando \texttt{ColumnTransformer}, \texttt{SimpleImputer}, \texttt{StandardScaler} y \texttt{OneHotEncoder}. Esta integración aseguró que todas las transformaciones se estimaran exclusivamente con los datos de entrenamiento de cada pliegue durante la validación cruzada. El ajuste de hiperparámetros se llevó a cabo mediante \texttt{GridSearchCV}, definiendo rejillas específicas para cada algoritmo y utilizando el coeficiente de determinación ($R^2$) como métrica de selección.

Los modelos evaluados incluyen implementaciones de \textit{gradient boosting} (\texttt{XGBoost}, \texttt{LightGBM}, \texttt{CatBoost} y \texttt{GradientBoostingRegressor}), métodos basados en \textit{bagging} (\texttt{RandomForestRegressor}, \texttt{BaggingRegressor}), así como modelos de distinta naturaleza como \texttt{MLPRegressor}, \texttt{KNeighborsRegressor}, \texttt{LinearSVR}, \texttt{AdaBoostRegressor} y \texttt{BayesianRidge}. Para cada modelo se calcularon de forma sistemática las métricas de rendimiento sobre el conjunto de prueba: $R^2$, RMSE y MAE, junto con estadísticas adicionales del error absoluto (mediana y moda), almacenándose los resultados en estructuras tabulares para su análisis comparativo.

\subsection{Implementación del \textit{stacking}}

La agregación de modelos mediante \textit{stacking} se implementó de forma manual utilizando utilidades básicas de \texttt{scikit-learn}, con el objetivo de mantener un control estricto sobre el flujo de entrenamiento y validación. A partir de los mejores modelos individuales se generaron predicciones fuera de pliegue (\textit{out-of-fold}, OOF) sobre el conjunto de entrenamiento, empleando el mismo esquema de validación cruzada (\texttt{GroupKFold}).

Estas predicciones OOF se organizaron en matrices de meta-variables mediante \texttt{numpy} y se utilizaron como entrada para el entrenamiento de los metamodelos. En paralelo, cada modelo base se reentrenó sobre la totalidad del conjunto de entrenamiento para generar las correspondientes predicciones sobre el conjunto de test, que se emplearon posteriormente para la evaluación final del \textit{stack}.

Los metamodelos considerados incluyen \texttt{LinearRegression}, \texttt{Ridge}, \\
\texttt{GradientBoostingRegressor}, \texttt{RandomForestRegressor}, \texttt{SVR} con kernel lineal y \texttt{MLPRegressor}. Antes de su ajuste, las meta-variables se estandarizaron mediante \texttt{StandardScaler}, integrando este paso en un \texttt{Pipeline} específico del segundo nivel. La evaluación del \textit{stacking} se realizó exclusivamente sobre el conjunto de test independiente, calculando las métricas habituales ($R^2$, RMSE y MAE) para cada combinación de modelos base y metamodelo.


\subsection{Datos finales de entrenamiento}
\label{subsec:datos_finales_entrenamiento}

Tras aplicar los criterios de elegibilidad y filtrado descritos en la
Sección~\ref{subsec:filtrado_registros}, el conjunto de datos final utilizado
para el ajuste de los modelos queda compuesto por:

\begin{itemize}
    \item \textbf{IFN2:} Total de parcelas = \textbf{88.696}
          \begin{itemize}
              \item Casos con $c4 > c$: \textbf{31.428}
              \item Casos con $carbono\_bruto4 > carbono\_bruto$: \textbf{32.403}
          \end{itemize}

    \item \textbf{IFN3:} Total de parcelas = \textbf{171.157}
          \begin{itemize}
              \item Casos con $fccarb > 20$: \textbf{158.434}
              \item Casos con $fccarb > 20$ y $c4 > c$: \textbf{57.401}
              \item Casos con $fccarb > 20$ y $carbono\_bruto4 > carbono\_bruto$: \textbf{76.617}
          \end{itemize}
\end{itemize}

La Tabla~\ref{tab:resumen_variables} resume las principales estadísticas
descriptivas de las variables utilizadas en el modelado, adicionalmente en la Figura~\ref{fig:hist_c4} se muestra la distribución de las mismas.

\begin{table}[htbp]
    \centering
    \caption{Estadísticos descriptivos del conjunto de datos depurado.}
    \label{tab:resumen_variables}
    \begin{tabular}{lrrrrr}
        \toprule
        Variable        & N        & Media   & Desv.\ estándar & Mín.     & Máx.       \\
        \midrule
        carbono\_bruto4 & 136\,325 & 24.6168 & 35.8198         & 0.000327 & 420.498829 \\
        carbono\_bruto  & 114\,485 & 15.9326 & 26.3052         & 0        & 359.805707 \\
        c4              & 105\,714 & 38.3789 & 47.0348         & 0.484695 & 883.462735 \\
        c               & 92\,372  & 23.4399 & 34.9622         & 0        & 842.739088 \\
        periodo         & 105\,709 & 18.3167 & 6.4853          & 0        & 34         \\
        \bottomrule
    \end{tabular}
\end{table}

Se observa que la variable \texttt{carbono\_bruto4} presenta una media de 24.62 y una desviación estándar de 35.82, mientras que la variable
\texttt{c4} muestra valores notablemente superiores (media de 38.38 y desviación estándar de 47.03). La variable \texttt{c4} es más dispersa y heterogénea que \texttt{carbono\_bruto4}. En general, una mayor variabilidad en la variable objetivo se traduce en un problema de predicción más complejo, ya que el modelo debe capturar relaciones más inestables y sujetas a mayor ruido.

Por tanto, incluso antes de evaluar los modelos, es razonable esperar que una misma familia de algoritmos obtenga valores de $R^2$ más elevados y errores más bajos (RMSE, MAE) al predecir \texttt{carbono\_bruto4}, cuya estructura estadística es menos dispersa, que al predecir \texttt{c4}.

\begin{figure}[htbp]
    \centering
    \includegraphics[width=1\textwidth]{figuras/histograma_c.png}
    \caption{Distribución de las variables \texttt{c4} y \texttt{carbono\_bruto4} en el conjunto depurado.}
    \label{fig:hist_c4}
\end{figure}

Observamos una clara distribución asimétrica a la izquierda con una larga cola en ambas variables. No existe un factor de escala único que lleve de una variable a la otra porque la generalización a hectárea tiene en cuenta la densidad forestal particular de cada parcela.

\subsubsection{Efecto del periodo sobre el carbono}
\label{subsec:resultados_periodo_anova}

La influencia del \textit{periodo} sobre las variables de carbono se evaluó
mediante ANOVA de un factor. Los análisis realizados \
muestran que el \textit{periodo} ejerce un efecto significativo sobre ambas
variables. En \texttt{c4} se obtuvo un estadístico
\(F = 143.49\) \((p < 0.001)\), mientras que en \texttt{carbono\_bruto4}
el valor fue \(F = 161.08\) \((p < 0.001)\).
Estos resultados indican que las diferencias observadas entre periodos
no son aleatorias, sino que reflejan variaciones sistemáticas asociadas al
momento de muestreo, confirmando que el \textit{periodo} constituye un factor
explicativo relevante en la dinámica del carbono forestal.

\clearpage \thispagestyle{empty} \mbox{} \clearpage

\section{Entrenamiento y validación}

El proceso de entrenamiento se estructuró en varias fases orientadas a optimizar tanto la selección de variables predictoras como la robustez del modelo final. En primer lugar, se llevó a cabo una etapa de \textbf{selección de variables}, en la que se evaluaron distintos subconjuntos de características definidos por bloques temáticos con significado ecológico y funcional. Para esta tarea se adoptó un enfoque sistemático basado en la comparación del desempeño predictivo de las distintas combinaciones mediante el algoritmo \texttt{CatBoost}, seleccionado tras pruebas preliminares que mostraron su alta capacidad de ajuste y estabilidad frente a la heterogeneidad de los datos. En todas las configuraciones se mantuvo constante la variable objetivo (carbono capturado) y los parámetros del modelo, de modo que las variaciones en el coeficiente de determinación ($R^2$) y el error cuadrático medio (RMSE) reflejaran exclusivamente la contribución informativa de cada bloque. Los resultados de esta fase son preliminares ya que se emplearon entrenamientos más sencillos (sin validación cruzada para la selección de hiperparámetros).

Las configuraciones analizadas incorporaron progresivamente variables relacionadas con las características de la especie, las propiedades edáficas, el terreno, las condiciones climáticas y los índices de vegetación. A partir de los resultados obtenidos, se identificaron los bloques con mayor aporte marginal al rendimiento del modelo, priorizando aquellos cuya inclusión mejoró consistentemente el $R^2$ sin aumentar de forma significativa la complejidad o redundancia del conjunto de predictores.

En una segunda fase, se procedió al \textbf{entrenamiento comparativo de modelos}, implementando un conjunto de algoritmos de aprendizaje supervisado con el fin de contrastar su capacidad predictiva. Cada modelo fue entrenado bajo las mismas condiciones experimentales, utilizando las configuraciones de variables seleccionadas en la fase anterior. Esta comparación permitió identificar los algoritmos con mejor ajuste global y menor error de predicción, destacando de nuevo el desempeño de \texttt{CatBoost}.

Posteriormente, se implementó una estrategia de \textbf{stacking}, combinando las predicciones de los modelos individuales mediante un metamodelo de segundo nivel, con el objetivo de aprovechar la complementariedad entre los distintos enfoques y mejorar la capacidad de generalización.


%--------------------------------------------------
\subsection{Elección de variables}

\subsubsection{Resultados de la selección de variables manual}

La selección manual de variables partió de una organización temática del conjunto de
predictores, agrupando las variables según el tipo de información ecológica,
estructural o climática que representan. Esta clasificación permitió estructurar
el proceso de reducción dimensional en torno a los siguientes bloques conceptuales:

\begin{itemize}
    \item \textbf{Bloque de variables fijas}: describe la estructura básica de la
          masa forestal y los atributos esenciales de identificación y caracterización
          general de cada parcela.
    \item \textbf{Bloque de variables de especie}: recoge información relativa a la composición, estado y características específicas de las formaciones forestales.
    \item \textbf{Bloque sustrato}: integra variables edáficas y de manejo susceptibles de variar en el tiempo.
    \item \textbf{Bloque de terreno}: agrupa propiedades físicas del medio que permanecen estables a escala temporal de inventarios (pendiente, orientación, tipo de suelo, etc.).
    \item \textbf{Bloque climático resumido}: representado por el índice de aridez de Martonne, que sintetiza la interacción entre temperatura y precipitación.
    \item \textbf{Bloque climático detallado}: incluye métricas estacionales explícitas de temperatura y precipitación.
    \item \textbf{Bloque de índices de vegetación}: recoge información espectral relacionada con el estado hídrico, vigor y actividad fotosintética de la vegetación.
\end{itemize}

En total, la base de datos contenía inicialmente 445 variables candidatas distribuidas entre estos bloques temáticos. Tras aplicar el procedimiento de selección manual, apoyado en criterios estadísticos, ecológicos y en la comparación del rendimiento del modelo, el conjunto se redujo a 44 variables representativas. Las variables finalmente seleccionadas dentro de cada bloque fueron las siguientes:

\begin{itemize}
    \item \textbf{Bloque de variables fijas}:
          \texttt{especie\_id}, \texttt{tipo\_especie}, \texttt{grupo\_id}, \texttt{periodo}, \texttt{radio}, \texttt{ocupa},
          \texttt{npies\_1}, \texttt{npies\_2}, \texttt{npies\_5},
          \texttt{npies\_10}, \texttt{npies\_15}, \texttt{npies\_20},
          \texttt{npies\_25}, \texttt{npies\_30}, \texttt{npies\_35},
          \texttt{npies\_40}, \texttt{npies\_45}, \texttt{npies\_50},
          \texttt{npies\_55}, \texttt{npies\_60}, \texttt{npies\_65},
          \texttt{npies\_70}.

    \item \textbf{Bloque de variables de especie}:
          \texttt{estado\_id}, \texttt{fccarb}, \texttt{disesp\_id}.

    \item \textbf{Bloque sustrato (dinámico)}:
          \texttt{modcomb\_id}, \texttt{nivel2\_id},
          \texttt{tratmasa\_id}.

    \item \textbf{Bloque de terreno}:
          \texttt{rocosidad\_id},
          \texttt{orientacion\_cat}, \texttt{elevacion}, \texttt{pendiente\_cat}.

    \item \textbf{Bloque climático resumido (Martonne)}:
          \texttt{martonneidx\_id}.

    \item \textbf{Bloque climático detallado (temperatura y precipitación)}:
          \texttt{skt\_mean\_primavera}, \texttt{skt\_mean\_verano},
          \texttt{skt\_std\_primavera}, \texttt{skt\_std\_verano},
          \texttt{pr\_sum\_invierno}, \texttt{pr\_sum\_otoño},
          \texttt{pr\_sum\_primavera}, \texttt{pr\_sum\_verano}.

    \item \textbf{Bloque de índices de vegetación}:
          \texttt{gndvi\_mean\_verano}, \texttt{ndii\_mean\_primavera},
          \texttt{gndvi\_std\_primavera}, \texttt{evi\_mean\_primavera}.
\end{itemize}

Este proceso permitió sintetizar la información original manteniendo una representación equilibrada de todos los ámbitos ecológicos implicados en la estimación del carbono.

La comparación de modelos entrenados con combinaciones incrementales de bloques mostró que todos ellos aportan información relevante, siguiendo el orden de contribución aproximado: \textit{variables fijas} $>$ \textit{variables de especie} $>$ \textit{sustrato} $>$ \textit{terreno} $>$ \textit{índices de vegetación} $>$ \textit{Martonne} $>$ \textit{temperatura y precipitación}. Es decir, la mayor parte de la capacidad predictiva se explica por la estructura y composición de la masa forestal, mientras que las condiciones edáficas, topográficas y climáticas actúan como moduladores adicionales de la acumulación de carbono.

\subsubsection{Selección de variables mediante \textit{Featurewiz}}

Aplicado al conjunto completo de predictores, \textit{Featurewiz} seleccionó \textbf{67 variables}. El patrón resultante muestra una clara preferencia por dos grandes grupos: (i) \textbf{índices de vegetación} derivados de Sentinel-2 y (ii) \textbf{variables térmicas estacionales}. El algoritmo retuvo numerosas estadísticas de NDII, EVI, GNDVI y NDVI (medias, máximos, mínimos, medianas y desviaciones estándar), especialmente durante primavera y verano, reflejando la relevancia del estado hídrico y el vigor fotosintético en la estimación del carbono.

Asimismo, se seleccionaron múltiples métricas de temperatura del aire y del suelo (\texttt{t2m\_*}, \texttt{skt\_*}, \texttt{stl\_*}) y diversas variables de precipitación (\texttt{pr\_sum\_*}, \texttt{pr\_max\_*}, \texttt{pr\_min\_*}), lo que muestra sensibilidad del método a las condiciones climáticas estacionales. El índice de aridez de Martonne también fue seleccionado, aportando una medida sintetizada del balance térmico-hídrico.

Finalmente, el algoritmo incluyó un conjunto contenido pero representativo de variables estructurales (número de pies por clase diamétrica), de especie y de terreno, indicando que dichas variables aportan información complementaria necesaria para la predicción.

\subsubsection{Selección de variables mediante \textit{mRMR}}

El método \textit{mRMR} seleccionó un total de \textbf{50 variables}, priorizando aquellas con alta información mutua respecto al carbono y baja redundancia entre sí. El conjunto final integra predictores estructurales (identificación de especie, radio, clases diamétricas, orientación y pendiente), variables topográficas y edáficas (rocosidad, tipos de suelo), métricas climáticas estacionales (temperatura del aire y del suelo, índice de Martonne) e índices de vegetación representativos del estado estacional de la copa.

La presencia sistemática de valores medios, máximos y medianos de NDII, GNDVI y EVI en verano y primavera confirma que la actividad fotosintética y el estado hídrico son predictores directos del carbono almacenado. De igual modo, la selección de múltiples métricas térmicas refleja la relevancia de los pulsos climáticos sobre la productividad forestal.

En conjunto, mRMR produjo un conjunto compacto y equilibrado, asegurando diversidad informativa y evitando redundancias, lo que lo convierte en un complemento eficaz a los métodos anteriores.

\subsubsection{Discusión de la selección de variables}
De los tres conjuntos de variables seleccionados se mantuvo la selección manual al demostrar un mejor rendimiento con mayor simplicidad como se aprecia en la tabla \ref{tab:comparativa_modelos}.

TODO: Esto igual debería ir en resultados?

\begin{table}[htbp]
    \centering
    \caption{Comparación de configuraciones de selección de variables y rendimiento del modelo CatBoost sobre los datos del IFN 2-3 y 4 para predecir \texttt{c4}.}
    \label{tab:comparativa_modelos}
    \footnotesize
    \begin{tabular}{l l r r r r r}
        \toprule
        Configuración & Modelo   & $n_\text{vars}$ & $R^{2}$ & RMSE  & MAE   & Moda error (aprox.) \\
        \midrule
        Manual        & CatBoost & 44              & 0.80    & 21.77 & 11.48 & 1                   \\
        mRMR          & CatBoost & 67              & 0.79    & 21.91 & 11.69 & 1                   \\
        FeatureWiz    & CatBoost & 50              & 0.72    & 25.65 & 13.08 & 2                   \\
        \bottomrule
    \end{tabular}
\end{table}


\subsection{Ensamblado tipo \textit{stacking} de modelos de regresión}

Con el objetivo de estudiar el compromiso entre diversidad del ensamble, coste computacional y rendimiento, se definieron cinco configuraciones de modelos base (Tabla~\ref{tab:stack_configs}). Los modelos AdaBoost, BayesianNN, SVR, MLP y KNN se descartaron como candidatos.

\begin{table}[htbp]
    \centering
    \caption{Configuraciones de modelos base para \textit{stacking}.}
    \label{tab:stack_configs}
    \begin{tabular}{cl}
        \toprule
        \textbf{Config.} & \textbf{Modelos base} \\
        \midrule
        1 & LightGBM, Random Forest \\
        2 & CatBoost, Random Forest, GBDT \\
        3 & LightGBM, XGBoost, GBDT \\
        4 & CatBoost, LightGBM, Random Forest, GBDT \\
        5 & CatBoost, LightGBM, XGBoost, Random Forest, GBDT, BaggedDT \\
        \bottomrule
    \end{tabular}
\end{table}

\begin{itemize}
    \item \textbf{Configuración 1:} es la configuración más simple. LightGBM compite con CatBoost en rendimiento, mientras que Random Forest aporta un sesgo diferente al basarse en bagging en lugar de boosting. Esta configuración sirve como referencia de un ensamble muy ligero, con bajo coste computacional y, al mismo tiempo, razonablemente diverso.
    
    \item \textbf{Configuración 2:} combina un modelo de boosting basado en manejo robusto de variables categóricas (CatBoost) con Random Forest (bagging de árboles) y GBDT (boosting clásico). La idea es mezclar enfoques de bagging y boosting, manteniendo un número moderado de modelos y una buena diversidad estructural.
    
    \item \textbf{Configuración 3:} agrupa únicamente modelos de la familia de \textit{gradient boosting}. El objetivo es analizar el efecto de combinar variantes de un mismo paradigma y evaluar hasta qué punto diferentes implementaciones de boosting proporcionan suficiente diversidad como para ser beneficiosa en un ensamble.
    
    \item \textbf{Configuración 4:} reduce el número de modelos en comparación con la configuración siguiente (que incluye todos los modelos competitivos), eliminando XGBoost y BaggedDT, que aportan menos mejora marginal respecto a sus alternativas (LightGBM y Random Forest). Esta combinación mantiene una buena diversidad con menor complejidad y coste computacional.
    
    \item \textbf{Configuración 5:} incluye todos los modelos con rendimiento competitivo. Esta configuración es la más rica en términos de variedad de arquitecturas, aunque también la más costosa computacionalmente y potencialmente más propensa al sobreajuste si no se controla adecuadamente.
\end{itemize}

El objetivo es que el meta-modelo reciba como entradas predicciones de alta calidad y suficientemente diversas, en lugar de introducir ruido procedente de modelos débiles.

Sobre las predicciones apiladas de cada configuración se entrenan distintos meta-modelos $g(\cdot)$, definidos en la Tabla~\ref{tab:meta_modelos}.

\begin{table}[htbp]
    \centering
    \caption{Meta-modelos utilizados en el \textit{stacking} junto con sus parámetros.}
    \label{tab:meta_modelos}
    \footnotesize
    \begin{tabular}{lp{8cm}}
        \toprule
        \textbf{Meta-modelo} & \textbf{Parámetros} \\
        \midrule
        Gradient Boosting   & Configuración por defecto \\
        Regresión Lineal    & Sin regularización \\
        Ridge               & Regularización L2 con validación cruzada ($\alpha \in \{0.01, 0.1, 1, 10, 100\}$) \\
        Random Forest       & 50 árboles \\
        SVR                 & Kernel lineal \\
        MLP                 & Una capa oculta con 50 neuronas, 500 iteraciones máximas \\
        \bottomrule
    \end{tabular}
\end{table}

Estos meta-modelos representan diferentes formas de combinar las predicciones de los modelos base:

\begin{itemize}
    \item \textbf{Modelos lineales} (Regresión Lineal y Ridge): permiten comprobar si una combinación lineal de las predicciones base es suficiente para mejorar el rendimiento. Ridge añade regularización L2 para controlar el sobreajuste.
    \item \textbf{Modelos no lineales basados en árboles} (GradientBoostingRegressor, RandomForestRegressor): pueden capturar interacciones complejas entre las predicciones de los modelos base, a costa de una mayor complejidad.
    \item \textbf{Modelos de \textit{kernel}} (SVR con kernel lineal): permiten una combinación robusta y, en algunos casos, menos sensible a valores extremos en las predicciones.
    \item \textbf{Red neuronal (MLPRegressor)}: introduce una capa adicional de flexibilidad, capaz de aproximar combinaciones no lineales complejas entre las salidas de los modelos base.
\end{itemize}

Al evaluar todas las combinaciones de \texttt{stack\_configs} con los diferentes \texttt{meta\_models}, se obtiene un conjunto de ensambles apilados que permiten estudiar de forma sistemática:
(i) qué subconjuntos de modelos base son más complementarios, y (ii) qué tipo de meta-modelo aprovecha mejor la información contenida en sus predicciones.






\clearpage \thispagestyle{empty} \mbox{} \clearpage

\section{Resultados}
% Aquí es donde analizas lo que significan tus resultados.
% - Interpretación de los resultados: ¿Qué significan los valores de tus métricas? ¿El modelo es bueno, aceptable, o tiene problemas?
% - Comparación con la literatura: ¿Cómo se comparan tus resultados con estudios similares en el campo? ¿Superas, igualas o quedas por debajo de lo esperado? ¿Por qué?
% - Implicaciones: ¿Qué implicaciones tienen tus hallazgos para la predicción de créditos de carbono, la gestión de dehesas, o la toma de decisiones empresariales?
% - Limitaciones del estudio: ¿Qué aspectos no pudo abordar tu modelo o tu metodología? (ej. tamaño del dataset, calidad de los datos, alcance geográfico, factores no considerados).
% - Áreas de mejora: ¿Cómo se podría mejorar el modelo o la investigación en el futuro? (ej. más datos, diferentes modelos de IA, considerar otras variables, integrar datos de otras fuentes).
% - Relevancia práctica: ¿Cómo puede ser utilizado este modelo en el mundo real?

%--------------------------------------------------
\subsection{Resultados}
\label{subsec:resultados_modelos}

En esta sección se presentan los resultados obtenidos por los modelos descritos en la Sección~\ref{subsec:modelosevaluados}. Las tablas de resultados completas se pueden consultar en el \ref{anexo:resultados}.

En conjunto, los resultados obtenidos a lo largo de las cuatro configuraciones de entrenamiento analizadas (IFN3 o IFN2 y 3 como explicativos / variable objetivo en tC o tC/ha) muestran un comportamiento notablemente estable y coherente entre versiones, tanto en términos de capacidad predictiva como de generalización. De forma sistemática, los modelos basados en árboles de decisión y \textit{gradient boosting} son los que alcanzan los mejores niveles de rendimiento, destacando de manera consistente CatBoost y LightGBM como las alternativas más competitivas entre los modelos individuales, independientemente del inventario empleado o de la forma en que se expresa la variable objetivo.

Un aspecto especialmente relevante es la alta similitud entre los valores de $R^2$ obtenidos en validación cruzada y en el conjunto de test, lo que indica que los modelos presentan una buena capacidad de generalización y no muestran síntomas apreciables de sobreajuste. Esta estabilidad se observa tanto en los escenarios con mayor volumen de información (IFN2+IFN3) como en aquellos más simples (IFN3), reforzando la robustez de los enfoques basados en árboles frente a variaciones en la disponibilidad de datos.

La incorporación de esquemas de \textit{stacking} no produce incrementos sustanciales en el coeficiente de determinación respecto a los mejores modelos individuales. No obstante, sí se aprecia una mejora sistemática en el error absoluto medio (MAE), con reducciones que oscilan aproximadamente entre 210 y 371 kg de carbono (o kg/ha), dependiendo del escenario considerado. Esta reducción, aunque moderada en términos relativos, resulta relevante desde un punto de vista práctico, ya que implica predicciones más precisas en el rango de error típico y justifica la consideración del \textit{stacking} como una estrategia complementaria.

En cuanto a la estructura de los ensambles, los mejores resultados se obtienen cuando se combinan modelos base de alta calidad y naturaleza similar (principalmente variantes de \textit{gradient boosting}) y se emplean metamodelos con complejidad moderada, como MLP o SVR lineal. Por el contrario, los \textit{stacks} con pocos modelos base o aquellos que incorporan metamodelos excesivamente flexibles, como Random Forest en el segundo nivel, tienden a ofrecer un rendimiento inferior, probablemente debido a la baja dimensionalidad del espacio de meta-predictores y a un ajuste innecesario del ruido residual.

En síntesis, los resultados confirman que los modelos individuales basados en árboles constituyen una solución sólida y eficiente, mientras que el \textit{stacking} aporta mejoras incrementales principalmente en términos de reducción del error medio.

\begin{table}[H]
\centering
\scriptsize
\caption{Comparación sintética del rendimiento de los modelos según inventarios utilizados y variable objetivo.}
\label{tab:comparativa_global_modelos}
\begin{tabular}{lllrcccc}
\toprule
\textbf{IFN} & \textbf{Variable objetivo} & \textbf{Modelo} & \textbf{Modelos} & $\boldsymbol{R^2}$ & \textbf{RMSE} & \textbf{MAE} \\
\midrule
2 y 3 & tC/ha & LightGBM & 1  & 0.79 & 22.77 & 11.65 \\
2 y 3 & tC/ha & stack1 + MLP & 6   & 0.79 & 22.39 & 11.32 \\
\midrule
2 y 3 & tC & CatBoost & 1  & 0.84 & 13.85 & 6.61 \\
2 y 3 & tC & stack1 + MLP & 6    & 0.85 & 13.76 & 6.40 \\
\midrule
3 & tC/ha & CatBoost & 1  & 0.8598 & 17.7087 & 9.2504 \\
3 & tC/ha & stack1 + MLP & 6     & 0.8656 & 17.3380 & 8.8789 \\
\midrule
3 & tC & LightGBM & 1  & 0.9091 & 10.6623 & 5.4774 \\
3 & tC & stack1 + MLP & 6   & 0.9140 & 10.3723 & 5.2515 \\
\bottomrule
\end{tabular}
\end{table}

La Tabla~\ref{tab:comparativa_global_modelos} sintetiza el rendimiento de los mejores modelos identificados en cada una de las cuatro líneas de entrenamiento consideradas, permitiendo una comparación directa entre inventarios utilizados, variable objetivo y complejidad del modelo. En todos los escenarios se observa un patrón consistente: los modelos individuales basados en \textit{gradient boosting} (LightGBM o CatBoost) ofrecen un rendimiento sólido, que se ve ligeramente mejorado mediante la incorporación de esquemas de \textit{stacking}.

Cuando se emplean conjuntamente los inventarios IFN2 e IFN3 y se predice la variable normalizada en tC/ha, el rendimiento del modelo individual (LightGBM) y del \textit{stack} es prácticamente equivalente en términos de $R^2$, si bien el \textit{stacking} logra una reducción apreciable del MAE, pasando de 11.65 a 11.32 tC/ha. Un comportamiento análogo se observa al predecir carbono total (tC) con IFN2 e IFN3, donde CatBoost alcanza ya valores elevados de $R^2$ (0.84), y el \textit{stack} introduce una mejora moderada pero consistente tanto en $R^2$ como en los errores (RMSE y MAE).

En los escenarios basados exclusivamente en IFN3, los niveles de rendimiento son, en general, superiores. Para la variable en tC/ha, CatBoost explica cerca del 86\% de la varianza observada, mientras que el \textit{stack} incrementa ligeramente este valor y reduce el MAE en aproximadamente 0.37 tC/ha. De forma aún más clara, al predecir carbono total (tC), LightGBM alcanza un $R^2$ superior a 0.91, y el \textit{stacking} vuelve a aportar una mejora incremental, reduciendo el error absoluto medio hasta valores en torno a 5.25 tC.

En conjunto, estos resultados confirman que la mayor ganancia del \textit{stacking} no reside tanto en aumentos sustanciales del $R^2$, sino en una reducción sistemática del error medio, lo que se traduce en predicciones más precisas en términos absolutos. Al mismo tiempo, la tabla pone de manifiesto que los enfoques basados en árboles constituyen una base extremadamente robusta, sobre la que los ensambles apilados actúan como un refinamiento adicional más que como un cambio de paradigma.

\subsection{Síntesis de resultados}
\label{subsec:resultados_sintesis}

A partir del análisis realizado, pueden resumirse las principales conclusiones en los siguientes puntos:

\begin{itemize}
    \item El conjunto de datos depurado muestra una variables objetivos marcadas con gran variabilidad:
          \texttt{carbono\_bruto4} presenta menor dispersión (SD $\approx 36$ tC/ha)
          que \texttt{c4} (SD $\approx 47$ tC/ha), lo que anticipa un problema predictivo más complejo
          para esta última.

    \item El análisis ANOVA confirma que el \textit{periodo} tiene un efecto estadísticamente
          significativo sobre ambas variables de carbono, evidenciando la existencia de variaciones
          temporales sistemáticas relevantes para su modelización.

    \item Entre las estrategias de selección de variables evaluadas (manual, FeatureWiz y mRMR),
          la selección manual, basada en bloques temáticos con coherencia ecológica, ofrece el mejor
          equilibrio entre simplicidad y rendimiento, superando en precisión y error a las selecciones
          automáticas.

    \item Los bloques de variables más informativos son, en orden aproximado de importancia:
          estructura de la masa forestal, características de especie, condiciones edáficas y topográficas,
          índices de vegetación e información climática estacional. La mayor parte del poder predictivo se
          concentra en las características estructurales y de especie.

    \item Los modelos individuales muestran que los métodos basados en árboles y
          \textit{gradient boosting} (CatBoost, LightGBM, XGBoost y GBDT) alcanzan el mejor rendimiento
          global, con valores de $R^2$ de hasta $0.85$ y errores moderados (inferiores al $50\%$ de
          la desviación típica de la variable).

    \item CatBoost destaca como el mejor modelo individual, gracias a su capacidad para capturar
          relaciones no lineales y manejar adecuadamente la complejidad y heterogeneidad de los datos.

    \item Métodos como AdaBoost, KNN o BayesianNN muestran un rendimiento sustancialmente inferior,
          lo que los descarta como candidatos eficaces para este tipo de predicción.

    \item Las técnicas de \textit{stacking} aportan mejoras sistemáticas en el error absoluto medio de los modelos,
          reduciendolos, en la mejor configuración, entorno a un 5\%.
    
    \item El rendimiento del \textit{stacking} depende del meta-modelo: los modelos lineales
          (Regresión Lineal y Ridge) ofrecen combinaciones estables y robustas; los meta-modelos Random
          Forest tienden al sobreajuste; y los meta-modelos moderadamente no lineales (SVR y MLP)
          proporcionan las mayores mejoras.

    \item El modelo desarrollado es capaz de predecir, a partir de las características estructurales,
          ecológicas y ambientales de un cultivo forestal, la cantidad de carbono almacenado
          en un horizonte temporal de entre 5 y 30 años con un nivel elevado de precisión.
          El mejor modelo obtenido se construye mediante un metamodelo \texttt{MLP} combinando los modelos
          \texttt{CatBoost,LightGBM,XGBoost, Random Forest, BaggedDT} y \texttt{GBDT} y alcanza un coeficiente de determinación de
          \textbf{$R^2 = 0.85$}, junto con un error típico de \textbf{RMSE = 13.76 tC} y un error
          medio absoluto de \textbf{MAE = 6.40 tC}.
\end{itemize}

\clearpage \thispagestyle{empty} \mbox{} \clearpage

\section{Discusión}
\label{sec:discusion}

\vspace{1cm}
\hrule
\vspace{1cm}


En primer lugar se analizarán los modelos ``globales'', es decir, aquellos entrenados con todos los datos (IFN2 e IFN3) al mismo tiempo. Posteriormente se analizarán los modelos entrenador solo con un inventario y se comparán con los globales.


\subsection{Modelos globales}

\subsubsection{Variable \texttt{c4} (en toneladas de carbono por hectárea)}
\vspace{0.3cm}
\paragraph{Modelos base}

La variable \texttt{periodo} (el número de años entre la medición y la predicción) tiene una gran importancia en el estudio de los modelos. Es por esto que en la Figura \ref{fig:LightGBM_rmse_y_pct_rmse_cuantiles} se incluyen métricas en función de esta variable, donde se muestra la evolución del RMSE y el \%RMSE en cuantiles en función de la variable \texttt{periodo}, junto con un histograma que nos permite visualizar la cantidad de datos en el conjunto de entrenamiento para cada valor de \texttt{periodo}.

\begin{figure}
      \centering
      \begin{subfigure}[b]{0.43\textwidth}
            \centering
            \includegraphics[width=\textwidth]{figuras/09_discusion/c4_LightGBM_metrics.pdf}
            \caption{Evolución del error absoluto y el error absoluto porcentual en cuantiles}
            \label{fig:LightGBM_rmse_y_pct_rmse_cuantiles}
      \end{subfigure}
      \hfill
      \begin{subfigure}[b]{0.48\textwidth}
            \centering
            \includegraphics[width=\textwidth]{figuras/09_discusion/c4_LightGBM_density.pdf}
            \caption{Densidad de predicciones frente a valores reales}
            \label{fig:density_LightGBM_0-300}
      \end{subfigure}
      \caption{Análisis del modelo LightGBM para la variable \texttt{c4}. (a) Evolución de métricas de error en función del periodo. (b) Densidad de predicciones frente a valores reales.}
      \label{fig:LightGBM_analisis_combined}
\end{figure}
  
% Tanto en la Figura \ref{fig:LightGBM_rmse_cuantiles} como en la Figura \ref{fig:LightGBM_rmse_porcentual_cuantiles} se observa el mismo comportamiento: las mejores predicciones se obtienen en aquellos casos en los que los datos de entrenamiento provienen del IFN3. Esto se ve claramente en la forma en la que aumentan los errores (sobre todo los del cuantil $90$) cuando se intenta predecir valores para un periodo mayor a $17$ años, valor a partir del cuál los datos del IFN3 se acaban. Observamos que a partir de los $17$ años aumentan en gran medida los valores grandes de error (cuantil $90$) junto con los valores grandes - medios (cuantil $75$). No obstante, los valores medianos y los pequeños - medios (cuantil $25$) se mantienen relativamente estables, aumentando ligeramente con el periodo.

Podemos observar algo que podíamos prever: la dificultad de la predicción depende de una combinación entre cuán lejos se desea realizar la predicción y la calidad y cantidad de los datos de entrenamiento. Empezando en terreno del IFN3, podemos observar que para los primeros años ($5$, y $6$), pese a tener pocos datos, los errores son comedidos. A medida que aumenta el periodo (entre $10$ y $17$ años) los errores medios y medianos se mantienen relativamente estables pese a disponer de, en general, más datos de entrenamiento. Los errores grandes extremos (cuantil $90$) experimentan un aumento generalizado. Cuando entramos en el rango de años donde los datos de entrenamiento proceden del IFN2 principalmente (el rango entre $20$ y $30$ años), aunque la mediana y los valores pequeños - medios (cuantil $25$) se mantienen estables, los valores grandes - medios (cuantil $75$) y los grandes (cuantil $90$) aumentan en gran medida. Esto es un indicativo de que, aun disponiendo de una gran cantidad de datos de entrenamiento (especialmente para los años $27$ y $28$), la combinación entre la peor calidad de los datos (comparando siempre con el IFN3) y la predicción a valores lejanos dificultan la predicción del modelo. Además, como podemos observar en la Figura \ref{fig:distribucion_variables_combinado}, los valores de los primeros años del IFN2 presentan una gran variabilidad, siendo en su mayoría valores más altos que los del IFN3 para esos mismos años, y con una mayor variabilidad. Esto, unido a que la cantidad de datos para esos años no es excesivamente grande, hace que el modelo obtenga peores predicciones.

El hecho de que los valores medianos se mantengan relativamente estables a lo largo de todos los periodos indica que el modelo asimila correctamente el comportamiento de la variable objetivo tanto con datos del IFN2 como del IFN3. Por otro lado, que los errores del cuantil $90$ sean tan elevados se explica por varios factores: la alta variabilidad del conjunto de datos, el menor número de variables de campo recogidas en el IFN2 (lo que reduce la información disponible) y la mayor dificultad inherente a predecir a horizontes temporales lejanos. Esta combinación hace que los casos particulares o atípicos sean menos reconocibles por el modelo, especialmente cuando proceden del IFN2, donde la capacidad de discriminación es menor que en el IFN3. La variabilidad mencionada se observa claramente en la Figura~\ref{fig:density_LightGBM_0-300}, que muestra los valores predichos frente a los reales para el modelo con mejores métricas, junto con un histograma de distribución.

\paragraph{Modelos de stacking}

La incorporación de esquemas de \textit{stacking} no produce incrementos sustanciales, aunque sí sistemáticos, en el coeficiente de determinación respecto a los mejores modelos individuales. Fijándonos en la Tabla \ref{tab:stack_ifn2_ifn3c_resultados} podemos observar que los modelos de stacking presentan mejores métricas en la gran mayoría de casos, siempre comparando con los modelos bases. Los mejores parámetros los obtiene el modelo de stacking con configuración 5, que mantiene todos los modelos con rendimiento competitivo, junto con el metamodelo de la red neuronal.

El hecho de que todos los ensembles mejoren a los modelos base es un síntoma de que esta mejora no es una excepción estadística, sino una mejora real. Esto es, no se trata de que las métricas sean mejores por estocasticidad de los parámetros del metamodelo, sino porque el montaje realmente mejora el resultado final. Obviamente, reentrenar los metamodelos con otros parámetros haría variar las métricas, pero el hecho es que la función de mejorar las predicciones realmente se alcanza con los ensembles. No obstante, la mejora es pequeña, lo que puede hacer que según el caso se prefiera la simplicidad de emplear uno de los modelos base en comparación con uno de los ensembles. 

En la Figura \ref{fig:Stack5_MLP_rmse_y_pct_rmse_cuantiles} se muestra la evolución del RMSE y el RMSE porcentual en cuantiles para el modelo ensemble con mejores métricas, el stacking con configuración 5 y metamodelo MLP. Podemos observar que el comportamiento es prácticamente idéntico a aquel del LightGBM, el que obtuvo mejores métricas de los modelos base. 
Por otro lado, en la Figura \ref{fig:Stack5_MLP_density} podemos observar el gráfico de puntos de las predicciones frente a los valores reales para el modelo de stacking con configuración 5 y metamodelo MLP para la variable \texttt{c4} (tC/ha).


\begin{figure}
      \centering
      \begin{subfigure}[b]{0.43\textwidth}
            \centering
            \includegraphics[width=\textwidth]{figuras/09_discusion/c4_Stack5_MLP_metrics.pdf}
            \caption{Evolución del error absoluto y el error absoluto porcentual en cuantiles}
            \label{fig:Stack5_MLP_rmse_y_pct_rmse_cuantiles}
      \end{subfigure}
      \hfill
      \begin{subfigure}[b]{0.48\textwidth}
            \centering
            \includegraphics[width=\textwidth]{figuras/09_discusion/c4_Stack5_MLP_density.pdf}
            \caption{Gráfico de puntos de las predicciones frente a los valores reales}
            \label{fig:Stack5_MLP_density}
      \end{subfigure}
      \caption{Análisis del modelo Stacking (Conf. 5, MLP) para la variable \texttt{c4}. (a) Evolución de métricas de error. (b) Densidad de predicciones frente a valores reales.}
      \label{fig:Stack5_MLP_analisis_combined}
\end{figure}




En cuanto a la estructura de los ensambles, los mejores resultados se obtienen cuando se combinan modelos base de alta calidad y naturaleza similar (principalmente variantes de \textit{gradient boosting}) y se emplean metamodelos con complejidad moderada, como MLP o SVR lineal. Por el contrario, los \textit{stacks} con pocos modelos base o aquellos que incorporan metamodelos excesivamente flexibles, como Random Forest en el segundo nivel, tienden a ofrecer un rendimiento inferior, probablemente debido a la baja dimensionalidad del espacio de meta-predictores o a un sobreajuste innecesario del ruido residual.



\subsubsection{Variable \texttt{carbono\_bruto4} (en toneladas de carbono)}
\vspace{0.3cm}
\paragraph{Modelos base}

De igual forma que en los apartados anteriores, en la Figura \ref{fig:CatBoost_combined} se muestra la evolución del RMSE y el \%RMSE en cuantiles en función de la variable \texttt{periodo} para el modelo CatBoost (aquel que obtuvo mejores métricas entre los modelos base) con IFN2 e IFN3 como explicativos para la variable en toneladas de carbono, junto con el histograma de distribución de los datos de entrenamiento en función de la variable periodo.

\begin{figure}
      \centering
      \begin{subfigure}[b]{0.43\textwidth}
            \centering
            \includegraphics[width=\textwidth]{figuras/09_discusion/carbono_bruto4_CatBoost_metrics.pdf}
            \caption{Evolución del error absoluto y el error absoluto porcentual en cuantiles}
            \label{fig:CatBoost_combined}
      \end{subfigure}
      \hfill
      \begin{subfigure}[b]{0.48\textwidth}
            \centering
            \includegraphics[width=\textwidth]{figuras/09_discusion/carbono_bruto4_CatBoost_density.pdf}
            \caption{Gráfico de puntos de las predicciones frente a los valores reales}
            \label{fig:CatBoost_density}
      \end{subfigure}
      \caption{Análisis del modelo CatBoost para la variable \texttt{carbono\_bruto4}. (a) Evolución de métricas de error. (b) Densidad de predicciones frente a valores reales.}
      \label{fig:CatBoost_analisis_combined}
\end{figure}

Atendiendo a la Figura \ref{fig:CatBoost_combined} observamos el mismo comportamiento cualitativo que en el caso de la variable \texttt{c4} (con la particularidad de que cambian las unidades y el rango de valores): el error entre los cuantiles $25$ y $75$ se mantiene en valores relativamente pequeños y constantes, si bien es verdad que empieza siendo menor y va aumentando ligeramente a medida que lo hace la variable periodo. Los errores extremos (cuantil $90$) siguen siendo altos. De hecho, si nos fijamos en los valores más altos del RMSE porcentual para el cuantil $90$, observamos que se alcanzan valores notablemente más altos que en el caso de la variable \texttt{c4}, llegando hasta el $130\%$ cuando para \texttt{c4} apenas se superó el $100\%$ para el peor de los casos. Esto es de destacar, ya que las métricas globales para las predicciones de la variable \texttt{carbono\_bruto4} (ver Tabla \ref{tab:resultados_modelos_base_tc_resultados}) son sistemáticamente mejores que las de la variable \texttt{c4} (ver Tabla \ref{tab:resultados_modelos_base_THA_resultados}). Esto es, la variable \texttt{carbono\_bruto4} (toneladas) es más fácil de predecir en general para los modelos que la variable \texttt{c4} (toneladas por hectárea), pero ocurre lo contrario en los casos particulares, donde los errores más altos se disparan de una manera más exagerada que en el mismo caso para la variable \texttt{c4}. También se observa el aumento del error al pasar del rango de años del IFN2 a IFN3, causado por el aumento de la variabilidad de los datos del segundo inventario (ver Figura \ref{fig:distribucion_variables_combinado}). 

Al igual que en los apartados anteriores, en la Figura \ref{fig:CatBoost_density} podemos observar un scatter plot de las predicciones frente a los valores reales para el modelo CatBoost del caso que estamos considerando.




\paragraph{Modelos con stacking}

Los resultados obtenidos en este apartado son muy similares a aquellos que se han obtenido con los modelos stacking para la variable \texttt{c4} (tC/ha). Esto es, encontramos una mejora sistemática en todos los modelos de stacking frente a los base, pero esa mejora es pequeña. Esto causa que los resultados sean estrictamente mejores, como se puede ver en la Tabla \ref{tab:stack_ifn2_ifn3_tc_resultados} comparando con las métricas de los modelos base de la Tabla \ref{tab:resultados_modelos_base_tc_resultados}. De nuevo, que la mejora de los modelos stacking respecto a los base sea sistemática indica que no nos encontramos frente a un incidente estadístico, sino que el formato del stacking es capaz de mejorar las predicciones de los modelos escogiendo las mejores decisiones de cada uno de ellos. No obstante, como ya se comentó anteriormente, la complejidad de entrenar cinco modelos además del metamodelo puede resultar incómoda frente a la posibilidad de usar, por ejemplo, el mejor de los modelos individuales.

El modelo ensemble que mejores métricas proporciona para la predicción de la variable \texttt{carbono\_bruto4} (tC) es la configuración 5 y metamodelo MLP, al igual que ocurrió con la variable \texttt{c4} (tC/ha). La evolución del RMSE y el RMSE porcentual en cuantiles en función de la variable \texttt{periodo} para este modelo se puede visualizar en la Figura \ref{fig:Stack5_MLP_combined_rmse_y_pct_rmse_cuantiles}.

\begin{figure}
      \centering
      \begin{subfigure}[b]{0.43\textwidth}
            \centering
            \includegraphics[width=\textwidth]{figuras/09_discusion/carbono_bruto4_Stack5_MLP_metrics.pdf}
            \caption{Evolución del error absoluto y el error absoluto porcentual en cuantiles}
            \label{fig:Stack5_MLP_combined_rmse_y_pct_rmse_cuantiles}
      \end{subfigure}
      \hfill
      \begin{subfigure}[b]{0.48\textwidth}
            \centering
            \includegraphics[width=\textwidth]{figuras/09_discusion/carbono_bruto4_Stack5_MLP_density.pdf}
            \caption{Gráfico de puntos de las predicciones frente a los valores reales}
            \label{fig:Stack5_MLP_density_carbono_bruto4}
      \end{subfigure}
      \caption{Análisis del modelo Stacking (Conf. 5, MLP) para la variable \texttt{carbono\_bruto4}. (a) Evolución de métricas de error. (b) Densidad de predicciones frente a valores reales.}
      \label{fig:Stack5_MLP_carbono_bruto4_analisis_combined}
\end{figure}


Por otro lado, en la Figura \ref{fig:Stack5_MLP_density_carbono_bruto4} se muestra un scatter plot de las predicciones frente a los valores reales para el modelo de stacking con configuración 5 y metamodelo MLP y la variable \texttt{carbono\_bruto4} (tC).

\subsection{Modelos individuales entrenados solo con el IFN2 o IFN3.}

Al comparar el rendimiento de los modelos ``globales'' (entrenados con el conjunto combinado de IFN2 e IFN3) frente a los modelos ``locales'' (entrenados exclusivamente con IFN2 o IFN3), se observa un comportamiento diferenciado según el inventario de prueba. En el caso del IFN3, los modelos locales superan a los globales en términos de métricas de precisión. Por el contrario, para el IFN2, los modelos globales muestran un desempeño superior.

Una posible explicación para este fenómeno radica en la calidad y cantidad de información contenida en cada inventario. Los datos del IFN2, al ser menos informativos o presentar menos variables, parecen beneficiarse del aprendizaje conjunto con los del IFN3, lo que permite a los modelos capturar patrones adicionales y mejorar su capacidad de generalización sobre el segundo inventario. En cambio, para el IFN3, cuyos datos poseen una mayor calidad intrínseca, la inclusión de registros del IFN2 durante el entrenamiento podría estar introduciendo ruido o variabilidad no deseada, lo que acaba penalizando la precisión de las predicciones del modelo final sobre este conjunto.

\begin{figure}[htbp]
      \centering
      \begin{subfigure}[b]{0.48\textwidth}
            \centering
            \includegraphics[width=\textwidth]{figuras/09_discusion/Consolidado_IFN2_c4_paper.pdf}
            \caption{Resultados para \texttt{c4} en el IFN2.}
            \label{fig:sub1}
      \end{subfigure}
      \hfill
      \begin{subfigure}[b]{0.48\textwidth}
            \centering
            \includegraphics[width=\textwidth]{figuras/09_discusion/Consolidado_IFN3_c4_paper.pdf}
            \caption{Resultados para \texttt{c4} en el IFN3.}
            \label{fig:sub2}
      \end{subfigure}

      \vspace{0.5cm}

      \begin{subfigure}[b]{0.48\textwidth}
            \centering
            \includegraphics[width=\textwidth]{figuras/09_discusion/Consolidado_IFN2_carbono_bruto4_paper.pdf}
            \caption{Resultados para \texttt{carbono\_bruto4} en el IFN2.}
            \label{fig:sub3}
      \end{subfigure}
      \hfill
      \begin{subfigure}[b]{0.48\textwidth}
            \centering
            \includegraphics[width=\textwidth]{figuras/09_discusion/Consolidado_IFN3_carbono_bruto4_paper.pdf}
            \caption{Resultados para \texttt{carbono\_bruto4} en el IFN3.}
            \label{fig:sub4}
      \end{subfigure}
      \caption{Comparación del error absoluto entre mejores modelos base globales vs. locales (arriba), mejores modelos de stacking globales vs. locales (medio) y mejor modelo base vs. mejor modelo de stacking (abajo).}
      \label{fig:comparativa_mejores}
\end{figure}

En la Figura \ref{fig:comparativa_mejores} se muestra una comparativa entre los mejores modelos base globales vs. locales (arriba), mejores modelos de stacking globales vs. locales (medio) y mejor modelo base vs. mejor modelo de stacking (abajo). Observamos que en general los errores menores para el IFN2 los cometen los modelos globales, mientras que para el IFN3 son los locales los que realizan un mejor trabajo.


% \subsubsection{Comportamiento global de los modelos}

\subsubsection{Asimilación del comportamiento de las variables objetivo}

Con las métricas que hemos planteado y el análisis que se ha hecho en esta sección parece claro que, si bien los resultados no son ni mucho menos perfectos, los modelos son capaces de entender la lógica y el significado de cada variable para obtener una predicción con un grado de acierto que depende de la naturaleza de la instancia que se trate. Otra prueba que podemos hacer para comprobar si los resultados de los modelos son lógicos es la que mostramos en la Figura \ref{fig:sensibilidad_escenarios}.
Esta figura muestra la evolución de varios casos particulares a medida que avanza el tiempo entre medida y predicción. Los casos seleccionados son reales dentro de los datos disponibles. Se seleccionaron con la idea de mostrar cómo se comportan los modelos para configuraciones de carbono inicial más o menos comunes. Para ello se sumaron, para cada instancia de los datos, las variables \texttt{npies_x}, y luego se seleccionaron los valores más cercanos a los percentiles 10, 35, 65 y 90 para la variable resultante. Así evitamos posibles problemas fruto de seleccionar ejemplos poco realistas. 


\begin{figure}
      \centering
      \begin{subfigure}[b]{0.48\textwidth}
            \centering
            \includegraphics[width=\textwidth]{figuras/09_discusion/sensibilidad_escenarios_c4_P10_P35_P65_P90.pdf}
            \caption{Cantidad de carbono en toneladas por hectárea para varios escenarios.}
            \label{fig:sensibilidad_escenarios_c4}
      \end{subfigure}
      \hfill
      \begin{subfigure}[b]{0.48\textwidth}
            \centering
            \includegraphics[width=\textwidth]{figuras/09_discusion/sensibilidad_escenarios_carbono_bruto4_P10_P35_P65_P90.pdf}
            \caption{Cantidad de carbono en toneladas para varios escenarios.}
            \label{fig:sensibilidad_escenarios_carbono_bruto4}
      \end{subfigure}
      \caption{Evolución de la cantidad de carbono en toneladas para varios escenarios.}
      \label{fig:sensibilidad_escenarios}
\end{figure}

Podemos señalar las siguientes características:
\begin{itemize}
      \item Las líneas de los modelos base y los ensembles para cada caso están muy cerca una de la otra, lo que concuerda con las métricas obtenidas, que eran muy similares entre sí.
      \item La tendencia general es ascendente, es decir, los modelos capturan de forma correcta el comportamiento del crecimiento de los árboles con el tiempo para los años estudiados.
      \item En el año $17$ tenemos un crecimiento abrupto generalizado, el cual es más exagerado cuanto más pequeño es el percentil del caso de estudio. Esto es sencillo de entender mirando a los datos, ya que los valores de la variable a predecir experimentan un aumento generalizado al pasar de los años del IFN2 a aquellos del IFN3, como se puede ver en la Figura \ref{fig:distribucion_variables_combinado}. Este cambio (que el modelo empiece a ingerir datos del IFN2 en lugar del IFN3) ocurre principalmente en el año $17$. Esto puede verse en el histograma de la Figura \ref{fig:Stack5_MLP_combined_rmse_y_pct_rmse_cuantiles}, por ejemplo. La razón por la que ocurre el salto a percentiles pequeños es porque estos valores pequeños en los datos del IFN2 apenas existen o son menos comunes. El modelo ha visto que para esos años los valores de carbono son, en general, mayores, y eso es lo que predice. De hecho, en la Figura \ref{fig:distribucion_variables_combinado} se observa que para los años iniciales de los datos del IFN2 los valores de carbono aumentan para luego disminuir, y esto es justo lo que vemos en las predicciones: un aumento y una posterior disminución.
      \item También podemos ver una subida abrupta en muchos casos al pasar del año $29$ al $30$, sobre todo en el caso de la variable en toneladas de carbono bruto (Figura \ref{fig:sensibilidad_escenarios_carbono_bruto4}). Estos años coinciden con una carencia de datos, como podemos ver en el histograma de la Figura \ref{fig:distribucion_variables_combinado}. Es de esperar que las predicciones de esos años sean menos fiables, sobre todo siendo un valor extremo. 
\end{itemize}


\subsubsection{Rendimiento de los modelos en función del valor de la variable objetivo}

Si se analizan las métricas globales de las tablas de la sección \ref{sec:resultados} nos damos cuenta de que los valores del RMSE son notablemente mayores que los valores de MAE, concretamente cerca del doble en la mayoría de casos. Debido a que el RMSE penaliza mucho más los errores grandes que el MAE, esto es indicativo de la presencia de outliers en el error. Ya hemos visto a lo largo de esta sección que los mayores errores ocurren en el rango de años en que el IFN2 es la principal fuente de datos como podemos ver en las Figuras \ref{fig:LightGBM_rmse_y_pct_rmse_cuantiles}, \ref{fig:Stack5_MLP_rmse_y_pct_rmse_cuantiles}, \ref{fig:CatBoost_combined} y \ref{fig:Stack5_MLP_combined_rmse_y_pct_rmse_cuantiles}, y la principal conclusión que podemos sacar es que la calidad o cantidad de información contenida en este inventario es peor. No obstante, esto no responde del todo a la pregunta de en qué casos el modelo predice mejor, salvo la obviedad de que la predicción será mejor cuanto más cerca esté en el tiempo y más datos haya de ese año. Es por esto que la Figura \ref{fig:SMAPE_RMSE_deciles_c4_carbono_bruto4} nos puede ser útil. Esta figura muestra la distribución de errores, concretamente el RMSE y el SMAPE, en función de la variable objetivo para \texttt{c4} y \texttt{carbono\_bruto4} en deciles para los modelos base.

\begin{figure}
      \centering
      \begin{subfigure}[b]{0.48\textwidth}
            \centering
            \includegraphics[width=\textwidth]{figuras/09_discusion/c4_base_models_SMAPE_RMSE_percentiles.pdf}
            \caption{Distribución de errores en función de la variable objetivo para \texttt{c4}.}
            \label{fig:distribucion_c4_combinado}
      \end{subfigure}
      \hfill
      \begin{subfigure}[b]{0.48\textwidth}
            \centering
            \includegraphics[width=\textwidth]{figuras/09_discusion/carbono_bruto4_base_models_SMAPE_RMSE_percentiles.pdf}
            \caption{Distribución de errores en función de la variable objetivo para \texttt{carbono\_bruto4}.}
            \label{fig:distribucion_carbono_bruto4_combinado}
      \end{subfigure}
      \caption{Distribución del SMAPE y RMSE en función de la variable objetivo para \texttt{c4} y \texttt{carbono\_bruto4} en deciles.}
      \label{fig:SMAPE_RMSE_deciles_c4_carbono_bruto4}
\end{figure}


Rápidamente nos damos cuenta de lo siguiente: las predicciones para valores pequeños de carbono objetivo (primeros deciles) son malas (SMAPE grande) pese a ser aquellas con un error absoluto menor (RMSE pequeño). Luego, a medida que aumenta el carbono objetivo, el RMSE aumenta ligeramente mientras que el SMAPE disminuye mucho más rápido, llegando a valores cercanos al $25\%$. En ambos casos los mejores resultados se logran cuando el carbono objetivo es mayor, pese a que observamos un aumento significativo del RMSE entre el penúltimo y último decil. Como esto se produce para todos los modelos, sugiere que la dificultad de predicción no es homogénea para todas las situaciones o cantidades de carbono, y que pese a ser los percentiles de menor carbono aquellos con un RMSE menor, la precisión de los modelos no es la suficiente como para hacer una predicción decente en situaciones con poca biomasa. Esto era predecible, ya que un error absoluto mayor para una parcela de bosque mayor no implica necesariamente un peor error relativo.

También es llamativo el hecho de que los deciles iniciales tienen unos rangos de carbono muy cercanos, sobre todo en el caso de la variable \texttt{carbono\_bruto4}, donde el primer decil abarca desde $0.0003$ tC hasta $0.002$ tC. Esta gran cantidad de valores no sirve a los modelos para obtener una buena predicción, por otra parte. Las densidades mayores para valores pequeños de carbono se ven claramente en las Figuras \ref{fig:density_LightGBM_0-300}, \ref{fig:Stack5_MLP_density}, \ref{fig:CatBoost_density} y \ref{fig:Stack5_MLP_density_carbono_bruto4}.

\vspace{1cm}
\hrule 
\vspace{1cm}



% Comienzo parte Maider
% \subsection{Conjunto de datos de entrenamiento}


% TODO: Revisar teiendo en cuenta los resultados completos
% TODO: comentar siguiente imagen


% \begin{figure}
%       \centering
%       \begin{subfigure}[b]{0.48\textwidth}
%             \centering
%             \includegraphics[width=\textwidth]{figuras/09_discusion/distribucion_c4_combinado.pdf}
%             \caption{Distribución de la variable \texttt{c4} (toneladas por hectárea).}
%             \label{fig:distribucion_c4_combinado}
%       \end{subfigure}
%       \hfill
%       \begin{subfigure}[b]{0.48\textwidth}
%             \centering
%             \includegraphics[width=\textwidth]{figuras/09_discusion/distribucion_carbono_bruto4_combinado.pdf}
%             \caption{Distribución de la variable \texttt{carbono\_bruto4} (toneladas por hectárea).}
%             \label{fig:distribucion_carbono_bruto4_combinado}
%       \end{subfigure}
%       \caption{Distribución de las variables a predecir en percentiles.}
%       \label{fig:distribucion_variables_combinado}
% \end{figure}

% Esta Figura muestra la evolución de varios casos particulares a medida que avanza el tiempo entre medida y predicción. Los casos seleccionado son reales dentro de los datos disponibles. Se seleccionaron con la idea de mostrar cómo se comportan los modelos para configuraciones de carbono inicial más o menos comunes. Para ello se sumaron, para cada instancia de los datos, las variables \texttt{npies_x}, y luego se seleccionaron los valores más cercanos a los percentiles 10, 35, 65 y 90 para la variable resultante. Así evitamos posibles problemas de seleccionar ejemplos poco realistas. 

% Podemos señalar las siguientes características:
% \begin{itemize}
%       \item Las líneas de los modelos base y los ensembles para cada caso están muy cercas una de la otra, lo que concuerda con las métricas obtenidas, que eran muy similares entre sí.
%       \item La tendencia general es ascendente, es decir, los modelos campturas de forma correcta el comportamiento de crecimiento de los árboles con el tiempo.
%       \item 



% Uno de los resultados más consistentes del presente trabajo es que los modelos entrenados utilizando exclusivamente el IFN3 como conjunto de variables explicativas alcanzan un rendimiento sistemáticamente superior al de aquellos entrenados de forma conjunta con IFN2 e IFN3, tanto para la predicción del carbono total (tC) como del carbono normalizado por superficie (tC/ha). Esta mejora se manifiesta de forma clara en valores más elevados de $R^2$ y en reducciones apreciables de las métricas de error (RMSE y MAE), y se observa de manera estable en todos los algoritmos evaluados.

% Una explicación plausible de este comportamiento está relacionada con la calidad y homogeneidad de los datos. El IFN3 fue realizado aproximadamente una década después del IFN2, incorporando avances metodológicos y tecnológicos relevantes en la recogida de información de campo, así como protocolos más refinados para la medición de variables estructurales y de estado de las masas forestales. A esto se une el aumento de variables que se miden en cada apeo de las parcelas. Esta mayor precisión y cantidad en las variables explicativas reduce el ruido inherente al proceso de modelización y facilita el aprendizaje de relaciones más consistentes entre predictores y variable objetivo. En este contexto, la inclusión de datos procedentes del IFN2 podría introducir heterogeneidad adicional asociada a diferencias metodológicas entre inventarios, lo que penaliza el rendimiento predictivo global.

% Por otro lado, el número de observaciones disponibles para el entrenamiento difiere entre configuraciones. El entrenamiento exclusivo con datos del IFN3 se realiza sobre un conjunto de menor tamaño y que, además, incorpora filtros de calidad más estrictos, (filtro por \texttt{fccarb}$>20$), que no está disponible en el IFN2. A pesar de estas diferencias en tamaño muestral y criterios de selección, los resultados no muestran indicios de sobreajuste en ninguno de los casos. 

% En conjunto, estos resultados ponen de manifiesto que, en este caso de estudio, la calidad y coherencia temporal del inventario parecen tener un impacto más relevante en el rendimiento predictivo que la simple agregación de información procedente de inventarios previos. Este hallazgo es especialmente relevante de cara a futuras aplicaciones operativas, ya que sugiere que modelos entrenados sobre inventarios recientes y metodológicamente homogéneos pueden ofrecer estimaciones más precisas y fiables del carbono forestal, incluso cuando se dispone de menos fuentes de información histórica.

% A modo de conclusión, los resultados obtenidos indican que la elección del conjunto de entrenamiento debe adaptarse al horizonte temporal de predicción considerado. En particular, el modelo entrenado exclusivamente con datos del IFN3 resulta más adecuado para predicciones a medio plazo, en un horizonte temporal aproximado de entre 9 y 17 años, donde la mayor calidad y coherencia metodológica de este inventario se traduce en estimaciones más precisas y estables del carbono forestal. Por el contrario, cuando el objetivo es realizar predicciones a más largo plazo, con horizontes temporales superiores y que pueden extenderse hasta los 30 años, el uso combinado de datos procedentes del IFN2 y del IFN3 se hace obligatorio, ya que ofrece una base temporal más amplia que permite capturar mejor la evolución de las masas forestales en periodos prolongados. En este contexto, aunque el rendimiento predictivo sea ligeramente inferior, la integración de ambos inventarios aporta robustez frente a escenarios de extrapolación temporal, lo que hace recomendable su empleo para proyecciones de largo plazo.

% \subsection{Variable objetivo}
% En la interpretación de los resultados es fundamental contextualizar la diferencia entre las dos variables objetivo empleadas en el estudio: el carbono expresado en toneladas absolutas (\texttt{tC}) y el carbono normalizado por superficie (\texttt{tC/ha}). Tal y como se observa en la Tabla~\ref{tab:resumen_variables}, la variable \texttt{carbono_bruto4} (tC) muestra una media inferior, pero una elevada dispersión relativa, con un rango amplio y valores extremos asociados a parcelas con estructuras muy heterogéneas. Por su parte, la variable \texttt{c4} (tC/ha) presenta una media más alta y una variabilidad absoluta mayor.

% Estas diferencias estructurales tienen implicaciones directas sobre la capacidad predictiva de los modelos. En términos generales, la variable expresada en tC resulta más sencilla de modelizar, ya que integra implícitamente la superficie y reduce parte de la variabilidad introducida por la normalización por hectárea. Como consecuencia, los modelos entrenados para predecir carbono total alcanzan sistemáticamente valores más altos de $R^2$ y errores más bajos (RMSE y MAE) que aquellos orientados a la predicción de tC/ha. Esto indica que una mayor fracción de la varianza es explicada por las variables explicativas disponibles cuando la respuesta se expresa en términos absolutos.

% En cambio, la predicción en tC/ha constituye un problema más exigente desde el punto de vista estadístico, al amplificar la heterogeneidad intra-parcela y la influencia de factores locales no completamente capturados por los predictores. No obstante, esta variable resulta especialmente relevante para aplicaciones de comparación espacial, evaluación de productividad y análisis de eficiencia en el secuestro de carbono, lo que justifica su inclusión a pesar de presentar métricas de ajuste ligeramente inferiores. En conjunto, los resultados ponen de manifiesto que la elección de la variable objetivo debe alinearse con el objetivo final del análisis, asumiendo el compromiso existente entre interpretabilidad ecológica y rendimiento predictivo.

% \subsection{Distribución del error}
% El análisis conjunto de las métricas globales de los modelos predictivos muestra de forma consistente que el valor del RMSE es aproximadamente el doble del MAE. Dado que el RMSE penaliza de manera más severa los errores de gran magnitud, esta diferencia indica que, aunque el error medio absoluto se mantiene en niveles moderados, existen observaciones concretas en las que los modelos cometen desviaciones significativamente mayores.

% Este patrón sugiere que la dificultad predictiva no es homogénea a lo largo de todo el rango de la variable objetivo, sino que se concentra en determinados valores o contextos específicos. En este sentido, resulta necesario complementar la evaluación con métricas relativas como el SMAPE, que permiten analizar el error en proporción a la magnitud de la variable, así como con representaciones gráficas —como los diagramas de dispersión entre valores observados y predichos— que facilitan la identificación visual de sesgos, heterocedasticidad o rangos problemáticos del modelo.

% En la Figura~\ref{fig:dispersion_densidad_c4} se representa la dispersión entre los valores observados y las predicciones obtenidas por el modelo LightGBM (el de mayor R$^2$) para la variable objetivo \texttt{c4} (tC/ha)empleando como explicativos los inventarios IFN2 y IFN3. También se incluye un histograma de la densidad de puntos.

% TODO: hace falta otra igual para carbono_bruto4. No se que modelo es el de la foto. Actualizar la imagen. Poner el caption de la foto en CASTELLANO.

% Aunque \texttt{c4} presenta un rango muy amplio, desde valores próximos a cero hasta aproximadamente $880$~tC/ha, la figura pone de manifiesto que la mayor concentración de observaciones se sitúa en el intervalo comprendido entre $0$ y $200$~tC/ha. En este rango, que además concentra la mayor parte de la masa de datos, la nube de puntos se alinea de forma clara en torno a la diagonal identidad.

% Si bien existen casos puntuales en los que las predicciones se desvían notablemente de los valores reales, especialmente en los extremos superiores del rango, la estructura general del gráfico indica que el modelo reproduce de manera consistente la relación media entre valores observados y predichos. Este comportamiento es coherente con la diferencia observada entre RMSE y MAE, y refuerza la idea de que los errores más elevados se concentran en un subconjunto reducido de observaciones, mientras que el ajuste es sólido en las regiones donde se acumula la mayor densidad de datos.

% \begin{figure}
% \centering
% \includegraphics[width=0.8\textwidth]{figuras/08_resultados/density_LightGBM_0-300.pdf}
% \caption{Dispersión de las predicciones frente a los valores reales para la variable objetivo \texttt{c4}. Se representa únicamente el rango $[0,300]$~tC.}
% \label{fig:dispersion_densidad_c4}
% \end{figure}

% Este comportamiento se analiza de forma explícita en la Figura~\ref{fig:smape_c4}, donde se representan, por intervalos de la variable objetivo \texttt{c4}, los valores de SMAPE y RMSE obtenidos por los distintos modelos base, junto con un histograma que muestra el número de observaciones disponibles en cada rango. Esta figura permite evaluar simultáneamente la magnitud del error y su dependencia de la densidad de datos a lo largo del dominio de la variable.

% Los resultados confirman que el error relativo no es homogéneo. En los rangos intermedios de carbono, donde existe una combinación favorable de volumen de datos, comportamiento más estable del sistema y una cantidad de carbono suficientemente grande como para permitir un margen de error razonable (no es lo mismo un error de una toneladas en una parcela de 2 toneladas totales que en otra de 100 toneladas), el SMAPE alcanza valores mínimos, bajando del $20\%$ de error. En cambio, para valores bajos de \texttt{c4}, el SMAPE es elevado pese a la abundancia de observaciones, lo que refleja una alta variabilidad relativa en este régimen y una mala precisión de los modelos. Finalmente, en los valores más altos de carbono, el incremento del SMAPE coincide con una fuerte reducción del número de datos, evidenciando que la escasez de observaciones en los extremos de la distribución limita la capacidad de generalización del modelo.

% \begin{figure}
% \centering
% \includegraphics[width=0.8\textwidth]{figuras/09_discusion/c4_base_models_SMAPE_RMSE_combined_bins.pdf}
% \caption{SMAPE y RMSE por rangos de la variable objetivo \texttt{c4} para los modelos base, junto con la distribución del número de observaciones en cada intervalo.}
% \label{fig:smape_c4}
% \end{figure}

% TODO: Tal vez conviene hablar aqui sobre como de bien predice según PERIODO. Un grafico y tal. 


\clearpage \thispagestyle{empty} \mbox{} \clearpage

\section{Conclusiones}
% Este es un resumen conciso de tus hallazgos clave.
% - Reafirma el objetivo principal y si se logró.
% - Destaca los resultados más significativos.
% - Reitera la contribución principal de tu trabajo.
% - Enfatiza la importancia y el impacto potencial de tu modelo.
% - Evita introducir nueva información aquí.

El objetivo de este trabajo es la obtención de un modelo de Inteligencia Artificial capaz de predecir el carbono que una cierta parcela de terreno forestada o reforestada capturará en un cierto periodo de tiempo. Para ello se han recogido datos de tierra (Inventario Forestal Nacional \cite{ifn}), datos meteorológicos \cite{era5land} e imágenes satelitales \cite{landsat5_data} con los que se han entrenado varios modelos para intentar predecir el carbono capturado por las parcelas presenten en las iteraciones $2$ y $3$ del Inventario Forestal Nacional, comparando el resultado con la última de las iteraciones, la $4$. Las predicciones se hicieron para dos configuraciones distintas: usando como datos explicativos únicamente los del inventario $2$ y usando como datos explicativos los de los inventarios $3$ y $4$. A su vez, para cada caso se realizó la predicción de dos variables objetivo: la predicción del carbono en toneladas por hectárea (tC/ha) y la predicción del carbono en toneladas (tC). Los resultados para los mejores modelo en cada caso están recogidos en la Tabla \ref{poner_tabla}.


Con estos resultados podemos afirmar que disponemos de datos suficientes y de suficiente calidad para entrenar modelos capaces de predecir el carbono capturado con un error aceptable. 
\clearpage \thispagestyle{empty} \mbox{} \clearpage

\section{Recomendaciones para Futuras Investigaciones}

% A partir de los resultados obtenidos y de las limitaciones identificadas durante el desarrollo de este trabajo, se proponen a continuación varias líneas de investigación que podrían contribuir a mejorar y ampliar el alcance del modelo desarrollado.

% En primer lugar, sería recomendable ampliar y diversificar la base de datos empleada. La incorporación de futuras ediciones del Inventario Forestal Nacional permitiría reforzar la dimensión temporal del conjunto de entrenamiento y evaluar con mayor detalle la estabilidad del modelo ante horizontes temporales más largos. Así mismo, la extensión del estudio a otras regiones bioclimáticas, tanto dentro como fuera del ámbito nacional, permitiría analizar la capacidad de generalización del modelo y su adaptabilidad a contextos ecológicos distintos.

% En relación con las variables explicativas, futuras investigaciones podrían explorar la inclusión de nuevas fuentes de información, como datos de teledetección de mayor resolución espacial o temporal (por ejemplo, LIDAR aéreo o satelital). Del mismo modo, la incorporación explícita de variables relacionadas con perturbaciones (incendios, plagas, siembras, talas\dots) podría mejorar la capacidad del modelo para capturar dinámicas no lineales en la acumulación de carbono.

% Desde el punto de vista metodológico, sería de interés evaluar arquitecturas de aprendizaje más avanzadas, como modelos de deep learning especializados en series temporales o enfoques híbridos que combinen modelos mecanicistas de crecimiento forestal con técnicas de aprendizaje automático. Así mismo, el análisis sistemático de la incertidumbre asociada a las predicciones, por ejemplo, mediante enfoques bayesianos o técnicas de quantile regression, permitiría proporcionar intervalos de confianza, un aspecto especialmente relevante para aplicaciones vinculadas a la certificación de créditos de carbono.

% Otra línea de trabajo prometedora consiste en profundizar en la interpretabilidad de los modelos. El uso de técnicas explicativas avanzadas podría facilitar una comprensión más detallada del papel de cada variable en la predicción final, reforzando la confianza de técnicos y gestores en el uso del modelo y favoreciendo su adopción en contextos operativos.

% Por último, desde una perspectiva aplicada, sería recomendable desarrollar herramientas que faciliten la transferencia del modelo a usuarios finales. Esto podría materializarse en una interfaz gráfica o plataforma web que permita introducir escenarios de plantación y obtener estimaciones de captura de carbono de forma directa. En este contexto, también podría explorarse la integración del modelo con sistemas de registro y trazabilidad, como tecnologías de blockchain, para apoyar la gestión y certificación de créditos de carbono de manera transparente y verificable.

% En conjunto, estas líneas de investigación futura permitirían consolidar y ampliar el impacto del modelo propuesto, reforzando su utilidad científica, técnica y aplicada en el ámbito de la gestión forestal y la mitigación del cambio climático.
\clearpage \thispagestyle{empty} \mbox{} \clearpage

\input{sections/12_agradecimientos}
\clearpage \thispagestyle{empty} \mbox{} \clearpage

% ---------------------------------------------------------------------------
% Referencias
% ---------------------------------------------------------------------------
\printbibliography[title={Referencias}]

% ---------------------------------------------------------------------------
% Anexos (opcional - comentar si no se necesitan)
% ---------------------------------------------------------------------------
\appendix
\section{Anexos}
\small

\subsection{Anexo: Origen y cálculo de las variables \texttt{ca} y \texttt{cr}}\label{sec:Carbono}

Las variables \texttt{ca} (carbono arbóreo) y \texttt{cr} (carbono radical) incluidas en la base de datos del \textit{Inventario Forestal Nacional} (IFN4) derivan de las ecuaciones alométricas de biomasa desarrolladas por el \textit{Instituto Nacional de Investigación y Tecnología Agraria y Alimentaria} (INIA), en particular por \textit{Gregorio Montero} y \textit{Ricardo Ruiz-Peinado} \cite{montero2009, ruizpeinado2011}. Estas ecuaciones fueron elaboradas a partir de datos de campo obtenidos mediante talas y pesadas directas de árboles de distintas especies representativas de la flora forestal española.


Cada ecuación estima la biomasa seca (en kilogramos) de los diferentes componentes del árbol en función del diámetro normal (\textit{D}, en cm, medido a 1,3 m del suelo) y la altura total (\textit{H}, en m). Para cada especie o grupo de especies similares se dispone de ecuaciones específicas de la forma:
\[
    W_i = a_i \cdot D^{b_i} \cdot H^{c_i}
\]

donde $W_i$ representa la biomasa del componente $i$ (fuste, corteza, ramas, hojas, raíces, etc.), y $a_i$, $b_i$ y $c_i$ son coeficientes empíricos obtenidos mediante regresión no lineal. En los casos en que una especie no dispone de ecuación propia, se utiliza la de otra especie considerada análoga por similitud morfológica o ecológica.


Los componentes de biomasa definidos en el IFN4 incluyen \cite{miteco_ifn4_manual}:

\begin{itemize}
    \item $W_s$: biomasa del fuste (kg),
    \item $W_c$: biomasa de la corteza del fuste (kg),
    \item $W_{b7}$: biomasa de ramas mayores de 7 cm de diámetro (kg),
    \item $W_{b2-7}$: biomasa de ramas entre 2 y 7 cm de diámetro (kg),
    \item $W_{b0.5-2}$: biomasa de ramas entre 0,5 y 2 cm de diámetro (kg),
    \item $W_t$: biomasa de ramas menores de 0,5 cm de diámetro (kg),
    \item $W_h$: biomasa de hojas (kg),
    \item $W_{db}$: biomasa de ramas muertas (kg),
    \item $W_T = W_s + W_c + W_{b7} + W_{b2-7} + W_{b0.5-2} + W_t + W_h$: biomasa aérea total (kg),
    \item $W_r$: biomasa radical (raíces, kg).
\end{itemize}


\noindent
A partir de estas ecuaciones, el cálculo de biomasa y carbono en el IFN4 se realiza de la siguiente forma:

\begin{enumerate}
    \item \textbf{Biomasa por árbol (kg):} en la tabla \texttt{Mayores\_exs} se incluyen las medidas de diámetro y altura de cada pie. Aplicando las ecuaciones alométricas correspondientes se obtiene la biomasa aérea ($W_T$) y radical ($W_r$) para cada árbol.

    \item \textbf{Conversión a carbono (kg):} se aplica un factor de conversión estándar de 0.5, según las directrices del IPCC \cite{ipcc2006}, de forma que:
          \[
              \text{CA} = 0.5 \times W_T, \quad \text{CR} = 0.5 \times W_r
          \]

    \item \textbf{Expansión a valores por hectárea (t/ha):} los valores por árbol se convierten a toneladas por hectárea mediante un \textit{factor de expansión} (\textit{Fac}), que refleja la densidad de árboles por unidad de superficie dentro de cada clase diamétrica y especie. Este factor se calcula en función del número de pies inventariados y la superficie de muestreo, permitiendo expresar los resultados en términos comparables de biomasa o carbono por hectárea.

    \item \textbf{Agregación por clases diamétricas y especie:} finalmente, en la tabla \texttt{Parcelas\_exs} se agrupan los valores por parcela, especie y clase diamétrica (CD), sumando las contribuciones individuales ya expandidas. El resultado son los valores medios de biomasa y carbono por hectárea (\textit{t/ha}) para cada combinación de parcela y especie.
\end{enumerate}


El mismo procedimiento se aplica tanto a la biomasa aérea (para obtener \texttt{ca}) como a la biomasa radical (para \texttt{cr}). De esta forma, \textbf{\texttt{ca} Y \texttt{cr} representan el carbono almacenado en la biomasa viva, aérea y subterránea respectivamente, expresado en toneladas de carbono por hectárea (t/ha)}.


Este enfoque metodológico se ajusta a las recomendaciones del \textit{IPCC Guidelines for National Greenhouse Gas Inventories} \cite{ipcc2006}, garantizando la coherencia con los métodos de reporte de carbono a nivel internacional y facilitando la comparación de los resultados con otros estudios y marcos regulatorios.


\subsection{Anexo: Estado de las Poblaciones (\texttt{estado\_id})}\label{sec:EstadoIFN34}

Se determinará las fases de desarrollo de las \textit{poblaciones} codificándose de la siguiente forma:

\begin{enumerate}
    \item \textbf{Repoblado}. Conjunto de pies que desde el estrato herbáceo llega hasta el subarbustivo y los pies inician la tangencia de copas.
    \item \textbf{Monte bravo}. Comprende desde el estrato y clase de edad anterior hasta el momento en que por efecto del crecimiento, los pies empiezan a perder las ramas inferiores; es decir que en esta clase de edad, las ramas se encuentran a lo largo de todo el fuste.
    \item \textbf{Latizal}. Comprende desde la clase anterior hasta que los pies tienen 20 cm de diámetro normal; es decir, el diámetro de su fuste, medido a la altura de 1,30 m del suelo.
    \item \textbf{Fustal}. Se caracteriza esta clase de edad, porque sus pies tienen diámetros normales superiores a 20 cm.
\end{enumerate}


\subsection{Anexo: Forma Principal de Masa (IFN3 e IFN4: \texttt{fpmasa\_id})}\label{sec:FPMasa}

\begin{enumerate}
    \item \textbf{Coetánea}. Cuando al menos el 90\% de sus pies tienen la misma edad individual. Ejemplo típico: las repoblaciones.
    \item \textbf{Regular}. Cuando al menos el 90\% de sus pies pertenecen a la misma clase artificial de edad o misma clase diamétrica en su defecto.
    \item \textbf{Semirregular}. Cuando al menos el 90\% de sus pies pertenecen a dos clases artificiales de edad cíclicamente contiguas o dos clases diamétricas contiguas en su defecto.
    \item \textbf{Irregular}. Cuando no se cumplen las condiciones anteriores, es decir, cuando en cualquier parte de la masa existen pies más o menos mezclados, de todas las clases de edad que tiene la masa o de varias clases diamétricas en su defecto.
\end{enumerate}


\subsection{Anexo: Tratamiento de la Masa (IFN3 e IFN4: \texttt{tratmasa\_id})}\label{sec:tratmasa}

\begin{enumerate}
    \item \textbf{Monte alto}. Cuando todos los pies proceden de semilla.
    \item \textbf{Monte medio}. Cuando coexisten pies de la misma especie, unos procedentes de semilla (brinzales) y otros de brote (chirpiales).
    \item \textbf{Monte bajo}. Cuando todos los pies proceden de brote de cepa o de raíz.
\end{enumerate}


\subsection{Anexo: Origen de la Masa (IFN3 e IFN4: \texttt{orgmasa\_id})}\label{sec:OrgMasa}

\begin{enumerate}
    \item \textbf{Natural}. Bosque desarrollado espontáneamente, sin intervención humana directa.
    \item \textbf{Artificial}. Plantado intencionadamente por el ser humano.
    \item \textbf{Naturalizado}. Bosque originalmente plantado pero que ha evolucionado hacia una estructura más similar a un bosque natural.
\end{enumerate}


\subsection{Anexo: Tipo de Suelo (\texttt{tipsuelo1\_id}, \texttt{tipsuelo2\_id}, \texttt{tipsuelo3\_id})}\label{sec:TipSuelo}

Se utilizará la siguiente codificación para el tipo de suelo, diferenciando tres variables:

\vspace{1em}
\noindent
\textbf{Tipo de suelo (I):} \textbf{Presencia de sales, yesos o hidromorfía}

\begin{enumerate}
    \item \textbf{No se observan sales, yesos ni procesos de fidromorfía.}
    \item \textbf{Suelo salino.} Si presenta al menos dos de las siguientes características:
          \begin{itemize}
              \item Presencia de eflorescencias en la superficie o a distintas profundidades.
              \item Existencia de plantas halófitas.
              \item Zonas llanas o endorreicas con climas secos que provocan gran evaporación.
          \end{itemize}


    \item \textbf{Suelo yesífero.} Si presenta alguna de las siguientes características:
          \begin{itemize}
              \item Presencia de materia yesífera en superficie o a distintas profundidades.
              \item Existencia de plantas gipsófilas.
          \end{itemize}


    \item \textbf{Suelo hidromorfo.} Si el suelo presenta síntomas de hidromorfía acusada, cumpliendo al menos dos de las siguientes:
          \begin{itemize}
              \item Zona encharcada permanente o casi permanentemente de forma natural.
              \item Zona llana o endorreica con climas húmedos.
              \item Grietas en verano si no hay encharcamiento.
              \item Presencia de vegetación indicadora de hidromorfismo.
          \end{itemize}
\end{enumerate}

Identificandose las siguientes:
\begin{itemize}
    \item Formaciones vegetales indicadoras de hidromorfía:
          \begin{itemize}
              \item Ribereñas: \textit{saucedas, mimbreras, alisedas}.
              \item Brezales con \textit{Erica ciliaris, Erica tetralix}.
              \item Turberas arboladas (excepto Cornisa Cantábrica y Pirineos).
              \item Turberas de montaña con \textit{Sphagnum, Erica tetralix}.
              \item Cervunales con \textit{Nardus stricta}.
              \item Carrizales y espadañares (\textit{Phragmites, Tipha, Cladium}).
              \item Juncales (\textit{Scirpus, Juncus}).
              \item Pastizales con cárices (\textit{Carex spp.}).
              \item Marismas.
          \end{itemize}
    \item Formaciones vegetales gipsófilas:
          \begin{itemize}
              \item Aznallar: matorral de \textit{Ononis tridentata}.
              \item Tomillares gipsófilos con:
                    \begin{itemize}
                        \item \textit{Lepidium subulatum}
                        \item \textit{Gypsophila spp.}
                        \item \textit{Matthiola fruticulosa}
                    \end{itemize}
          \end{itemize}
    \item   Formaciones vegetales indicadoras de suelos salinos:
          \begin{itemize}
              \item Salicorniales: matas leñosas crasas (Salicornia, Arthrocnemum, Halozylon).
              \item Bosques halófitos del género \textit{Tamarix}.
              \item Saladar o sosar: predominio de \textit{Suaeda vera}.
              \item Saladar blanco: predominio de \textit{Atriplex halimus}.
          \end{itemize}
\end{itemize}


\vspace{1em}
\noindent
\textbf{Tipo de suelo (II y III):} \textbf{Composición del suelo (calizo o silíceo)}

\begin{enumerate}
    \item \textbf{Suelo calizo.} Más del 50\,\% de la vertical del perfil da efervescencia con ácido clorhídrico.

          \begin{itemize}
              \item \textbf{Moderadamente básico:} pH en superficie $\leq$ 8.5.
              \item \textbf{Fuertemente básico:} pH en superficie > 8.5.
          \end{itemize}

    \item \textbf{Suelo silíceo.} Menos del 50\,\% de la vertical del perfil da efervescencia.

          \begin{itemize}
              \item \textbf{Moderadamente ácido:} pH $\geq$ 5.5.
              \item \textbf{Fuertemente ácido:} pH < 5.5.
          \end{itemize}
\end{enumerate}


\subsection{Anexo: Rocosidad (\texttt{rocosidad\_id}) }\label{sec:Rocosid}
Se considerará el conjunto de la parcela clasificando la rocosidad según la siguiente codificación:
\begin{enumerate}
    \item \textbf{Sin pedregosidad}: la superficie de la parcela está completamente cubierta de vegetación.
    \item \textbf{Poco pedregoso}: cuando la superficie de la parcela cubierta por rocas coherentes es menor del 25\,\%.
    \item \textbf{Pedregoso}: cuando la superficie rocosa está comprendida entre el 25\,\% y el 50\,\%.
    \item \textbf{Muy pedregoso}: cuando la superficie rocosa se sitúa entre el 50\,\% y el 75\,\%.
    \item \textbf{Roquedo}: cuando la superficie de rocas es mayor del 75\,\%. En este caso, no se tomará ningún dato adicional correspondiente a suelos.
\end{enumerate}


\subsection{Anexo: Textura del Suelo (\texttt{textura\_id})}\label{sec:textura}

Se clasificará en función de la siguiente codificación:

\begin{enumerate}
    \item \textbf{Suelo arenoso.} Si los cilindros se deshacen sin apenas formarse.
    \item \textbf{Suelo franco.} Es posible hacer cilindros gruesos pero no delgados.
    \item \textbf{Suelo arcilloso.} Se consiguen cilindros de unos 5 mm de diámetro.
\end{enumerate}



\subsection{Anexo: Contenido en Materia Orgánica (IFN3 e IFN4: \texttt{matorg\_id})}\label{sec:MatOrg}

Según la siguiente clasificación:

\begin{enumerate}
    \item \textbf{Suelo muy humífero.} Cuando a 15 cm la pureza es menor de 4, o cuando la capa de broza sea de espesor mayor de 5 cm y a 15 cm de profundidad la pureza sea menor de 6.
    \item \textbf{Suelo moderadamente humífero.} Cuando a 15 cm la pureza sea menor de 6 con capa de broza nula o de escaso espesor, o cuando dicha capa tenga espesor mayor de 5 cm y a 15 cm de profundidad la pureza sea igual o mayor de 6.
    \item \textbf{Suelo poco humífero.} En los restantes casos.
\end{enumerate}


\subsection{Anexo: Modelo de Combustible (IFN3 e IFN4: \texttt{modcomb\_id})}\label{sec:modComb}
Se determinará la clase de combustible que es más probable que propague el fuego si hubiese un incendio en la zona, hasta un máximo de 60m: pasto, matorral, hojarasca de bosque o deshechos o restos de corta. Se determinará el modelo de combustible a partir de la siguiente clave:

\footnotesize
\begin{longtable}{|p{2.5cm}|c|p{9cm}|}
    \caption{Descripción de los modelos de combustible del Inventario Forestal Nacional, clasificados por grupo funcional.}
    \label{tab:modelos_combustible}                                                                                                                                                                                                                    \\
    \hline
    \rowcolor[HTML]{D9EAD3} \textbf{GRUPO}   & \textbf{MOD.} & \textbf{DESCRIPCIÓN DEL MODELO}                                                                                                                                                         \\
    \hline
    \endfirsthead
    \hline
    \rowcolor[HTML]{D9EAD3} \textbf{GRUPO}   & \textbf{MOD.} & \textbf{DESCRIPCIÓN DEL MODELO}                                                                                                                                                         \\
    \hline
    \endhead
    \hline
    \multicolumn{3}{r}{\emph{Continúa en la siguiente página}}                                                                                                                                                                                         \\
    \endfoot
    \hline
    \endlastfoot

    Pastos                                   & 1             & Pasto fino, seco y bajo, que recubre completamente el suelo. Puede aparecer algunas plantas leñosas dispersas ocupando menos de 1/3 de la superficie.                                   \\
                                             & 2             & Pasto fino, seco y bajo, que recubre completamente el suelo. Las plantas leñosas dispersas cubren de 1/3 a 2/3 de la superficie; pero la propagación del fuego se realiza por el pasto. \\
                                             & 3             & Pasto grueso, denso, seco y alto (> 1 m). Puede haber algunas plantas leñosas dispersas. Los campos de cereales son representativos de este modelo.                                     \\
    \hline
    Matorral                                 & 4             & Matorral o plantación joven muy densa; de más de 2 m de altura; con ramas muertas en su interior. Propagación del fuego por las copas de las plantas.                                   \\
                                             & 5             & Matorral disperso, denso y verde, de menos de 1 m de altura. Propagación del fuego por la hojarasca, el pasto, las ramillas y el matorral.                                              \\
                                             & 6             & Parecido al modelo 5, pero con especies más inflamables, de mayor talla, pudiéndose encontrar ramas gruesas en el suelo. Propagación del fuego con vientos moderados a fuertes.         \\
                                             & 7             & Matorral de especies muy inflamables; de 0.5 a 2 m de altura, situado como sotobosque en masas de coníferas.                                                                            \\
    \hline
    Hojarasca bajo arbolado                  & 8             & Bosque denso, sin matorral. Propagación del fuego por la hojarasca muy compacta, formada por acículas cortas (5 cm o menos) o por hojas planas no muy grandes.                          \\
                                             & 9             & Parecido al modelo 8, pero con hojarasca menos compacta, formada por acículas largas y rígidas (P. pinaster) o follaje de frondosas de hoja grande, caducas (castaño o robles).         \\
                                             & 10            & Bosque con gran cantidad de leña y árboles caídos, como consecuencia de vendavales, plagas intensas, etc.                                                                               \\
    \hline
    Restos de corta y operaciones selvícolas & 11            & Bosque claro y fuertemente aclarado. Restos de poda o aclareo ligeros (diámetro < 7.5 cm).                                                                                              \\
                                             & 12            & Predominio de los restos sobre el arbolado. La hojarasca y el matorral presente ayudarán a la propagación del fuego.                                                                    \\
                                             & 13            & Grandes acumulaciones de restos gruesos y pesados, cubriendo todo el suelo.                                                                                                             \\
\end{longtable}
\normalsize


\subsection{Anexo: Distribución Espacial (\texttt{disesp\_id})}\label{sec:disEsp}

La disposición de la vegetación en el espacio se clasificará según la siguiente codificación:

\begin{enumerate}
    \item \textbf{Uniforme.} Cuando el estrato arbóreo presenta continuidad en el espacio.

    \item \textbf{Diseminada en bosquetes aislados.} Cuando la masa arbórea se encuentra dividida en porciones que tienen una superficie inferior a 0,5 ha.

    \item \textbf{Diseminada en individuos aislados.} Cuando se trata de dehesas.

    \item[9.] \textbf{Otras o no se sabe.} En caso diferente a los anteriores o si se desconoce el dato exacto.
\end{enumerate}


\subsection{Anexo: Composición Específica (\texttt{comesp\_id})}\label{anexo:compesp}

En función de las especies presentes:

\begin{enumerate}
    \item \textbf{Masas homogéneas o puras}. Masas monoespecíficas con una única especie arbórea. La normativa española precisa que una masa es monoespecífica o pura cuando al menos el 90\% de los pies pertenecen a la misma especie.

    \item \textbf{Masas heterogéneas o mezcladas pie a pie}. Masas de diferentes especies que se juntan o bien se entremezclan por golpes o grupos, siempre que tengan una altura similar.

    \item \textbf{Masas heterogéneas o mezcladas con subpiso}. Las dos o más especies mezcladas, cuando alcancen el estado adulto y la estabilidad, presentarán alturas diferentes.

    \item[9.] \textbf{Otras o no se sabe}. En caso diferente a los anteriores o desconocer el dato exacto.
\end{enumerate}


\subsection{Anexo: Manifestaciones Erosivas (\texttt{merosiva\_id})}\label{sec:ManERo}

Se observará la parcela y sus alrededores hasta una distancia de 60 metros desde el centro, y se codificará la existencia de manifestaciones erosivas según la siguiente clave:

\begin{enumerate}
    \item \textbf{No hay ninguna manifestación.}

    \item \textbf{Cuellos de raíces al descubierto:} los cuellos de las raíces están visibles, con acumulación de residuos aguas arriba de los tallos y obstáculos, así como abundancia superficial de piedras.

    \item \textbf{Presencia de regueros:} canales paralelos de erosión con una profundidad máxima de un palmo (aproximadamente 20 cm).

    \item \textbf{Cárcavas y barrancos en V:} erosión lineal más profunda que los regueros, con forma de ``V''.

    \item \textbf{Cárcavas y barrancos en U:} erosión avanzada con formas suavizadas y amplias en ``U''.

    \item \textbf{Deslizamientos del terreno:} desplazamientos de masas de tierra, ladera o materiales del suelo.
\end{enumerate}



\subsection{Anexo: Nivel de usos del suelo  (IFN3 e IFN4: \texttt{nivel1\_id})}\label{sec:nivel1}

\begin{enumerate}
    \item \textbf{Monte.} Toda superficie en la que vegetan especies arbóreas, arbustivas, de matorral o herbáceas, ya sea espontáneamente o procedan de siembra o plantación, siempre que no sean características de cultivo agrícola o fueran objeto del mismo.
    \item \textbf{Agrícola.} Territorio o ecosistema poblado con siembras o plantaciones de herbáceas y/o leñosas, anuales o plurianuales que se laborea con una fuerte intervención humana, puede estar poblado por especies forestales de fruto (flor, hojas o en el futuro biomasa) siempre que la intervención humana sea importante. Incluye las dehesas, montes huecos o montes adehesados de base cultivo, siempre que la fracción de cabida cubierta de los árboles sea inferior al 5\%.
    \item \textbf{Artificial.} Territorio o ecosistemas dominado por edificios, parques urbanos (aunque estén poblados de árboles), viveros fuera de los montes (aunque sean de especies forestales), carreteras (salvo las vías de servicio de los montes) u otras construcciones humanas que tengan superficies continuas.
    \item \textbf{Humedal.} Lo constituyen las lagunas, charcas, zonas húmedas, marismas y corrientes discontinuas de agua en las que, al menos durante 6 meses del año, esté presente dicho líquido.
    \item \textbf{Agua.} Es la parte de la tierra constituida por ríos, lagos, embalses, canales o estanques con superficies continuas de más de 0.26 ha y con agua prácticamente todo el año.
\end{enumerate}


\subsection{Anexo: Nivel morfoestructural  (IFN3 e IFN4: \texttt{nivel2\_id})}\label{sec:nivel2}
Para el nivel de usos del suelo Monte se definirán los siguientes niveles morfoestructurales.

\begin{enumerate}
    \item \textbf{Monte arbolado.} Territorio o ecosistema con especies forestales arbóreas como manifestación vegetal de estructura vertical dominante y con una fracción de cabida cubierta igual o superior al 20\%; incluye dehesas con base cultivo o pastizal con labores siempre que la fracción arbolada supere el 20\%, y excluye terrenos con fuerte intervención humana para obtener frutos, hojas, flores o varas.

    \item \textbf{Monte arbolado ralo.} Terreno de uso forestal con especies arbóreas forestales dominantes y fracción de cabida cubierta entre el 10\% y 20\% (incluido el 10\%, excluido el 20\%); también aplica a terrenos con matorral o pastizal natural como dominantes, pero con presencia importante de árboles forestales, incluyendo dehesas de base de cultivo.

    \item \textbf{Monte temporalmente desarbolado.} Terreno que fue monte arbolado recientemente y que casi con seguridad volverá a estar cubierto de árboles en un futuro próximo.

    \item \textbf{Monte desarbolado.} Terreno con matorral y/o pastizal natural o débil intervención humana como cobertura dominante, con fracción de cabida cubierta por árboles forestales inferior al 5\%.

    \item \textbf{Monte sin vegetación superior.} Terreno de uso forestal que no está poblado por vegetales superiores debido a condiciones actuales de suelo, clima o topografía, aunque podría estarlo en otras circunstancias.

    \item \textbf{Árboles fuera del monte.} Incluye riberas arboladas no estructuradas con los montes, bosquetes de menos de 2.500 m\textsuperscript{2}, alineaciones de especies arbóreas o arbustivas de menos de 25 m de anchura, y árboles sueltos en terreno forestal.

    \item \textbf{Monte arbolado disperso.} Terreno forestal con especies arbóreas dominantes y fracción de cabida cubierta entre el 5\% y el 10\% (incluido el 5\%, excluido el 10\%); también terrenos con matorral o pastizal como cobertura dominante pero con presencia significativa de árboles forestales, incluyendo dehesas de base cultivo.
\end{enumerate}


\subsection{Anexo: Código de los grupos taxonómicos de las especies (\texttt{grupo\_id})}\label{sec:gruposespecies}
\begin{table}[htbp]
    \centering
    \small
    \renewcommand{\arraystretch}{1.1}
    \setlength{\tabcolsep}{6pt}
    \caption{Relación de códigos de grupo taxonómico utilizados en la variable \texttt{grupo\_id}.}
    \begin{tabular}{rl|rl}
        \toprule
        \textbf{Código} & \textbf{Grupo taxonómico} & \textbf{Código} & \textbf{Grupo taxonómico} \\
        \midrule
        7               & Acacia                    & 69              & Phoenix                   \\
        15              & Crataegus                 & 73              & Betula                    \\
        19              & Coníferas                 & 77              & Tilia                     \\
        20              & Pinos                     & 78              & Sorbus                    \\
        31              & Abies                     & 79              & Platanus                  \\
        35              & Larix                     & 80              & Laurisilva                \\
        40              & Quercus                   & 91              & Buxus                     \\
        53              & Tamarix                   & 93              & Pistacia                  \\
        57              & Salix                     & 94              & Laurus                    \\
        58              & Populus                   & 95              & Prunus                    \\
        60              & Eucalyptus                & 99              & Frondosas                 \\
        65              & Ilex                      & 399             & Morus                     \\
        68              & Arbutus                   & 455             & Fraxinus                  \\
        917             & Cedrus                    & 936             & Cupressus                 \\
        937             & Juniperus                 & 956             & Ulmus                     \\
        975             & Juglans                   & 976             & Acer                      \\
        997             & Sambucus                  &                 &                           \\
        \bottomrule
    \end{tabular}
\end{table}

\subsection{Anexo: Resultados}\label{anexo:resultados}

\subsubsection{Ifn2 e Ifn3 como explicativo para \texttt{carbono\_bruto4} (tC)}
\begin{table}[htbp]
    \centering
    \caption{Resumen del rendimiento de los modelos para la predicción de la variable de carbono en tC con el conjunto de datos que emplea IFN2 e IFN3 como explicativos.}
    \label{tab:resultados_modelos23c}
    \begin{tabular}{lrrrr}
        \toprule
        Modelo            & CV $R^2$ & Test $R^2$ & Test RMSE (tC) & Test MAE (tC) \\
        \midrule
        \textbf{CatBoost} & 0.8405   & 0.8454     & 13.7701        & 6.7100        \\
        LightGBM          & 0.8399   & 0.8418     & 13.9258        & 6.6542        \\
        XGBoost           & 0.8394   & 0.8403     & 13.9937        & 6.6695        \\
        GBDT              & 0.8372   & 0.8372     & 14.1270        & 6.6973        \\
        MLP               & 0.8294   & 0.8343     & 14.2546        & 7.0064        \\
        BaggedDT          & 0.8184   & 0.8218     & 14.7801        & 7.2181        \\
        Random Forest     & 0.8125   & 0.8163     & 15.0091        & 7.1317        \\
        SVR               & 0.7987   & 0.7988     & 15.7059        & 6.6313        \\
        BayesianNN        & 0.7703   & 0.7701     & 16.7897        & 8.8954        \\
        KNN               & 0.7380   & 0.7415     & 17.8022        & 8.1109        \\
        AdaBoost          & 0.4814   & 0.4802     & 25.2456        & 20.8181       \\
        \bottomrule
    \end{tabular}
\end{table}

Se mantienen las conclusiones extraidas en el análisis de resultados realizado para el modelo entrenado
con los mismos datos pero variable objetivo \texttt{c4} (tC/ha), esto es:
\begin{itemize}
    \item Los modelos presentan una buena capacidad de generalización.
    \item Los modelos basados en árboles de decisión y en \textit{gradient boosting} son los que ofrecen, en general, el mejor equilibrio entre capacidad predictiva y estabilidad.
    \item Algoritmos como AdaBoost o KNN muestran un rendimiento claramente inferior
\end{itemize}

En particular, \textbf{CatBoost} destaca como el modelo con mejor rendimiento global,
alcanzando un $R^2 = 0.8454$ y un RMSE de $13.77$ tC/ha. Estos valores implican que el modelo
es capaz de explicar una proporción sustancial de la variabilidad del carbono en las parcelas,
reduciendo el error típico de predicción a menos de la mitad de la variabilidad natural de la
variable (SD $\approx 36$ tC). Esto indica que, dentro de la complejidad inherente al problema,
CatBoost logra capturar de manera más eficaz las relaciones no lineales presentes en los datos.



\subsubsection{Resultados del \textit{stacking} frente a los modelos individuales}
\begin{table}[htbp]
    \centering
    \small
    \footnotesize
        \begin{tabular}{lcccc}
            \hline
            \textbf{Stack}             & \textbf{Bases} & \textbf{Test $R^2$} & \textbf{RMSE}    & \textbf{MAE}    \\
            \hline
            % ---------- STACK 1 ----------
            stack1\_\_GradientBoosting & 6              & 0.8423              & 13.9071          & 6.5327          \\
            stack1\_\_LinearRegression & 6              & 0.8424              & 13.9005          & 6.6250          \\
            stack1\_\_Ridge            & 6              & 0.8424              & 13.9005          & 6.6250          \\
            stack1\_\_RandomForest     & 6              & 0.8113              & 15.2118          & 7.2916          \\
            stack1\_\_SVR              & 6              & 0.8386              & 14.0695          & 6.5204          \\
            \textbf{stack1\_\_MLP}     & 6              & 0.8432              & 13.8638          & 6.4806          \\
            \hline
            % ---------- STACK 2 ----------
            stack2\_\_GradientBoosting & 4              & 0.8435              & 13.8538          & 6.5280          \\
            stack2\_\_LinearRegression & 4              & 0.8439              & 13.8326          & 6.5844          \\
            stack2\_\_Ridge            & 4              & 0.8439              & 13.8326          & 6.5844          \\
            stack2\_\_RandomForest     & 4              & 0.8267              & 14.5758          & 7.0146          \\
            stack2\_\_SVR              & 4              & 0.8405              & 13.9833          & 6.4897          \\
            \textbf{stack2\_\_MLP}     & 4              & 0.8442              & 13.8206          & 6.5289          \\
            \hline
            % ---------- STACK 3 ----------
            stack3\_\_GradientBoosting & 3              & 0.8479              & 13.6581          & 6.4281          \\
            stack3\_\_LinearRegression & 3              & 0.8463              & 13.7272          & 6.5942          \\
            stack3\_\_Ridge            & 3              & 0.8463              & 13.7272          & 6.5942          \\
            stack3\_\_RandomForest     & 3              & 0.8301              & 14.4327          & 6.9187          \\
            stack3\_\_SVR              & 3              & 0.8417              & 13.9308          & 6.4787          \\
            \textbf{stack3\_\_MLP}     & 3              & 0.8481              & 13.6451          & 6.3674          \\
            \hline
            % ---------- STACK 4 ----------
            stack4\_\_GradientBoosting & 3              & 0.8481              & 13.6470          & 6.4214          \\
            stack4\_\_LinearRegression & 3              & 0.8471              & 13.6915          & 6.5631          \\
            stack4\_\_Ridge            & 3              & 0.8471              & 13.6915          & 6.5630          \\
            stack4\_\_RandomForest     & 3              & 0.8345              & 14.2450          & 6.8175          \\
            stack4\_\_SVR              & 3              & 0.8428              & 13.8817          & 6.4555          \\
            \textbf{stack4\_\_MLP}     & \textbf{3}     & \textbf{0.8484}     & \textbf{13.6336} & \textbf{6.4327} \\
            \hline
            % ---------- STACK 5 ----------
            stack5\_\_GradientBoosting & 2              & 0.8479              & 13.6540          & 6.4258          \\
            stack5\_\_LinearRegression & 2              & 0.8471              & 13.6899          & 6.5610          \\
            stack5\_\_Ridge            & 2              & 0.8471              & 13.6899          & 6.5610          \\
            stack5\_\_RandomForest     & 2              & 0.8382              & 14.0828          & 6.7136          \\
            stack5\_\_SVR              & 2              & 0.8428              & 13.8832          & 6.4518          \\
            \textbf{stack5\_\_MLP}     & \textbf{2}     & \textbf{0.8482}     & \textbf{13.6420} & \textbf{6.5120} \\
            \hline
        \end{tabular}
    \caption{Resultados de las diferentes configuraciones de stacking utilizando IFN2 e IFN3 como explicativos de la variable en tC.}
    \label{tab:stack_ifn2_ifn3carb}
\end{table}


Los resultados recogidos en la Tabla~\ref{tab:stack_ifn2_ifn3carb} muestran que el \textit{stacking} no alcanza resultados tan
satisfactorios como en \ref{tab:stack_ifn2_ifn3c}. Esta técnica permite mejorar ligeramente el rendimiento respecto a
los mejores modelos individuales basados en árboles y \textit{gradient boosting}. En concreto, mientras que CatBoost
obtiene un $R^2$ de 0.8454 y un RMSE de 13.77 tC,  \textit{stacking} alcanza $R^2$ en torno a 0.8484 y reduce el RMSE hasta 13.6336 tC.


Se observa aun patrón claro: en todas las configuraciones,
los mejores resultados se obtienen cuando el meta-modelo es una red neuronal MLP
(\texttt{stack3\_\_MLP}, \texttt{stack4\_\_MLP}, \texttt{stack5\_\_MLP}), seguido muy de cerca
por los meta-modelos basados en \textit{gradient boosting}.

Tomamos como mejor combinación el modelo \texttt{stack5\_\_MLP}, esto es, la combinación \texttt{MLP}
de los modelos LightGB y Random Forest. Aunque la mejora en términos de R2 de este modelo respecto de
CatBoost es ligera ($\varDelta 0.0003$), la variación el MAE alcanza $\varDelta 0.0667$ unidades, lo
cual se traduce en una diferencia de 67kg de error medio. Ofrece un buen equilibrio entre una mejora
mediocre y un aumento ligero de la complejidad del problema.


\subsubsection{Ifn2 como explicativo para \texttt{c4} (tC/ha)}
\begin{table}[htbp]
    \centering
    \caption{Resumen del rendimiento de los modelos para la predicción de la variable de carbono en tC/ha con el conjunto de datos que emplea IFN2 como explicativo.}
    \label{tab:resultados_modelos_ifn2}
    \begin{tabular}{lrrrr}
        \toprule
        Modelo        & CV $R^2$ & Test $R^2$ & Test RMSE (tC/ha) & Test MAE (tC/ha) \\
        \midrule
        Random Forest & 0.8081   & 0.8268     & 19.1845           & 10.1126          \\
        XGBoost       & 0.8400   & 0.8581     & 17.3623           & 9.0307           \\
        CatBoost      & 0.8410   & 0.8614     & 17.1604           & 9.0277           \\
        LightGBM      & 0.8406   & 0.8563     & 17.4713           & 8.9638           \\
        GBDT          & 0.8368   & 0.8598     & 17.2612           & 9.2273           \\
        BaggedDT      & 0.8105   & 0.8293     & 19.0472           & 10.0722          \\
        AdaBoost      & 0.4580   & 0.4974     & 32.6789           & 25.2755          \\
        KNN           & 0.7433   & 0.7567     & 22.7382           & 11.9747          \\
        MLP           & 0.8158   & 0.8266     & 19.1954           & 10.7453          \\
        SVR           & 0.7318   & 0.7374     & 23.6220           & 10.9064          \\
        BayesianNN    & 0.7672   & 0.7766     & 21.7868           & 11.9026          \\
        \bottomrule
    \end{tabular}
\end{table}

Se mantienen las conclusiones extraidas en el análisis de resultados realizado para el modelo entrenado
con la misma variable objetivo pero los datos del IFN2 e IFN3 como explicativos (\ref{tab:resultados_modelos23c}):
\begin{itemize}
    \item Los modelos presentan una buena capacidad de generalización.
    \item Los modelos basados en árboles de decisión y en \textit{gradient boosting} son los que ofrecen, en general, el mejor equilibrio entre capacidad predictiva y estabilidad.
    \item Algoritmos como AdaBoost o KNN muestran un rendimiento claramente inferior
\end{itemize}

En particular, \textbf{CatBoost} destaca como el modelo con mejor rendimiento global,
alcanzando un $R^2 = 0.8614$ y un RMSE de $17.1604$ tC/ha. Estos valores implican que el modelo
es capaz de explicar una proporción sustancial de la variabilidad del carbono en las parcelas,
reduciendo el error típico de predicción a menos de la mitad de la variabilidad natural de la
variable (SD $\approx 47$ tC/ha).

\begin{table}[htbp]
    \centering
    \small
    \footnotesize
        \begin{tabular}{lcccc}
            \hline
            \textbf{Stack}             & \textbf{Bases} & \textbf{Test $R^2$} & \textbf{RMSE}    & \textbf{MAE}    \\
            \hline
            stack1\_\_GradientBoosting & 6              & 0.8649              & 16.9454          & 8.7640          \\
            stack1\_\_LinearRegression & 6              & 0.8650              & 16.9342          & 8.7602          \\
            stack1\_\_Ridge            & 6              & 0.8650              & 16.9342          & 8.7602          \\
            stack1\_\_RandomForest     & 6              & 0.8633              & 17.0456          & 9.1193          \\
            stack1\_\_SVR              & 6              & 0.8608              & 17.1971          & 8.5929          \\
            stack1\_\_MLP              & 6              & \textbf{0.8673}     & \textbf{16.7925} & \textbf{8.6644} \\
            \hline
            stack2\_\_GradientBoosting & 4              & 0.8641              & 16.9945          & 8.7752          \\
            stack2\_\_LinearRegression & 4              & 0.8653              & 16.9197          & 8.7680          \\
            stack2\_\_Ridge            & 4              & 0.8653              & 16.9196          & 8.7680          \\
            stack2\_\_RandomForest     & 4              & 0.8573              & 17.4123          & 9.1919          \\
            stack2\_\_SVR              & 4              & 0.8614              & 17.1598          & 8.6112          \\
            stack2\_\_MLP              & 4              & \textbf{0.8675}     & \textbf{16.7764} & \textbf{8.7338} \\
            \hline
            stack3\_\_GradientBoosting & 3              & 0.8594              & 17.2836          & 8.8767          \\
            stack3\_\_LinearRegression & 3              & 0.8625              & 17.0935          & 8.8214          \\
            stack3\_\_Ridge            & 3              & 0.8625              & 17.0935          & 8.8214          \\
            stack3\_\_RandomForest     & 3              & 0.8527              & 17.6940          & 9.5033          \\
            stack3\_\_SVR              & 3              & 0.8591              & 17.2998          & 8.6736          \\
            stack3\_\_MLP              & 3              & \textbf{0.8639}     & 17.0060          & 8.8602          \\
            \hline
            stack4\_\_GradientBoosting & 3              & 0.8646              & 16.9618          & 8.9358          \\
            stack4\_\_LinearRegression & 3              & 0.8645              & 16.9659          & 8.9309          \\
            stack4\_\_Ridge            & 3              & 0.8645              & 16.9660          & 8.9309          \\
            stack4\_\_RandomForest     & 3              & 0.8495              & 17.8810          & 9.6129          \\
            stack4\_\_SVR              & 3              & 0.8604              & 17.2203          & 8.7753          \\
            stack4\_\_MLP              & 3              & \textbf{0.8668}     & 16.8260          & 8.8613          \\
            \hline
            stack5\_\_GradientBoosting & 2              & 0.8534              & 17.6522          & 8.9497          \\
            stack5\_\_LinearRegression & 2              & 0.8569              & 17.4362          & 8.9133          \\
            stack5\_\_Ridge            & 2              & 0.8569              & 17.4362          & 8.9133          \\
            stack5\_\_RandomForest     & 2              & 0.8233              & 19.3788          & 10.1032         \\
            stack5\_\_SVR              & 2              & 0.8545              & 17.5846          & 8.7686          \\
            stack5\_\_MLP              & 2              & \textbf{0.8561}     & 17.4875          & 8.8770          \\
            \hline
        \end{tabular}
    \caption{Resultados de las diferentes configuraciones de stacking utilizando IFN2 como conjunto explicativo de la variable en tC/ha.}
    \label{tab:stack_ifn2carb}
\end{table}

En este caso (Tabla \cite{tab:stack_ifn2carb}) la técnica de \textit{stacking} no ofrece mejoras que compensen el incremento en la complejidad del modelo. Destaca
el uso de MLP como metamodelo.

\subsubsection{Ifn2 como explicativo para \texttt{carbono\_bruto4} (tC)}

\begin{table}[htbp]
    \centering
    \caption{Resumen del rendimiento de los modelos para la predicción de la variable de carbono en tC con el conjunto de datos que emplea IFN2 como explicativo.}
    \label{tab:resultados_modelos2carb}
    \begin{tabular}{lrrrr}
        \toprule
        Modelo            & CV $R^2$        & Test $R^2$      & Test RMSE (tC)   & MAE (tC)        \\
        \midrule
        Random Forest     & 0.8587          & 0.8607          & 13.0457          & 6.4514          \\
        XGBoost           & 0.8966          & 0.8957          & 11.2890          & 5.5822          \\
        \textbf{CatBoost} & \textbf{0.8974} & \textbf{0.8976} & \textbf{11.1825} & \textbf{5.5840} \\
        LightGBM          & 0.8974          & 0.9007          & 11.0128          & 5.4472          \\
        GBDT              & 0.8940          & 0.8912          & 11.5309          & 5.7745          \\
        BaggedDT          & 0.8700          & 0.8721          & 12.5015          & 6.3180          \\
        AdaBoost          & 0.5702          & 0.5765          & 22.7439          & 19.2913         \\
        KNN               & 0.7803          & 0.7892          & 16.0461          & 7.9430          \\
        MLP               & 0.8860          & 0.8890          & 11.6427          & 6.3052          \\
        SVR               & 0.8155          & 0.8159          & 14.9943          & 7.1427          \\
        BayesianNN        & 0.8348          & 0.8344          & 14.2244          & 7.8642          \\
        \bottomrule
    \end{tabular}
\end{table}

Una vez más, se mantienen las conclusiones extraidas en el análisis de resultados realizado para el modelo entrenado
con la misma variable objetivo pero los datos del IFN2 e IFN3 como explicativos (\ref{tab:resultados_modelos2carb}):
\begin{itemize}
    \item Los modelos presentan una buena capacidad de generalización.
    \item Los modelos basados en árboles de decisión y en \textit{gradient boosting} son los que ofrecen, en general, el mejor equilibrio entre capacidad predictiva y estabilidad.
    \item Algoritmos como AdaBoost o KNN muestran un rendimiento claramente inferior
\end{itemize}

En particular, \textbf{CatBoost} destaca como el modelo con mejor rendimiento global,
alcanzando un $R^2 = 0.8976$ y un RMSE de $11.1825$ tC. Estos valores implican que el modelo
es capaz de explicar una proporción sustancial de la variabilidad del carbono en las parcelas,
reduciendo el error típico de predicción a menos de la mitad de la variabilidad natural de la
variable (SD $\approx 36$ tC).

\begin{table}[htbp]
    \centering
    \small
    \footnotesize
        \begin{tabular}{lcccc}
            \hline
            \textbf{Stack}             & \textbf{Bases} & \textbf{Test $R^2$} & \textbf{RMSE}    & \textbf{MAE}    \\
            \hline
            stack1\_\_GradientBoosting & 6              & 0.9013              & 10.9803          & 5.2917          \\
            stack1\_\_LinearRegression & 6              & 0.9028              & 10.8960          & 5.3982          \\
            stack1\_\_Ridge            & 6              & 0.9028              & 10.8962          & 5.3982          \\
            stack1\_\_RandomForest     & 6              & 0.8946              & 11.3469          & 5.5493          \\
            stack1\_\_SVR              & 6              & 0.9017              & 10.9600          & 5.2741          \\
            stack1\_\_MLP              & 6              & \textbf{0.9043}     & \textbf{10.8148} & \textbf{5.2523} \\
            \hline
            stack2\_\_GradientBoosting & 4              & 0.9016              & 10.9648          & 5.2924          \\
            stack2\_\_LinearRegression & 4              & 0.9027              & 10.9000          & 5.4073          \\
            stack2\_\_Ridge            & 4              & 0.9027              & 10.9002          & 5.4074          \\
            stack2\_\_RandomForest     & 4              & 0.8938              & 11.3894          & 5.6335          \\
            stack2\_\_SVR              & 4              & 0.9016              & 10.9633          & 5.2807          \\
            stack2\_\_MLP              & 4              & \textbf{0.9039}     & \textbf{10.8337} & 5.2856          \\
            \hline
            stack3\_\_GradientBoosting & 3              & 0.8989              & 11.1106          & 5.4026          \\
            stack3\_\_LinearRegression & 3              & 0.9006              & 11.0195          & 5.4354          \\
            stack3\_\_Ridge            & 3              & 0.9006              & 11.0196          & 5.4354          \\
            stack3\_\_RandomForest     & 3              & 0.8868              & 11.7604          & 5.8714          \\
            stack3\_\_SVR              & 3              & 0.8996              & 11.0728          & 5.3318          \\
            stack3\_\_MLP              & 3              & \textbf{0.9002}     & \textbf{11.0388} & 5.4138          \\
            \hline
            stack4\_\_GradientBoosting & 3              & 0.8965              & 11.2463          & 5.4273          \\
            stack4\_\_LinearRegression & 3              & 0.8979              & 11.1665          & 5.5801          \\
            stack4\_\_Ridge            & 3              & 0.8979              & 11.1665          & 5.5801          \\
            stack4\_\_RandomForest     & 3              & 0.8861              & 11.7940          & 5.8454          \\
            stack4\_\_SVR              & 3              & 0.8960              & 11.2707          & 5.4679          \\
            stack4\_\_MLP              & 3              & \textbf{0.8993}     & \textbf{11.0889} & 5.4012          \\
            \hline
            stack5\_\_GradientBoosting & 2              & 0.8995              & 11.0811          & 5.3741          \\
            stack5\_\_LinearRegression & 2              & 0.9007              & 11.0163          & 5.4479          \\
            stack5\_\_Ridge            & 2              & 0.9006              & 11.0164          & 5.4479          \\
            stack5\_\_RandomForest     & 2              & 0.8813              & 12.0431          & 6.0737          \\
            stack5\_\_SVR              & 2              & 0.8988              & 11.1210          & 5.3609          \\
            stack5\_\_MLP              & 2              & \textbf{0.9011}     & \textbf{10.9941} & \textbf{5.3603} \\
            \hline
        \end{tabular}
    \caption{Resultados de las diferentes configuraciones de stacking utilizando IFN2 como conjunto explicativo de la variable en tC.}
    \label{tab:stack_ifn2_tc}
\end{table}

En este caso (Tabla \cite{tab:stack_ifn2_tc}) la técnica de \textit{stacking} no ofrece mejoras que compensen el
incremento en la complejidad del modelo, aunque es cierto que se rompe la barrera del $R^2>0.9$. Destaca
el uso de MLP como metamodelo.

De entre los modelos entrenados para predecir \texttt{carbono\_bruto4} (tC) con IFN3 como explicativo destaca
aquel entrenado con MLP como metamodelo para combinar \texttt{CatBoost, LightGBM, Random Forest} y \texttt{GBDT}, con un $R^2=0.9039$,
un $RMSE=10.8387$ y un $MAE=5.2856$.


\subsection{Anexo: Código de las especies (\texttt{especie\_id})}\label{sec:especies}

\footnotesize
\begin{longtable}{r p{4cm} p{5cm} c c}
    \caption{Relación de especies empleadas en el estudio y metadatos asociados.} \label{anexo:especies}                               \\
    \toprule
    \textbf{Cód.} & \textbf{Nombre}                   & \textbf{Sinonimia}                            & \textbf{Tipo} & \textbf{Grupo} \\
    \midrule
    \endfirsthead
    \caption[]{Relación de especies (continuación).}                                                                                   \\
    \toprule
    \textbf{Cód.} & \textbf{Nombre}                   & \textbf{Sinonimia}                            & \textbf{Tipo} & \textbf{Grupo} \\
    \midrule
    \endhead
    \midrule
    \multicolumn{5}{r}{\emph{Continúa en la siguiente página}}                                                                         \\
    \endfoot
    \bottomrule
    \endlastfoot
    307           & Acacia dealbata                   & Acacia dealbata                               & 1             & 7              \\
    207           & Acacia melanoxylon                & Acacia melanoxylon                            & 1             & 7              \\
    7             & Acacia spp.                       & -                                             & 1             & 7              \\
    392           & Gleditsia triacanthos             & Acacia gleditsia                              & 1             & 7              \\
    92            & Robinia pseudoacacia              & Acacia robinia                                & 1             & 7              \\
    292           & Sophora japonica                  & Acacia sofora                                 & 1             & 7              \\
    515           & Crataegus azarolus                & Espino                                        & 1             & 15             \\
    415           & Crataegus laciniata               & Majoleto                                      & 1             & 15             \\
    315           & Crataegus laevigata               & Espino majuelo                                & 1             & 15             \\
    215           & Crataegus monogyna                & Majuelo                                       & 1             & 15             \\
    15            & Crataegus spp.                    & -                                             & 1             & 15             \\
    30            & Mezcla de coníferas               & Coníferas | excepto pinos                     & 0             & 19             \\
    19            & Otras coníferas                   & -                                             & 0             & 19             \\
    29            & Otros pinos                       & -                                             & 0             & 20             \\
    20            & Pinos                             & -                                             & 0             & 20             \\
    27            & Pinus canariensis                 & -                                             & 0             & 20             \\
    24            & Pinus halepensis                  & -                                             & 0             & 20             \\
    25            & Pinus nigra                       & Pinus laricio | Pinus clusiana                & 0             & 20             \\
    26            & Pinus pinaster                    & Pinus maritima                                & 0             & 20             \\
    23            & Pinus pinea                       & -                                             & 0             & 20             \\
    28            & Pinus radiata                     & Pinus insignis                                & 0             & 20             \\
    21            & Pinus sylvestris                  & -                                             & 0             & 20             \\
    22            & Pinus uncinata                    & Pinus montana | Pinus mugo                    & 0             & 20             \\
    31            & Abies alba                        & Abies pectinata                               & 0             & 31             \\
    32            & Abies pinsapo                     & -                                             & 0             & 31             \\
    235           & Larix decidua                     & Alerce común                                  & 0             & 35             \\
    335           & Larix leptolepis                  & Larix kaempferi | Alerce leptolepis           & 0             & 35             \\
    35            & Larix spp.                        & -                                             & 0             & 35             \\
    435           & Larix x eurolepis                 & Alerce híbrido                                & 0             & 35             \\
    49            & Otros quercus                     & -                                             & 1             & 40             \\
    344           & Quercus alpestris                 & -                                             & 1             & 40             \\
    47            & Quercus canariensis               & Quercus lusitanica var. baetica               & 1             & 40             \\
    44            & Quercus faginea                   & Quercus lusitanica var. faginea               & 1             & 40             \\
    45            & Quercus ilex ssp. ballota         & Quercus rotundifolia                          & 1             & 40             \\
    245           & Quercus ilex ssp. ilex            & -                                             & 1             & 40             \\
    244           & Quercus lusitanica                & Quercus fruticosa | Quejigueta                & 1             & 40             \\
    42            & Quercus petraea                   & Quercus sessiliflora                          & 1             & 40             \\
    243           & Quercus pubescens                 & Quercus pubescens | Quercus humilis           & 1             & 40             \\
    43            & Quercus pyrenaica                 & Quercus toza                                  & 1             & 40             \\
    41            & Quercus robur                     & Quercus pedunculata                           & 1             & 40             \\
    48            & Quercus rubra                     & Quercus borealis                              & 1             & 40             \\
    46            & Quercus suber                     & -                                             & 1             & 40             \\
    253           & Tamarix canariensis               & Tarajal                                       & 1             & 53             \\
    53            & Tamarix spp.                      & -                                             & 1             & 53             \\
    257           & Salix alba                        & Sauce blanco                                  & 1             & 57             \\
    357           & Salix atrocinerea                 & Bardaguera                                    & 1             & 57             \\
    858           & Salix canariensis                 & Sauce canario                                 & 1             & 57             \\
    557           & Salix cantabrica                  & Sauce cantábrico                              & 1             & 57             \\
    657           & Salix caprea                      & Sauce cabruno                                 & 1             & 57             \\
    757           & Salix elaeagnos                   & Sarga                                         & 1             & 57             \\
    857           & Salix fragilis                    & Mimbre                                        & 1             & 57             \\
    957           & Salix purpurea                    & Mimbrera                                      & 1             & 57             \\
    57            & Salix spp.                        & -                                             & 1             & 57             \\
    51            & Populus alba                      & -                                             & 1             & 58             \\
    58            & Populus nigra                     & -                                             & 1             & 58             \\
    52            & Populus tremula                   & -                                             & 1             & 58             \\
    258           & Populus x canadensis              & Populus x euroamericana                       & 1             & 58             \\
    62            & Eucalyptus camaldulensis          & Eucalyptus rostrata                           & 1             & 60             \\
    61            & Eucalyptus globulus               & -                                             & 1             & 60             \\
    364           & Eucalyptus gomphocephalus         & Eucalipto gonfo                               & 1             & 60             \\
    64            & Eucalyptus nitens                 & -                                             & 1             & 60             \\
    464           & Eucalyptus robusta                & -                                             & 1             & 60             \\
    264           & Eucalyptus viminalis              & Eucalipto viminalis                           & 1             & 60             \\
    63            & Otros eucaliptos                  & -                                             & 1             & 60             \\
    65            & Ilex aquifolium                   & -                                             & 1             & 65             \\
    82            & Ilex canariensis                  & -                                             & 1             & 65             \\
    282           & Ilex platyphylla                  & Naranjero                                     & 1             & 65             \\
    268           & Arbutus canariensis               & Madroño canario                               & 1             & 68             \\
    68            & Arbutus unedo                     & -                                             & 1             & 68             \\
    469           & Phoenix canariensis               & Palmera                                       & 1             & 69             \\
    69            & Phoenix spp.                      & -                                             & 1             & 69             \\
    273           & Betula alba                       & Betula verrucosa | Abedul pubescens           & 1             & 73             \\
    373           & Betula pendula                    & Betula hispanica | Abedul péndula             & 1             & 73             \\
    73            & Betula spp.                       & -                                             & 1             & 73             \\
    277           & Tilia cordata                     & Tilo cordata                                  & 1             & 77             \\
    377           & Tilia platyphyllos                & Tilo común                                    & 1             & 77             \\
    77            & Tilia spp.                        & -                                             & 1             & 77             \\
    278           & Sorbus aria                       & Mostajo                                       & 1             & 78             \\
    378           & Sorbus aucuparia                  & Serbal de cazadores                           & 1             & 78             \\
    778           & Sorbus chamaemespilus             & Serbal chame                                  & 1             & 78             \\
    478           & Sorbus domestica                  & Serbal común                                  & 1             & 78             \\
    678           & Sorbus latifolia                  & Serbal de hoja ancha                          & 1             & 78             \\
    78            & Sorbus spp.                       & -                                             & 1             & 78             \\
    578           & Sorbus torminalis                 & Serbal torminal                               & 1             & 78             \\
    79            & Platanus hispanica                & Platanus hybrida                              & 1             & 79             \\
    279           & Platanus orientalis               & Plátano oriental                              & 1             & 79             \\
    80            & Laurisilva                        & -                                             & 1             & 80             \\
    89            & Otras laurisilvas                 & -                                             & 1             & 80             \\
    291           & Buxus balearica                   & Boj de Baleares                               & 1             & 91             \\
    91            & Buxus sempervirens                & -                                             & 1             & 91             \\
    293           & Pistacia atlantica                & Cornicabra canaria                            & 1             & 93             \\
    93            & Pistacia terebinthus              & Cornicabra                                    & 1             & 93             \\
    294           & Laurus azorica                    & Laurel canario                                & 1             & 94             \\
    94            & Laurus nobilis                    & Laurel                                        & 1             & 94             \\
    395           & Prunus avium                      & Cerezo silvestre                              & 1             & 95             \\
    495           & Prunus lusitanica                 & Loro | hija                                   & 1             & 95             \\
    595           & Prunus padus                      & Prunus                                        & 1             & 95             \\
    295           & Prunus spinosa                    & Espino negro                                  & 1             & 95             \\
    95            & Prunus spp.                       & Prunus                                        & 1             & 95             \\
    70            & Mezcla de frondosas de gran porte & Frondosas de gran porte (H.t. > 10 m)         & 1             & 99             \\
    90            & Mezcla de pequeñas frondosas      & Frondosas de pequeño porte (H.t. $\leq$ 10 m) & 1             & 99             \\
    99            & Otras frondosas                   & Otras frondosas                               & 1             & 99             \\
    499           & Morus alba                        & Morera                                        & 1             & 399            \\
    599           & Morus nigra                       & Morera                                        & 1             & 399            \\
    399           & Morus spp.                        & Morera                                        & 1             & 399            \\
    55            & Fraxinus angustifolia             & -                                             & 1             & 455            \\
    255           & Fraxinus excelsior                & Fresno excelsior                              & 1             & 455            \\
    355           & Fraxinus ornus                    & Fresno orno                                   & 1             & 455            \\
    955           & Fraxinus spp.                     & Fresnos                                       & 1             & 455            \\
    17            & Cedrus atlantica                  & -                                             & 0             & 917            \\
    217           & Cedrus deodara                    & Cedrus deodara                                & 0             & 917            \\
    317           & Cedrus libani                     & Cedrus libani                                 & 0             & 917            \\
    917           & Cedrus spp.                       & Cedrus spp.                                   & 0             & 917            \\
    337           & Juniperus cedrus                  & Enebro canario                                & 0             & 917            \\
    236           & Cupressus arizonica               & Ciprés arizónica                              & 0             & 936            \\
    336           & Cupressus lusitanica              & Ciprés lambertiana                            & 0             & 936            \\
    436           & Cupressus macrocarpa              & Ciprés americano                              & 0             & 936            \\
    36            & Cupressus sempervirens            & -                                             & 0             & 936            \\
    936           & Cupressus spp.                    & Cipres                                        & 0             & 936            \\
    37            & Juniperus communis                & -                                             & 0             & 937            \\
    237           & Juniperus oxycedrus               & Enebro oxicedro                               & 0             & 937            \\
    39            & Juniperus phoenicea               & -                                             & 0             & 937            \\
    239           & Juniperus sabina                  & Sabina rastrera                               & 0             & 937            \\
    937           & Juniperus spp.                    & Enebros y sabinas                             & 0             & 937            \\
    38            & Juniperus thurifera               & -                                             & 0             & 937            \\
    238           & Juniperus turbinata               & Sabina canaria                                & 0             & 937            \\
    256           & Ulmus glabra                      & Ulmus montana                                 & 1             & 956            \\
    56            & Ulmus minor                       & Ulmus campestris                              & 1             & 956            \\
    356           & Ulmus pumila                      & Olmo pumilo                                   & 1             & 956            \\
    956           & Ulmus spp.                        & Olmo                                          & 1             & 956            \\
    275           & Juglans nigra                     & Nogal                                         & 1             & 975            \\
    75            & Juglans regia                     & -                                             & 1             & 975            \\
    975           & Juglans spp.                      & -                                             & 1             & 975            \\
    76            & Acer campestre                    & -                                             & 1             & 976            \\
    276           & Acer monspessulanum               & Arce de Montpelier                            & 1             & 976            \\
    376           & Acer negundo                      & Negundo fraxinifolia | Arce negundo           & 1             & 976            \\
    476           & Acer opalus                       & Arce ópalus                                   & 1             & 976            \\
    676           & Acer platanoides                  & Arce platanoide                               & 1             & 976            \\
    576           & Acer pseudoplatanus               & Arce seudoplátano                             & 1             & 976            \\
    976           & Acer spp.                         & Arces                                         & 1             & 976            \\
    97            & Sambucus nigra                    & Saúco negro                                   & 1             & 997            \\
    297           & Sambucus racemosa                 & Saúco racemosa                                & 1             & 997            \\
    997           & Sambucus spp.                     & -                                             & 1             & 997            \\
    11            & Ailanthus altissima               & Ailanthus glandulosa                          & 1             & -              \\
    54            & Alnus glutinosa                   & -                                             & 1             & -              \\
    2             & Amelanchier ovalis                & Guillomo                                      & 1             & -              \\
    88            & Apollonias barbujana              & Apollonias canariensis                        & 1             & -              \\
    98            & Carpinus betulus                  & Carpe                                         & 1             & -              \\
    72            & Castanea sativa                   & Castanea vesca                                & 1             & -              \\
    13            & Celtis australis                  & -                                             & 1             & -              \\
    67            & Ceratonia siliqua                 & -                                             & 1             & -              \\
    18            & Chamaecyparis lawsoniana          & -                                             & 0             & -              \\
    369           & Chamaerops humilis                & Palmito                                       & 1             & -              \\
    9             & Cornus sanguinea                  & -                                             & 1             & -              \\
    74            & Corylus avellana                  & -                                             & 1             & -              \\
    569           & Dracaena draco                    & Drago                                         & 1             & -              \\
    83            & Erica arborea                     & -                                             & 1             & -              \\
    283           & Erica scoparia                    & Tejo | brezo arbóreo escopario                & 1             & -              \\
    5             & Euonymus europaeus                & -                                             & 1             & -              \\
    71            & Fagus sylvatica                   & -                                             & 1             & -              \\
    299           & Ficus carica                      & Higuera                                       & 1             & -              \\
    3             & Frangula alnus                    & Rhamnus frangula                              & 1             & -              \\
    1             & Heberdenia bahamensis             & Heberdenia excelsa                            & 1             & -              \\
    12            & Malus sylvestris                  & -                                             & 1             & -              \\
    60            & Mezcla de eucaliptos              & Eucaliptos                                    & 1             & -              \\
    50            & Mezcla de árboles de ribera       & Árboles ripícolas                             & 1             & -              \\
    81            & Myrica faya                       & -                                             & 1             & -              \\
    281           & Myrica rivasmartinezii            & -                                             & 1             & -              \\
    6             & Myrtus communis                   & -                                             & 1             & -              \\
    87            & Ocotea phoetens                   & -                                             & 1             & -              \\
    66            & Olea europaea                     & Olea oleaster                                 & 1             & -              \\
    59            & Otros árboles ripícolas           & -                                             & 1             & -              \\
    84            & Persea indica                     & -                                             & 1             & -              \\
    8             & Phillyrea latifolia               & -                                             & 1             & -              \\
    86            & Picconia excelsa                  & Notelaea excelsa                              & 1             & -              \\
    33            & Picea abies                       & Picea excelsa                                 & 0             & -              \\
    289           & Pleiomeris canariensis            & Delfino                                       & 1             & -              \\
    34            & Pseudotsuga menziesii             & Pseudotsuga douglasii                         & 0             & -              \\
    16            & Pyrus spp.                        & -                                             & 1             & -              \\
    40            & Quercus                           & -                                             & 1             & -              \\
    4             & Rhamnus alaternus                 & Aladierno                                     & 1             & -              \\
    389           & Rhamnus glandulosa                & Sanguino                                      & 1             & -              \\
    96            & Rhus coriaria                     & Zumaque                                       & 1             & -              \\
    457           & Salix babylonica                  & Sauce llorón                                  & 1             & -              \\
    85            & Sideroxylon marmulano             & -                                             & 1             & -              \\
    10            & Sin asignar                       & Sin asignar                                   & 1             & -              \\
    14            & Taxus baccata                     & -                                             & 0             & -              \\
    219           & Tetraclinis articulata            & Tetraclinis articulata                        & 0             & -              \\
    319           & Thuja spp.                        & Thuja                                         & 0             & -              \\
    489           & Visnea mocanera                   & Mocan                                         & 1             & -              \\
\end{longtable}




\end{document}